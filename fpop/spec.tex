\documentclass{article}
\usepackage{amsmath}
\usepackage{amssymb}
\usepackage{amsthm}
\usepackage{mathpartir}

\newcommand{\KVAR}{\kappa}
\newcommand{\ONE}{\circ}
\newcommand{\MANY}{\ast}

\newcommand{\TASS}[1]{#1\colon\!}

\newcommand{\TVAR}{\alpha}
\newcommand{\TALL}[2]{\forall\TASS{#1}#2.}
\newcommand{\KALL}[1]{\forall#1.}

\newcommand{\LAM}[2]{\lambda\TASS{#1}#2.}
\newcommand{\APP}[1]{#1\,}
\newcommand{\TLAM}[2]{\Lambda\TASS{#1}#2.}
\newcommand{\TAPP}[2]{#1\,[#2]}
\newcommand{\KLAM}[1]{\Lambda#1.}
\newcommand{\KAPP}[2]{#1\,\{#2\}}

\newcommand{\TENV}{\Gamma}
\newcommand{\TENVEMPTY}{\Diamond}

\newcommand{\SPLIT}[3]{#1 \Join #2 = #3}

\newcommand\stepsto{\longrightarrow}

\newtheorem{lemma}{Lemma}

\title{$F^{\ONE}$ with subkinding}
\author{Peter Thiemann}

\begin{document}
\maketitle
Syntax (perhaps more expressions are needed)
\begin{align*}
  k &::= \KVAR \mid \ONE \mid \MANY & \text{kinds, where }  \MANY \sqsubseteq \ONE \\
    &\mid k \to k & \text{constructor kinds}\\
    &\mid \KALL\KVAR k \mid k \le k \Rightarrow k & \text{universal kinds}\\
  t &::= \TVAR \mid t \stackrel{k}{\to} t \mid \TALL\TVAR k t \mid \KALL\KVAR t \mid k \le k \Rightarrow t & \text{types} \\
    & \mid \LAM\TVAR k t \mid \APP tt \mid \KLAM\KVAR t \mid \KAPP t k  & \text{constructors} \\
  e &::= x \mid \LAM x t e \mid \APP ee \mid \TLAM \TVAR k v \mid \TAPP et \mid \KLAM \KVAR v \mid \KAPP ek & \text{expressions} \\
  v &::=  \LAM x t e \mid  \TLAM \TVAR k v \mid \KLAM\KVAR v & \text{values}
  \\
  \TENV &::=
          \TENVEMPTY
          \mid \TENV, \TASS x t
          \mid \TENV, \TASS \TVAR k
          \mid \TENV, k \le k
          \mid \TENV, \KVAR
          & \text{environments}
\end{align*}
%
Kind formation
\begin{mathpar}
  \inferrule{\TENV \models \MANY}{\TENV, \KVAR \vdash \KVAR }

  \inferrule{\TENV \models \MANY}{\TENV \vdash \ONE}

  \inferrule{\TENV \models \MANY}{\TENV \vdash \MANY}

  \inferrule{\TENV \vdash k_1 \\ \TENV \vdash k_2}{\TENV \vdash k_1 \to k_2}

  \inferrule{\TENV, \KVAR \vdash k}{\TENV \vdash \KALL\KVAR k}

  \inferrule{\TENV, k_1 \le k_2 \vdash k_3}{\TENV \vdash k_1 \le k_2 \Rightarrow k_3}
\end{mathpar}
Subkinding
\begin{mathpar}
  \inferrule{\TENV \models \MANY\\k_1 \sqsubseteq k_2}{\TENV \vdash k_1 \le k_2}

  \inferrule{\TENV \models \MANY}{\TENV, k_1 \le k_2 \vdash k_1 \le k_2}

  \inferrule{\TENV \vdash k_2 \le k_1 \\ \TENV \vdash k_1' \le k_2'}{\TENV \vdash k_1 \to k_1' \le k_2 \to k_2'}

  \inferrule{\TENV, \KVAR \vdash k_1 \le k_2}{\TENV \vdash \KALL\KVAR k_1 \le \KALL\KVAR k_2}

  \inferrule[HOW?]{}{\TENV \vdash (k_1 \le k_2 \Rightarrow k_3) \le (k_1' \le k_2' \Rightarrow k_3')}

  \inferrule{\TENV \vdash k}{\TENV \vdash k \le k}

  \inferrule{\TENV \vdash k_1 \le k_2 \\\TENV \vdash k_2 \le k_3}{\TENV \vdash k_1 \le k_3}
\end{mathpar}
Kinding
\begin{mathpar}
  \inferrule{\TENV \vdash t : k \\ \TENV \vdash k \le k' }{\TENV \vdash t : k'}

  \inferrule{\TENV \vdash t_1 : k_1 \\ \TENV \vdash t_2 : k_2}{ \TENV \vdash t_1 \stackrel{k}{\to} t_2 : k}

  \inferrule{}{\TENV, \TVAR:k \vdash \TVAR:k}

  \inferrule{\TENV \vdash k\\ \TENV, \TVAR:k \vdash t:k' \\ \TVAR \notin \TENV}{\TENV \vdash \TALL\TVAR k t : k'}

  \inferrule{\TENV, \KVAR \vdash t: k \\ \KVAR \notin \TENV}{\TENV \vdash \KALL\KVAR t : k}

  \inferrule{\TENV \vdash k_1, k_2 \\
    \TENV, k_1 \le k_2 \vdash t : k}{\TENV \vdash k_1 \le k_2 \Rightarrow t : k}

  \inferrule{\TENV, \TVAR:k \vdash t : k'}{\TENV \vdash \LAM\TVAR k t : k \to k'}

  \inferrule{\TENV \vdash t_1 : k_2 \to k_1 \\ \TENV \vdash t_2 : k_2}{\TENV \vdash \APP{t_1}{t_2} : k_1}

  \inferrule{\TENV,\KVAR \vdash t : k}{\TENV \vdash \KLAM\KVAR t : \KALL\KVAR k}

  \inferrule{\TENV \vdash t : \KALL\KVAR k' \\ \TENV \vdash k}{\TENV \vdash \KAPP t k : k'[\KVAR\mapsto k]}
\end{mathpar}
Environment formation for $k \sqsubseteq \ONE$
\begin{mathpar}
  \inferrule{}{\TENVEMPTY \models k}

  \inferrule{\TENV \models k \\ \TENV \vdash x:t \\ \TENV \vdash t : k' \\ \TENV \vdash k' \le k }{
    \TENV, \TASS x t \models k}

  \inferrule{\TENV \models k \\ \TENV \vdash k' \\ \TVAR\notin\TENV }{\TENV, \TASS\TVAR{k'} \models k}

  \inferrule{\TENV \models k \\ \TENV \vdash k', k''}{\TENV, k' \le k'' \models k}

  \inferrule{\TENV \models k \\ \KVAR\notin \TENV}{\TENV, \KVAR \models k}
\end{mathpar}
Type conversion (congruence rules omitted)
\begin{mathpar}
  \inferrule{\TENV, \TASS\TVAR k' \vdash t:k \\ \TENV \vdash t' : k' }{
    \TENV \vdash \APP{(\LAM\TVAR {k'} t)}{t'} = t[\TVAR \mapsto t'] : k}

  \inferrule{
    \TENV, \KVAR \vdash t : k \\ \TENV \vdash k'
  }{
    \TENV \vdash \KAPP{(\KLAM\KVAR t)}{k'} = t[\KVAR \mapsto k'] : k[\KVAR \mapsto k']}
\end{mathpar}
Environment splitting
\begin{mathpar}
  \inferrule{}{
    \SPLIT{\TENVEMPTY}{\TENVEMPTY}{\TENVEMPTY}}

  \inferrule{
    \SPLIT{\TENV_1}{\TENV_2}{\TENV} \\
    \TENV \vdash t : k \\
    \TENV \vdash k \le \MANY
  }{
    \SPLIT{\TENV_1, \TASS x t}{\TENV_2, \TASS x t}{\TENV, \TASS x t}}

  \inferrule{
    \SPLIT{\TENV_1}{\TENV_2}{\TENV} 
  }{
    \SPLIT{\TENV_1, \TASS x t}{\TENV_2}{\TENV, \TASS x t}}

  \inferrule{
    \SPLIT{\TENV_1}{\TENV_2}{\TENV} 
  }{
    \SPLIT{\TENV_1}{\TENV_2, \TASS x t}{\TENV, \TASS x t}}

  \inferrule{
    \SPLIT{\TENV_1}{\TENV_2}{\TENV}
  }{
    \SPLIT{\TENV_1, \TASS \TVAR k}{\TENV_2, \TASS \TVAR k}{\TENV, \TASS \TVAR k}}

  \inferrule{
    \SPLIT{\TENV_1}{\TENV_2}{\TENV}
  }{
    \SPLIT{\TENV_1, k \le k'}{\TENV_2, k \le k'}{\TENV, k \le k'}}

  \inferrule{
    \SPLIT{\TENV_1}{\TENV_2}{\TENV}
  }{
    \SPLIT{\TENV_1, \KVAR}{\TENV_2, \KVAR}{\TENV, \KVAR}}
\end{mathpar}
Typing rules
\begin{mathpar}
  \inferrule[Conv]{
    \SPLIT{\TENV_1}{\TENV_2}{\TENV} \\
    \TENV_1 \vdash e:t \\ \TENV_2 \vdash t = t' : k }{\TENV \vdash e : t'}

  \inferrule[Var]{\TENV \models \MANY }{\TENV, x:t \vdash x:t }

  \inferrule[Lam]
  {\TENV \models k \\ \TENV, x:t \vdash e : t'}
  {\TENV \vdash \LAM xte : t \stackrel{k}\to t'}

  \inferrule[App]
  { \SPLIT{\TENV_1}{\TENV_2}{\TENV} \\
    \TENV_1 \vdash e : t' \stackrel{k}\to t \\
    \TENV_2 \vdash e' : t'}
  {\TENV \vdash \APP ee' : t}

  \inferrule[TLam]
  {\TENV, \TASS\TVAR k \vdash v : t }
  {\TENV \vdash \TLAM\TVAR k v : \TALL \TVAR k t}

  \inferrule[TApp]
  {
    \SPLIT{\TENV_1}{\TENV_2}{\TENV} \\
    \TENV_1 \vdash e : \TALL \TVAR k t' \\
    \TENV_2 \vdash t : k     
  }
  {\TENV \vdash \TAPP e t : t'[\TVAR \mapsto t]}

  \inferrule[KLam]
  {\TENV, \KVAR \vdash e : t \\ \KVAR \notin \TENV}
  {\TENV \vdash \KLAM\KVAR e : \KALL\KVAR t}

  \inferrule[KApp]
  {
    \SPLIT{\TENV_1}{\TENV_2}{\TENV} \\
    \TENV_1 \vdash e : \KALL\KVAR t \\ \TENV_2 \vdash k}
  {\TENV \vdash \KAPP e k : t[\KVAR \mapsto k]}

  \inferrule[CIntro]
  {\TENV,   k_1 \le k_2 \vdash e : t}
  {\TENV \vdash e :  k_1 \le k_2 \Rightarrow t}

  \inferrule[CElim]
  {\TENV \vdash e :  k_1 \le k_2 \Rightarrow t \\ \TENV \vdash k_1 \le k_2}
  {\TENV \vdash e : t}
\end{mathpar}

Simple small-step semantics (for type preservation) with
$e[x \mapsto v]$ standing for capture-avoiding substitution of $v$ for
$x$ in $e$. Call-by-value as in $F^{\ONE}$.
\begin{mathpar}
  \inferrule{}{\APP{(\LAM x t e)}v \stepsto e[x \mapsto v]}

  \inferrule{}{\TAPP{(\TLAM \TVAR k v)}t \stepsto v[\TVAR \mapsto t]}

  \inferrule{}{\KAPP{(\KLAM \KVAR v)}k \stepsto v[\KVAR \mapsto k]}

  \inferrule{e_1 \stepsto e_1'
  }{\APP{e_1}{e_2} \stepsto \APP{e_1'}{e_2}}

  \inferrule{e \stepsto e'}{\APP v e \stepsto \APP v e'}

  \inferrule{e \stepsto e'}{\TAPP{e} t \stepsto \TAPP{e'} t}

  \inferrule{e \stepsto e'}{\KAPP{e} k \stepsto \KAPP{e'} k}
\end{mathpar}

\clearpage
\begin{lemma}[Value Substitution]
  Suppose that $\TENV, x:t_x \vdash e : t$
  and $\TENVEMPTY \vdash v : t'$
  and $\TENVEMPTY \vdash t_x = t' : k'$.
  Then $\TENV \vdash e[x \mapsto v] : t$.
\end{lemma}
\begin{proof}
  As the conversion assumes the empty environment, it cannot be
  affected by adding further assumptions.

  The proof proceeds by induction on the derivation of  $\TENV, x:t_x
  \vdash e : t$ and produces a derivation for the term after
  substitution. The only interesting case is the one for the
  \TirName{Var} rule when the variable is $x$:
  \begin{mathpar}
    \inferrule[Var]{\TENV \models \MANY }{\TENV, x:t_x \vdash x:t_x }
  \end{mathpar}
  By unrestricted weakening we have that
  \begin{gather*}
    \TENV \vdash v : t' \\
    \TENV \vdash t_x = t' : k'
  \end{gather*}
  and hence by \TirName{Conv} and symmetry of $=$, we have the desired outcome
  \begin{gather*}
    \TENV \vdash v : t_x
  \end{gather*}
  
\end{proof}
\begin{lemma}[Type Preservation]~\\
  Suppose that $\TENVEMPTY \vdash e : t$ and $e \stepsto e'$.
  Then $\TENVEMPTY \vdash e' : t$.
\end{lemma}
\begin{proof}
  \textbf{Case} ${\APP{(\LAM x t e)}v \stepsto e[x \mapsto v]}$.

  By assumption $\TENVEMPTY \vdash \APP{(\LAM x {t_x} e)}v : t$.
  It is not straightforward to invert this judgment as the typing is
  not syntax driven. Thus, there may be a sequence of \TirName{Conv},
  \TirName{CIntro}, and \TirName{CElim} rules on top of the
  \TirName{App} rule. However, we can rearrange the typing derivation
  by moving all such rules into the first (function) subterm of the
  \TirName{App} rule (which requires congruence of conversion as well
  as additional lemmas proving that $t' \to (k_1\le k_2 \Rightarrow t) = k_1\le k_2
  \Rightarrow t' \to t$).

  Thus we can apply inversion to find
  \begin{gather}
    \label{eq:1}
    \TENVEMPTY \vdash \LAM x {t_x} e : t' \stackrel{k}\to t
    \\
    \label{eq:2}
    \TENVEMPTY \vdash v : t'
  \end{gather}
  Here $t'$ is a type \emph{convertible} to $t_x$ as \TirName{Lam} may
  not be the last rule applied to obtain \eqref{eq:1}. There could be
  a sequence of \TirName{CIntro} and \TirName{CElim} rules, but they
  would have to cancel out as \TirName{Lam} expects an arrow type:
  \begin{gather}
    \label{eq:3}
    \TENVEMPTY \vdash t_x = t' : k' \text{ where }k' \sqsubseteq \ONE
  \end{gather}
\end{proof}

TODO: More elaborate semantics that tracks linearity.
Templates in $F^{\ONE}$, work by Morris, Bernardi, and others.
\end{document}

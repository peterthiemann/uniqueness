\documentclass{article}
\usepackage{amsmath}
\usepackage{amssymb}
\usepackage{amsthm}
\usepackage{mathpartir}

\newcommand{\KVAR}{\kappa}
\newcommand{\ONE}{\circ}
\newcommand{\MANY}{\ast}

\newcommand{\TASS}[1]{#1\colon\!}

\newcommand{\TVAR}{\alpha}
\newcommand{\TALL}[2]{\forall\TASS{#1}#2.}
\newcommand{\KALL}[1]{\forall#1.}

\newcommand{\LAM}[2]{\lambda\TASS{#1}#2.}
\newcommand{\APP}[1]{#1\,}
\newcommand{\TLAM}[2]{\Lambda\TASS{#1}#2.}
\newcommand{\TAPP}[2]{#1\,[#2]}
\newcommand{\KLAM}[1]{\Lambda#1.}
\newcommand{\KAPP}[2]{#1\,\{#2\}}

\newcommand{\KENV}{\Delta}
\newcommand{\KENVEMPTY}{\Diamond}
\newcommand{\TENV}{\Gamma}
\newcommand{\TENVEMPTY}{\Diamond}

\newcommand{\SPLIT}[3]{#1 \Join #2 = #3}

\newcommand\stepsto{\longrightarrow}

\newtheorem{lemma}{Lemma}

\title{$F^{\ONE}$ with subkinding}
\author{Peter Thiemann}

\begin{document}
\maketitle
Syntax (perhaps more expressions are needed)
\begin{align*}
  k &::= \KVAR \mid \ONE \mid \MANY & \text{kinds, where }  \MANY \sqsubseteq \ONE \\
    &\mid k \to k & \text{constructor kinds}\\
    &\mid \KALL\KVAR k & \text{universal kinds}\\
    &\mid k \le k \Rightarrow k & \text{constrained kinds}\\
  t &::= \TVAR \mid t \stackrel{k}{\to} t \mid \TALL\TVAR k t \mid \KALL\KVAR t \mid k \le k \Rightarrow t & \text{types} \\
    & \mid \LAM\TVAR k t \mid \APP tt \mid \KLAM\KVAR t \mid \KAPP t k  & \text{constructors} \\
  e &::= x \mid \LAM x t e \mid \APP ee \mid \TLAM \TVAR k v \mid \TAPP et \mid \KLAM \KVAR v \mid \KAPP ek & \text{expressions} \\
  v &::=  \LAM x t e \mid  \TLAM \TVAR k v \mid \KLAM\KVAR v & \text{values}
  \\
  \TENV &::=
          \TENVEMPTY
          \mid \TENV, \TASS x t
                                    & \text{type environments}
  \\
  \KENV &::= \KENVEMPTY
          \mid \TENV, \TASS \TVAR k
          \mid \TENV, k \le k
          \mid \TENV, \KVAR
                                    & \text{kind environments}
\end{align*}
%
Kind environment formation
\begin{mathpar}
  \inferrule{}{\KENVEMPTY \models }

  \inferrule{\KENV \models \\ \KENV \vdash k \\ \TVAR\notin\KENV }{\KENV, \TASS\TVAR{k} \models}

  \inferrule{\KENV \models  \\ \KENV \vdash k_1, k_2}{\KENV, k_1 \le k_2 \models}

  \inferrule{\KENV \models \\ \KVAR\notin \TENV}{\KENV, \KVAR \models }
\end{mathpar}
Kind formation
\begin{mathpar}
  \inferrule{\KENV \models}{\KENV, \KVAR \vdash \KVAR }

  \inferrule{\KENV \models}{\KENV \vdash \ONE}

  \inferrule{\KENV \models}{\KENV \vdash \MANY}

  \inferrule{\KENV \vdash k_1, k_2}{\KENV \vdash k_1 \to k_2}

  \inferrule{\KENV, \KVAR \vdash k}{\KENV \vdash \KALL\KVAR k}

  \inferrule{\KENV, k_1 \le k_2 \vdash k_3}{\KENV \vdash k_1 \le k_2 \Rightarrow k_3}
\end{mathpar}
Subkinding
\begin{mathpar}
  \inferrule{\KENV \models\\k_1 \sqsubseteq k_2}{\KENV \vdash k_1 \le k_2}

  \inferrule{\KENV \vdash \KVAR}{\KENV \vdash \KVAR \le \KVAR}

  \inferrule{\KENV, k_1 \le k_2 \models}{\KENV, k_1 \le k_2 \vdash k_1 \le k_2}

  \inferrule{\KENV \vdash k_2 \le k_1 \\ \KENV \vdash k_1' \le k_2'}{\KENV \vdash k_1 \to k_1' \le k_2 \to k_2'}

  \inferrule{\KENV, \KVAR \vdash k_1 \le k_2}{\KENV \vdash \KALL\KVAR k_1 \le \KALL\KVAR k_2}

  \inferrule[HOW?]{}{\KENV \vdash (k_1 \le k_2 \Rightarrow k_3) \le (k_1' \le k_2' \Rightarrow k_3')}

  \inferrule{\KENV \vdash k_1 \le k_2 \\\KENV \vdash k_2 \le k_3}{\KENV \vdash k_1 \le k_3}
\end{mathpar}
Kinding
\begin{mathpar}
  \inferrule[KSub]{\KENV \vdash t : k \\ \KENV \vdash k \le k' }{\KENV \vdash t : k'}

  \inferrule[KFun]{\KENV \vdash t_1 : \ONE \\ \KENV \vdash t_2 : \ONE}{ \KENV \vdash t_1 \stackrel{k}{\to} t_2 : k}

  \inferrule[KVar]{}{\KENV, \TVAR:k \vdash \TVAR:k}

  \inferrule[KTAll]{\KENV \vdash k\\ \KENV, \TVAR:k \vdash t:k' \\ \TVAR \notin \KENV}{\KENV \vdash \TALL\TVAR k t : k'}

  \inferrule[KKAll]{\KENV, \KVAR \vdash t: k \\ \KVAR \notin \KENV}{\KENV \vdash \KALL\KVAR t : k}

  \inferrule[KConst]{\KENV \vdash k_1, k_2 \\
    \KENV, k_1 \le k_2 \vdash t : k}{\KENV \vdash k_1 \le k_2 \Rightarrow t : k}

  \inferrule[KTLam]{\KENV, \TVAR:k \vdash t : k'}{\KENV \vdash \LAM\TVAR k t : k \to k'}

  \inferrule[KTApp]{\KENV \vdash t_1 : k_2 \to k_1 \\ \KENV \vdash t_2 : k_2}{\KENV \vdash \APP{t_1}{t_2} : k_1}

  \inferrule[KKAll]{\KENV,\KVAR \vdash t : k}{\KENV \vdash \KLAM\KVAR t : \KALL\KVAR k}

  \inferrule[KKApp]{\KENV \vdash t : \KALL\KVAR k' \\ \KENV \vdash k}{\KENV \vdash \KAPP t k : k'[\KVAR\mapsto k]}
\end{mathpar}
Type environment formation
\begin{mathpar}
  \inferrule{\KENV \models \\ \KENV \vdash k \le \ONE }{\KENV; \TENVEMPTY \models k}

  \inferrule{\KENV; \TENV \models k \\  \KENV \vdash t : k' \\ \KENV \vdash k' \le k \\ x \notin \TENV }{
    \KENV; \TENV, \TASS x t \models k}
\end{mathpar}
Type conversion (congruence rules omitted)
\begin{mathpar}
  \inferrule[Conv-Beta]{\KENV, \TASS\TVAR k' \vdash t:k \\ \KENV \vdash t' : k' }{
    \KENV \vdash \APP{(\LAM\TVAR {k'} t)}{t'} = t[\TVAR \mapsto t'] : k}

  \inferrule[Conv-KSubst]{
    \KENV, \KVAR \vdash t : k \\ \KENV \vdash k'
  }{
    \KENV \vdash \KAPP{(\KLAM\KVAR t)}{k'} = t[\KVAR \mapsto k'] : k[\KVAR \mapsto k']}
\end{mathpar}
Type environment splitting
\begin{mathpar}
  \inferrule{}{
    \KENV \vdash \SPLIT{\TENVEMPTY}{\TENVEMPTY}{\TENVEMPTY}}

  \inferrule{
    \KENV \vdash \SPLIT{\TENV_1}{\TENV_2}{\TENV} \\
    \KENV \vdash t : k \\
    \KENV \vdash k \le \MANY
  }{
    \KENV \vdash \SPLIT{\TENV_1, \TASS x t}{\TENV_2, \TASS x t}{\TENV, \TASS x t}}

  \inferrule{
    \KENV \vdash \SPLIT{\TENV_1}{\TENV_2}{\TENV} 
  }{
    \KENV \vdash \SPLIT{\TENV_1, \TASS x t}{\TENV_2}{\TENV, \TASS x t}}

  \inferrule{
    \KENV \vdash \SPLIT{\TENV_1}{\TENV_2}{\TENV} 
  }{
    \KENV \vdash \SPLIT{\TENV_1}{\TENV_2, \TASS x t}{\TENV, \TASS x t}}
\end{mathpar}
Typing rules
\begin{mathpar}
  \inferrule[Conv]{
    \KENV; \TENV \vdash e:t \\ \KENV \vdash t = t' : k }{\KENV; \TENV \vdash e : t'}

  \inferrule[Var]{ \KENV; \TENV \models \MANY \\ \KENV \vdash t : k \\ \KENV \vdash k \le \ONE }{\KENV; \TENV, x:t \vdash x:t }

  \inferrule[Lam]
  {\KENV; \TENV \models k \\ \KENV; \TENV, x:t \vdash e : t'}
  {\KENV;\TENV \vdash \LAM xte : t \stackrel{k}\to t'}

  \inferrule[App]
  { \KENV \vdash \SPLIT{\TENV_1}{\TENV_2}{\TENV} \\
    \KENV;\TENV_1 \vdash e : t' \stackrel{k}\to t \\
    \KENV; \TENV_2 \vdash e' : t'}
  { \KENV; \TENV \vdash \APP ee' : t}

  \inferrule[TLam]
  {\KENV, \TASS\TVAR k; \TENV \vdash v : t \\ \TVAR \notin \KENV}
  {\KENV; \TENV \vdash \TLAM\TVAR k v : \TALL \TVAR k t}

  \inferrule[TApp]
  {
    \KENV; \TENV \vdash e : \TALL \TVAR k t' \\
    \KENV \vdash t : k     
  }
  { \KENV; \TENV \vdash \TAPP e t : t'[\TVAR \mapsto t]}

  \inferrule[KLam]
  { \KENV, \KVAR; \TENV \vdash e : t \\ \KVAR \notin \KENV}
  { \KENV; \TENV \vdash \KLAM\KVAR e : \KALL\KVAR t}

  \inferrule[KApp]
  { \KENV; \TENV \vdash e : \KALL\KVAR t \\ \KENV \vdash k}
  { \KENV; \TENV \vdash \KAPP e k : t[\KVAR \mapsto k]}

  \inferrule[CIntro]
  { \KENV,   k_1 \le k_2; \TENV \vdash e : t}
  { \KENV; \TENV \vdash e :  k_1 \le k_2 \Rightarrow t}

  \inferrule[CElim]
  { \KENV; \TENV \vdash e :  k_1 \le k_2 \Rightarrow t \\ \KENV \vdash k_1 \le k_2}
  { \KENV; \TENV \vdash e : t}
\end{mathpar}

Simple small-step semantics (for type preservation) with
$e[x \mapsto v]$ standing for capture-avoiding substitution of $v$ for
$x$ in $e$. Call-by-value as in $F^{\ONE}$.
\begin{mathpar}
  \inferrule{}{\APP{(\LAM x t e)}v \stepsto e[x \mapsto v]}

  \inferrule{}{\TAPP{(\TLAM \TVAR k v)}t \stepsto v[\TVAR \mapsto t]}

  \inferrule{}{\KAPP{(\KLAM \KVAR v)}k \stepsto v[\KVAR \mapsto k]}

  \inferrule{e_1 \stepsto e_1'
  }{\APP{e_1}{e_2} \stepsto \APP{e_1'}{e_2}}

  \inferrule{e \stepsto e'}{\APP v e \stepsto \APP v e'}

  \inferrule{e \stepsto e'}{\TAPP{e} t \stepsto \TAPP{e'} t}

  \inferrule{e \stepsto e'}{\KAPP{e} k \stepsto \KAPP{e'} k}
\end{mathpar}

\clearpage
\begin{lemma}[Weakening]
  Let $\mathcal{A}$ range over assumptions in kind environments.
  Let $\mathcal{J}$ range over judgments in the context of a kind environment. 

  If $\KENV \vdash \mathcal{J}$ and $\KENV, \mathcal{A} \models$, then $\KENV, \mathcal{A} \vdash \mathcal{J}$.
\end{lemma}
\begin{lemma}[Unrestricted Weakening]
  Suppose that $\KENV; \TENV \vdash e : t$, $x\notin \TENV$, and $\KENV \vdash t_x : \MANY$.
  Then  $\KENV; \TENV, x : t_x \vdash e : t$.
\end{lemma}
\begin{lemma}[Value Substitution]\label{lemma:value-substitution}
  Suppose that $\KENV; \TENV, x:t_x \vdash e : t$
  and $\KENV; \TENVEMPTY \vdash v : t'$
  and $\KENV \vdash t_x = t' : k'$.
  Then $\KENV; \TENV \vdash e[x \mapsto v] : t$.
\end{lemma}
\begin{proof}
  As the conversion assumes the empty environment, it cannot be
  affected by adding further assumptions.

  The proof proceeds by induction on the derivation of  $\KENV; \TENV, x:t_x
  \vdash e : t$ and produces a derivation for the term after
  substitution. The only interesting case is the one for the
  \TirName{Var} rule when the variable is $x$:
  \begin{mathpar}
    \inferrule[Var]{ \KENV;\TENV \models \MANY \\ \KENV \models t_x : k' \\ \KENV \vdash k' \le \ONE }{\TENV, x:t_x \vdash x:t_x }
  \end{mathpar}
  By unrestricted weakening we have that
  \begin{gather*}
    \KENV; \TENV \vdash v : t' \\
    \KENV \vdash t_x = t' : k'
  \end{gather*}
  and hence by \TirName{Conv} and symmetry of $=$, we have the desired outcome
  \begin{gather*}
    \KENV; \TENV \vdash v : t_x
  \end{gather*}
\end{proof}
\begin{lemma}[Inversion for Function Type]\label{lemma:inversion-function}
  If $\KENV; \TENVEMPTY \vdash \LAM x {t_x} e : t_f$,
  then there is some $n\ge0$ and $k_{i1}$, $k_{i2}$ (for $1\le i\le n$) such that 
  \begin{gather}
    \KENV \vdash t_f = k_{11}\le k_{12}\Rightarrow \dots k_{n1}\le k_{n2} \Rightarrow t_x \stackrel{k}\to t : k'
    \\
    \KENV; \TASS x{t_x} \vdash e : t
    \\
    \KENV \vdash k' \le \ONE
    \\
    \KENV \vdash k \le \ONE
    \mathrm{.}
  \end{gather}
\end{lemma}
\begin{proof}
  Induction on the derivation of
  $\KENV; \TENVEMPTY \vdash \LAM x {t_x} e : t_f$.

  \textbf{Case} \TirName{Lam}. Rule inversion yields $\KENV \models$,
  $\KENV \vdash k \le \ONE$, $t_f = t_x \stackrel{k}\to t$, and
  $\KENV; \TASS x{t_x} \vdash e : t$. By reflexivity of conversion
  $\KENV \vdash t_f = t_x \stackrel{k}\to t : k'$, for some
  $\KENV \vdash k' \le \ONE$, and the claim holds for $n=0$.

  \textbf{Case} \TirName{Conv}. Rule inversion yields
  $\KENV; \TENVEMPTY \vdash \LAM x {t_x} e : t_1$ and
  $\KENV \vdash t_1 = t_f : k'$ with $\KENV \vdash k' \le
  \ONE$. Conclude by induction and by transitivity of conversion.

  \textbf{Case} \TirName{CIntro}. Rule inversion yields that
  $t_f = k_1 \le k_2 \Rightarrow t_1$ and
  $\KENV, k_1 \le k_2; \TENVEMPTY \vdash e : t_1$. By induction,
  $\KENV \vdash t_1 = k_{11}\le k_{12}\Rightarrow \dots k_{n1}\le
  k_{n2} \Rightarrow t_x \stackrel{k}\to t : k'$ so by congruence
  $\KENV \vdash t_f = k_1 \le k_2 \Rightarrow k_{11}\le k_{12}\Rightarrow \dots k_{n1}\le
  k_{n2} \Rightarrow t_x \stackrel{k}\to t : k'$, which proves the claim.

  \textbf{Case} \TirName{CElim}. Rule inversion yields that
  $\KENV; \TENVEMPTY \vdash \LAM x {t_x} e : t_1$ with
  $t_1 = k_1 \le k_2 \Rightarrow t_f$ and $\KENV \vdash k_1 \le
  k_2$. By induction
  $\KENV \vdash t_1 = k_1 \le k_2 \Rightarrow k_{11}\le
  k_{12}\Rightarrow \dots k_{n1}\le k_{n2} \Rightarrow t_x
  \stackrel{k}\to t : k'$ so that by congruence
  $\KENV \vdash t_f = k_{11}\le
  k_{12}\Rightarrow \dots k_{n1}\le k_{n2} \Rightarrow t_x
  \stackrel{k}\to t : k'$, which proves the claim.
\end{proof}
\begin{lemma}[Type Preservation]~\\
  Suppose that $\KENV; \TENVEMPTY \vdash e : t$ and $e \stepsto e'$.
  Then $\KENV;\TENVEMPTY \vdash e' : t$.
\end{lemma}
\begin{proof}
  In each case we proceed by induction on the derivation of the
  assumed judgment.  If the top-level rule is \TirName{Conv},
  \TirName{CIntro}, and \TirName{CElim}, then the claim holds by
  induction.

  \textbf{Case} ${\APP{(\LAM x {t_x} e)}v \stepsto e[x \mapsto v]}$.

  By assumption $\KENV;\TENVEMPTY \vdash \APP{(\LAM x {t_x} e)}v : t$.


  If the top-level rule is \TirName{App} rule, we can apply inversion to find
  \begin{gather}
    \label{eq:1}
    \KENV; \TENVEMPTY \vdash \LAM x {t_x} e : t' \stackrel{k}\to t
    \\
    \label{eq:2}
    \KENV; \TENVEMPTY \vdash v : t'
  \end{gather}
  By Lemma~\ref{lemma:inversion-function}, there exists an $n\ge0$ and $k_{i1}$, $k_{i2}$ (for $1\le i\le n$) such that
  \begin{gather}
    \KENV \vdash t' \stackrel{k}\to t = k_{11}\le k_{12}\Rightarrow \dots k_{n1}\le k_{n2} \Rightarrow t_x \stackrel{k}\to t : k'
    \\
    \KENV; \TASS x{t_x} \vdash e : t
  \end{gather}
  Hence, $n=0$ and 
  \begin{gather}
    \KENV \vdash t' \stackrel{k}\to t = t_x \stackrel{k}\to t : k'
  \end{gather}
  from which we can follow
  \begin{gather}
    \label{eq:3}
    \KENV \vdash t_x = t' : k' \text{ where } \KENV \vdash k' \le \ONE
  \end{gather}
  With these assumptions, we apply value substitution
  (Lemma~\ref{lemma:value-substitution}) to obtain the result.

  \textbf{Case} $\TAPP{(\TLAM \TVAR k v)}{t'} \stepsto v[\TVAR \mapsto t']$.

  By assumption
  $\KENV;\TENVEMPTY \vdash \TAPP{(\TLAM \TVAR k v)}{t'} :
  t$. Inversion of the \TirName{TApp} rule yields
  \begin{gather}
    \KENV; \TENVEMPTY \vdash \TLAM \TVAR k v : \TALL\TVAR k t_0 \\
    \KENV \vdash t' : k \\
    t = t_0 [\TVAR\mapsto t']
  \end{gather}

  \textbf{!!!TBC}
\end{proof}

TODO: More elaborate semantics that tracks linearity.
Templates in $F^{\ONE}$, work by Morris, Bernardi, and others.
\end{document}

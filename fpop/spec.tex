\documentclass{article}
\usepackage{amsmath}
\usepackage{amssymb}
\usepackage{amsthm}
\usepackage{mathpartir}

\newcommand{\Addr}{a}

\newcommand{\KVAR}{\kappa}
\newcommand{\ONE}{\circ}
\newcommand{\MANY}{\ast}

\newcommand{\KINDU}{\textbf{U}}
\newcommand{\KINDA}{\textbf{A}}

\newcommand{\BORROW}[1][\iota]{\&^{#1}\,}

\newcommand{\TASS}[1]{#1\colon\!}

\newcommand{\TVAR}{\alpha}
\newcommand{\TALL}[2]{\forall\TASS{#1}#2.}
\newcommand{\KALL}[1]{\forall#1.}

\newcommand{\LAM}[3][{}]{\lambda^{#1}\TASS{#2}#3.}
\newcommand{\APP}[1]{#1\,}
\newcommand{\TLAM}[3][{}]{\Lambda^{#1}\TASS{#2}#3.}
\newcommand{\TAPP}[2]{#1\,[#2]}
\newcommand{\KLAM}[2][{}]{\Lambda^{#1}#2.}
\newcommand{\KAPP}[2]{#1\,\{#2\}}

\newcommand{\KENV}{\Delta}
\newcommand{\KENVEMPTY}{\Diamond}
\newcommand{\TENV}{\Gamma}
\newcommand{\TENVEMPTY}{\Diamond}

\newcommand{\SPLIT}[3]{#1 \ltimes #2 = #3} % used to be \bowtie

\newcommand\stepsto{\longrightarrow}

\newtheorem{lemma}{Lemma}

\title{$F^{\ONE}$ with subkinding}
\author{Peter Thiemann}

\begin{document}
\maketitle
Syntax and rules updated --- possibly inconsistent

Syntax (perhaps more expressions are needed)
\begin{align*}
  k &::= \KVAR \mid \ONE \mid \MANY & \text{kinds, where }  \MANY \sqsubseteq \ONE \\
    &\mid k \to k & \text{constructor kinds}\\
    &\mid k \le k \Rightarrow k & \text{constrained kinds}\\
    &\mid \KALL\KVAR k & \text{universal kinds}\\
    % &\mid k \le k \Rightarrow k & \text{constrained kinds}\\
  t &::= \TVAR \mid t \stackrel{k}{\to} t \mid \TALL\TVAR k t \mid \KALL\KVAR t \mid k \le k \Rightarrow t & \text{types} \\
    & \mid \LAM\TVAR k t \mid \APP tt \mid \KLAM\KVAR t \mid \KAPP t k  & \text{constructors} \\
  e &::= x \mid \LAM[k] x t e \mid \APP ee \mid \TLAM[k] \TVAR k v \mid \TAPP et \mid \KLAM[k] \KVAR v \mid \KAPP ek & \text{expressions} \\
  v &::=  \LAM[k] x t e \mid  \TLAM[k] \TVAR k v \mid \KLAM[k]\KVAR v & \text{values}
  \\
  \TENV &::=
          \TENVEMPTY
          \mid \TENV, \TASS x t
                                    & \text{type environments}
  \\
  \KENV &::= \KENVEMPTY
          \mid \TENV, \TASS \TVAR k
          \mid \TENV, k \le k
          \mid \TENV, \KVAR
                                    & \text{kind environments}
\end{align*}
%
Kind environment formation
\begin{mathpar}
  \inferrule{}{\KENVEMPTY \models }

  \inferrule{\KENV \models \\ \KENV \vdash k \\ \TVAR\notin\KENV }{\KENV, \TASS\TVAR{k} \models}

  \inferrule{\KENV \models  \\ \KENV \vdash k_1, k_2}{\KENV, k_1 \le k_2 \models}

  \inferrule{\KENV \models \\ \KVAR\notin \TENV}{\KENV, \KVAR \models }
\end{mathpar}
Kind formation
\begin{mathpar}
  \inferrule{\KENV, \KVAR, \KENV' \models}{\KENV, \KVAR, \KENV' \vdash \KVAR }

  \inferrule{\KENV \models}{\KENV \vdash \ONE}

  \inferrule{\KENV \models}{\KENV \vdash \MANY}

  \inferrule{\KENV \vdash k_1, k_2}{\KENV \vdash k_1 \to k_2}

  \inferrule{\KENV, \KVAR \vdash k}{\KENV \vdash \KALL\KVAR k}
  % 
  % \inferrule{\KENV, k_1 \le k_2 \vdash k_3}{\KENV \vdash k_1 \le k_2 \Rightarrow k_3}
\end{mathpar}
Subkinding
\begin{mathpar}
  \inferrule{\KENV \models\\k_1 \sqsubseteq k_2}{\KENV \vdash k_1 \le k_2}

  \inferrule{\KENV \vdash \KVAR}{\KENV \vdash \KVAR \le \KVAR}

  \inferrule{\KENV, k_1 \le k_2, \KENV' \models}{\KENV, k_1 \le k_2, \KENV' \vdash k_1 \le k_2}

  \inferrule{\KENV \vdash k_2 \le k_1 \\ \KENV \vdash k_1' \le k_2'}{\KENV \vdash k_1 \to k_1' \le k_2 \to k_2'}

  \inferrule{\KENV, \KVAR \vdash k_1 \le k_2}{\KENV \vdash \KALL\KVAR k_1 \le \KALL\KVAR k_2}

  % \inferrule[HOW?]{}{\KENV \vdash (k_1 \le k_2 \Rightarrow k_3) \le (k_1' \le k_2' \Rightarrow k_3')}
  %
  \inferrule{\KENV \vdash k_1 \le k_2 \\\KENV \vdash k_2 \le k_3}{\KENV \vdash k_1 \le k_3}
\end{mathpar}
Kinding
\begin{mathpar}
  \inferrule[KSub]{\KENV \vdash t : k \\ \KENV \vdash k \le k' }{\KENV \vdash t : k'}

  \inferrule[KFun]{\KENV \vdash t_1 : \ONE \\ \KENV \vdash t_2 : \ONE}{ \KENV \vdash t_1 \stackrel{k}{\to} t_2 : k}

  \inferrule[KVar]{}{\KENV, \TVAR:k, \KENV' \vdash \TVAR:k}

  \inferrule[KTAll]{\KENV \vdash k\\ \KENV, \TVAR:k \vdash t:k' \\ \TVAR \notin \KENV}{\KENV \vdash \TALL\TVAR k t : k'}

  \inferrule[KKAll]{\KENV, \KVAR \vdash t: k \\ \KVAR \notin \KENV}{\KENV \vdash \KALL\KVAR t : k}

  \inferrule[KConst]{\KENV \vdash k_1, k_2 \\
    \KENV, k_1 \le k_2 \vdash t : k}{\KENV \vdash k_1 \le k_2 \Rightarrow t : k}

  \inferrule[KTLam]{\KENV, \TVAR:k \vdash t : k'}{\KENV \vdash \LAM\TVAR k t : k \to k'}

  \inferrule[KTApp]{\KENV \vdash t_1 : k_2 \to k_1 \\ \KENV \vdash t_2 : k_2}{\KENV \vdash \APP{t_1}{t_2} : k_1}

  \inferrule[KKAll]{\KENV,\KVAR \vdash t : k}{\KENV \vdash \KLAM\KVAR t : \KALL\KVAR k}

  \inferrule[KKApp]{\KENV \vdash t : \KALL\KVAR k' \\ \KENV \vdash k}{\KENV \vdash \KAPP t k : k'[\KVAR\mapsto k]}
\end{mathpar}
Type environment formation
\begin{mathpar}
  \inferrule[TEStart]{\KENV \models \\ \KENV \vdash k \le \ONE }{\KENV; \TENVEMPTY \models k}

  \inferrule[TEAssume]{\KENV; \TENV \models k \\  \KENV \vdash t : k' \\ \KENV \vdash k' \le k \\ x \notin \TENV }{
    \KENV; \TENV, \TASS x t \models k}
\end{mathpar}
Type conversion (congruence rules omitted)
\begin{mathpar}
  \inferrule[Conv-Beta]{\KENV, \TASS\TVAR k' \vdash t:k \\ \KENV \vdash t' : k' }{
    \KENV \vdash \APP{(\LAM\TVAR {k'} t)}{t'} = t[\TVAR \mapsto t'] : k}

  \inferrule[Conv-KSubst]{
    \KENV, \KVAR \vdash t : k \\ \KENV \vdash k'
  }{
    \KENV \vdash \KAPP{(\KLAM\KVAR t)}{k'} = t[\KVAR \mapsto k'] : k[\KVAR \mapsto k']}
\end{mathpar}
Type environment splitting (sequential)
\begin{mathpar}
  \inferrule{}{
    \KENV \vdash \SPLIT{\TENVEMPTY}{\TENVEMPTY}{\TENVEMPTY}}

  \inferrule{
    \KENV \vdash \SPLIT{\TENV_1}{\TENV_2}{\TENV} \\
    \KENV \vdash \SPLIT{t_1}{t_2}{t} \\
  }{
    \KENV \vdash \SPLIT{(\TENV_1, \TASS x t_1)}{(\TENV_2, \TASS x t_1)}{(\TENV, \TASS x t)}}

  \inferrule{
    \KENV \vdash \SPLIT{\TENV_1}{\TENV_2}{\TENV} 
  }{
    \KENV \vdash \SPLIT{(\TENV_1, \TASS x t)}{\TENV_2}{(\TENV, \TASS x t)}}

  \inferrule{
    \KENV \vdash \SPLIT{\TENV_1}{\TENV_2}{\TENV} 
  }{
    \KENV \vdash \SPLIT{\TENV_1}{(\TENV_2, \TASS x t)}{(\TENV, \TASS x t)}}
\end{mathpar}
Type splitting
\begin{mathpar}
  \inferrule{
    \KENV \vdash t : k \\
    \KENV \vdash k \le \KINDU_\infty
  }{
    \KENV \vdash \SPLIT{t}{t}{t}
  }

  \inferrule{
    \KENV \vdash t : k \\
    \KENV \vdash k \le \KINDA
  }{
    \KENV \vdash \SPLIT{\BORROW t}{t}{t}
  }

  \inferrule{
    \KENV \vdash t : k \\
    \KENV \vdash k \le \KINDA
  }{
    \KENV \vdash \SPLIT{\BORROW[i] t}{\BORROW[!]t}{\BORROW[!]t}
  }
\end{mathpar}
Typing rules
\begin{mathpar}
  \inferrule[Conv]{
    \KENV; \TENV \vdash e:t:k \\ \KENV \vdash t = t' : k
  }{\KENV; \TENV \vdash e : t': k}

  \inferrule[Var]{ \KENV; \TENV, \TENV' \models \MANY \\ \KENV \vdash t : k \\ \KENV \vdash k \le \ONE }{\KENV; \TENV, x:t, \TENV' \vdash x:t:k }

  \inferrule[Lam]
  {\KENV; \TENV \models k \\ \KENV; \TENV, x:t \vdash e : t':k'}
  {\KENV;\TENV \vdash \LAM[k] xte : t \stackrel{k}\to t':k}

  \inferrule[App]
  { \KENV \vdash \SPLIT{\TENV_1}{\TENV_2}{\TENV} \\
    \KENV;\TENV_1 \vdash e : t' \stackrel{k}\to t:k \\
    \KENV; \TENV_2 \vdash e' : t':k' \\
    \KENV \vdash t:k''
  }
  { \KENV; \TENV \vdash \APP ee' : t : k''}

  \inferrule[TLam]
  {\KENV, \TASS\TVAR k; \TENV \vdash v : t : k' \\ \TVAR \notin \KENV}
  {\KENV; \TENV \vdash (\TLAM[k']\TVAR k v) : (\TALL \TVAR k t) : k'}

  \inferrule[TApp]
  {
    \KENV; \TENV \vdash e : (\TALL \TVAR k t') : k' \\
    \KENV \vdash t : k 
  }
  { \KENV; \TENV \vdash \TAPP e t : t'[\TVAR \mapsto t] : k' }

  \inferrule[KLam]
  { \KENV, \KVAR; \TENV \vdash e : t : k \\ \KVAR \notin \KENV}
  { \KENV; \TENV \vdash \KLAM[k]\KVAR e : \KALL\KVAR t : k}

  \inferrule[KApp]
  { \KENV; \TENV \vdash e : \KALL\KVAR t : k' \\ \KENV \vdash k}
  { \KENV; \TENV \vdash \KAPP e k : t[\KVAR \mapsto k] : k'[\KVAR \mapsto k]}

  \inferrule[CIntro]
  { \KENV,   k_1 \le k_2; \TENV \vdash e : t}
  { \KENV; \TENV \vdash e :  k_1 \le k_2 \Rightarrow t}

  \inferrule[CElim]
  { \KENV; \TENV \vdash e :  k_1 \le k_2 \Rightarrow t \\ \KENV \vdash k_1 \le k_2}
  { \KENV; \TENV \vdash e : t}
\end{mathpar}

Simple small-step semantics (for type preservation) with
$e[x \mapsto v]$ standing for capture-avoiding substitution of $v$ for
$x$ in $e$. Call-by-value as in $F^{\ONE}$.
\begin{mathpar}
  \inferrule[V-Beta]{}{\APP{(\LAM[k] x t e)}v \stepsto e[x \mapsto v]}

  \inferrule[T-Beta]{}{\TAPP{(\TLAM[k'] \TVAR k v)}t \stepsto v[\TVAR \mapsto t]}

  \inferrule[K-Beta]{}{\KAPP{(\KLAM \KVAR v)}k \stepsto v[\KVAR \mapsto k]}
  \\
  \inferrule[App-Left]{e_1 \stepsto e_1'
  }{\APP{e_1}{e_2} \stepsto \APP{e_1'}{e_2}}

  \inferrule[App-Right]{e \stepsto e'}{\APP v e \stepsto \APP v e'}

  \inferrule[TApp-Left]{e \stepsto e'}{\TAPP{e} t \stepsto \TAPP{e'} t}

  \inferrule[KApp-Left]{e \stepsto e'}{\KAPP{e} k \stepsto \KAPP{e'} k}
\end{mathpar}

\clearpage
\begin{lemma}[Weakening]
  Let $\mathcal{A}$ range over assumptions in kind environments.
  Let $\mathcal{J}$ range over judgments in the context of a kind environment. 

  If $\KENV \vdash \mathcal{J}$ and $\KENV, \mathcal{A} \models$, then $\KENV, \mathcal{A} \vdash \mathcal{J}$.
\end{lemma}
\begin{lemma}[Unrestricted Weakening]
  Suppose that $\KENV; \TENV \vdash e : t$, $x\notin \TENV$, and $\KENV \vdash t_x : \MANY$.
  Then  $\KENV; \TENV, x : t_x \vdash e : t$.
\end{lemma}
\begin{lemma}[Value Substitution]\label{lemma:value-substitution}
  Suppose that $\KENV; \TENV, x:t_x \vdash e : t$
  and $\KENV; \TENVEMPTY \vdash v : t'$
  and $\KENV \vdash t_x = t' : k'$.
  Then $\KENV; \TENV \vdash e[x \mapsto v] : t$.
\end{lemma}
\begin{proof}
  As the conversion assumes the empty environment, it cannot be
  affected by adding further assumptions.

  The proof proceeds by induction on the derivation of  $\KENV; \TENV, x:t_x
  \vdash e : t$ and produces a derivation for the term after
  substitution. The only interesting case is the one for the
  \TirName{Var} rule when the variable is $x$:
  \begin{mathpar}
    \inferrule[Var]{ \KENV;\TENV \models \MANY \\ \KENV \models t_x : k' \\ \KENV \vdash k' \le \ONE }{\TENV, x:t_x \vdash x:t_x }
  \end{mathpar}
  By unrestricted weakening we have that
  \begin{gather*}
    \KENV; \TENV \vdash v : t' \\
    \KENV \vdash t_x = t' : k'
  \end{gather*}
  and hence by \TirName{Conv} and symmetry of $=$, we have the desired outcome
  \begin{gather*}
    \KENV; \TENV \vdash v : t_x
  \end{gather*}
\end{proof}
\begin{lemma}[Type Substitution]\label{lemma:type-substitution}
  Suppose that $\KENV \vdash t' : k'$.
  \begin{enumerate}
  \item\label{item:1} If $\KENV, \TASS\TVAR {k'} \vdash t : k$, then
    $\KENV \vdash t[\TVAR \mapsto t'] : k$.
  \item\label{item:3} If $\KENV, \TASS\TVAR {k'}; \TENV \models k$,
    then $\KENV; \TENV[\TVAR \mapsto t'] \models k$.
  \item\label{item:5} If $\KENV, \TASS\TVAR {k'} \vdash t_1 = t_2 : k$,
    then $\KENV \vdash t_1[\TVAR \mapsto t'] = t_2[\TVAR \mapsto t'] : k$.
  \item\label{item:4} If
    $\KENV, \TASS\TVAR {k'} \vdash \SPLIT{\TENV_1}{\TENV_2}{\TENV}$,
    then $\KENV \vdash \SPLIT{\TENV_1[\TVAR \mapsto t']}{\TENV_2[\TVAR \mapsto t']}{\TENV[\TVAR \mapsto t']}$.
  \item\label{item:2} If $\KENV, \TASS\TVAR {k'}; \TENV \vdash e : t$, then
    $\KENV; \TENV[\TVAR \mapsto t'] \vdash e[\TVAR \mapsto t'] : t[\TVAR \mapsto t']$.
  \end{enumerate}
\end{lemma}
\begin{proof}
  \textbf{Item}~\ref{item:1} is proved by induction on the derivation of
  $\KENV, \TASS\TVAR {k'} \vdash t : k$.

  The only interesting rule is \TirName{KVar}:
  $\KENV, \TASS\TVAR {k'} \vdash \TVAR : k'$. As
  $\TVAR[\TVAR \mapsto t'] = t'$, we have $\KENV \vdash t' : k'$ by
  assumption.

  All remaining cases are immediate by appeal to the inductive hypothesis.

  \textbf{Item}~\ref{item:3} is proved by induction on the derivation of
  $\KENV, \TASS\TVAR {k'}; \TENV \models k$.

  \textbf{Case} \TirName{TEStart}. Immediate.

  \textbf{Case} \TirName{TEAssume}. From $\KENV, \TASS\TVAR {k'}; \TENV, \TASS x t \models k$ inversion yields
  \begin{gather}
    \label{eq:4}
    \KENV, \TASS\TVAR {k'}; \TENV \models k \\
    \KENV, \TASS\TVAR {k'} \vdash t : k'' \\
    \KENV, \TASS\TVAR {k'} \vdash k'' \le k \\
    x \notin \TENV
  \end{gather}
  By induction from~\eqref{eq:4}
  \begin{gather}
    \KENV; \TENV[\TVAR \mapsto t'] \models k
  \end{gather}
  By Item~\ref{item:1}
  \begin{gather}
    \KENV \vdash t[\TVAR\mapsto t'] : k''
  \end{gather}
  As $\TVAR\notin k'', k$
  \begin{gather}
    \KENV \vdash k'' \le k
  \end{gather}
  By \TirName{TEAssume}
  \begin{gather}
    \KENV; (\TENV, \TASS x t)[\TVAR \mapsto t'] \models k
  \end{gather}

  \textbf{Item}~\ref{item:5} is proved by induction on the derivation of
  $\KENV, \TASS\TVAR{ k'} \vdash t_1 = t_2 : k$.

  \textbf{Item}~\ref{item:4} is proved by induction on the derivation of splitting.

  \textbf{Item}~\ref{item:2} is proved by induction on the derivation of
  $\KENV, \TASS\TVAR k; \TENV \vdash e : t$.

  \textbf{Case} \TirName{Conv}: immediate by Item~\ref{item:5}.

  \textbf{Case} \TirName{Var}: immediate by Item~\ref{item:1} and Item~\ref{item:3}.

  \textbf{Case} \TirName{Lam}: by Item~\ref{item:3} and induction.

  \textbf{Case} \TirName{App}: by Item~\ref{item:4} and induction.

  The remaining cases present no new problems.
\end{proof}
Kind substitution can lead to unsatisfiable constraints. That does not
mean that kind substitution is bad, but that a value with a bad kind
substitution cannot be used as the unsatisfiable sonstraint cannot be
eliminated by \TirName{CElim}!
\begin{lemma}[Kind Substitution]\label{lemma:kind-substitution}
  Suppose that $\KENV \vdash k$.
  \begin{enumerate}
  \item If $\KENV, \KVAR \vdash t : k'$,
    then $\KENV \vdash t[\KVAR \mapsto k] : k'[\KVAR \mapsto k]$.
  \item If $\KENV, \KVAR \vdash t = t' : k'$,
    then $\KENV \vdash t[\KVAR \mapsto k] = t'[\KVAR \mapsto k] : k'[\KVAR \mapsto k]$.
  \item If $\KENV, \KVAR ; \TENV \models k'$,
    then $\KENV; \TENV[\KVAR \mapsto k] \models k'[\KVAR \mapsto k]$.
  \item If $\KENV,\KVAR \vdash k_1 \le k_2$,
    then $\KENV \vdash k_1[\KVAR \mapsto k] \le k_2[\KVAR \mapsto k]$.
  \item If $\KENV, \KVAR \vdash \SPLIT{\TENV_1}{\TENV_2}{\TENV}$,
    then $\KENV \vdash \SPLIT{\TENV_1[\KVAR \mapsto k]}{\TENV_2[\KVAR \mapsto k]}{\TENV[\KVAR \mapsto k]}$. 
  \item If $\KENV, \KVAR; \TENV \vdash e : t$,
    then $\KENV; \TENV[\KVAR \mapsto k] \vdash e[\KVAR \mapsto k] : t[\KVAR \mapsto k]$.
  \end{enumerate}
\end{lemma}
\begin{proof}
  \textbf{!!! TODO !!!}
\end{proof}
\begin{lemma}[Inversion for Function Type]\label{lemma:inversion-function}
  If $\KENV; \TENVEMPTY \vdash \LAM x {t_x} e : t_f$,
  then there is some $n\ge0$ and $k_{i1}$, $k_{i2}$ (for $1\le i\le n$) such that 
  \begin{gather}
    \KENV \vdash t_f = k_{11}\le k_{12}\Rightarrow \dots k_{n1}\le k_{n2} \Rightarrow t_x \stackrel{k}\to t : k'
    \\
    \KENV; \TASS x{t_x} \vdash e : t
    \\
    \KENV \vdash k' \le \ONE
    \\
    \KENV \vdash k \le \ONE
    \mathrm{.}
  \end{gather}
\end{lemma}
\begin{proof}
  Induction on the derivation of
  $\KENV; \TENVEMPTY \vdash \LAM x {t_x} e : t_f$.

  \textbf{Case} \TirName{Lam}. Rule inversion yields $\KENV \models$,
  $\KENV \vdash k \le \ONE$, $t_f = t_x \stackrel{k}\to t$, and
  $\KENV; \TASS x{t_x} \vdash e : t$. By reflexivity of conversion
  $\KENV \vdash t_f = t_x \stackrel{k}\to t : k'$, for some
  $\KENV \vdash k' \le \ONE$, and the claim holds for $n=0$.

  \textbf{Case} \TirName{Conv}. Rule inversion yields
  $\KENV; \TENVEMPTY \vdash \LAM x {t_x} e : t_1$ and
  $\KENV \vdash t_1 = t_f : k'$ with $\KENV \vdash k' \le
  \ONE$. Conclude by induction and by transitivity of conversion.

  \textbf{Case} \TirName{CIntro}. Rule inversion yields that
  $t_f = k_1 \le k_2 \Rightarrow t_1$ and
  $\KENV, k_1 \le k_2; \TENVEMPTY \vdash e : t_1$. By induction,
  $\KENV \vdash t_1 = k_{11}\le k_{12}\Rightarrow \dots k_{n1}\le
  k_{n2} \Rightarrow t_x \stackrel{k}\to t : k'$ so by congruence
  $\KENV \vdash t_f = k_1 \le k_2 \Rightarrow k_{11}\le k_{12}\Rightarrow \dots k_{n1}\le
  k_{n2} \Rightarrow t_x \stackrel{k}\to t : k'$, which proves the claim.

  \textbf{Case} \TirName{CElim}. Rule inversion yields that
  $\KENV; \TENVEMPTY \vdash \LAM x {t_x} e : t_1$ with
  $t_1 = k_1 \le k_2 \Rightarrow t_f$ and $\KENV \vdash k_1 \le
  k_2$. By induction
  $\KENV \vdash t_1 = k_1 \le k_2 \Rightarrow k_{11}\le
  k_{12}\Rightarrow \dots k_{n1}\le k_{n2} \Rightarrow t_x
  \stackrel{k}\to t : k'$ so that by congruence
  $\KENV \vdash t_f = k_{11}\le
  k_{12}\Rightarrow \dots k_{n1}\le k_{n2} \Rightarrow t_x
  \stackrel{k}\to t : k'$, which proves the claim.
\end{proof}
\begin{lemma}[Inversion for Type Abstraction]\label{lemma:inversion-universal}
  Suppose that $\KENV; \TENVEMPTY \vdash \TLAM\TVAR k v : t_a$.
  Then there is some $n\ge0$ and $k_{i1}$, $k_{i2}$ (for $1\le i\le n$) such that 
  \begin{gather}
    \KENV \vdash t_a = k_{11}\le k_{12}\Rightarrow \dots k_{n1}\le k_{n2} \Rightarrow \TALL \TVAR k t' : k'
    \\
    \KENV, \TASS \TVAR{k}; \TENVEMPTY \vdash v : t'
    \\
    \KENV \vdash k' \le \ONE
    \mathrm{.}
  \end{gather}
\end{lemma}
\begin{proof}
  Analogous to Lemma~\ref{lemma:inversion-function}.
\end{proof}
\begin{lemma}[Inversion for Kind Abstraction]\label{lemma:inversion-kind-abstraction}
  Suppose that $\KENV; \TENVEMPTY \vdash \KLAM\KVAR v : t_a$.
  Then there is some $n\ge0$ and $k_{i1}$, $k_{i2}$ (for $1\le i\le n$) such that 
  \begin{gather}
    \KENV \vdash t_a = k_{11}\le k_{12}\Rightarrow \dots k_{n1}\le k_{n2} \Rightarrow \KALL \KVAR t' : k'
    \\
    \KENV, \KVAR; \TENVEMPTY \vdash v : t'
    \\
    \KENV \vdash k' \le \ONE
    \mathrm{.}
  \end{gather}
\end{lemma}
\begin{proof}
  Analogous to Lemma~\ref{lemma:inversion-function}.
\end{proof}
\begin{lemma}[Type Preservation]~\\
  Suppose that $\KENV; \TENVEMPTY \vdash e : t$ and $e \stepsto e'$.
  Then $\KENV;\TENVEMPTY \vdash e' : t$.
\end{lemma}
\begin{proof}
  In each case we proceed by induction on the derivation of the
  assumed judgment.  If the top-level rule is \TirName{Conv},
  \TirName{CIntro}, or \TirName{CElim}, then the claim holds by
  induction.

  \textbf{Case} ${\APP{(\LAM x {t_x} e)}v \stepsto e[x \mapsto v]}$.

  By assumption $\KENV;\TENVEMPTY \vdash \APP{(\LAM x {t_x} e)}v : t$.


  If the top-level rule is \TirName{App} rule, we can apply inversion to find
  \begin{gather}
    \label{eq:1}
    \KENV; \TENVEMPTY \vdash \LAM x {t_x} e : t' \stackrel{k}\to t
    \\
    \label{eq:2}
    \KENV; \TENVEMPTY \vdash v : t'
  \end{gather}
  By Lemma~\ref{lemma:inversion-function}, there exists an $n\ge0$ and $k_{i1}$, $k_{i2}$ (for $1\le i\le n$) such that
  \begin{gather}
    \KENV \vdash t' \stackrel{k}\to t = k_{11}\le k_{12}\Rightarrow \dots k_{n1}\le k_{n2} \Rightarrow t_x \stackrel{k}\to t : k'
    \\
    \KENV; \TASS x{t_x} \vdash e : t
  \end{gather}
  Hence, $n=0$ and 
  \begin{gather}
    \KENV \vdash t' \stackrel{k}\to t = t_x \stackrel{k}\to t : k'
  \end{gather}
  from which we can follow
  \begin{gather}
    \label{eq:3}
    \KENV \vdash t_x = t' : k' \text{ where } \KENV \vdash k' \le \ONE
  \end{gather}
  With these assumptions, we apply value substitution
  (Lemma~\ref{lemma:value-substitution}) to obtain the result.

  \textbf{Case} $\TAPP{(\TLAM \TVAR k v)}{t'} \stepsto v[\TVAR \mapsto t']$.

  By assumption
  $\KENV;\TENVEMPTY \vdash \TAPP{(\TLAM \TVAR k v)}{t'} :
  t$. Inversion of the \TirName{TApp} rule yields
  \begin{gather}
    \KENV; \TENVEMPTY \vdash \TLAM \TVAR k v : \TALL\TVAR k t_0 \\
    \KENV \vdash t' : k \\
    t = t_0 [\TVAR\mapsto t']
  \end{gather}
  By Lemma~\ref{lemma:inversion-universal}, there exists an $n\ge0$ and $k_{i1}$, $k_{i2}$ (for $1\le i\le n$) such that
  \begin{gather}
    \KENV \vdash \TALL\TVAR k t_0 = k_{11}\le k_{12}\Rightarrow \dots k_{n1}\le k_{n2} \Rightarrow \TALL\TVAR k t_1 : k_1
    \\
    \label{eq:5}
    \KENV, \TASS \TVAR{k}; \TENVEMPTY \vdash v : t_1
  \end{gather}
  Hence, $n=0$ and
  \begin{gather}
    \KENV \vdash \TALL\TVAR k t_0 = \TALL\TVAR k t_1 : k_1
  \end{gather}
  from which we can follow
  \begin{gather}
    \label{eq:6}
    \KENV, \TASS\TVAR k \vdash t_0 = t_1 : k_1
  \end{gather}
  By Lemma~\ref{lemma:type-substitution} (type substitution) applied to~\eqref{eq:5}
  \begin{gather}
    \label{eq:7}
    \KENV; \TENVEMPTY \vdash v[\TVAR \mapsto t'] : t_1[\TVAR \mapsto t']
  \end{gather}
  By Lemma~\ref{lemma:type-substitution} (type substitution) applied to~\eqref{eq:6}
  \begin{gather}
    \label{eq:8}
    \KENV \vdash t_0[\TVAR \mapsto t'] = t_1[\TVAR \mapsto t'] : k_1
  \end{gather}
  By \TirName{Conv} applied to~\eqref{eq:7} and~\eqref{eq:8}
  \begin{gather}
    \KENV; \TENVEMPTY \vdash v[\TVAR \mapsto t'] : t
  \end{gather}
  as required.

  \textbf{Case} $\KAPP{(\KLAM \KVAR v)}k \stepsto v[\KVAR \mapsto k]$.

  By inversion of rule \TirName{KApp}, we obtain from
  $\KENV; \TENVEMPTY \vdash \KAPP{(\KLAM \KVAR v)}k : t_e$ that
  \begin{gather}
    t_e =  t[\KVAR \mapsto k] \\
    \label{eq:9}
    \KENV; \TENVEMPTY \vdash \KLAM \KVAR v : \KALL\KVAR t \\
    \KENV \vdash k
  \end{gather}
  By Lemma~\ref{lemma:inversion-kind-abstraction} applied to~\eqref{eq:9} we find that
  \begin{gather}
    \KENV \vdash
    \KALL\KVAR t
    = k_{11}\le k_{12}\Rightarrow \dots k_{n1}\le k_{n2} \Rightarrow
    \KALL \KVAR t' : k'     \\
    \label{eq:10}
    \KENV, \KVAR; \TENVEMPTY \vdash v : t'
  \end{gather}
  so that $n=0$ and
  \begin{gather}
    \KENV \vdash
    \KALL\KVAR t
    = 
    \KALL \KVAR t' : k'
  \end{gather}
  and thus
  \begin{gather}\label{eq:11}
    \KENV, \KVAR \vdash
    t
    = 
    t' : k'
  \end{gather}
  Applying kind substitution to~\eqref{eq:10} and~\eqref{eq:11} yields
  \begin{gather}
    \KENV; \TENVEMPTY \vdash v[\KVAR \mapsto k] : t'[\KVAR \mapsto k]
    \\
    \KENV \vdash
    t[\KVAR \mapsto k]
    = 
    t'[\KVAR \mapsto k] : k'[\KVAR \mapsto k]
  \end{gather}
  and thus by \TirName{Conv} the desired
  \begin{gather}
    \KENV; \TENVEMPTY \vdash v[\KVAR \mapsto k] : t_e
  \end{gather}

  \textbf{Case} reduction in context by rules \TirName{App-Left},
  \TirName{App-Right}, \TirName{TApp-Left}, and \TirName{KApp-Left}. The result is
  immediate by inversion of the respective rule and then by appeal to the inductive hypothesis.
\end{proof}

TODO: 
Templates in $F^{\ONE}$, work by Morris, Bernardi, and others.

\section{Allocating Semantics}

A more elaborate big-step semantics that allocates all values on the
heap and removes linear values after they are used.  The evaluation
judgement is thus $H, e \Downarrow H, \Addr$ where $H$ is a heap that
binds addresses $\Addr$ to values, which are abstractions that may
contain addresses themselves, but no free variables.

\begin{mathpar}
  \inferrule{}{H, \LAM[k] x t e \Downarrow H+ (\Addr \mapsto \LAM[k] x t e), \Addr}

  \inferrule{
    H_1, e_1 \Downarrow H_2, \Addr_1 \\
    H_2 (\Addr_1) = \LAM[k]x t e \\
    H_2' = H_2 \text{ if $k=\MANY$ else } H_2 \setminus \Addr_1
    \\\\
    H_2', e_2 \Downarrow H_3, \Addr_2 \\
    H_3, e[x \mapsto \Addr_2] \Downarrow H_4, \Addr
  }{H_1, \APP{e_1}{e_2} \Downarrow H_4, \Addr}

  \inferrule{}{
    H, \TLAM[k']\TVAR k v \Downarrow H+ (\Addr \mapsto \TLAM[k']\TVAR k v), \Addr
  }

  \inferrule{
    H, e \Downarrow H', \Addr' \\
    H' (\Addr') = \TLAM[k']\TVAR k v \\
    H'' = H' \text{ if $k'=\MANY$ else } H' \setminus \Addr'
  }{
    H, \TAPP e t \Downarrow H''+ (\Addr \mapsto v[\TVAR \mapsto t]), \Addr
  }

  \inferrule{}{
    H, \KLAM[k]\KVAR v \Downarrow H+ (\Addr \mapsto \KLAM[k]\KVAR v), \Addr
  }

  \inferrule{
    H, e \Downarrow H', \Addr' \\
    H' (\Addr') = \KLAM[k']\KVAR v \\
    H'' = H' \text{ if $k'=\MANY$ else } H' \setminus \Addr'
  }{
    H, \KAPP e k \Downarrow H''+ (\Addr\mapsto v[\KVAR\mapsto k]),
    \Addr
  }
\end{mathpar}

\end{document}

%%% Local Variables:
%%% mode: latex
%%% TeX-master: t
%%% End:

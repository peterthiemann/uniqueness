
\documentclass[xcolor=svgnames,11pt]{beamer}

\usepackage[T1]{fontenc}
\usepackage[utf8]{inputenc}
\usepackage[french, english]{babel}
\selectlanguage{english}

\usepackage{graphicx}

% Math
\usepackage{amsmath}
% \usepackage{amsfonts}
\usepackage{amssymb}
\usepackage{amsthm}
% \usepackage{mathrsfs}
\usepackage{mathtools}
\usepackage{textcomp}
% \usepackage{textgreek}

% Specialized packages
% \usepackage{syntax} % Grammar definitions
\usepackage{verbatim}
\usepackage{listings} % Code
\usepackage{xspace} % Useful for macros
\usepackage{natbib}% Good citations and bibliography
\usepackage{mathpartir} % Syntax trees
\usepackage{colortbl}
\usepackage{hhline}
\usepackage{multicol}%multicolumn lists
\usepackage{pifont}%% check mark

% \usepackage{mathptmx}
% \usepackage[scaled=0.9]{helvet}
\usepackage{beramono}

\bibliographystyle{ACM-Reference-Format}
\citestyle{acmauthoryear}
\usepackage{appendixnumberbeamer}

\usetheme{metropolis}
\beamertemplatenavigationsymbolsempty
\setbeamercovered{transparent=20}

\usepackage{wasysym}

\usepackage{tikz}
\usetikzlibrary{decorations.text,backgrounds,positioning,shapes,
  shadings,shadows,arrows,decorations.markings,calc,fit,fadings,
  tikzmark
}

\def\HUGE{\fontsize{35pt}{15pt}\selectfont}

\newcommand\htag[1]{\shortintertext{\textbf{#1}}}

\newcommand\TODO[1]{{\ \\\color{red}\large\textbf{TODO} #1}\\}

\newcommand\ocaml{OCaml\xspace}
\newcommand\affe{Affe\xspace}

%%% Local Variables:
%%% mode: latex
%%% TeX-master: "main"
%%% End:


%% Syntax

% Kinds
\newcommand\lk{\leq}
\newcommand\klin{\circ}
\newcommand\kun{\ast}
\newcommand\karr{\operatorname{\rightarrow}}
\newcommand\kvar{\kappa}
\newcommand\K[1]{\mathrm{#1}}

% Types
\newcommand\tvar{\alpha}
\newcommand\tarr[1]{\operatorname{\xrightarrow{#1}}}
\newcommand\T[1]{\mathrm{#1}}
\newcommand\qual[2]{#1 \operatorname{\Rightarrow} #2}

\newcommand\instanciate[2]{#1 \operatorname{\succ} #2}
\newcommand\generalize[3]{\operatorname{\text{Gen}}(#1,#2,#3)}

% Expressions
\newcommand\lam[2]{\lambda #1 . #2}
\newcommand\elet{\mathtt{let}}
\newcommand\ein{\mathtt{in}}
\newcommand\letin[3]{\elet\ #1 = #2\ \ein\ #3}
\newcommand\fix[1]{\mathtt{fix}\ #1}
\newcommand\app[2]{(#1\ #2)}


%% Typing environments
\newcommand\E{\Gamma}
\newcommand\bvar[2]{(#1 : #2)}
\newcommand\bty[3]{(\T{#1} : #2 = #3)}
\newcommand\bineq[2]{(#1 \lk #2)}

\newcommand\esplit[2]{#1 \operatorname{\Join} #2}

%% Judgements
\newcommand\BAR{\operatorname{|}}
\newcommand\dash{\operatorname{\vdash}}
\newcommand\hastype[4]{#1 \BAR #2 \dash #3 : #4}
\newcommand\haskind[4]{#1 \BAR #2 \dash #3 \lk #4}
\newcommand\subkind[4]{#1 \dash\ \bineq{#3}{#4}}
\newcommand\envkind[3]{#1 \dash #2 \lk #3}

%% Reductions

\newcommand\subst[3]{#3[#1\rightarrow#2]}
\newcommand\fresh{\operatorname{\text{fresh}}}
\newcommand\closure[3]{\lambda^{#1} #2 . #3}
\newcommand\ered[4]{#3\ \operatorname{\Downarrow^{#1}_{#2}}\ #4}

%%% Local Variables:
%%% mode: latex
%%% TeX-master: "main"
%%% End:



\lstset{
  tabsize=4,
  aboveskip={0.5\baselineskip},
  belowcaptionskip=0.5\baselineskip,
  columns=fixed,
  showstringspaces=false,
  extendedchars=true,
  breaklines=true,
  frame=none,
  basicstyle=\small\ttfamily,
  keywordstyle=\bfseries,
  commentstyle=\color{gray},
  % identifierstyle=\color{blue!80!black},
  stringstyle=\color{purple},
  numbersep=5pt,
  numberstyle=\tiny\color{gray},
  escapeinside={(*@}{@*)},
  emphstyle=\color{green!60!black}\bfseries,
  emphstyle={[2]\color{blue!60!black}\bfseries},
  language=[Objective]Caml,
  % numbers=left,
}

\title{Kindly Bent To Free Us}
\author{Gabriel \textsc{Radanne} \and Peter \textsc{Thiemann}}

\begin{document}

\frame[plain]{\titlepage}

\begin{frame}[fragile,fragile]

\begin{lstlisting}
val Tls_lwt.of_t : Tls_lwt.Unix.t -> in * out
(* Turn a file descr into input/output channels *)
\end{lstlisting}\pause
\begin{lstlisting}
let fd : Tls_lwt.Unit.t = .....
let input, output = Tls_lwt.of_t fd
... (* read some things *)
let%lwt () = Lwt_io.close input in (*@\pause@*)
...
let%lwt c = Lwt_io.write output "thing" in (*Oups*)
...
\end{lstlisting}
  \pause

  The default behavior is to close the underlying file description when a channel is closed.
\end{frame}

\begin{frame}
  Many partial solutions
  \begin{itemize}
  \item Closures
  \item Monads
  \item Existential types
  \item \dots
  \end{itemize}
  \pause

  What we really need is to enforce linearity.
\end{frame}

\begin{frame}
  Many places in \ocaml where enforcing linearity is useful:
  \begin{itemize}
  \item IO (File handle, channels, network connections, \dots)
  \item Protocols (With session types! Mirage libraries)
  \item One-shot continuations (effects!)
  \item Transient data-structures
  \item C-style ``struct parsing''
  \item \dots
  \end{itemize}
\end{frame}

\begin{frame}
  Goals:
  \begin{itemize}
  \item Complete and principal type inference
  \item Impure strict context
  \item Works well with type abstraction
  \item Play balls with various other ongoing works (Effects, Resource polymorphism, \dots)
  \end{itemize}

  \pause
  Non Goals:
  \begin{itemize}
  \item Support every linear code pattern under the sun
  \item Design associated compiler optimisations/GC integration (yet)
  \end{itemize}
\end{frame}


\section{The \lang language}

\begin{frame}[fragile]{Types and Behaviors}
  In \lang, the behavior of a variable is determined
  by its type:
\begin{lstlisting}
type channel : A (* channel is Affine! *)
\end{lstlisting}\pause
\begin{lstlisting}
let with_file s f = 
  let c = open_channel s
  let c = f c in
  close_channel c
val with_file : string -> (channel -> channel)
\end{lstlisting}\pause
\begin{lstlisting}
let () = 
  let r = ref None in
  with_file "thing"
    (fun c -> r := Some c ; c) (* (*@\color{red} \ding{56}  No!@*) *)
\end{lstlisting}
\end{frame}

\begin{frame}[fragile]{Inference at work}
  Infer unrestricted in case of duplication:
\begin{lstlisting}
let f = fun c -> r := Some c ; c
val f : ('a : U) . 'a -> 'a
\end{lstlisting}
\end{frame}

\begin{frame}{The kinds so far}

  So far, two kinds:
  \begin{description}
  \item[A] Affine types: can be used at most once
  \item[U] Unrestricted types
  \end{description}

  Additionally, we have:
  \begin{align*}
    \kun &\leq \kaff
  \end{align*}
\end{frame}

\defverbatim{\KA}{
\begin{lstlisting}
let f = fun a -> fun b -> (a, b)
val f : 'a -> 'b -> 'a * 'b (* ? *)
\end{lstlisting}
}
\defverbatim{\KB}{
\begin{lstlisting}
let f = fun a -> fun b -> (a, b)
val f : ('a : 'k) => 'a -> 'b -{'k}> 'a * 'b
\end{lstlisting}
}

\begin{frame}[fragile]{Inference at work}
  What about closures?
  \alt<2>{\KB}{\KA}
\end{frame}


\defverbatim{\AppA}{
\begin{lstlisting}
let app f x = f x
val app :
  'k1 < 'k2 =>
  ('a -{'k1}> 'b) -> 'a -{'k2}> 'b
\end{lstlisting}
}
\defverbatim{\AppB}{
\begin{lstlisting}
let app f x = f x
val app :
  ('a -{'k}> 'b) -> 'a -{'k}> 'b
\end{lstlisting}
}

\begin{frame}[fragile]{Inference at work}
  \AppA
\end{frame}


\defverbatim{\Comp}{
\begin{lstlisting}
let compose f g x = f (g x)
val compose : 
  ('b -{?}> 'a) -> ('c -{?}> 'b) -{?}> 'c -{?}> 'a
\end{lstlisting}
}
\defverbatim{\CompA}{
\begin{lstlisting}
let compose f g x = f (g x)
val compose : 
  ('k2 < 'k) & ('k1 < 'k) & ('k1 < 'k3) =>
  ('b -{'k1}> 'a) -> ('c -{'k2}> 'b) -{'k3}> 
  'c -{'k}> 'a
\end{lstlisting}
}
\defverbatim{\CompB}{
\begin{lstlisting}
let compose f g x = f (g x)
val compose : 
  ('k1 < 'k) =>
  ('b -{'k1}> 'a) -> ('c -{'k}> 'b) -{'k1}> 
  'c -{'k}> 'a
\end{lstlisting}
}

% \begin{frame}[fragile]{Inference at work}
%   What about more complicated cases ?
%   \only<1>{\Comp}
%   \only<2>{\CompA}
%   \only<3>{\CompB}
% \end{frame}


\defverbatim{\RefA}{
\begin{lstlisting}
type ('a : U) ref : U = ...
\end{lstlisting}
}
\defverbatim{\RefB}{
\begin{lstlisting}
type ref : U -> U = ...
\end{lstlisting}
}

\begin{frame}[fragile]{Closer look at type declarations}
  You can annotate the kinds on type declarations.

  Vanilla OCaml references are fully unrestricted:
  \only<1>{\RefA}
  \only<2-3>{\RefB}
  \pause\pause

  We can also have constraints on kinds. The pair type operator:
\begin{lstlisting}
type * : (k1 < k) & (k2 < k) => k1 -> k2 -> k
\end{lstlisting}
\end{frame}


\begin{frame}[fragile]{More interesting example}

  Mixing with abstraction:
\begin{lstlisting}
module AffineArray : sig
  type -'a w : A
  val create :
    ('a : U) . int -> 'a -> 'a w
  val set : 'a w -> int -{A}> 'a -{A}> 'a w

  type +'a r : U
  val freeze : 'a w -> 'a r
  val get : int -> 'a r -> 'a 
end
\end{lstlisting}
\end{frame}

\section{The calculus}

\begin{frame}
  \frametitle{The calculus}
  \begin{columns}[t]
    \column{0.5\textwidth}
    \begin{align*}
      \htag{Expressions}
      e ::=&\ c\ |\ x\ |\ \app{e}{e'}\ |\ \lam{x}{e}\\
      % |&\ \fix{e}\tag{Fixpoint}\\
      |&\ \letin{x}{e}{e'}\\
      |&\ \introK{K}{e}\ |\ \elimK{K}{e}\\
    \end{align*}
    \column{0.5\textwidth}
    \begin{align*}
      \htag{Type Expressions}
      \tau ::=&\ \tvar\ |\ \tau\tarr{k}\tau\ |\ \tapp{t}{(\tau^*)}\\
      k ::=&\ \kvar\ |\ \ell \in {\mathcal L}
    \end{align*}
  \end{columns}
\end{frame}

\begin{frame}
  \frametitle{The calculus \hfill Constraints}

  Constraints are only acceptable in schemes:
  \begin{align*}
    \schm ::=&\ \forall\kvar^*\forall\bvar{\alpha}{k}^*.(\qual{C}{\tau})\\
    \kschm ::=&\ \forall\kvar^*.(\qual{C}{k_i^* \karr k})\\
  \end{align*}
  The constraint language in schemes is limited to list of inequalities:
  \begin{align*}
    C ::=&\ \Cleq{k}{k'}^*
  \end{align*}
\end{frame}

\newcommand\tikzcoord[2]%
{\tikz[remember picture,baseline=(#1.base)]{\node[coordinate] (#1) {#2};}}
\begin{frame}
  \frametitle{Typing}
  \centering

  \begin{align*}
    \inferW
      {\tikzmark{Sv}{\addlin{\Sv}}}
      {(\tikzmark{C}{C},\tikzmark{psi}{\psi})}
      {\tikzmark{E}{\Gamma}}
      {\tikzmark{e}{e}}
      {\tikzmark{t}{\tau}}
  \end{align*}

  \tikzstyle{link}=[->, >=latex, thick]
  \begin{tikzpicture}[remember picture, overlay]
    \node[above right=1.2 and 1.2 of pic cs:e, anchor=south] (expr)
    {\bf Expression};
    \draw[link] (expr) to[out=-100,in=80] ([shift={(0.8ex,1.2ex)}]pic cs:e);
    
    \node[above left=1.2 and 0.2 of pic cs:E] (env) {\bf Environment};
    \draw[link] (env) to[out=-80,in=100] ([shift={(0.8ex,1.9ex)}]pic cs:E);
    
    \node[below right=0.6 and 0.8 of pic cs:t] (ty) {Type};
    \draw[link] (ty) to[out=170,in=-70] ([shift={(0.8ex,-0.4ex)}]pic cs:t);
    
    \node[below right=1.2 and 0.8 of pic cs:psi] (psi) {Unifier};
    \draw[link] (psi) to[out=100,in=-80] ([shift={(0.8ex,-0.4ex)}]pic cs:psi);
    
    \node[below left=1.2 and 0 of pic cs:C] (C) {Constraints};
    \draw[link] (C) to[out=100,in=-80] ([shift={(0.8ex,-0.4ex)}]pic cs:C);
    
    \node[below left=0 and 0.9 of pic cs:Sv] (Sv) {\addlin{Usage Map}};
    \draw[link,blue] (Sv) to[out=20,in=180] ([shift={(0ex,0.6ex)}]pic cs:Sv);
  \end{tikzpicture}
\end{frame}

\begin{frame}
  \frametitle{Typing \hfill Tracking linearity}

  Variables can be kind-polymorphic and all their instances might not have the same kinds.

  $\implies$ We must track the kinds of all use-sites for each variable.\pause

  Use maps ($\Sv$) associates variables to multisets of kinds
  and are equipped with three operations:
  \begin{align*}
    \Sv&\cap\Sv'& \Sv&\cup\Sv' & \Sv &\leq k
  \end{align*}
\end{frame}

\begin{frame}
  \frametitle{Typing \hfill Tracking linearity}
  
  When typechecking (e1, e2):
  \begin{itemize}[<+->]
  \item $\inferW{\Sv_1}{(C_1,\psi_1)}{\Gamma}{e_1}{\tau_1}$
  \item $\inferW{\Sv_2}{(C_2,\psi_2)}{\Gamma}{e_2}{\tau_2}$
  \item Add  $(\Sv_1\cap\Sv_2\leq \kun)$ to the constraints
  \item \dots
  \item Return $\Sv_1\cup\Sv_2$
  \end{itemize}

\end{frame}


\newcommand\lub\bigvee
\newcommand\glb\bigwedge
\newcommand\CL{{\mathcal C_{\mathcal L}}}

\begin{frame}
  \frametitle{Constraints}

  A slightly more general context: $\CL$ is the constraint system:
  \begin{align*}
    C ::=&\ \Cleq{\tau_1}{\tau_2}\ |\ \Cleq{k_1}{k_2}\ |\ C_1 \Cand C_2\ |\ \Cproj{\alpha}{C}
  \end{align*}

  where $k ::=\kvar\ |\ \ell \in {\mathcal L}$ and
  $(\mathcal L, \leq_{\mathcal L})$ is a complete lattice.
  \pause

  Respect, among other things:
  \begin{mathpar}
    \inferrule{\ell \leq_{\mathcal L} \ell'}{\entail{}{\Cleq{\ell}{\ell'}}}
    \and
    \inferrule{}{\entail{}{\Cleq{k}{\ell^\top}}}
    \and
    \inferrule{}{\entail{}{\Cleq{\ell^\bot}{k}}}
  \end{mathpar}
  
\end{frame}

\begin{frame}
  \frametitle{Constraints \hfill Normalization}

  Example : $\lam{f}{\lam{x}{(\app{f}{x},x)}}$

  Raw constraints:
  \begin{gather*}
    (\tvar_f : \kvar_f)
    (\tvar_x : \kvar_x)\dots\\
    (\tvar_f \leq \gamma \tarr{\kvar_1} \beta )
    \Cand
    (\gamma \leq \tvar_x)
    \Cand
    (\beta * \tvar_x \leq \alpha_r)
    \Cand
    (\kvar_x \leq \kun)
  \end{gather*}\pause

  We unify the types and discover new constraints:
  \begin{gather*}
    \alpha_r =
    (\gamma \tarr{\kvar_3} \beta) \tarr{\kvar_2} \gamma \tarr{\kvar_1} \beta * \gamma\\
    (\kvar_x \leq \kun)
    \Cand
    (\kvar_\gamma \leq \kvar_x)
    \Cand
    (\kvar_x \leq \kvar_r)
    \Cand
    (\kvar_\beta \leq \kvar_r)
    \Cand
    (\kvar_3 \leq \kvar_f)
    \Cand
    (\kvar_f \leq \kvar_1)
  \end{gather*}
\end{frame}

\begin{frame}[plain]
  \centering
  \begin{gather*}
    (\gamma : \kvar_\gamma) (\beta : \kvar_\beta).\ 
    (\gamma \tarr{\kvar_3} \beta) \tarr{\kvar_2} \gamma \tarr{\kvar_1} \beta * \gamma\\
    \visible<7->{\kvar_\gamma = \kvar_x = \kun}
    \visible<13>{\Cand \kvar_3 \leq \kvar_1}
  \end{gather*}
  \begin{tikzpicture}[every edge/.style=[link,->,>=latex,thick]]
    \node<-11> (U) {$\kun$} ;
    \node<-7>[above left=of U] (x) {$\kvar_x$} ;
    \node<-10>[above=of x] (r) {$\kvar_r$};
    \node<-7>[left=of x] (g) {$\kvar_\gamma$} ;
    \node[right=of x] (b) {$\kvar_\beta$} ;
    \node[right=of b] (3) {$\kvar_3$} ;
    \node<-10>[above=of 3] (f) {$\kvar_f$} ;
    \node[above=of f] (1) {$\kvar_1$} ;

    \draw<1> (x) -> (U) ;
    \draw<2-7> (x) to[bend right] (U) ;
    \draw<-7> (x) -> (r) ;
    \draw<-10> (b) -> (r) ;
    \draw<-7> (g) -> (x) ;
    \draw<-10> (3) -> (f) ;
    \draw<-10> (f) -> (1) ;

    \only<2-11>{\begin{scope}[dashed,gray]
        \draw<-7> (U) to[bend right] (x) ;
        \draw (U) to (b) ;
        \draw<-7> (U) to[bend left] (g) ;
        \draw<-10> (U) to (r) ;
        \draw (U) to (1) ;
        \draw (U) to (3) ;
        \draw<-10> (U) to (f) ;
      \end{scope}}
    \visible<3-11>{\begin{scope}[dashed,gray]
      \node[above left= 0.2 and 1 of 1] (A) {$\kaff$} ;
      \draw<-7> (x) to (A) ;
      \draw (b) to (A) ;
      \draw<-7> (g) to (A) ;
      \draw<-10> (r) to (A) ;
      \draw (1) to (A) ;
      \draw (3) to (A) ;
      \draw<-10> (f) to (A) ;
      \draw (U) to[out=0,in=0,looseness=1.5] (A) ;
    \end{scope}}

  \visible<4-5>{
    \begin{scope}[scale=0.7,xscale=0.9,transform shape, xshift=6.5cm,yshift=2cm]
      \node (a1) {$\kvar$};
      \node[above right=of a1] (a2) {$\kvar_1$};
      \node[above left=of a1] (a3) {$\kvar_2$};
      \node[above=of a2] (l2) {$\ell_1$};
      \node[above=of a3] (l3) {$\ell_2$};
      \draw (a1) to (a2) ;
      \draw (a1) to (a3) ;
      \draw (a2) to (l2) ;
      \draw (a3) to (l3) ;
    \visible<5>{\begin{scope}[dashed,gray]
      \node[above=of a1] (l1) {$\ell_1 \wedge \ell_2$};
      \draw (a1) to (l1) ;
      \draw (l1) to (l2) ;
      \draw (l1) to (l3) ;
    \end{scope}}
      \node[scale=1.3, xscale=1.1,draw=Grey, rectangle, rounded corners=10pt, fit=(a1) (l2) (l3)] {};
    \end{scope}
  }

  \begin{scope}[on background layer]
    \visible<6-7>{
      \node[fill=green!20,draw=green,
      inner sep=-1mm,ellipse,rotate fit=-20,fit=(U) (x) (g)] {};
    }
    \visible<9-10>{
      \node[fill=red!20,draw=red, inner sep=-2pt, circle,fit=(r)] {};
      \node[fill=red!20,draw=red, inner sep=-2pt, circle,fit=(f)] {};
    }
  \end{scope}

  \draw<10-13> (3) to[bend right] (1) ;

  \end{tikzpicture}
  
\end{frame}

\begin{frame}
  \frametitle{Constraints \hfill Normalization}

  Normalization is complete and principal.
  \begin{gather*}
    \lam{f}{\lam{x}{(\app{f}{x},x)}}: \\
    \forall \kvar_\beta \kvar_1 \kvar_2 \kvar_3
    (\gamma : \kun) (\beta : \kvar_\beta).\ %
    \qual
    {\Cleq{\kvar_3}{\kvar_1}}
    {(\gamma \tarr{\kvar_3} \beta) \tarr{\kvar_2} \gamma \tarr{\kvar_1} \beta * \gamma}
  \end{gather*}
  

  
\end{frame}

\defverbatim{\CompAA}{
\begin{lstlisting}
let compose f g x = f (g x)
val compose : 
  ('k1 < 'k) & ('k1 < 'k3) =>
  ('b -{'k1}> 'a) -> ('c -{'k}> 'b) -{'k3}> 
  'c -{'k}> 'a
\end{lstlisting}
}


\begin{frame}[fragile]
  \frametitle{Constraints \hfill Simplification rules}

  Well known simplifications on constraints:
  \begin{itemize}
  \item Replace variable in positive position by their lower bound
  \item Replace variable in negative position by their upper bound
  \end{itemize}

  \begin{align*}
    \only<2>{
      \forall \kvar_\beta \kvar_1 \kvar_2 \kvar_3
      (\gamma : \kun) (\beta : \kvar_\beta).&
      \qual
      {\Cleq{\kvar_3}{\kvar_1}}
      {(\gamma \tarr{\kvar_3} \beta) \tarr{\kvar_2} \gamma \tarr{\kvar_1} \beta * \gamma}}
    \only<3>{
      \forall \kvar_\beta \kvar_1 \kvar_3
      (\gamma : \kun) (\beta : \kvar_\beta).&
      \qual
      {\Cleq{\kvar_3}{\kvar_1}}
      {(\gamma \tarr{\kvar_3} \beta) \tarr{} \gamma \tarr{\kvar_1} \beta * \gamma}}
    \only<4->{
      \forall \kvar_\beta \kvar
      (\gamma : \kun) (\beta : \kvar_\beta).&
      {(\gamma \tarr{\kvar} \beta) \tarr{} \gamma \tarr{\kvar} \beta * \gamma}}
  \end{align*}
  \visible<5>{
    $\implies$ Unfinished, need to investigate principality
  }
\end{frame}

\begin{frame}
  \frametitle{Constraints}
  The only requirement is that $\ell^\bot = \kun$.\pause

  \begin{itemize}
  \item $\kaff$ doesn't appear in the typing rules.

    It only comes from the buitins and/or the type declarations.
  \item The lattice doesn't have to be finite.
  \item The constraint language can be expanded further.
  \end{itemize}
\end{frame}

\begin{frame}
  \frametitle{Conclusion}

  I presented a somewhat minimalistic approach to add linear types to an existing ML language (like \ocaml).

  \begin{itemize}
  \item Based on kinds and constraints
  \item Works with type abstraction and modules
  \item Support type inference
  \item Doesn't break the whole ecosystem
  \end{itemize}

  The system is still small. We must look at concrete code pattern used in \ocaml and decide how to support them.
  
  Area of future work:
  \begin{itemize}
  \item Explore various interactions with modules
  \item Borrowing
  \item Better control-flow interactions
  \end{itemize}
\end{frame}

\begin{frame}[plain,standout]
  \Huge{\texttt{Close(Talk)}}
\end{frame}

\begin{frame}
  \frametitle{Really??}

  \begin{quote}
    Do you really think adding kinds, subkinding and qualified types to \ocaml is a good idea?
  \end{quote}\pause
  Yes, I do!\pause
  \begin{itemize}
  \item Qualified types are coming for modular implicits anyway.
  \item Having proper kinds would fix many weirdness (rows, \dots) and
    enable nice extensions (units of measures).\pause
  \item I could make Eliom even better with them! \smiley{}
  \end{itemize}
\end{frame}


\section{Going further}


\begin{frame}{Current area of work}

  \begin{enumerate}
  \item Richer type system
  \item Modules
  \item Borrowing
  \item Prototype cool APIs with it
  \end{enumerate}
  
\end{frame}

\begin{frame}
  \frametitle{Constraints \hfill Extensions}
  Constraints in a similar style have been applied to:
  \begin{itemize}
  \item (Relaxed) value restriction
  \item GADTs
  \item Rows
  \item Type elaboration
  \item \dots
  \end{itemize}
\end{frame}

\begin{frame}
  \frametitle{Modules}

  Several distinct problematic:
  \begin{itemize}
  \item \alert<2>{Type abstraction} \only<2>{{\color{green}\ding{52}}
      
      Can declare unrestricted types and expose them as Affine.
    }
  \item \alert<3>{Linear/affine values in modules} \only<3>{

      Behave like tuples: take the LUB of the kinds of the exposed values.

      What about values that are not exposed? They don't matter!
    }
  \item \alert<4>{Functors}\only<4>{

      What happens if a functor takes a module containing affine values?
      
      $\implies$ We need kind annotation on the functor arrow\dots \frownie{}
    }
  \item \alert<5->{Separate compilation}\only<5->{

      What about linear/affine constants?

      $\implies$ Should probably be forbidden\dots

      \visible<6>{But what about \lstinline{stdout} ?}
    }
  \end{itemize}
  
\end{frame}

\begin{frame}[fragile,fragile,fragile]
  \frametitle{Borrowing}

  Borrowing seem essential to express many patterns found in \ocaml.\pause

  Read-only borrows, in \texttt{CCHashTrie}:
\begin{lstlisting}
val add_mut : id -> key -> 'a -> 'a t -> 'a t
(* add_mut ~id k v m behaves like add k v m, except
   it will mutate in place whenever possible. *)
\end{lstlisting}
  \pause
%   Read-only borrows, in \texttt{duff}
% \begin{lstlisting}
% val make: ?copy:bool -> Cstruct.t -> t
% (* [make ?copy raw] returns a Rabin's fingerprint of [raw]. [make] expects to take the
%    ownership of [raw]. *)
% \end{lstlisting}

  Mutable borrows, in \texttt{lacaml}:
\begin{lstlisting}
val Lacaml.D.sycon : 
  ... -> ?iwork:Common.int32_vec -> mat -> float
(* iwork is an optional preallocated work buffer *)
\end{lstlisting}
  
\end{frame}

\begin{frame}[fragile]{Borrowing}

  ``Resource Polymorphism'' has the following lattice:
  \begin{center}
  \begin{tikzpicture}
    \node (Copy) {Copy};
    \node (Own) at (2,2) {Own};
    \node (Copy@r) at (-2,1) {Copy@r};
    \node (Seq@r) at (-1,2) {Seq@r};
    \node (Own@r) at (0,3) {Own@r};

    \draw (Copy) -- (Own) -- (Own@r) -- (Seq@r) -- (Copy@r) -- (Copy) ;
  \end{tikzpicture}
\end{center}
  % \pause
  It would requires:
  \begin{itemize}
  \item More syntactic annotations
  \item Regions
  \end{itemize}
  
\end{frame}


\begin{frame}
  Which kind of linearity?
  \begin{itemize}
  \item \alert<2>{Ownership approaches}
    
    \only<2>{
      Suitable to imperative languages (Rust, \dots).
    }
  \item \alert<3>{Capabilities and typestates}
    
    \only<3>{
      Often use in Object-Oriented contexts (Wyvern, Plaid, Hopkins Objects Group, \dots).
    }
  \item \alert<4,6>{Substructural type systems}
    
    \only<4>{
      Many variations, mostly in functional languages:
      \begin{itemize}
      \item Inspired directly from linear logic (Linear Haskell, Walker, \dots)
      \item Uniqueness (Clean)
      \item Kinds (Alms, Clean, F$^\circ$)
      \item Constraints (Quill)
      \end{itemize}
    }
  \item \alert<5>{\dots}
    
    \only<5>{Mix of everything: Mezzo}
  \end{itemize}
\end{frame}

\begin{frame}
  \frametitle{The HM(X) framework}
  HM(X) \citep{DBLP:journals/tapos/OderskySW99} is a framework
  to build an HM type system (with inference) based on a given constraint system.

  We provide two additions:
  \begin{itemize}
  \item A small extension of HM(X) that tracks kinds and linearity
  \item An appropriate constraint system
  \end{itemize}
\end{frame}


\bibliography{biblio}

\end{document}

%%% Local Variables:
%%% mode: latex
%%% TeX-master: t
%%% End:

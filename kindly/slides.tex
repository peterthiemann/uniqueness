
\documentclass[xcolor=svgnames,11pt]{beamer}

\usepackage[T1]{fontenc}
\usepackage[utf8]{inputenc}
\usepackage[french, english]{babel}
\selectlanguage{english}

\usepackage{graphicx}

% Math
\usepackage{amsmath}
% \usepackage{amsfonts}
\usepackage{amssymb}
\usepackage{amsthm}
% \usepackage{mathrsfs}
\usepackage{mathtools}
\usepackage{textcomp}
% \usepackage{textgreek}

% Specialized packages
% \usepackage{syntax} % Grammar definitions
\usepackage{verbatim}
\usepackage{listings} % Code
\usepackage{xspace} % Useful for macros
\usepackage{natbib}% Good citations and bibliography
\usepackage{mathpartir} % Syntax trees
\usepackage{colortbl}
\usepackage{hhline}
\usepackage{multicol}%multicolumn lists
\usepackage{pifont}%% check mark

% \usepackage{mathptmx}
% \usepackage[scaled=0.9]{helvet}
\usepackage{beramono}

\bibliographystyle{ACM-Reference-Format}
\citestyle{acmauthoryear}
\usepackage{appendixnumberbeamer}

\usetheme{metropolis}
\beamertemplatenavigationsymbolsempty
\setbeamercovered{transparent=20}

\usepackage{wasysym}

\usepackage{tikz}
\usetikzlibrary{decorations.text,backgrounds,positioning,shapes,
  shadings,shadows,arrows,decorations.markings,calc,fit,fadings,
  tikzmark
}

\def\HUGE{\fontsize{35pt}{15pt}\selectfont}

\newcommand\htag[1]{\shortintertext{\textbf{#1}}}

\newcommand\TODO[1]{{\ \\\color{red}\large\textbf{TODO} #1}\\}

%%% Local Variables:
%%% mode: latex
%%% TeX-master: "main"
%%% End:

\usepackage{xcolor}
%\definecolor{butter}{HTML}{FCE94F}
%\definecolor{butter}{HTML}{EDD400}
\definecolor{butter}{HTML}{C4A000}
%\definecolor{orange}{HTML}{FCAF3E}
%\definecolor{orange}{HTML}{F57900}
\definecolor{orange}{HTML}{CE5C00}
%\definecolor{chocolate}{HTML}{E9B96E}
%\definecolor{chocolate}{HTML}{C17D11}
\definecolor{chocolate}{HTML}{8F5902}
%\definecolor{chameleon}{HTML}{8AE234}
%\definecolor{chameleon}{HTML}{73D216}
\definecolor{chameleon}{HTML}{4E9A06}
%\definecolor{skyblue}{HTML}{729FCF}
%\definecolor{skyblue}{HTML}{3465A4}
\definecolor{skyblue}{HTML}{204A87}
%\definecolor{plum}{HTML}{AD7FA8}
%\definecolor{plum}{HTML}{75507B}
\definecolor{plum}{HTML}{5C3566}
%\definecolor{scarletred}{HTML}{EF2929}
%\definecolor{scarletred}{HTML}{CC0000}
\definecolor{scarletred}{HTML}{A40000}
%\definecolor{lightalu}{HTML}{EEEEEC}
%\definecolor{lightalu}{HTML}{D3D7CF}
\definecolor{lightalu}{HTML}{BABDB6}
%\definecolor{darkalu}{HTML}{888A85}
%\definecolor{darkalu}{HTML}{555753}
\definecolor{darkalu}{HTML}{2E3436}

\newcommand{\kwstyle}{}

\lstset{
% backgroundcolor=\color{},
        basicstyle=\scriptsize\ttfamily,
% breakatwhitespace=false,
  breaklines=true,
% captionpos=b,
        aboveskip=5pt,
        belowskip=5pt,
        xleftmargin=0pt,
  commentstyle=\color{orange},
  % columns=[c]spaceflexible,
  % flexiblecolumns=false,
% deletekeywords={...},
% extendedchars=true,
% frame=l,
% keepspaces=true,
  keywordstyle=\kwstyle,
  language=Caml,
% morekeywords={*,...},
% numbers=left,
% numbersep=5pt,
% numberstyle=\color{},
% rulecolor=\color{},
% showspaces=false,
% showstringspaces=false,
% showtabs=false,
% stepnumber=2,
  stringstyle=\color{plum},
  tabsize=2,
  numberstyle=\tiny\color{gray},
  escapeinside={(*@}{*)},
  numbers=left,
  numbersep=2pt,
% title=\lstname,
  keywordstyle=[1]\kwstyle\color{chameleon},
  keywordstyle=[2]\kwstyle\color{scarletred},
  keywordstyle=[3]\kwstyle\color{skyblue},
  keywordstyle=[4]\kwstyle\color{butter},
  keywordstyle=[5]\kwstyle\color{skyblue},
  keywordstyle=[6]\kwstyle\color{skyblue},
  keywordstyle=[7]\kwstyle\color{chameleon},
  keywordstyle=[8]\kwstyle\color{butter},
  keywordstyle=[9]\kwstyle\color{butter},
  keywords=[1]{let,val,method,in,and,rec,private,virtual,constraint},
  keywords=[2]{type,open,class,module,exception,external},
  keywords=[3]{fun,function,functor,match,try,with},
  keywords=[4]{as,when,of},
  keywords=[5]{if,then,else},
  keywords=[6]{begin,end,object,struct,sig,for,while,do,done,to,downto},
  keywords=[7]{true,false},
  keywords=[8]{include,inherit,initializer},
  keywords=[9]{new,ref,mutable,lazy,assert,raise},
  keywords=[10]{lin,aff,un},
}

\lstset{literate=
  % {0}{{{\kwstyle\color{plum}0}}}1 {0.}{{{\kwstyle\color{plum}0.}}}2
  % {1}{{{\kwstyle\color{plum}1}}}1 {1.}{{{\kwstyle\color{plum}1.}}}2
  % {2}{{{\kwstyle\color{plum}2}}}1 {2.}{{{\kwstyle\color{plum}2.}}}2
  % {3}{{{\kwstyle\color{plum}3}}}1 {3.}{{{\kwstyle\color{plum}3.}}}2
  % {4}{{{\kwstyle\color{plum}4}}}1 {4.}{{{\kwstyle\color{plum}4.}}}2
  % {5}{{{\kwstyle\color{plum}5}}}1 {5.}{{{\kwstyle\color{plum}5.}}}2
  % {6}{{{\kwstyle\color{plum}6}}}1 {6.}{{{\kwstyle\color{plum}6.}}}2
  % {7}{{{\kwstyle\color{plum}7}}}1 {7.}{{{\kwstyle\color{plum}7.}}}2
  % {8}{{{\kwstyle\color{plum}8}}}1 {8.}{{{\kwstyle\color{plum}8.}}}2
  % {9}{{{\kwstyle\color{plum}9}}}1 {9.}{{{\kwstyle\color{plum}9.}}}2
  % {->}{{{\kwstyle\color{chameleon}->}}}2
  {->}{{{$\tarr{}$}}}2
  {->.}{{{$\multimap$}}}2
  {=>}{{{$\Rightarrow{}$}}}1
  {<=}{{$\le$}}1
  % {un}{{{$\kun$}}}1
  % {lin}{{{$\klin$}}}1
  % {aff}{{{$\kaff$}}}1
  {un}{{{$\texttt{un}$}}}2
  {lin}{{{$\texttt{lin}$}}}3
  {aff}{{{$\texttt{aff}$}}}3
  {_r}{{${}_r$}}1
  {_r1}{{${}_{r+1}$}}2
  {-\{'k\}>}{{{$\tarr{\kvar}$}}}2
  {-\{'k_1\}>}{{{$\tarr{\kvar_1}$}}}2
  % {-\{un\}>}{{{$\tarr{\kun}$}}}3
  % {-\{lin\}>}{{{$\tarr{\klin}$}}}3
  % {-\{aff\}>}{{{$\tarr{\kaff}$}}}3
  {-\{un\}>}{{{$\tarr{\texttt{un}}$}}}3
  {-\{lin\}>}{{{$\tarr{\texttt{lin}}$}}}3
  {-\{aff\}>}{{{$\tarr{\texttt{aff}}$}}}3
  {-\{aff_r1\}>}{{{$\tarr{\texttt{aff}_{r+1}}$}}}5
  {-\{un_r1\}>}{{{$\tarr{\texttt{un}_{r+1}}$}}}5
  {-\{k\}>}{{{$\tarr{k}$}}}3
  {'a}{{{$\alpha$}}}1
  {tt}{{{$\tau$}}}1
  {kk}{{{$k$}}}1
  {'b}{{{$\beta$}}}1
  {'c}{{{$\gamma$}}}1
  {'k}{{{$\kvar$}}}1
  {'S}{{{$\mathcal S$}}}1
  {'T}{{{$\mathcal T$}}}1
  {`}{{{\lq}}}1
  {_1}{{{${}_1$}}}1
  {_2}{{{${}_2$}}}1
  {\{|}{{{\bfseries\kwstyle\color{gray}\{|}}}2
  {|\}}{{{\bfseries\kwstyle\color{gray}|\}}}}2
  {\\E}{$\exists$}1
}

%%% Local Variables:
%%% mode: latex
%%% TeX-master: "main"
%%% End:

\newcommand\lang{Affe\xspace}

%% Syntax

% Kinds
\newcommand\lk{\leq}
\newcommand\klin{\mathbf{L}}
\newcommand\kaff{\mathbf{A}}
\newcommand\kun{\mathbf{U}}
\newcommand\karr{\operatorname{\rightarrow}}
\newcommand\kvar{\kappa}
\newcommand\K[1]{\mathrm{#1}}
\newcommand\loc{\ell}

% Lattice
\newcommand\lub\bigvee
\newcommand\glb\bigwedge
\newcommand\Lat{\mathcal L}
\newcommand\CL{{\mathcal C_{\Lat}}}

% Types
\newcommand\T[1]{\mathrm{#1}}
\newcommand\tvar{\alpha}
\newcommand\schm{\sigma}
\newcommand\kschm{\theta}
\newcommand\tarr[1]{{\xrightarrow{#1}}}
\newcommand\tapp[2]{#2\ \T{#1}}
\newcommand\qual[2]{#1 \operatorname{\Rightarrow} #2}
\newcommand\tyPair[3][{}]{#2 \times^{#1} #3}

\newcommand\instanciate[2]{#1 \operatorname{\succ} #2}
\newcommand\generalize[3]{\operatorname{\text{gen}}(#1,#2,#3)}

\newcommand\unif{\mathrm{\psi}}
\newcommand\meet{\sqcup}
\newcommand\meeti{\bigsqcup}
\newcommand\mostgeneral{\sqcup}

\newcommand\tydecl[4]{(\T{#1} : #2 = #3\ \mathtt{of}\ #4)}

% Expressions
\newcommand\LAM[1]{\Lambda #1 .}
\newcommand\APP[2]{#1[ #2]}

\newcommand\ilam[5]{\lambda[#1; #2 \mid #3 \Rightarrow #4]#5.}
\newcommand\ivar[3]{\APP {#1} {#2; #3}}

\newcommand\lam[2][{}]{\lambda^{#1} #2.}
\newcommand\elet{\mathtt{let}}
\newcommand\ematch{\mathtt{match}}
\newcommand\ein{\mathtt{in}}
\newcommand\letin[3]{\elet\ #1 = #2\ \ein\ #3}
\newcommand\matchin[4][\etransfm]{\ematch_{#1}\ #2 = #3\ \ein\ #4}
\newcommand\introPair[3][{}]{({#2},{#3})^{#1}}
\newcommand\introK[2]{\app{#1}{#2}}
\newcommand\elimK[2]{\mathtt{elim}_{#1}\ #2}
\newcommand\fix[1]{\mathtt{fix}\ #1}
\newcommand\app[2]{(#1\ #2)}
\newcommand\borrow[2][\BORROW]{{\&}^{#1}#2}
\newcommand\borrowty[3][\BORROW]{{\&}^{#1}(#2,#3)}
\newcommand\region[2]{\{\!|#2|\!\}^n_{#1}}
\newcommand\regionS[1]{\{\!|#1|\!\}}

\newcommand\sborrow[2][\BORROW]{#2{#1}}
\newcommand\etransfm{\phi}
\newcommand\transfm[1]{\etransfm(#1)}


\newcommand\create[1]{\ensuremath{\mathtt{create}\ #1}}
\newcommand\rss[1]{[#1]}
\newcommand\observe[1]{\ensuremath{\mathtt{observe}\ #1}}
\newcommand\update[2]{\ensuremath{\mathtt{update}\ #1\ #2}}
\newcommand\destroy[1]{\ensuremath{\mathtt{destroy}\ #1}}

%% Region annotation

\newcommand\Rannot[4][n]{#2 \rightsquigarrow_{#1} #3, #4}
\newcommand\getBorrows[3]{#1 \oplus #2 = #3}

%% Constraints
\newcommand\Ctrue{\operatorname{True}}
\newcommand\Cempty{\cdot}
\newcommand\Ckind[2]{(#1 : #2)}
\newcommand\Cleq[2]{(#1 \leq #2)}
\newcommand\Ceq[2]{(#1 = #2)}
\newcommand\Cand{\wedge}
\newcommand\Cproj[2]{\exists #1.#2}
\newcommand\Weaken{\operatorname{Weaken}}

%% Typing environments
\newcommand\Eempty{\cdot}
\newcommand\E{\Gamma}
\newcommand\bkind[1]{(#1)}
\newcommand\bnone{\emptyset}
\newcommand\svar[3][\BORROW]{[#2 : #3]_{#1}}
\newcommand\bvar[2]{(#1 : #2)}
\newcommand\bshadow[2][\BORROW]{(#2)_{#1}}
\newcommand\bty[3]{(\T{#1} : #2 = #3)}
\newcommand\bineq[2]{(#1 \lk #2)}

\newcommand\esplit[2]{#1 \operatorname{\ltimes} #2}
\newcommand\lsplit[4]{#1 \vdash_e #2 = #3 \ltimes #4}
\newcommand\bsplit[4]{#1 \Lleftarrow #2 = #3 \ltimes #4}
\newcommand\lregion[4]{#1 \vdash_e #3 \rightsquigarrow_n^{#2} #4}
\newcommand\bregion[4]{#1 \Lleftarrow #3 \rightsquigarrow_n^{#2} #4}
\newcommand\fv[1]{\operatorname{fv}(#1)}

%% Sets
\newcommand\Sempty{\cdot}
\newcommand\Sone[2]{\left\{#1 \to #2\right\}}
\newcommand\Sunion{\operatorname{\cup}}
\newcommand\Sinter{\operatorname{\cap}}
\newcommand\Sv{\Sigma}
\newcommand\Sdel[1]{\setminus \{#1\}}
\newcommand\Sonly[1]{\big|_{#1}}
\newcommand\Sadd[1]{\cup \{#1\}}

%% Judgements
\newcommand\BAR{\operatorname{|}}
\newcommand\dash{\operatorname{\vdash}}
\newcommand\entail[2]{#1 \operatorname{\vdash_e} #2}
\newcommand\equivC{\operatorname{=_e}}
\newcommand\inferS[4]{#1 \BAR #2 \operatorname{\vdash_s} #3 : #4}
\newcommand\inferW[5]{#1 \BAR #2 \BAR #3 \operatorname{\vdash_w} #4 : #5}
\newcommand\inferSK[4]{#1 \BAR #2 \operatorname{\vdash_s} #3 : #4}
\newcommand\inferK[4]{#1 \BAR #2 \operatorname{\vdash_w} #3 : #4}

% subkinding
\newcommand\inferSS[4]{#1 \BAR #2 \operatorname{\vdash_s} #3 \le #4}


\newcommand\tyval[3]{#1 \BAR #2 \operatorname{\vdash_t} #3}
\newcommand\schval[3]{#1 \BAR #2 \operatorname{\vdash_t} #3}

\newcommand\normalize[2]{\operatorname{normalize}(#1,#2)}

\newcommand\Dom[1]{\operatorname{dom} (#1)}


%% Reductions

\newcommand\subst[3]{#3[#1\rightarrow#2]}
\newcommand\fresh{\operatorname{\text{fresh}}}
\newcommand\closure[3]{\lambda^{#1} #2 . #3}
\newcommand\ered[4]{#3\ \operatorname{\Downarrow^{#1}_{#2}}\ #4}

\newcommand\IF[3]{\text{if } (#1) \text{ then } #2 \text{ else } #3}

\newcommand\MBORROW{m}
\newcommand\IBORROW{i}
\newcommand\BORROW{\textit{b}}
\newcommand\Addr\rho           %addresses
\newcommand\Loc\ell             %locations

\newcommand\blob\bullet

\newcommand\Multi[2][{}]{\overline{#2_{#1}}}

%% Misc
\newcommand\addlin[1]{{\color{blue}{#1}}}

%%
\newcommand\CType[1]{\operatorname{\text{CType}} (#1)} %constant type
\newcommand\IType[2]{\operatorname{\text{IType}} (\T{#1}, #2)} % implementation type
\newcommand\Bcompatible{\sim}
\newcommand\VEnv\gamma          %value environments
\newcommand\Store\delta % {s}
\newcommand\SE\Delta            % store typing
\newcommand\Perm\pi

\newcommand\TimeOut{\operatorname{TimeOut}}
\newcommand\Ok[1]{\operatorname{Ok} (#1)}
\newcommand\Nat{\mathbf{N}}
\newcommand\Addresses[1]{\operatorname{addr} (#1)}
\newcommand\Reach[2]{\operatorname{reach}_{#1} (#2)}
\newcommand\Writeable[2]{\operatorname{writeable}_{#1} (#2)}

\newcommand\Active[1]{{#1}_{\bullet}}
\newcommand\MutableBorrows[1]{{#1}_{\circledast}}
\newcommand\ImmutableBorrows[1]{{#1}_{\circ}}

\newcommand\Matches[2]{#1 \gets #2}        %pattern matching

%%% Local Variables:
%%% mode: latex
%%% TeX-master: "main"
%%% End:



\lstset{
  tabsize=4,
  aboveskip={0.5\baselineskip},
  belowcaptionskip=0.5\baselineskip,
  basicstyle=\scriptsize\ttfamily,
}

\title{Kindly Bent To Free Us}
\author{\textbf{Gabriel \textsc{Radanne}}
  \and Hannes \textsc{Saffrich}
  \and Peter \textsc{Thiemann}}

\begin{document}

\frame[plain]{\titlepage}

\begin{frame}[fragile]

Simplified API from the library \texttt{ocaml-tls}:
  
\begin{lstlisting}
val channels : Tls.fd -> in_channel * out_channel
(* Turn a file descr into input/output channels *)
\end{lstlisting}\pause
\begin{lstlisting}
let fd : Tls.fd = .....
let input, output = Tls.channels fd
let x = read_stuff input in
let () = close input in (*@\pause*)
...
let c = write output "thing" in (*Oups*)
...
\end{lstlisting}
  \pause

  The default behavior is to close the underlying file descriptor when a channel is closed.

  This bug was found in the wild, and then fixed in \texttt{ocaml-tls}.
\end{frame}

\begin{frame}
  Many partial solutions
  \begin{itemize}
  \item Closures
  \item Monads
  \item Existential types
  \item \dots
  \end{itemize}
  \pause

  What we really need is to enforce linearity.
\end{frame}

\begin{frame}
  Many places in \ocaml where enforcing linearity is useful:
  \begin{itemize}
  \item IO (File handle, channels, network connections, \dots)
  \item Protocols (With session types! Mirage libraries)
  \item One-shot continuations (effects!)
  \item Transient data-structures
  \item C-style ``struct parsing''
  \item \dots
  \end{itemize}
\end{frame}

\begin{frame}
  Goals:
  \begin{itemize}
  \item Complete and principal type inference
  \item Impure and strict context
  \item Support both functional and imperative styles
  \item Works well with type abstraction
  % \item Play balls with various other ongoing works (Effects, Resource polymorphism, \dots)
  \end{itemize}

  \pause
  Non Goals:
  \begin{itemize}
  \item Support every linear code pattern under the sun
  \item Design associated compiler optimisations/GC integration (yet)
  \end{itemize}
\end{frame}


\section{The \lang language}

\lstMakeShortInline[keepspaces,basicstyle=\small\ttfamily]@


\begin{frame}[t,fragile]

  Let's create an @LinArray@ API together!
  
  In \lang, the behavior of a variable is determined
  by its type:
\begin{lstlisting}
module LinArray : sig
  type ('a : un) t : lin (* LinArrays are linear! *)
  val create : int -> 'a -> 'a t
  val free : 'a t -> unit
end
\end{lstlisting}
\begin{onlyenv}<2>
\begin{lstlisting}
let main () =
  let a = LinArray.create 3 "foo" (* : string t *)
  .... (* a is linear *)
  LinArray.free a ;
\end{lstlisting}

No type annotation!
\end{onlyenv}
\begin{onlyenv}<3>
\begin{lstlisting}
let main () =
  let a = LinArray.create 3 "foo" (* : string t *)
  .... (* a is linear *)
  LinArray.free a ;
  f a (* (*@\color{red}\ding{56} No!*) *)
\end{lstlisting}
\end{onlyenv}

\end{frame}


\begin{frame}[t,fragile]

\begin{onlyenv}<1-2>
  How to read the array ?
\begin{lstlisting}
module LinArray : sig
  type ('a : un) t : lin
  val create : int -> 'a -> 'a t
  val free : 'a t -> unit
  val get : 'a t * int -> 'a (* ? *)
end
\end{lstlisting}\pause
\begin{lstlisting}
let main () =
  let a = LinArray.create 3 "foo"
  let x = LinArray.get (a, 2) in
  LinArray.free a  (* (*@\color{red}\ding{56} No!*) *)
  print x
\end{lstlisting}

This doesn't work!
\end{onlyenv}
\begin{onlyenv}<3>
  How to read the array ?
\begin{lstlisting}
module LinArray : sig
  type ('a : un) t : lin
  val create : int -> 'a -> 'a t
  val free : 'a t -> unit
  val get : 'a t * int -> 'a * 'a t (* ?? *)
end
\end{lstlisting}
\begin{lstlisting}
let main () =
  let a = LinArray.create 3 "foo"
  let x, a = LinArray.get (a, 2) in
  LinArray.free a ;
  print x
\end{lstlisting}
This works, but is inconvenient!
\end{onlyenv}
\begin{onlyenv}<4>
  How to read the array ?
\begin{lstlisting}
module LinArray : sig
  type ('a : un) t : lin
  val create : int -> 'a -> 'a t
  val free : 'a t -> unit
  val get : &('a t) * int -> 'a
end
\end{lstlisting}
\begin{lstlisting}
let main () =
  let a = LinArray.create 3 "foo"
  let x = LinArray.get (&a, 2) in (* Borrow *)
  LinArray.free a
\end{lstlisting}
We use borrows!

We temporarily give @&a@ to @LinArray.get@.
\end{onlyenv}
\end{frame}

\begin{frame}[fragile]{A recap on borrows}

  Borrows allow to lend usage of something to someone else.

  There are different types of borrows:
  \begin{itemize}
  \item Shared borrows @&a@ which are \emph{Unrestricted} (@un@)
  \item Exclusive borrows @&!a@ which are \emph{Affine} (@aff@)
  \end{itemize}

  We cannot use a borrow of @a@ and @a@ itself at the same time.\\
  A borrow must not escape.
\end{frame}

\begin{frame}[t,fragile]
\begin{lstlisting}
module LinArray : sig
  type ('a : un) t : lin
  val create : int -> 'a -> 'a t
  val free : 'a t -> unit
  val get : &('a t) * int -> 'a
  val set : &!('a t) * int * 'a -> unit
end
\end{lstlisting}

\begin{onlyenv}<1>
\begin{lstlisting}
let main () =
  let a = create 3 "foo"
  let x = get (&a, 0) ^ get (&a, 1) in
    (* (*@\color{DarkGreen}\ding{52}*) Multiple Shared borrows *)
  set (&!a, 2, x);
    (* (*@\color{DarkGreen}\ding{52}*) One Exclusive borrow *)
  free a
\end{lstlisting}
\end{onlyenv}

\begin{onlyenv}<2>
\begin{lstlisting}
let main () =
  let a = create 3 "foo"
  f (a, &a, 42)
  (* (*@\color{red} \ding{56}*) Using a and a borrow simultaneously! *)
\end{lstlisting}
\end{onlyenv}

\begin{onlyenv}<3>
\begin{lstlisting}
let main () =
  let a = create 3 "foo"
  f (&!a, &a, 42)
  (* (*@\color{red} \ding{56}*) Conflicting borrows *)
\end{lstlisting}
\end{onlyenv}

\end{frame}


\begin{frame}[t,fragile]
\begin{lstlisting}
module LinArray : sig
  type ('a : un) t : lin
  val create : int -> 'a -> 'a t
  val free : 'a t -> unit
  val get : &('a t) * int -> 'a
  val set : &!('a t) * int * 'a -> unit
end
\end{lstlisting}

A slightly bigger piece of code:

\begin{onlyenv}<1-2>
\begin{lstlisting}
let mk_fib_array n =
  let a = create n 1 in
  for i = 2 to n - 1 do
    let x = get (&a, i-1) + get (&a, i-2) in
    set (&!a, i, x)
  done;
  a
# mk_fib_array : int -> int Array.t
\end{lstlisting}

Still no type annotations: everything is inferred.\pause

Borrows must not escape $\implies$ What is their scope ?
\end{onlyenv}
\begin{onlyenv}<3>
\begin{lstlisting}
let mk_fib_array n =
  let a = create n 1 in
  for i = 2 to n - 1 do {|
    let x = {| get (&a, i-1) + get (&a, i-2) |} in
    set (&!,a, i) x
  |} done;
  a
# mk_fib_array : int -> int Array.t
\end{lstlisting}

A borrow cannot escape a region @{| .... |}@.\\
Regions are inferred automatically, but can be manually provided.
\end{onlyenv}

\end{frame}


\begin{frame}[fragile]{Closures}

  Closures can capture linear and affine values:

\begin{lstlisting}
let a = LinArray.create 10 "foo"
let f i = LinArray.set(&!a,i,"bar")
\end{lstlisting}

If @f@ can be used multiple times, we violate the usage of @&!a@.\pause

We infer:
\begin{lstlisting}
val f : int -{aff}> unit
\end{lstlisting}

Arrows are annotated with a kind (here, \emph{Affine}) denoting their use.

@->@ is equivalent to @-{un}>@.

\end{frame}


\begin{frame}[t,fragile]{Inference and polymorphism}

  So far, we have seen limited polymorphism.

  What is the type of @compose@ ?

\begin{lstlisting}
let compose f g x = f (g x)
\end{lstlisting}

  The type of @compose f g@ depends on the linearity of @f@ and @g@.
\begin{onlyenv}<2>
\begin{lstlisting}
val compose :
  ('b -{'k_1}> 'a) -> ('c -{'k_2}> 'b) -{?}> 'c -{?}> 'a
\end{lstlisting}\pause

We would expect something of the form $\kappa_1 \sqcup \kappa_2$
\end{onlyenv}

\begin{onlyenv}<3>
\begin{lstlisting}
val compose :
  ('k_1 <= 'k_2) =>
  ('b -{'k_1}> 'a) -> ('c -{'k_2}> 'b) -{'k_1}> 'c -{'k_2}> 'a
\end{lstlisting}

We use kind inequalities and subkinding to express such constraints.

This type is the most general and is inferred.
\end{onlyenv}

\end{frame}


\begin{frame}[fragile]{A more general API}

We can now generalize @LinArray@ to arbitrary content:

\begin{lstlisting}
module LinArray : sig
  type ('a : 'k) t : lin
  val create : ('a : un) => int -> 'a -> 'a t
  val init : (int -> 'a) -> int -> 'a t
  
  val free : ('a : aff) => 'a t -> unit
  
  val length : &('a t) -> int
  
  val get : ('a : un) => &('a t) * int -> 'a
  val set : ('a : aff) => &!('a t) * int * 'a -> unit
end
\end{lstlisting}

Each operation quantifies the type of element it accepts.\pause

What about iterations ?

\end{frame}

\begin{frame}[fragile]{Iterators and linearity}

  A naive fold function only works on unrestricted elements
\begin{lstlisting}
val fold :
  ('a : un) => ('a -> 'b -> 'b) -> 'a LinArray.t -> 'b -> 'b
\end{lstlisting}\pause

Ideally, we would like to borrow the element while folding \dots\\
But the borrow shouldn't be captured!\pause

\begin{lstlisting}
val fold :
  ('b:'k),('k <= aff_r) =>
  (&(aff_r1,'a) -> 'b -{aff_r1}> 'b) -> &('k_1,'a LinArray.t) -> 'b -{'k_1}> 'b
\end{lstlisting}

We can express such types using region variables.

\end{frame}

% \begin{frame}[fragile]{Inference at work}
%   Infer unrestricted in case of duplication:
% \begin{lstlisting}
% let f = fun c -> r := Some c ; c
% val f : ('a : U) . 'a -> 'a
% \end{lstlisting}
% \end{frame}

% \begin{frame}{The kinds so far}

%   So far, two kinds:
%   \begin{description}
%   \item[A] Affine types: can be used at most once
%   \item[U] Unrestricted types
%   \end{description}

%   Additionally, we have:
%   \begin{align*}
%     \kun &\leq \kaff
%   \end{align*}
% \end{frame}

% \defverbatim{\KA}{
% \begin{lstlisting}
% let f = fun a -> fun b -> (a, b)
% val f : 'a -> 'b -> 'a * 'b (* ? *)
% \end{lstlisting}
% }
% \defverbatim{\KB}{
% \begin{lstlisting}
% let f = fun a -> fun b -> (a, b)
% val f : ('a : 'k) => 'a -> 'b -{'k}> 'a * 'b
% \end{lstlisting}
% }

% \begin{frame}[fragile]{Inference at work}
%   What about closures?
%   \alt<2>{\KB}{\KA}
% \end{frame}


% \defverbatim{\AppA}{
% \begin{lstlisting}
% let app f x = f x
% val app :
%   'k1 < 'k2 =>
%   ('a -{'k1}> 'b) -> 'a -{'k2}> 'b
% \end{lstlisting}
% }
% \defverbatim{\AppB}{
% \begin{lstlisting}
% let app f x = f x
% val app :
%   ('a -{'k}> 'b) -> 'a -{'k}> 'b
% \end{lstlisting}
% }

% \begin{frame}[fragile]{Inference at work}
%   \AppA
% \end{frame}


% \defverbatim{\Comp}{
% \begin{lstlisting}
% let compose f g x = f (g x)
% val compose :
%   ('b -{?}> 'a) -> ('c -{?}> 'b) -{?}> 'c -{?}> 'a
% \end{lstlisting}
% }
% \defverbatim{\CompA}{
% \begin{lstlisting}
% let compose f g x = f (g x)
% val compose :
%   ('k2 < 'k) & ('k1 < 'k) & ('k1 < 'k3) =>
%   ('b -{'k1}> 'a) -> ('c -{'k2}> 'b) -{'k3}>
%   'c -{'k}> 'a
% \end{lstlisting}
% }
% \defverbatim{\CompB}{
% \begin{lstlisting}
% let compose f g x = f (g x)
% val compose :
%   ('k1 < 'k) =>
%   ('b -{'k1}> 'a) -> ('c -{'k}> 'b) -{'k1}>
%   'c -{'k}> 'a
% \end{lstlisting}
% }

% \begin{frame}[fragile]{Inference at work}
%   What about more complicated cases ?
%   \only<1>{\Comp}
%   \only<2>{\CompA}
%   \only<3>{\CompB}
% \end{frame}


% \defverbatim{\RefA}{
% \begin{lstlisting}
% type ('a : U) ref : U = ...
% \end{lstlisting}
% }
% \defverbatim{\RefB}{
% \begin{lstlisting}
% type ref : U -> U = ...
% \end{lstlisting}
% }

% \begin{frame}[fragile]{Closer look at type declarations}
%   You can annotate the kinds on type declarations.

%   Vanilla OCaml references are fully unrestricted:
%   \only<1>{\RefA}
%   \only<2-3>{\RefB}
%   \pause\pause

%   We can also have constraints on kinds. The pair type operator:
% \begin{lstlisting}
% type * : (k1 < k) & (k2 < k) => k1 -> k2 -> k
% \end{lstlisting}
% \end{frame}


% \begin{frame}[fragile]{More interesting example}

%   Mixing with abstraction:
% \begin{lstlisting}
% module AffineArray : sig
%   type -'a w : A
%   val create :
%     ('a : U) . int -> 'a -> 'a w
%   val set : 'a w -> int -{A}> 'a -{A}> 'a w

%   type +'a r : U
%   val freeze : 'a w -> 'a r
%   val get : int -> 'a r -> 'a
% end
% \end{lstlisting}
% \end{frame}

\section{A glimpse at the theory}


\begin{frame}{A glimpse at the theory}
  In the rest of this talk, we will take a closer look at:
  \begin{itemize}
  \item Kinds and constraints
  \item Inference
  \end{itemize}
\end{frame}

\begin{frame}[fragile]{More precise syntax}

  Let's clarify some syntax:

  \begin{itemize}
  \item
    Kind constants are composed of a ``quality'' (unrestricted $\kun$, Affine $\kaff$, Linear $\klin$) and a ``level'' $n\in\Nat$.
  \item
    Borrows are noted
    $\borrow[\kaff]{a}$ (Exclusive) and $\borrow[\kun]{a}$ (Shared).
  \item
    Borrow types are annotated with their kind: $\borrowty{k}{\tau}$.
  \item
    Regions annotated
    with their ``nesting'' and inner borrows.
  \end{itemize}

  Example of code:
\[
  \setlength{\jot}{-1pt}
  \begin{aligned}
  \lam{a}{}\region[1]{\Sone a\MBORROW}{\ &%
  \letin{x}{\app{f}{\borrow[\kaff]{a}}}{}\\
  &\region[2]{\Sone x\MBORROW}{g\ (\borrow[\kaff]{x})};\\
  &\region[2]{\Sone x\IBORROW}{f\ (\borrow[\kun]{x})\ (\borrow[\kun]{x})}\\
  &\!\!\!}
\end{aligned}
\]
\end{frame}

\begin{frame}{Subkinding lattice}

  \lang has subkinding. Kind constants respects the following lattice:
  
  \begin{center}
  \begin{tikzpicture}
    [->,auto,semithick, node distance=1.4cm, every node/.style={scale=1.1}]
    \node(U) {$\kun_0$} ;
    \node(A) [above left of=U] {$\kaff_0$} ;
    \node(L) [above left of=A] {$\klin_0$} ;
    \node(Un) [above right of=U] {$\kun_n$} ;
    \node(An) [above left of=Un] {$\kaff_n$} ;
    \node(Ln) [above left of=An] {$\klin_n$} ;
    \node(Uinf) [above right of=Un] {$\kun_\infty$} ;
    \node(Ainf) [above left of=Uinf] {$\kaff_\infty$} ;
    \node(Linf) [above left of=Ainf] {$\klin_\infty$} ;
    \path
    (U) edge (A)
    (A) edge (L)
    (Un) edge (An)
    (An) edge (Ln)
    (Uinf) edge (Ainf)
    (Ainf) edge (Linf)
    ;
    \path[dashed]
    (U) edge (Un)
    (A) edge (An)
    (L) edge (Ln)
    (Un) edge (Uinf)
    (An) edge (Ainf)
    (Ln) edge (Linf)
    ;
  \end{tikzpicture}
\end{center}
\end{frame}


% \begin{frame}
%   \frametitle{The calculus}
%   \begin{columns}[t]
%     \column{0.5\textwidth}
%     \begin{align*}
%       \htag{Expressions}
%       e ::=&\ c\ |\ x\ |\ \app{e}{e'}\ |\ \lam{x}{e}\\
%       % |&\ \fix{e}\tag{Fixpoint}\\
%       |&\ \letin{x}{e}{e'}\\
%       |&\ \introK{K}{e}\ |\ \elimK{K}{e}\\
%     \end{align*}
%     \column{0.5\textwidth}
%     \begin{align*}
%       \htag{Type Expressions}
%       \tau ::=&\ \tvar\ |\ \tau\tarr{k}\tau\ |\ \tapp{\tcon}{(\tau^*)}\\
%       k ::=&\ \kvar\ |\ \ell \in {\mathcal L}
%     \end{align*}
%   \end{columns}
% \end{frame}

\begin{frame}{Kinds during typing}

  A traditional linear rule for pairs:

  \begin{mathpar}
    \inferrule
    { \Gamma = \Gamma_1 \ltimes \Gamma_2 \\
      \Gamma_1 \vdash e_1 : \tau_1 \\
      \Gamma_2 \vdash e_2 : \tau_2
    }
    { \Gamma \vdash (e_1, e_2) : \tyPair{\tau_1}{\tau_2} }
  \end{mathpar}

  How to take kinds into account ?
  
\end{frame}

\begin{frame}{Kinds during typing}
  
  We propagate constraints:
  \begin{mathpar}
    \inferrule
    { \color{blue}{\lsplit{C}{\E}{\E_1}{\E_2}} \\
      \inferS{C}{\E_1}{e_1}{\tau_1} \\
      \inferS{C}{\E_2}{e_2}{\tau_2}
    }
    { \inferS{C}{\E}{\introPair{e_1}{e_2}}{\tyPair{\tau_1}{\tau_2}} }
  \end{mathpar}
  \pause
  
  And use a constraint-aware split:
  \begin{center}
    \begin{tabular}
      {@{}>{$}r<{$}@{ $\vdash_e$ }
      >{$}c<{$}@{ $=$ }
      >{$}c<{$}@{ $\ltimes$ }
      >{$}c<{$}r}

      \Cleq{\schm}{\kun_\infty}
      &\bvar{x}{\schm}&\bvar{x}{\schm}&\bvar{x}{\schm}
      &Both\\\pause

      {\Ctrue}&{B_x}&{B_x}&{\bnone}
      &Left\\
      {\Ctrue}&{B_x}&{\bnone}&{B_x}
      &Right
    \end{tabular}\\
    $\vdots$
  \end{center}
\end{frame}


\begin{frame}{Constraints}

  Constraints are a list of inequalities: $\Cleq{k}{k'}^*$

  We can only use constraints in schemes:
  \begin{align*}
    \schm ::=&\ \forall\kvar^*\forall\bvar{\alpha}{k}^*.(\qual{C}{\tau})
    &\text{Type schemes}\\
    \kschm ::=&\ \forall\kvar^*.(\qual{C}{k_i^* \karr k})
    &\text{Kind schemes}
  \end{align*}\pause

  We use these constraints to verify everything!
\end{frame}

\begin{frame}{Constraint and regions}

  Consider the following program : 
\[
  \setlength{\jot}{-1pt}
  \begin{aligned}
  &\letin{x}{\texttt{create} ()}{}\\
  &\region[n]{\Sone x\MBORROW}{g\ (\borrow[\kaff]{x})}\\
\end{aligned}
\]\pause

We deduce the following:
\begin{align*}
  &(x : \tau) \wedge (\borrow[\kaff]{x} : \borrowty{k}{\tau}) \wedge
  (\kaff_{n} \leq k) \wedge (k \leq \kaff_{\infty})\\\pause
  &(g : \borrowty[\kaff]{k}{\tau}\tarr{\kvar}\tau') \wedge (\tau' : k') \wedge
  (k' \leq \klin_{n-1})
\end{align*}
\pause
Finally, we must verify and normalize the constraints

\end{frame}


\subsection{Inference}

% \newcommand\tikzcoord[2]%
% {\tikz[remember picture,baseline=(#1.base)]{\node[coordinate] (#1) {#2};}}
% \begin{frame}
%   \frametitle{Typing}
%   \centering

%   \begin{align*}
%     \inferW
%       {\tikzmark{Sv}{\addlin{\Sv}}}
%       {(\tikzmark{C}{C},\tikzmark{psi}{\psi})}
%       {\tikzmark{E}{\Gamma}}
%       {\tikzmark{e}{e}}
%       {\tikzmark{t}{\tau}}
%   \end{align*}

%   \tikzstyle{link}=[->, >=latex, thick]
%   \begin{tikzpicture}[remember picture, overlay]
%     \node[above right=1.2 and 1.2 of pic cs:e, anchor=south] (expr)
%     {\bf Expression};
%     \draw[link] (expr) to[out=-100,in=80] ([shift={(0.8ex,1.2ex)}]pic cs:e);

%     \node[above left=1.2 and 0.2 of pic cs:E] (env) {\bf Environment};
%     \draw[link] (env) to[out=-80,in=100] ([shift={(0.8ex,1.9ex)}]pic cs:E);

%     \node[below right=0.6 and 0.8 of pic cs:t] (ty) {Type};
%     \draw[link] (ty) to[out=170,in=-70] ([shift={(0.8ex,-0.4ex)}]pic cs:t);

%     \node[below right=1.2 and 0.8 of pic cs:psi] (psi) {Unifier};
%     \draw[link] (psi) to[out=100,in=-80] ([shift={(0.8ex,-0.4ex)}]pic cs:psi);

%     \node[below left=1.2 and 0 of pic cs:C] (C) {Constraints};
%     \draw[link] (C) to[out=100,in=-80] ([shift={(0.8ex,-0.4ex)}]pic cs:C);

%     \node[below left=0 and 0.9 of pic cs:Sv] (Sv) {\addlin{Usage Map}};
%     \draw[link,blue] (Sv) to[out=20,in=180] ([shift={(0ex,0.6ex)}]pic cs:Sv);
%   \end{tikzpicture}
% \end{frame}

% \begin{frame}
%   \frametitle{Typing \hfill Tracking linearity}

%   Variables can be kind-polymorphic and all their instances might not have the same kinds.

%   $\implies$ We must track the kinds of all use-sites for each variable.\pause

%   Use maps ($\Sv$) associates variables to multisets of kinds
%   and are equipped with three operations:
%   \begin{align*}
%     \Sv&\cap\Sv'& \Sv&\cup\Sv' & \Sv &\leq k
%   \end{align*}
% \end{frame}

% \begin{frame}
%   \frametitle{Typing \hfill Tracking linearity}

%   When typechecking (e1, e2):
%   \begin{itemize}[<+->]
%   \item $\inferW{\Sv_1}{(C_1,\psi_1)}{\Gamma}{e_1}{\tau_1}$
%   \item $\inferW{\Sv_2}{(C_2,\psi_2)}{\Gamma}{e_2}{\tau_2}$
%   \item Add  $(\Sv_1\cap\Sv_2\leq \kun)$ to the constraints
%   \item \dots
%   \item Return $\Sv_1\cup\Sv_2$
%   \end{itemize}

% \end{frame}


% \begin{frame}
%   \frametitle{Constraints}

%   A slightly more general context: $\CL$ is the constraint system:
%   \begin{align*}
%     C ::=&\ \Cleq{\tau_1}{\tau_2}\ |\ \Cleq{k_1}{k_2}\ |\ C_1 \Cand C_2\ |\ \Cproj{\alpha}{C}
%   \end{align*}

%   where $k ::=\kvar\ |\ \ell \in {\mathcal L}$ and
%   $(\mathcal L, \leq_{\mathcal L})$ is a complete lattice.
%   \pause

%   Respect, among other things:
%   \begin{mathpar}
%     \inferrule{\ell \leq_{\mathcal L} \ell'}{\entail{}{\Cleq{\ell}{\ell'}}}
%     \and
%     \inferrule{}{\entail{}{\Cleq{k}{\ell^\top}}}
%     \and
%     \inferrule{}{\entail{}{\Cleq{\ell^\bot}{k}}}
%   \end{mathpar}

% \end{frame}

\begin{frame}
  \frametitle{Constraints \hfill Normalization}

  Example : $\lam{f}{\lam{x}{(\app{f}{x},x)}}$

  Raw constraints:
  \begin{gather*}
    (\tvar_f : \kvar_f)
    (\tvar_x : \kvar_x)\dots\\
    (\tvar_f \leq \gamma \tarr{\kvar_1} \beta )
    \Cand
    (\gamma \leq \tvar_x)
    \Cand
    (\beta * \tvar_x \leq \alpha_r)
    \Cand
    (\kvar_x \leq \kun)
  \end{gather*}\pause

  We unify the types and discover new constraints:
  \begin{gather*}
    \alpha_r =
    (\gamma \tarr{\kvar_3} \beta) \tarr{\kvar_2} \gamma \tarr{\kvar_1} \beta * \gamma\\
    (\kvar_x \leq \kun)
    \Cand
    (\kvar_\gamma \leq \kvar_x)
    \Cand
    (\kvar_x \leq \kvar_r)
    \Cand
    (\kvar_\beta \leq \kvar_r)
    \Cand
    (\kvar_3 \leq \kvar_f)
    \Cand
    (\kvar_f \leq \kvar_1)
  \end{gather*}
\end{frame}

\begin{frame}[plain]
  \centering
  \begin{gather*}
    (\gamma : \kvar_\gamma) (\beta : \kvar_\beta).\
    (\gamma \tarr{\kvar_3} \beta) \tarr{\kvar_2} \gamma \tarr{\kvar_1} \beta * \gamma\\
    \visible<7->{\kvar_\gamma = \kvar_x = \kun}
    \visible<13>{\Cand \kvar_3 \leq \kvar_1}
  \end{gather*}
  \begin{tikzpicture}[every edge/.style=[link,->,>=latex,thick]]
    \node<-11> (U) {$\kun$} ;
    \node<-7>[above left=of U] (x) {$\kvar_x$} ;
    \node<-10>[above=of x] (r) {$\kvar_r$};
    \node<-7>[left=of x] (g) {$\kvar_\gamma$} ;
    \node[right=of x] (b) {$\kvar_\beta$} ;
    \node[right=of b] (3) {$\kvar_3$} ;
    \node<-10>[above=of 3] (f) {$\kvar_f$} ;
    \node[above=of f] (1) {$\kvar_1$} ;

    \draw<1> (x) -> (U) ;
    \draw<2-7> (x) to[bend right] (U) ;
    \draw<-7> (x) -> (r) ;
    \draw<-10> (b) -> (r) ;
    \draw<-7> (g) -> (x) ;
    \draw<-10> (3) -> (f) ;
    \draw<-10> (f) -> (1) ;

    \only<2-11>{\begin{scope}[dashed,gray]
        \draw<-7> (U) to[bend right] (x) ;
        \draw (U) to (b) ;
        \draw<-7> (U) to[bend left] (g) ;
        \draw<-10> (U) to (r) ;
        \draw (U) to (1) ;
        \draw (U) to (3) ;
        \draw<-10> (U) to (f) ;
      \end{scope}}
    \visible<3-11>{\begin{scope}[dashed,gray]
      \node[above left= 0.2 and 1 of 1] (A) {$\klin$} ;
      \draw<-7> (x) to (A) ;
      \draw (b) to (A) ;
      \draw<-7> (g) to (A) ;
      \draw<-10> (r) to (A) ;
      \draw (1) to (A) ;
      \draw (3) to (A) ;
      \draw<-10> (f) to (A) ;
      \draw (U) to[out=0,in=0,looseness=1.5] (A) ;
    \end{scope}}

  \visible<4-5>{
    \begin{scope}[scale=0.7,xscale=0.9,transform shape, xshift=6.5cm,yshift=2cm]
      \node (a1) {$\kvar$};
      \node[above right=of a1] (a2) {$\kvar_1$};
      \node[above left=of a1] (a3) {$\kvar_2$};
      \node[above=of a2] (l2) {$\ell_1$};
      \node[above=of a3] (l3) {$\ell_2$};
      \draw (a1) to (a2) ;
      \draw (a1) to (a3) ;
      \draw (a2) to (l2) ;
      \draw (a3) to (l3) ;
    \visible<5>{\begin{scope}[dashed,gray]
      \node[above=of a1] (l1) {$\ell_1 \wedge \ell_2$};
      \draw (a1) to (l1) ;
      \draw (l1) to (l2) ;
      \draw (l1) to (l3) ;
    \end{scope}}
      \node[scale=1.3, xscale=1.1,draw=Grey, rectangle, rounded corners=10pt, fit=(a1) (l2) (l3)] {};
    \end{scope}
  }

  \begin{scope}[on background layer]
    \visible<6-7>{
      \node[fill=green!20,draw=green,
      inner sep=-1mm,ellipse,rotate fit=-20,fit=(U) (x) (g)] {};
    }
    \visible<9-10>{
      \node[fill=red!20,draw=red, inner sep=-2pt, circle,fit=(r)] {};
      \node[fill=red!20,draw=red, inner sep=-2pt, circle,fit=(f)] {};
    }
  \end{scope}

  \draw<10-13> (3) to[bend right] (1) ;

  \end{tikzpicture}

\end{frame}

\begin{frame}
  \frametitle{Constraints \hfill Normalization}

  Normalization is complete and principal.
  \begin{gather*}
    \lam{f}{\lam{x}{(\app{f}{x},x)}}: \\
    \forall \kvar_\beta \kvar_1 \kvar_2 \kvar_3
    (\gamma : \kun) (\beta : \kvar_\beta).\ %
    \qual
    {\Cleq{\kvar_3}{\kvar_1}}
    {(\gamma \tarr{\kvar_3} \beta) \tarr{\kvar_2} \gamma \tarr{\kvar_1} \beta * \gamma}
  \end{gather*}



\end{frame}

\defverbatim{\CompAA}{
\begin{lstlisting}
let compose f g x = f (g x)
val compose :
  ('k1 < 'k) & ('k1 < 'k3) =>
  ('b -{'k1}> 'a) -> ('c -{'k}> 'b) -{'k3}>
  'c -{'k}> 'a
\end{lstlisting}
}


\begin{frame}[fragile]
  \frametitle{Constraints \hfill Simplification rules}

  Well known simplifications on constraints:
  \begin{itemize}
  \item Replace variable in positive position by their lower bound
  \item Replace variable in negative position by their upper bound
  \end{itemize}

  \begin{align*}
    \only<2>{
      \forall \kvar_\beta \kvar_1 \kvar_2 \kvar_3
      (\gamma : \kun) (\beta : \kvar_\beta).&
      \qual
      {\Cleq{\kvar_3}{\kvar_1}}
      {(\gamma \tarr{\kvar_3} \beta) \tarr{\kvar_2} \gamma \tarr{\kvar_1} \beta * \gamma}}
    \only<3>{
      \forall \kvar_\beta \kvar_1 \kvar_3
      (\gamma : \kun) (\beta : \kvar_\beta).&
      \qual
      {\Cleq{\kvar_3}{\kvar_1}}
      {(\gamma \tarr{\kvar_3} \beta) \tarr{} \gamma \tarr{\kvar_1} \beta * \gamma}}
    \only<4->{
      \forall \kvar_\beta \kvar
      (\gamma : \kun) (\beta : \kvar_\beta).&
      {(\gamma \tarr{\kvar} \beta) \tarr{} \gamma \tarr{\kvar} \beta * \gamma}}
  \end{align*}
\end{frame}

% \begin{frame}
%   \frametitle{Constraints}
%   The only requirement is that $\ell^\bot = \kun$.\pause

%   \begin{itemize}
%   \item $\kaff$ doesn't appear in the typing rules.

%     It only comes from the buitins and/or the type declarations.
%   \item The lattice doesn't have to be finite.
%   \item The constraint language can be expanded further.
%   \end{itemize}
% \end{frame}

\begin{frame}
  \frametitle{Conclusion}

  I presented \lang:
  
  \begin{itemize}
  \item Support functional \emph{and} imperative programming styles\\
    thanks to linear types, borrows and regions.
  \item Novel use of kinds and constraints to verify these properties
  \item Complete and principal type inference
  \item Design compatible with \ocaml
  \end{itemize}\pause

  In the paper ``Kindly bent to free us'' (on Arxiv), you can find:
  \begin{itemize}
  \item Several examples of functional, imperative or mixed programming
  \item Complete account of the theory:
    \begin{itemize}
    \item A ``logical'' version of the type system
    \item A resource-aware semantics and the proof of soundness
    \item An inference algorithm based on HM(X) and the proofs of completeness/principality
    \end{itemize}
  \end{itemize}
\end{frame}


\begin{frame}{Future work}

  Area of future work:
  \begin{itemize}
  \item Mechanizing the formalization\\
    $\implies$ Ongoing work by Hannes Saffrich
  \item Design associated optimisations\\
    $\implies$ Collaboration with Guillaume Munch-Maccagnoni
  \item Investigate pattern matching
  \item Extend the expressivity further (at the price of inference ?)
  \end{itemize}
\end{frame}

\begin{frame}[plain,standout]
  \Huge{\texttt{Close(Talk)}}
\end{frame}

\begin{frame}
  \frametitle{Really??}

  \begin{quote}
    Do you really think adding kinds, subkinding and qualified types to \ocaml is a good idea?
  \end{quote}\pause
  Yes, I do!\pause
  \begin{itemize}
  \item Qualified types are coming for modular implicits anyway.
  \item Having proper kinds would fix many weirdness (rows, \dots) and
    enable nice extensions (units of measures).\pause
  \item I could make Eliom even better with them! \smiley{}
  \end{itemize}
\end{frame}


% \section{Going further}


% \begin{frame}{Current area of work}

%   \begin{enumerate}
%   \item Richer type system
%   \item Modules
%   \item Borrowing
%   \item Prototype cool APIs with it
%   \end{enumerate}

% \end{frame}

\begin{frame}
  \frametitle{Constraints \hfill Extensions}
  Constraints in a similar style have been applied to:
  \begin{itemize}
  \item (Relaxed) value restriction
  \item GADTs
  \item Rows
  \item Type elaboration
  \item \dots
  \end{itemize}
\end{frame}

\begin{frame}
  \frametitle{Modules}

  Several distinct problematic:
  \begin{itemize}
  \item \alert<2>{Type abstraction} \only<2>{{\color{green}\ding{52}}

      Can declare unrestricted types and expose them as Affine.
    }
  \item \alert<3>{Linear/affine values in modules} \only<3>{

      Behave like tuples: take the LUB of the kinds of the exposed values.

      What about values that are not exposed? They don't matter!
    }
  \item \alert<4>{Functors}\only<4>{

      What happens if a functor takes a module containing affine values?

      $\implies$ We need kind annotation on the functor arrow\dots \frownie{}
    }
  \item \alert<5->{Separate compilation}\only<5->{

      What about linear/affine constants?

      $\implies$ Should probably be forbidden\dots

      \visible<6>{But what about \lstinline{stdout} ?}
    }
  \end{itemize}

\end{frame}

% \begin{frame}[fragile,fragile,fragile]
%   \frametitle{Borrowing}

%   Borrowing seem essential to express many patterns found in \ocaml.\pause

%   Read-only borrows, in \texttt{CCHashTrie}:
% \begin{lstlisting}
% val add_mut : id -> key -> 'a -> 'a t -> 'a t
% (* add_mut ~id k v m behaves like add k v m, except
%    it will mutate in place whenever possible. *)
% \end{lstlisting}
%   \pause
% %   Read-only borrows, in \texttt{duff}
% % \begin{lstlisting}
% % val make: ?copy:bool -> Cstruct.t -> t
% % (* [make ?copy raw] returns a Rabin's fingerprint of [raw]. [make] expects to take the
% %    ownership of [raw]. *)
% % \end{lstlisting}

%   Mutable borrows, in \texttt{lacaml}:
% \begin{lstlisting}
% val Lacaml.D.sycon :
%   ... -> ?iwork:Common.int32_vec -> mat -> float
% (* iwork is an optional preallocated work buffer *)
% \end{lstlisting}

% \end{frame}

% \begin{frame}[fragile]{Borrowing}

%   ``Resource Polymorphism'' has the following lattice:
%   \begin{center}
%   \begin{tikzpicture}
%     \node (Copy) {Copy};
%     \node (Own) at (2,2) {Own};
%     \node (Copy@r) at (-2,1) {Copy@r};
%     \node (Seq@r) at (-1,2) {Seq@r};
%     \node (Own@r) at (0,3) {Own@r};

%     \draw (Copy) -- (Own) -- (Own@r) -- (Seq@r) -- (Copy@r) -- (Copy) ;
%   \end{tikzpicture}
% \end{center}
%   % \pause
%   It would requires:
%   \begin{itemize}
%   \item More syntactic annotations
%   \item Regions
%   \end{itemize}

% \end{frame}


\begin{frame}
  Which kind of linearity?
  \begin{itemize}
  \item \alert<2>{Ownership approaches}

    \only<2>{
      Suitable to imperative languages (Rust, \dots).
    }
  \item \alert<3>{Capabilities and typestates}

    \only<3>{
      Often use in Object-Oriented contexts (Wyvern, Plaid, Hopkins Objects Group, \dots).
    }
  \item \alert<4,6>{Substructural type systems}

    \only<4>{
      Many variations, mostly in functional languages:
      \begin{itemize}
      \item Inspired directly from linear logic (Linear Haskell, Walker, \dots)
      \item Uniqueness (Clean)
      \item Kinds (Alms, Clean, F$^\circ$)
      \item Constraints (Quill)
      \end{itemize}
    }
  \item \alert<5>{\dots}

    \only<5>{Mix of everything: Mezzo}
  \end{itemize}
\end{frame}

\begin{frame}
  \frametitle{The HM(X) framework}
  HM(X) \citep{DBLP:journals/tapos/OderskySW99} is a framework
  to build an HM type system (with inference) based on a given constraint system.

  We provide two additions:
  \begin{itemize}
  \item A small extension of HM(X) that tracks kinds and linearity
  \item An appropriate constraint system
  \end{itemize}
\end{frame}


\bibliography{biblio}

\end{document}

%%% Local Variables:
%%% mode: latex
%%% TeX-master: t
%%% End:

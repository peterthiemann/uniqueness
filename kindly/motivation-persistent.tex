\subsection{Solving sudokus with hybrid data-structures}

Recently introduced persistent data structures
permit transient mutations where
non-linear uses lead to degraded performance
\cite{DBLP:conf/ml/ConchonF07} or to
dynamic and static checks \cite{DBLP:journals/pacmpl/Puente17}.
In particular, persistent Hash-Array-Mapped-Tries (HAMT) have been used with similar
APIs in several non-pure functional languages (OCaml, Clojure, \dots).
Affine types help formalize the performance contract between the programmer
and the library, while borrows avoid the need to thread state explicitly,
as usually required by an API for immutable data types.
%

In this section, we present
a safe API for persistent arrays that support both immutable and mutable features,
and use it to implement a backtracking Sudoku solver.
% show how \lang allows us to define
The solver maintains an array to represent the state of
the game and uses backtracking when there are several choices to proceed.
The use of backtracking suggests a persistent data structure for the array.
However, only changes that correspond to a choice point need to use
the persistence mechanism, others may be implemented as
cheap in-place mutations.

\cref{sig:hybarray} contains an API @HYBARRAY@
that enables using mutable and immutable modifications to
the board through affine types and borrows.
The signature differs slightly from the @Array@
signature. For simplicity, the array elements are unrestricted, but
the structure itself remains linear so as to be implemented in terms of @Array@.
The in-place mutation function @set_mut@ with type
@&!('a t) * int * 'a -> unit@ works on a exclusive borrow  whereas the persistent
@set@ operation has type @&('a t) * int * 'a -> 'a t@. It
takes a shared borrow because it only reads from the
argument array, but returns a fresh,  modified structure.
The code in \cref{ex:cow} contains a very simple implementation of
@HYBARRAY@ that represents hybrid arrays
as regular arrays and uses copy-on-write for persistent
modifications. The function
@mapi: (int -> &'a -> 'b) -> &('a t) -> 'b t@
is a simple variation on @Array.map@ where the mapping function
takes the position of the element.
% A more elaborate implementation could use persistent arrays
% \cite{DBLP:conf/ml/ConchonF07}.


\begin{figure}[tp]
  \centering
  \begin{subfigure}[t]{0.45\linewidth}
\begin{lstlisting}
module type HYBARRAY = sig
  include ARRAY
  val set : &('a t) * int * 'a -> 'a t
  val set_mut : &!('a t) * int * 'a -> unit
end
\end{lstlisting}
  \end{subfigure}\hfill
  \begin{subfigure}[t]{0.55\linewidth}
\begin{lstlisting}
module Cow : HYBARRAY = struct
  include Array
  let set (a, i0, x0) =
    Array.mapi (fun i x -> if i = i0 then x0 else x) a
  let set_mut = Array.set
end
\end{lstlisting}
  \end{subfigure}
  \vspace{-15pt}
  \caption{Signature and Implementation of hybrid arrays}
  \label{sig:hybarray}
  \label{ex:cow}

  \begin{subfigure}[t]{0.45\linewidth}
  \lstinputlisting[linerange=1-13]{code/sukodu.affe}
  \end{subfigure}\hfill
  \begin{subfigure}[t]{0.55\linewidth}
  \lstinputlisting[linerange=14-31]{code/sukodu.affe}
  \end{subfigure}
  % \lstinputlisting[linerange=1-31,multicols=2]{code/sukodu.affe}
  \vspace{-15pt}
  \caption{Excerpt of the Sudoku solver}
  \label{ex:sudoku}
\end{figure}

Our implementation of a Sudoku solver (\cref{ex:sudoku}) performs
modifications that correspond to choice points using @set@,
which makes it trivial to come back to the previous version
of the array, while other modifications use @set_mut@, which
forbids uses of the previous version.

The board is represented as a 2D-matrix (\cref{line:boardtype}), where
the @Matrix@ type uses the same API as @PersistArray@
but with two indices.
Each cell contains an integer set that represent admissible solutions so far.
The Sudoku solver iterates over the cells and tries each possible solution (\cref{line:try_solution}).
When a value is picked for the current cell, we create a choice point,
change the cell with an immutable modification (\cref{line:immutset}), and propagate
the changes with the @propagate@ function.
The @propagate@ function uses direct mutation through a
exclusive borrow of the matrix as it need not preserve the previous
version of the board.
The implementation of @propagate@ is split into three parts
for lines, columns, and square, which are all very similar to function
@propagate_lines@ (\cref{line:propline}).

The typing ensures that the mutations do not compromise the state at
the choice point, because they operate on a new state @new_g@ created for one
particular branch of the choice.
As the @set@ function only requires an unrestricted shared borrow,
the closure @try_solution@ remains unrestricted even though
it captures the borrow @&g@.
The price is that @try_solution@ cannot escape from
@&g@'s region. In this example, inference places the region
around the @begin@/@end@ scope.
Hence, @try_solution@ can be used in
the iteration in \cref{line:iter}.
% As @&g@ is unrestricted, we could even use a parallel iteration
% instead of a sequential one.
As @g@ is linear we must free it outside of the region
before returning (\cref{line:free:g}).

While presented for copy-on-write arrays, the API
can easily be adapted to other persistent data structures with
transient mutability such as Relaxed-Radix Balance Vectors (RRB) \cite{DBLP:journals/pacmpl/Puente17}
or persistent HAMTs \cite{bagwell2001ideal,clojurehamt} to provide  a
convenient programming style without compromising performance.

% \subsection{Type abstraction and linearity violations}

% While linearity allows to verify many properties, it also prevents
% the implementation of functions that are nevertheless safe. Rust addresses this
% issue through the @unsafe@ construct, which allows to locally
% ignore linearity constraints. This is essential to build basic
% functions such as splitting strings.
% %
% \lang allows to partially emulate this behavior using type abstraction
% as provided by ML modules.
% In particular, data types can be internally defined as unrestricted but exposed
% as affine.

% \TODO{Show example with arrays}

%%% Local Variables:
%%% mode: latex
%%% TeX-master: "main"
%%% End:

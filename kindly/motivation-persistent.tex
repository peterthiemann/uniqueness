\subsection{Solving Sudokus with Hybrid Data-Structures}

This section presents an implementation of a
backtracking Sudoku solver using a safe API for persistent arrays that
supports both mutable updates and immutable versioning.
The implementation showcases safe mixing of resource allocation and
deallocation in the presence of exclusive (mutable) and immutable
borrows. It also demonstrates two new aspects: the interaction between
closures and borrows and the notion of reborrowing.
%% A function that closes over a borrow must not escape the borrow's region.

Recently introduced persistent data structures
permit transient mutations where
non-linear uses lead to degraded performance
\cite{DBLP:conf/ml/ConchonF07} or to
dynamic and static checks \cite{DBLP:journals/pacmpl/Puente17}.
In particular, persistent Hash-Array-Mapped-Tries (HAMT) have been used with similar
APIs in several non-pure functional languages (OCaml, Clojure, \dots).
Affine types help formalize the performance contract between the programmer
and the library, while borrows avoid the need to thread state explicitly,
as usually required by an API for immutable data types.
%

Our implementation of a backtracking Sudoku solver abstracts this scenario.
% show how \lang allows us to define
The solver maintains a two-dimensional array to represent the state of
the game and uses backtracking when there are several choices to proceed.
As choice points may be revisited several times, it seems advantageous
to select a persistent data structure for the array.
However, local changes between choice points may be implemented as
cheap in-place mutations.

\cref{sig:hybarray} contains an API @HYBARRAY@ along with an
implementation @CowArray@
that enables using mutable and immutable modifications to
the board through affine types and borrows.
The signature differs slightly from the @Array@
signature. As our application requires the @get@ function, the array
elements must be unrestricted, but
the structure itself remains linear so as to be implemented in terms of @Array@.
The in-place mutation function @set_mut@ with type
@&!('a t) * int * 'a -> unit@ works on an exclusive borrow  whereas the persistent
@set@ operation has type @&('a t) * int * 'a -> 'a t@. It
takes a shared borrow because it only reads from the
argument array, but returns a fresh,  modified structure.
The module @CowArray@, also in \cref{ex:cow}, contains a very simple
implementation of 
@HYBARRAY@ that represents hybrid arrays
as regular arrays and uses copy-on-write for persistent
modifications. The function
@mapi: (int * &'a -> 'b) * &('a t) -> 'b t@
is a simple variation on @Array.map@ where the mapping function also
takes the position of the element. Recall that @Array.map@ always
creates a new array for the result.
% A more elaborate implementation could use persistent arrays
% \cite{DBLP:conf/ml/ConchonF07}.
%
In the example code, we make use of a two-dimensional version
@Matrix@ of the @CowArray@ data\-structure. The only difference is
that the API functions @get@, @set@, and @set_mut@ now take two index
parameters of type @int@ instead of one. The internal working is exactly the same.


\begin{figure}[tp]
  \centering
  \begin{subfigure}[t]{0.45\linewidth}
\begin{lstlisting}
module type HYBARRAY = sig
  include ARRAY
  val set : &('a t) * int * 'a -> 'a t
  val set_mut : &!('a t) * int * 'a -> unit
end
\end{lstlisting}
  \end{subfigure}\hfill
  \begin{subfigure}[t]{0.55\linewidth}
\begin{lstlisting}
module CowArray : HYBARRAY = struct
  include Array
  let set (a, i0, x0) =
    Array.mapi ((fun (i, x) -> if i = i0 then x0 else x), a)
  let set_mut = Array.set
end
\end{lstlisting}
  \end{subfigure}
  \vspace{-15pt}
  \caption{Signature and Implementation of hybrid arrays}
  \label{sig:hybarray}
  \label{ex:cow}

  \begin{subfigure}[t]{0.45\linewidth}
  \lstinputlisting[linerange=1-13]{code/sukodu.affe}
  \end{subfigure}\hfill
  \begin{subfigure}[t]{0.55\linewidth}
  \lstinputlisting[linerange=14-31,firstnumber=last]{code/sukodu.affe}
  \end{subfigure}
  % \lstinputlisting[linerange=1-31,multicols=2]{code/sukodu.affe}
  \vspace{-10pt}
\begin{lstlisting}
val propagate : int -> int -> &!Matrix.t -> int -> unit
val solve : int -> int -> Matrix.t -> unit
\end{lstlisting}
  \vspace{-10pt}
  \caption{Excerpt of the Sudoku solver}
  \label{ex:sudoku}
\end{figure}

% Our implementation of a Sudoku solver (\cref{ex:sudoku}) performs
% modifications that correspond to choice points using @set@,
% which makes it trivial to come back to the previous version
% of the array, while other modifications use @set_mut@, which
% cannot be reverted.

Our implementation of a Sudoku solver (\cref{ex:sudoku}) represents
the board as a 2D-matrix (\cref{line:boardtype}).
Each cell contains an integer set of type @IntSet.t@ that represents
admissible solutions so far. This type is immutable, i.e.,
@IntSet.remove@ produces a new value.
%
The main functions are @solve@ and @propagate@ with the typings shown
on top of \cref{ex:sudoku}. The types of @propagate_line@ etc are the
same as for @propagate@. From the types, we can see that @solve@ takes
ownership of the board, whereas @propagate@ only takes a mutable
borrow. Hence, the @propagate@ functions can only modify or read the board,
whereas @solve@ has full control. 

The Sudoku solver @solve@ iterates over the cells and tries each possible
solution (\cref{line:try_solution}). 
When a value is picked for the current cell, it creates a choice point
in @new_board@, where the current cell is updated with an immutable modification (\cref{line:immutset}), and propagate
the changes with the @propagate@ function.
The @propagate@ function uses direct mutation through an
exclusive borrow of the matrix as it need not preserve the previous
version of the board.
The implementation of @propagate@ is split into three parts
for lines, columns, and square, which are all very similar to function
@propagate_lines@ (\cref{line:propline}).

The @board@ parameter of the @propagate@ function is an exclusive
borrow, it should be handled 
in an affine manner. To pass it safely to the three helper functions,
the body of @propagate@ \emph{reborrows} (i.e., it takes a borrow of a
borrow) the board three times in 
\crefrange{line:reborrow1}{line:reborrow3}.
The function @propagate_line@ also contains two reborrows of the
exclusive borrow @board@, an
immutable one (\cref{line:reborrow_i}) and an exclusive one
(\cref{line:reborrow_m}). It demonstrates the facility 
to take immutable borrows from exclusive ones.

The typing ensures that the mutations do not compromise the state at
the choice point, because they operate on a new state @new_board@ created for one
particular branch of the choice.
As the @set@ function only requires an unrestricted shared borrow,
the closure @try_solution@ remains unrestricted even though
it captures the borrow @&board@.
The price is that @try_solution@ cannot escape from
@&board@'s region. In this example, the inferred region corresponds to the
@begin@/@end@ scope. 
Hence, @try_solution@ can be used in
the iteration in \cref{line:iter}.
% As @&g@ is unrestricted, we could even use a parallel iteration
% instead of a sequential one.
As @board@ is linear we must free it outside of the region
before returning (\cref{line:free:g}).

While presented for copy-on-write arrays, the API
can easily be adapted to other persistent data structures with
transient mutability such as Relaxed-Radix Balance Vectors (RRB) \cite{DBLP:journals/pacmpl/Puente17}
or persistent HAMTs \cite{bagwell2001ideal,clojurehamt} to provide  a
convenient programming style without compromising performance.

% \subsection{Type abstraction and linearity violations}

% While linearity allows to verify many properties, it also prevents
% the implementation of functions that are nevertheless safe. Rust addresses this
% issue through the @unsafe@ construct, which allows to locally
% ignore linearity constraints. This is essential to build basic
% functions such as splitting strings.
% %
% \lang allows to partially emulate this behavior using type abstraction
% as provided by ML modules.
% In particular, data types can be internally defined as unrestricted but exposed
% as affine.

% \TODO{Show example with arrays}

%%% Local Variables:
%%% mode: latex
%%% TeX-master: "main"
%%% End:

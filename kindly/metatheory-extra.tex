\section{Proofs for Metatheory}
\label{sec:metatheory:proofs}

\begin{itemize}
\item Borrow compatibility
  $\Multi\IBORROW\Multi\MBORROW \Bcompatible \BORROW$,
  \begin{mathpar}
  \inferrule{}{
    \IBORROW\Multi\IBORROW\Multi\MBORROW \Bcompatible \IBORROW
  }

  \inferrule{}{
    \MBORROW\Multi\MBORROW \Bcompatible \MBORROW
  }
  \end{mathpar}
\item Store typing $ \vdash \Store : \SE$,
  \begin{mathpar}
    \inferrule{
      (\forall \Loc \in \Dom\Store)~
      \SE \vdash \Store (\Loc) : \SE (\Loc)
    }{ \vdash \Store : \SE }
  \end{mathpar}
\item Relating storables to type schemes $\SE \vdash W : \schm$
  \begin{mathpar}
  \inferrule{
    (\exists \E)~ \SE \vdash \VEnv : \E
    \\
    \inferS{C}{\E, (x:\tau_2)}{e}{\tau_1}
  }{
    \SE \vdash (\VEnv, \ilam {\Multi\kvar}{\Multi\tvar}Ckx{e})
    : \forall\Multi\kvar\forall\Multi{\bvar{\alpha}{k}}.(\qual{C}{\tau_2\tarr{k}\tau_1})
  }
  \end{mathpar}
\item Relating storables to types $ \SE \vdash W : \tau$
  \begin{mathpar}
    \inferrule{
      (\exists \E, C)~ \SE \vdash \VEnv : \E
      \\
      \inferS{C}{\E\bvar x{\tau_2})}{e}{\tau_1}
      \\
      \addlin{\entail{C}{\Cleq{\E}{k}}}
    }{
      \SE \vdash (\VEnv, \lam[k]xe) : \tau_2\tarr{k}\tau_1
    }

    \inferrule{
      \SE \vdash r_1 : \tau_1 \\
      \SE \vdash r_2 : \tau_2
    }{
      \SE \vdash \introPair[k]{r_1}{r_2} : \tyPair[k]{\tau_1}{\tau_2}
    }

    \inferrule{
      \SE \vdash r : \IType{t}{\Multi\tau}
    }{
      \SE \vdash {[r]} : \tapp{t}{\Multi\tau}
    }

    \inferrule{}{
      \SE \vdash \blob : \tau
    }
  \end{mathpar}
\item Relating  results to types $ \SE \vdash r : \tau$, 
\item Relating results to type schemes $\SE \vdash r : \schm$
  \begin{mathpar}
  \inferrule{}{ \SE \vdash c : \CType{c} }

  \inferrule{}{ \SE \vdash \ell : \SE (\ell) }

  \inferrule{
    \Multi\IBORROW\Multi\MBORROW \Bcompatible \BORROW \\
    \SE \vdash \Loc  : \tau
  }{  \SE \vdash
    \Multi\IBORROW\Multi\MBORROW\Loc : \borrow{\tau}}
  \end{mathpar}
\item Relating environments to contexts. Here we consider an
  environment $\VEnv = (\Active\VEnv, \MutableBorrows\VEnv,
  \ImmutableBorrows\VEnv, \Suspended\VEnv)$ as a quadruple
  consisting of the active entries in $\Active\VEnv$ and the
  entries for mutable borrows in $\MutableBorrows\VEnv$ and for
  immutable borrows in $\ImmutableBorrows\VEnv$, and suspended entries
  in $\Suspended\VEnv$. The
  suspended entries cannot be used directly, but they can be activated
  by borrowing.
  
  $\SE \vdash \Active\VEnv, \MutableBorrows\VEnv,
  \ImmutableBorrows\VEnv, \Suspended\VEnv : \E$
\end{itemize}
\begin{mathpar}
  \inferrule{}{\SE \vdash \Sempty, \Sempty, \Sempty, \Sempty : \Eempty}

  \inferrule{
    \SE \vdash \Active\VEnv, \MutableBorrows\VEnv,
    \ImmutableBorrows\VEnv, \Suspended\VEnv : \E
    \\ \SE \vdash r : \schm}
  {\SE \vdash \Active\VEnv[ x\mapsto r], \MutableBorrows\VEnv ,
    \ImmutableBorrows\VEnv, \Suspended\VEnv : \E\bvar x\schm }

  \inferrule{\SE \vdash \Active\VEnv, \MutableBorrows\VEnv,
    \ImmutableBorrows\VEnv, \Suspended\VEnv : \E \\
    \SE \vdash r : \schm}
  { \SE \vdash \Active\VEnv, \MutableBorrows\VEnv,
    \ImmutableBorrows\VEnv, \Suspended\VEnv[ x\mapsto r] : \E\svar x\schm^n }

  \inferrule{\SE \vdash \Active\VEnv, \MutableBorrows\VEnv,
    \ImmutableBorrows\VEnv, \Suspended\VEnv : \E \\ \SE \vdash
    \IBORROW r : \schm} 
  {\SE \vdash \Active\VEnv, \MutableBorrows\VEnv,
    \ImmutableBorrows\VEnv[ x\mapsto \IBORROW r], \Suspended\VEnv : \E\bvar{\borrow[\IBORROW] x}{\borrowty[\IBORROW] k{ \schm}} }

  \inferrule{\SE \vdash \Active\VEnv, \MutableBorrows\VEnv,
    \ImmutableBorrows\VEnv, \Suspended\VEnv : \E \\ \SE \vdash
    \MBORROW r : \schm}
  {\SE \vdash \Active\VEnv, \MutableBorrows\VEnv[ x\mapsto \MBORROW r],
    \ImmutableBorrows\VEnv : \E\bvar{\borrow[\MBORROW] x}{\borrowty[\MBORROW] k{ \schm}} }
\end{mathpar}
\paragraph{Extending environments and stores}
\begin{mathpar}
  \inferrule{}{\SE \le \SE}
  
  \inferrule{\SE \le \SE'}{\SE \le \SE' (\ell : \schm)}

  \inferrule{}{\Store\le\Store}

  \inferrule{\Store \le \Store' \\ \ell \notin \Dom\Store
  }{\Store \le \Store'[ \ell \mapsto W] }
\end{mathpar}

\begin{lemma}[Store Weakening]\label{lemma:store-weakening}
  $\SE \vdash \VEnv : \E$ and $\SE \le \SE'$ implies $\SE' \vdash
  \VEnv : \E$.
\end{lemma}

We write $\Addresses{\VEnv}$ for the function that extracts raw locations
from the range of the variable environment. It needs to be defined for
addresses, results, and environments.

\begin{align*}
  \Addresses{\Multi\IBORROW\Multi\MBORROW\Loc} &= \{ \Loc \} \\
  \Addresses{c} &= \{ \} \\
  \Addresses\Eempty &= \{\} \\
  \Addresses{\VEnv( x \mapsto r)} &= \Addresses\VEnv \cup \Addresses r
\end{align*}

We write $\Reach\Store\VEnv \subseteq \Dom\Store$ for the function
that computes all addresses reachable from $\Addresses\VEnv$
assuming that $\Addresses\VEnv \subseteq \Dom\Store$. It is defined in
two steps. First a helper function 
for addresses, results, storables, and environments.

\begin{align*}
  \RS\Store\Eempty &= \Eempty \\
  \RS\Store{\VEnv (x \mapsto r)} &= \RS\Store\VEnv \cup
                                      \RS\Store r \\
  \RS\Store\Addr &= \{ \Addr \}  \\
  \RS\Store c &= \{ \} \\
  \RS\Store{\text{STPOLY}(\VEnv, \silam
  {\Multi\kvar}{\Multi\tvar}Ckx{e})} &=
                                       \RS\Store\VEnv
  \\
  \RS\Store{\text{STCLOS}(\VEnv, \slam[k]xe)} &=
                                                   \RS\Store\VEnv
  \\
  \RS\Store{\text{STPAIR}\spair[k]{r_1}{r_2}} &=
                                                   \RS\Store{r_1}
                                                   \cup \RS\Store{r_2}
  \\
  \RS\Store{\text{STRSRC}(r)} &=
                                   \RS\Store r
  \\
  \RS\Store{\blob} &= \{ \}
\end{align*}

This set is closed transitively by store lookup: We define the set
$\Reach\Store\VEnv = \REACH$ as follow.
\begin{align*}
  \REACH &\supseteq \RS\Store\VEnv \\
  \REACH &\supseteq \RS\Store w & \text{if }
                                     \Multi\IBORROW\Multi\MBORROW\Loc
                                     \in \REACH \wedge w = \Store (\Loc)
\end{align*}


\begin{theorem}
  Suppose that
  \begin{itemize}
  \item $\inferS{C}{\E}{e}{\tau}$
  \item $\SE \vdash \VEnv : \E$
  \item $\vdash \Store : \SE$
  \item $\Addresses\Perm \subseteq \Dom\Store$
  \item $\Reach\Store{\Active\VEnv, \MutableBorrows\VEnv, \ImmutableBorrows\VEnv} \subseteq \Perm$
  % \item  $\VEnv'$ with $\Addresses{\VEnv'}
  %   \subseteq \Dom\Store$ and $\Dom\VEnv \cap \Dom{\VEnv'}=\emptyset$
  \item Incoming Resources: Let $\REACH = \Reach\Store\VEnv$.
    \begin{itemize}
    \item 
      For all $\Loc$ such that $\MultiNumber\Loc{\Reach{\Store}{\Active\VEnv}} >0$,
      if $\Affine{\SE}\Loc$ then $\MultiNumber\Loc\REACH= 1$
    \item For all $\Loc$ such that $
      \MultiNumber\Loc{\Reach{\Store}{\MutableBorrows\VEnv}} >0$, it
      must be that $\MultiNumber\Loc\REACH=1$.
    \item For all $\Loc$ such that $
      \MultiNumber\Loc{\Reach{\Store}{\Suspended\VEnv}} >0$, it
      must be that $\MultiNumber\Loc\REACH=1$.
    \end{itemize}
  \item  $i\in\Nat$ and $\Store, \Perm, \VEnv \vdash {e}
    \Downarrow^i R$ and $R\ne \TimeOut$.
  \end{itemize}
  Then,
  $\exists$ $\Store'$, $\Perm'$, $r'$, $\SE'$ such that
  \begin{itemize}
  \item
    $R = \Ok{\Store', \Perm', r'}$  
  \item $\SE \le \SE'$, $\Store \le \Store'$,
    $\vdash \Store' : \SE'$ 
  \item $\Addresses{\Perm'} \subseteq \Dom{\Store'}$
  \item $\SE' \vdash r' : \tau$
  \item $\Reach{\Store'}{r'} \subseteq \Perm'$
  \item Outside: For all $\Loc \in \Dom{\Store} \setminus
    \Reach{\Store'}{\VEnv}$ it must be that 
    $\Store' (\Loc) = \Store (\Loc)$
    and $\Loc\in\Perm \Leftrightarrow \Loc\in\Perm'$.
    %% must be \Store because \Perm has no idea of \Perm'
  \item Immutables: For all $\Loc \in
    \Reach{\Store'}{\ImmutableBorrows\VEnv}$ it must be that
    $\Loc\in\Dom\Store$, 
    $\Store' (\Loc) = \Store (\Loc)$
    and $\Loc\in\Perm \Leftrightarrow \Loc\in\Perm'$.
  \item Resources:
    Let $\REACH' =\Reach{\Store'}{\Active\VEnv}$.
    For all $\Loc$ such that $n= \MultiNumber\Loc\REACH >0$ and $n' =
    \MultiNumber\Loc{\REACH'}$, 
    \begin{itemize}
    \item if $\Affine{\SE}\Loc$ then $n=1$ and $n'\le 1$,
    \item if $\Linear{\SE}\Loc$ then $n=1$ and $n' = 0$,
    \item if $n'=0$, then $\Loc\notin\Perm'$.
    \end{itemize}
  \item Immutables: For all $\Loc \in \Reach
    {\Store'}{\Active\VEnv}$, if $\neg\Writeable{\SE}\Loc$ then
    $\Store' (\ell) = \Store (\ell)$ and $\Loc\in \Perm'$.
  \item No thin air permission: For all $\Loc\in \Perm'$, $\Loc
    \in \Perm \cup  \Dom{\Store'} \setminus \Dom{\Store}$.
  \end{itemize}
\end{theorem}

%\clearpage
\lstMakeShortInline[style=rule]@
\begin{proof}
  By induction on the evaluation of
  @eval \Store \Perm \VEnv i e@.

  The base case is trivial as
  @eval \Store \Perm \VEnv 0 e = \TimeOut@. 
  
  Let now $i>0$ and consider the different cases for expressions.

  \textbf{case $e$ of}
  \lstsemrule{sapp}

  By assumption $\ruleSDApp$.

  As \lstinline{sp}  is evidence for the split in the typing rule,
  we find that
  \begin{align}
    \label{eq:1}
    \SE \vdash \VEnv_1 : \E_1 && \SE \vdash \VEnv_2 : \E_2
  \end{align}

  Hence, we can establish the assumptions for the recursive call 
  @eval delta pi gamma_1 i' e_1@.
  \begin{itemize}
  \item  $\inferS{C}{\E_1}{e_1}{\tau_2\tarr{k}\tau_1}$ by inversion of typing
  \item $\SE \vdash \VEnv_1 : \E_1$ by \eqref{eq:1}
  \item $\vdash \Store : \SE$ by assumption
  \item $\Addresses\Perm \subseteq \Dom\Store$ by assumption 
  \item $
    \Reach\Store{\Active{\VEnv_1}, \MutableBorrows{\VEnv_1},
      \ImmutableBorrows{\VEnv_1}}
    \subseteq
    \Reach\Store{\Active\VEnv, \MutableBorrows\VEnv, \ImmutableBorrows\VEnv}
    \subseteq \Perm$
  \item Incoming Resources: For $\REACH_1 = \Reach\Store{\VEnv_1}$.
    \begin{itemize}
    \item 
      For all $\Loc$ such that $\MultiNumber\Loc{\Reach{\Store}{\Active{\VEnv_1}}} >0$,
      if $\Affine{\SE}\Loc$ then $\MultiNumber\Loc{\REACH_1}= 1$
    \item For all $\Loc$ such that $
      \MultiNumber\Loc{\Reach{\Store}{\MutableBorrows{\VEnv_1}}} >0$, it
      must be that $\MultiNumber\Loc{\REACH_1}=1$.
    \end{itemize}
  \item $i'<i$ and   @eval \Store \Perm gamma_1 i' e_1@ terminates
    with $R_1 \ne \TimeOut$.
  \end{itemize}
  Hence, we can apply the inductive hypothesis and obtain:
  \begin{gather}
    \label{eq:2}
    R_1 = \Ok{\Store_1, \Perm_1, r_1}
    \\
    \SE \le \SE_1 \qquad
    \Store \le \Store_1 \qquad
    \vdash \Store_1 :    \SE_1
    \\
    \Addresses{\Perm_1} \subseteq \Dom{\Store_1}
    \\\label{eq:3}
    \SE_1 \vdash r_1 : \tau_2\tarr{k}\tau_1
    \\\label{eq:10}
    \Reach{\Store_1}{r_1} \subseteq \Perm_1
  \end{gather}
  \begin{itemize}
  \item Outside: For all $\Loc \in \Dom{\Store} \setminus
    \Reach{\Store_1}{\VEnv_1}$ it must be that 
    $\Store_1 (\Loc) = \Store (\Loc)$
    and $\Loc\in\Perm \Leftrightarrow \Loc\in\Perm_1$ 
  \item Immutables: For all $\Loc \in
    \Reach{\Store_1}{\ImmutableBorrows{\VEnv_1}}$ it must be that
    $\Loc\in\Dom\Store$, 
    $\Store_1 (\Loc) = \Store (\Loc)$
    and $\Loc\in\Perm \Leftrightarrow \Loc\in\Perm_1$ 
  \item Resources:
    Let $\REACH_1 =\Reach{\Store_1}{\Active{\VEnv_1}}$.
    For all $\Loc$ such that $n= \MultiNumber\Loc{\REACH_1} >0$,
    \begin{itemize}
    \item if $\Affine{\SE_1}\Loc$ then $n\le 1$; if $n=0$, then $\Loc\notin\Perm_1$.
    \item if $\Linear{\SE_1}\Loc$ then $n=0$ and $\Loc\notin\Perm_1$.
    \end{itemize}
  \item Immutables: For all $\Loc \in \Reach
    {\Store_1}{\Active{\VEnv_1}}$, if $\neg\Writeable{\SE_1}\loc$ then
    $\Store_1 (\ell) = \Store (\ell)$ and $\Loc\in \Perm_1$.
  \item No thin air permission: For all $\Loc\in \Perm_1$, $\Loc
    \in \Perm \cup  \Dom{\Store_1} \setminus \Dom{\Store}$.
  \end{itemize}

  By inversion of \eqref{eq:3}, it must be that $r_1$ is a location
  $\Loc_1$ typed by $\SE_1 (\Loc_1) = \tau_2\tarr{k}\tau_1$.
  Hence, $w= \Store_1 (\Loc_1)$ with $\SE_1 \vdash w :
  \tau_2\tarr{k}\tau_1$, so that $w$ is a closure of the form
  $(\VEnv', k', x', e')$.

  By inversion of the store typing $\SE_1 \vdash   (\VEnv', k', x',
  e') : \tau_2\tarr{k}\tau_1$, we obtain some $\E'$ and $C'$ such that
  \begin{gather}
    \label{eq:4}
    \SE_1 \vdash \VEnv' : \E'
    \\\label{eq:5}
    \inferS{C'}{\E'\bvar{x'}{\tau_2}}{e'}{\tau_1}
    \\
    {\entail{C'}{\Cleq{\E'}{k'}}}
  \end{gather}

  Next, we establish the assumptions for the recursive call
  @eval delta_1 pi_1' gamma_2 i' e_2@
  \begin{itemize}
  \item  $\inferS{C}{\E_2}{e_2}{\tau_2}$ by inversion of typing
  \item $\SE_1 \vdash \VEnv_2 : \E_2$ by \eqref{eq:1} and Weakening
    (Lemma~\ref{lemma:store-weakening}).
  \item $\vdash \Store_1 : \SE_1$ by IH on $e_1$
  \item $\Addresses{\Perm_1'} \subseteq \Dom{\Store_1}$ by assumption
    and no-thin-air permission for $e_1$
  \item Show that for $\Active{\Theta_2} =
    \Reach{\Store_1}{\Active{\VEnv_2}}$, $\MutableBorrows{\Theta_2} =
    \Reach{\Store_1}{ \MutableBorrows{\VEnv_2}}$, and
    $\ImmutableBorrows{\Theta_2} = \Reach{\Store_1}{
      \ImmutableBorrows{\VEnv_2}}$ it holds that $\Active{\Theta_2}  \subseteq
    \Perm_1'$, $\MutableBorrows{\Theta_2} \subseteq \Perm_1'$, and
    $\ImmutableBorrows{\Theta_2} \subseteq \Perm_1'$.

    As these sets are disjoint, we can argue separately.
    Let $n = \MultiNumber\Loc{\Active{\Theta_2}}>0$.
    If $\Affine{\SE_1}\Loc$, then $\Loc \in \Dom{\Store} \setminus
    \Reach{\Store_1}{\VEnv_1}$. Hence, $\Loc\in\Perm_1$ iff
    $\Loc\in\Perm$. Moreover, $\Loc\ne\Loc_1$. Hence,
    $\Loc\in\Perm_1'$.

    If  $n = \MultiNumber\Loc{\MutableBorrows{\Theta_2}}>0$,
    then $\Loc$ is affine as it is a mutable borrow and
    $\Loc\in\Perm_1'$ by analogous argument.

    If  $n = \MultiNumber\Loc{\ImmutableBorrows{\Theta_2}}>0$,
    then its permission is never withdrawn an $\Loc\in\Perm_1'$.
  \item Incoming resources considering $\REACH_2$.
    \begin{itemize}
    \item 
      For all $\Loc$ such that $\MultiNumber\Loc{\Reach{\Store_1}{\Active{\VEnv_2}}} >0$,
      if $\Affine{\SE_2}\Loc$ then $\MultiNumber\Loc{\REACH_2}= 1$.
    \item For all $\Loc$ such that $
      \MultiNumber\Loc{\Reach{\Store_1}{\MutableBorrows{\VEnv_2}}} >0$, it
      must be that $\MultiNumber\Loc{\REACH_2}=1$.
    \end{itemize}
    In both cases the rationale is that this statement was true for
    the incoming environment $\VEnv$ (by assumption) and splitting did
    not pass affine resources to $\VEnv_1$. 
  \item $i'<i$ and  @eval delta_1 pi_1' gamma_2 i' e_2@ terminates
    with $R_2 \ne \TimeOut$.
  \end{itemize}
  Hence, we can apply the inductive hypothesis and obtain
  $\Store_2$, $\Perm_2$, $r_2$, $\SE_2$ such that
  \begin{gather}
    \label{eq:6}
    R_2 = \Ok{\Store_2, \Perm_2, r_2}
    \\\label{eq:7}
    \SE_1 \le \SE_2 \qquad
    \Store_1 \le \Store_2 \qquad
    \vdash \Store_2 :  \SE_2
    \\\label{eq:9}
    \Addresses{\Perm_2} \subseteq \Dom{\Store_2}
    \\\label{eq:8}
    \SE_2 \vdash r_2 : \tau_2
    \\\label{eq:11}
    \Reach{\Store_2}{r_2} \subseteq \Perm_2
  \end{gather}
  \begin{itemize}
  \item Outside: For all $\Loc \in \Dom{\Store_1} \setminus
    \Reach{\Store_2}{\VEnv_2}$ it must be that 
    $\Store_2(\Loc) = \Store_1 (\Loc)$
    and $\Loc\in\Perm_1' \Leftrightarrow \Loc\in\Perm_2$ 
  \item Immutables: For all $\Loc \in
    \Reach{\Store_2}{\ImmutableBorrows{\VEnv_2}}$ it must be that
    $\Loc\in\Dom{\Store_1}$, 
    $\Store_2 (\Loc) = \Store_1 (\Loc)$
    and $\Loc\in{\Perm_1'} \Leftrightarrow \Loc\in\Perm_2$ 
  \item Resources:
    Let $\REACH_2' =\Reach{\Store_2}{\Active{\VEnv_2}}$.
    For all $\Loc$ such that $n= \MultiNumber\Loc{\REACH_2} >0$ and $n' =
    \MultiNumber\Loc{\REACH_2'}$, 
    \begin{itemize}
    \item if $\Affine{\SE_2}\Loc$ then $n=1$ and $n'\le 1$,
    \item if $\Linear{\SE_2}\Loc$ then $n=1$ and $n' = 0$,
    \item if $n'=0$, then $\Loc\notin\Perm_2$.
    \end{itemize}
  \item Immutables: For all $\Loc \in \Reach
    {\Store_2}{\Active{\VEnv_2}}$, if $\neg\Writeable{\SE_2}\Loc$ then
    $\Store_2 (\ell) = \Store_1 (\ell)$ and $\Loc\in \Perm_2$.
  \item No thin air permission: For all $\Loc\in \Perm_2$, $\Loc
    \in \Perm_1' \cup  \Dom{\Store_2} \setminus \Dom{\Store_1}$.
  \end{itemize}
  
  Finally, we establish the assumptions for the recursive call 
  @eval delta_2 pi_2 gamma'(x'-:>r_2) i' e'@.
  To this end, let $\VEnv_3 = \VEnv' (x'\mapsto r_2)$.
  \begin{itemize}
  \item $\inferS{C'}{\E'\bvar{x'}{\tau_2}}{e'}{\tau_1}$

    which holds by \eqref{eq:5}.
  \item $\SE_2 \vdash \VEnv' (x' \mapsto r_2) : \E'\bvar{x'}{\tau_2} $

    By \eqref{eq:4}, $\SE_1\le\SE_2$ by \eqref{eq:7}, and weakening
    Lemma~\ref{lemma:store-weakening}, we have
    $\SE_2 \vdash \VEnv' : \E'$.
    It remains to show $\SE_2 \vdash r_2 : \tau_2$, which is
    \eqref{eq:8}.
  \item  $\vdash \Store_2 : \SE_2$\\
    holds by \eqref{eq:7}.
  \item  $\Addresses{\Perm_2} \subseteq \Dom{\Store_2}$\\
    by \eqref{eq:9}.
  \item $\Reach{\Store_2}{\Active{\VEnv_3}, \MutableBorrows{\VEnv_3}, \ImmutableBorrows{\VEnv_3}} \subseteq \Perm_2$\\
    This fact follows from (\ref{eq:10}) becase $\VEnv'$ is contained
    in the closure for $r_1$ and $\Store_1 \le \Store_2$. Moreover,
    either the closure is unreachable from $\VEnv_2$ or it is
    immutable, so that the permissions are retained to $\Perm_2$.
    For $r_2$ bound to $x'$, we obtain this from (\ref{eq:11}).
  \item Incoming Resources: Let $\REACH_3 = \Reach{\Store_2}{\VEnv_3}$.
    \begin{itemize}
    \item 
      For all $\Loc$ such that $\MultiNumber\Loc{\Reach{\Store_2}{\Active{\VEnv_3}}} >0$,
      if $\Affine{\SE_2}\Loc$ then $\MultiNumber\Loc{\REACH_3}= 1$
    \item For all $\Loc$ such that $
      \MultiNumber\Loc{\Reach{\Store_2}{\MutableBorrows{\VEnv_3}}} >0$, it
      must be that $\MultiNumber\Loc{\REACH_3}=1$.
    \end{itemize}
  \item $i'<i$ and  @eval delta_2 pi_2 gamma' i' e'@ terminates
    with $R_3 \ne \TimeOut$.
  % \item  $i\in\Nat$ and $\Store_2, \Perm_2, \VEnv' (x'\mapsto r_2) \vdash {e'}
  %   \Downarrow^i R_3$ and $R_3\ne \TimeOut$.
  \end{itemize}
  Hence, we can apply the inductive hypothesis and obtain
  $\Store_3$, $\Perm_3$, $r_3$, $\SE_3$ such that
  \begin{gather}
    R_3 = \Ok{\Store_3, \Perm_3, r_3}
    \\\label{eq:12}
    \SE_2 \le \SE_3 \qquad
    \Store_2 \le \Store_3 \qquad
    \vdash \Store_3 :  \SE_3
    \\\label{eq:13}
    \Addresses{\Perm_3} \subseteq \Dom{\Store_3}
    \\\label{eq:14}
    \SE_3 \vdash r_3 : \tau_1
    \\\label{eq:15}
    \Reach{\Store_3}{r_3} \subseteq \Perm_3
  \end{gather}
  \begin{itemize}
  \item Outside: For all $\Loc \in \Dom{\Store_2} \setminus
    \Reach{\Store_3}{\VEnv_3}$ it must be that 
    $\Store_3(\Loc) = \Store_2 (\Loc)$
    and $\Loc\in\Perm_2 \Leftrightarrow \Loc\in\Perm_3$ 
  \item Immutables: For all $\Loc \in
    \Reach{\Store_3}{\ImmutableBorrows{\VEnv_3}}$ it must be that
    $\Loc\in\Dom{\Store_2}$, 
    $\Store_3 (\Loc) = \Store_2 (\Loc)$
    and $\Loc\in{\Perm_2} \Leftrightarrow \Loc\in\Perm_3$ 
  \item Resources:
    Let $\REACH_3' =\Reach{\Store_3}{\Active{\VEnv_3}}$.
    For all $\Loc$ such that $n= \MultiNumber\Loc{\REACH_3} >0$ and $n' =
    \MultiNumber\Loc{\REACH_3'}$, 
    \begin{itemize}
    \item if $\Affine{\SE_2}\Loc$ then $n=1$ and $n'\le 1$,
    \item if $\Linear{\SE_2}\Loc$ then $n=1$ and $n' = 0$,
    \item if $n'=0$, then $\Loc\notin\Perm_3$.
    \end{itemize}
  \item Immutables: For all $\Loc \in \Reach
    {\Store_3}{\Active{\VEnv_2}}$, if $\neg\Writeable{\SE_2}\Loc$ then
    $\Store_3 (\ell) = \Store_2 (\ell)$ and $\Loc\in \Perm_3$.
  \item No thin air permission: For all $\Loc\in \Perm_3$, $\Loc
    \in \Perm_2 \cup  \Dom{\Store_3} \setminus \Dom{\Store_2}$.
  \end{itemize}
  From this we can finally conclude the inductive step.

  Let $\Store' = \Store_3$, $\Perm' = \Perm_3$, $r' = r_3$, and $\SE'
  = \SE_3$. Clearly $R' = R_3$ hence
  \begin{itemize}
  \item $R' = \Ok{\Store', \Perm', r'}$
  \item by transitivity, $\Store \le \Store'$ and $\SE \le \SE'$
  \item by (\ref{eq:12}), $\vdash \Store' : \SE'$
  \item by (\ref{eq:13}), $\Addresses{\Perm'} \subseteq \Dom{\Store'}$
  \item by (\ref{eq:14}), $\SE' \vdash r' : \tau_1$
  \item by (\ref{eq:15}), $\Reach{\Store'}{r'} \subseteq \Perm'$
  \item Outside: Let $\Loc \in \Dom{\Store} \setminus
    \Reach{\Store'}{\VEnv}$ and show that that 
    $\Store' (\Loc) = \Store (\Loc)$
    and $\Loc\in\Perm \Leftrightarrow \Loc\in\Perm'$.

    Inspecting the ``Outside'' cases for the subterms, we find that
    all of them only modify (permissions of) objects reachable from
    $\VEnv$ or newly allocated ones. Hence, the claim follows.
  \item Immutables: Let $\Loc \in
    \Reach{\Store'}{\ImmutableBorrows\VEnv}$ and show that
    $\Loc\in\Dom\Store$, 
    $\Store' (\Loc) = \Store (\Loc)$
    and $\Loc\in\Perm \Leftrightarrow \Loc\in\Perm'$.

    By definition of splitting, $\ImmutableBorrows{\VEnv_1} =
    \ImmutableBorrows{\VEnv_2} = \ImmutableBorrows\VEnv$. Moreover,
    $\VEnv_3$ refers to objects reachable from $\VEnv_1$ and $\VEnv_2$
    or that are new. By the inductive hypothesis on each
    subevaluation, contents and permissions of the immutable objects
    remain the same.
  \item Resources:
    Let $\REACH' =\Reach{\Store'}{\Active\VEnv}$.
    For all $\Loc$ such that $n= \MultiNumber\Loc\REACH >0$ and $n' =
    \MultiNumber\Loc{\REACH'}$, 
    \begin{itemize}
    \item if $\Affine{\SE}\Loc$ then $n=1$ and $n'\le 1$,
    \item if $\Linear{\SE}\Loc$ then $n=1$ and $n' = 0$,
    \item if $n'=0$, then $\Loc\notin\Perm'$.
    \end{itemize}
    Follows from the inductive hypotheses on the subevaluations.
  \item Immutables: For all $\Loc \in \Reach
    {\Store'}{\Active\VEnv}$, if $\neg\Writeable{\SE}\Loc$ then
    $\Store' (\ell) = \Store (\ell)$ and $\Loc\in \Perm'$.

    Follows from Immutable and Outside conditions of the subevaluations.
  \item No thin air permission: For all $\Loc\in \Perm'$, $\Loc
    \in \Perm \cup  \Dom{\Store'} \setminus \Dom{\Store}$.

    Immediate.
  \end{itemize}

%  \clearpage{}
  \textbf{case $e$ of}
  \lstsemrule{sregion}

  By assumption $\ruleSDRegion$.

  We need to establish the assumptions for the recursive call
  @eval delta pi' gamma i' e_1@.
  \begin{itemize}
  \item $\inferS{C}{\E'}{e}{\tau}$ by inversion of typing.
  \item To establish $\SE \vdash \VEnv' : \E'$, we find that
    $\lregion{C}{x}{\E}{\E'}$ only changes the $x$ entry of $\E$ to
    ${\bvar{\borrow[\BORROW]{x}}{\borrowty[\BORROW] k{\tau_x}}}$ in
    $\E'$ where $C \vdash_e {(\BORROW_n\lk k\lk\BORROW_\infty)}$. This
    change moves the entry for $x$ from $\Suspended\VEnv$ to either
    $\MutableBorrows\VEnv$ or $\ImmutableBorrows\VEnv$. The same
    happens in the semantics by mapping updating $\VEnv$ to $\VEnv'$. 
  \item $\vdash \Store : \SE$ by assumption.
  \item $\Addresses{\Perm'} \subseteq \Dom\Store$: we only change
    $\rho$ to $\rho'$, but their raw location is the same.
  \item $\Reach\Store{\Active{\VEnv'}, \MutableBorrows{\VEnv'},
      \ImmutableBorrows{\VEnv'}} \subseteq \Perm'$ because we
    ``activate'' the borrow for $x$ by putting $\rho'$ into $\Perm'$:
    The assumption $\Reach\Store{\Active\VEnv, \MutableBorrows\VEnv,
      \ImmutableBorrows\VEnv} \subseteq \Perm$ covers the remaining addresses.
  \item Incoming Resources: For $\REACH = \Reach\Store\VEnv$ we know that
    \begin{enumerate}
    \item\label{item:1} 
      For all $\Loc$ such that $\MultiNumber\Loc{\Reach{\Store}{\Active\VEnv}} >0$,
      if $\Affine{\SE}\Loc$ then $\MultiNumber\Loc\REACH= 1$
    \item\label{item:2} For all $\Loc$ such that $
      \MultiNumber\Loc{\Reach{\Store}{\MutableBorrows\VEnv}} >0$, it
      must be that $\MultiNumber\Loc\REACH=1$.
    \item\label{item:3} For all $\Loc$ such that $
      \MultiNumber\Loc{\Reach{\Store}{\Suspended\VEnv}} >0$, it
      must be that $\MultiNumber\Loc\REACH=1$.
    \end{enumerate}
    For $\REACH' = \Reach\Store{\VEnv'}$ we know that
    Item~\ref{item:1} and Item~\ref{item:3} do not change. Item~\ref{item:2} is not
    affected if $\BORROW=\IBORROW$. If $\BORROW=\MBORROW$, then we
    can deduce $\MultiNumber\Loc\REACH=1$ for the
    underlying location of $\rho'$ from Item~\ref{item:3}. 
  \item  $i'<i\in\Nat$ and @eval delta pi' gamma' i' e@ terminates
    with $R\ne \TimeOut$. 
  \end{itemize}
  Induction yields that
  \begin{itemize}
  \item
    $R_1 = \Ok{\Store_1, \Perm_1, r_1}$  
  \item $\SE \le \SE_1$, $\Store \le \Store_1$,
    $\vdash \Store_1 : \SE_1$ 
  \item $\Addresses{\Perm_1} \subseteq \Dom{\Store_1}$
  \item $\SE_1 \vdash r_1 : \tau$
  \item $\Reach{\Store_1}{r_1} \subseteq \Perm_1$
  \item Outside: For all $\Loc \in \Dom{\Store} \setminus
    \Reach{\Store_1}{\VEnv'}$ it must be that 
    $\Store_1 (\Loc) = \Store (\Loc)$
    and $\Loc\in\Perm' \Leftrightarrow \Loc\in\Perm_1$.
  \item Immutables: For all $\Loc \in
    \Reach{\Store_1}{\ImmutableBorrows{\VEnv'}}$ it must be that
    $\Loc\in\Dom\Store$, 
    $\Store_1 (\Loc) = \Store (\Loc)$
    and $\Loc\in\Perm' \Leftrightarrow \Loc\in\Perm_1$.
  \item Resources:
    Let $\REACH_1 =\Reach{\Store_1}{\Active{\VEnv'}}$.
    For all $\Loc$ such that $n= \MultiNumber\Loc{\REACH_1} >0$ and $n' =
    \MultiNumber\Loc{\REACH_1}$, 
    \begin{itemize}
    \item if $\Affine{\SE}\Loc$ then $n=1$ and $n'\le 1$,
    \item if $\Linear{\SE}\Loc$ then $n=1$ and $n' = 0$,
    \item if $n'=0$, then $\Loc\notin\Perm_1$.
    \end{itemize}
  \item Immutables: For all $\Loc \in \Reach
    {\Store_1}{\Active{\VEnv'}}$, if $\neg\Writeable{\SE}\Loc$ then
    $\Store_1 (\ell) = \Store (\ell)$ and $\Loc\in \Perm_1$.
  \item No thin air permission: For all $\Loc\in \Perm_1$, $\Loc
    \in \Perm' \cup  \Dom{\Store_1} \setminus \Dom{\Store}$.
  \end{itemize}
  From these outcomes, we prove the inductive step.
  Let $\Store' = \Store_1$, $\Perm_1' = \Perm_1-\rho'$, $r' = r_1$ and
  \begin{itemize}
  \item $R' = \Ok{\Store', \Perm_1', r'}$
  \item  $\SE \le \SE'$, $\Store \le \Store'$,
    $\vdash \Store' : \SE'$  by the subevaluation.
  \item $\Addresses{\Perm_1'} = \Addresses{\Perm_1-\rho'} \subseteq \Dom{\Store'}$
  \item $\SE' \vdash r' : \tau$
    %%% ok the next bit is fishy, but morally correct
  \item The constraint $\entail C {\Cleq{\tau}{\klin_{n-1}}}$
    along with the previous assumption on the kind $k$ of the type of $x$,
    $C \vdash_e {(\BORROW_n\lk k\lk\BORROW_\infty)}$, yields a
    contradiction, which shows that $\rho'$ cannot be reachable from $r'$.

    Hence, we conclude  $\Reach{\Store'}{r'} \subseteq \Perm_1- \rho'
    = \Perm_1'$.
  \item Outside: For all $\Loc \in \Dom{\Store} \setminus
    \Reach{\Store'}{\VEnv}$ it must be that 
    $\Store' (\Loc) = \Store (\Loc)$
    and $\Loc\in\Perm \Leftrightarrow \Loc\in\Perm_1'$. Immediate from
    the subevaluation as $\VEnv$ and $\VEnv'$ reach the same set of locations.
  \item Immutables: For all $\Loc \in
    \Reach{\Store'}{\ImmutableBorrows{\VEnv}}$ it must be that
    $\Loc\in\Dom\Store$, 
    $\Store' (\Loc) = \Store (\Loc)$
    and $\Loc\in\Perm \Leftrightarrow \Loc\in\Perm_1'$.
    This set can only be affected if $\rho'$ is an immutable
    borrow. But in this case, $\rho'$ is not reachable from
    $\ImmutableBorrows\VEnv$ so that the conclusion holds.
  \item Resources are not affected by this evaluation rule:
    Let $\REACH' =\Reach{\Store'}{\Active{\VEnv}}$.
    For all $\Loc$ such that $n= \MultiNumber\Loc{\REACH} >0$ and $n' =
    \MultiNumber\Loc{\REACH'}$, 
    \begin{itemize}
    \item if $\Affine{\SE}\Loc$ then $n=1$ and $n'\le 1$,
    \item if $\Linear{\SE}\Loc$ then $n=1$ and $n' = 0$,
    \item if $n'=0$, then $\Loc\notin\Perm_1$.
    \end{itemize}
  \item Immutables: For all $\Loc \in \Reach
    {\Store'}{\Active{\VEnv}}$, if $\neg\Writeable{\SE}\Loc$ then
    $\Store' (\ell) = \Store (\ell)$ and $\Loc\in \Perm_1'$.
  \item No thin air permission: For all $\Loc\in \Perm_1'$, $\Loc
    \in \Perm \cup  \Dom{\Store'} \setminus \Dom{\Store}$.
  \end{itemize}
%  \clearpage{}
  \textbf{case $e$ of}
  \lstsemrule{sborrow}

  By assumption
  \begin{enumerate}
  \item $\ruleSDBorrow$.
  \item\label{item:4} $\SE \vdash \VEnv : \E$
  \item\label{item:7} $\vdash \Store : \SE$
  \item\label{item:6} $\Addresses\Perm \subseteq \Dom\Store$
  \item\label{item:5} $\Reach\Store{\Active\VEnv, \MutableBorrows\VEnv, \ImmutableBorrows\VEnv} \subseteq \Perm$
  % \item  $\VEnv'$ with $\Addresses{\VEnv'}
  %   \subseteq \Dom\Store$ and $\Dom\VEnv \cap \Dom{\VEnv'}=\emptyset$
  \item Incoming Resources: Let $\REACH = \Reach\Store\VEnv$.
    \begin{itemize}
    \item 
      For all $\Loc$ such that $\MultiNumber\Loc{\Reach{\Store}{\Active\VEnv}} >0$,
      if $\Affine{\SE}\Loc$ then $\MultiNumber\Loc\REACH= 1$
    \item For all $\Loc$ such that $
      \MultiNumber\Loc{\Reach{\Store}{\MutableBorrows\VEnv}} >0$, it
      must be that $\MultiNumber\Loc\REACH=1$.
    \item For all $\Loc$ such that $
      \MultiNumber\Loc{\Reach{\Store}{\Suspended\VEnv}} >0$, it
      must be that $\MultiNumber\Loc\REACH=1$.
    \end{itemize}
  \item As $i>0$ @eval delta pi gamma i (Borrow (b,x))@ yields some $R\ne \TimeOut$.
  \end{enumerate}
  First, $\VEnv (x)$ yields an address $\rho$ by
  assumption~\ref{item:4}.

  Second, $\rho$ is a borrow address ending with $\BORROW$ also
  by~\ref{item:4}.
  
  By assumption~\ref{item:5}, $\rho \in \Perm$ as $\rho$ is either in
  $\MutableBorrows\VEnv$ or $\ImmutableBorrows\VEnv$.

  Hence,
  \begin{itemize}
  \item $R=\Ok{\Store, \Perm, \rho}$.
  \item $\SE \le \SE$, $\Store \le \Store$ trivially
  \item $\vdash \Store : \SE$ by assumption~\ref{item:7}.
  \item $\Addresses{\Perm} \subseteq \Dom{\Store}$ by
    assumption~\ref{item:6}.
  \item $\SE \vdash \rho : \borrowty{k}{\tau}$ by inversion of typing.
  \item $\Reach{\Store}{\rho} \subseteq \Perm$ by assumption~\ref{item:5}.
  \item Outside holds trivially: For all $\Loc \in \Dom{\Store} \setminus
    \Reach{\Store}{\VEnv}$ it must be that 
    $\Store (\Loc) = \Store (\Loc)$
    and $\Loc\in\Perm \Leftrightarrow \Loc\in\Perm$.
  \item Immutables holds trivially: For all $\Loc \in
    \Reach{\Store}{\ImmutableBorrows\VEnv}$ it must be that
    $\Loc\in\Dom\Store$, 
    $\Store (\Loc) = \Store (\Loc)$
    and $\Loc\in\Perm \Leftrightarrow \Loc\in\Perm$.
  \item Resources:
    Let $\REACH =\Reach{\Store}{\Active\VEnv}$.
    For all $\Loc$ such that $n= \MultiNumber\Loc\REACH >0$ and $n' =
    \MultiNumber\Loc{\REACH}$, 
    \begin{itemize}
    \item if $\Affine{\SE}\Loc$ then $n=1$ and $n'\le 1$,
    \item if $\Linear{\SE}\Loc$ then $n=1$ and $n' = 0$,
    \item if $n'=0$, then $\Loc\notin\Perm$.
    \end{itemize}

    This statement is void because $\REACH=\emptyset$ due to the
    constraint on $\E$ in the typing.
  \item Immutables: For all $\Loc \in \Reach
    {\Store}{\Active\VEnv}$, if $\neg\Writeable{\SE}\Loc$ then
    $\Store (\ell) = \Store (\ell)$ and $\Loc\in \Perm$.

    Void for the same reason as before.
  \item No thin air permission holds trivially: For all $\Loc\in \Perm$, $\Loc
    \in \Perm \cup  \Dom{\Store} \setminus \Dom{\Store} = \Perm$.
  \end{itemize}
  

\end{proof}

%%% Local Variables:
%%% mode: latex
%%% TeX-master: "main"
%%% End:

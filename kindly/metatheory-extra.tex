\section{Proofs for Metatheory}
\label{sec:metatheory:proofs}

\begin{itemize}
\item
  For simplicity, we only consider terms in A-normal forms following the grammar:
  \begin{align*}
    e ::=~& \ldots \mid \aiapp{x}{x'} \mid \aipair{k}{x}{x'} \mid \aimatchin\etransfm{x,y}{z}{e}
  \end{align*}
  Typing and semantics rules are unchanged.
\item Borrow qualifiers  $\BQ ::= \IBORROW_n \mid \MBORROW_n$ where
  $n\ge0$ is a region level. A vector of borrow qualifiers $\Multi\BQ$
  is wellformed if all $\IBORROW$s come before all $\MBORROW$s in the vector. 
\item Borrow compatibility
  $\Multi\BQ \Bcompatible \BQ$,
  \begin{mathpar}
  \inferrule{}{
    \BQ_n\Multi\BQ \Bcompatible \BQ_n
  }
  %
  % \inferrule{}{
  %   \MBORROW_n\Multi\MBORROW \Bcompatible \MBORROW_n
  % }
  \end{mathpar}
\item Store typing $ \vdash \Store : \SE$,
  \begin{mathpar}
    \inferrule{
      (\forall \Loc \in \Dom\Store)~~
      \SE \vdash \Store (\Loc) : \SE (\Loc)
    }{ \vdash \Store : \SE }
  \end{mathpar}
\item Relating storables to type schemes $\SE \vdash w : \schm$

  We write $\Disjoint\E$ for $\Active\VEnv$ and $\MutableBorrows\VEnv$
  and $\ImmutableBorrows\VEnv$ and $\Suspended\VEnv$ are all disjoint.
  \begin{mathpar}
  \inferrule{
    (\exists \E)~ \SE \vdash \VEnv : \E
    \\
    \Disjoint\E
    \\
    \inferS{C}{\E; \bvar x{\tau_2}}{e}{\tau_1}
    \\
    \Multi\tvar = \fv{\tau_1,\tau_2} \setminus \fv{\E}
  }{
    \SE \vdash (\VEnv, \ilam {\Multi\kvar}{\Multi\tvar}Ckx{e})
    : \forall\Multi\kvar\forall\Multi{\bvar{\tvar}{k}}.(\qual{C}{\tau_2\tarr{k}\tau_1})
  }
  \end{mathpar}
\item Relating storables to types $ \SE \vdash w : \tau$
  \begin{mathpar}
    \ruleStorableClosure
    
    \ruleStorablePair

    \ruleStorableResource

    \ruleStorableFreed
  \end{mathpar}
%\item Relating  results to types $ \SE \vdash r : \tau$,
\item Relating results to type schemes $\SE \vdash r : \schm$
  \begin{mathpar}
    \ruleResultConstant

    \ruleResultLocation

    \ruleResultBorrow
  \end{mathpar}
\item
We write $\Affine\SE\Loc$ to express that $\Loc$ points to a resource
that requires at least affine treatment. Borrow types do not appear in
store types as the store only knows about the actual resources.

Define  $\Affine\SE\Loc$ if one of the following cases holds:
\begin{itemize}
\item $\SE (\Loc) =
  \forall\Multi\kvar\forall\Multi{\bvar{\tvar}{k}}.(\qual{C}{\tau_2\tarr{k}\tau_1})$
  and $C \wedge (k \lk \kun_\infty)$ is contradictory;
\item $\SE (\Loc) = \tau_2\tarr{k}{\tau_1}$ and $\Cleq{\kaff}{k}$;
\item $\SE (\Loc) = \tyPair[k]{\tau_1}{\tau_2}$ and $\Cleq \kaff
  k$;
\item $\SE (\Loc) = \tapp{\tcon}{\Multi\tau}$.
% \item if $\SE (\Loc) = \borrow[\MBORROW]\tau$, then $\Loc$ is affine
%   \dots (THAT SHOULDN'T REALLY  BE A STORE TYPE)
% \item if $\SE (\Loc) = \borrow[\IBORROW]\tau$, then $\Loc$ is not affine
%   \dots (THAT SHOULDN'T REALLY  BE A STORE TYPE)
\end{itemize}
\item
We write $\Linear\SE\Loc$ to express that $\Loc$ points to a linear
resource.

Define  $\Linear\SE\Loc$ if one of the following cases holds:
\begin{itemize}
\item $\SE (\Loc) =
  \forall\Multi\kvar\forall\Multi{\bvar{\tvar}{k}}.(\qual{C}{\tau_2\tarr{k}\tau_1})$
  and $C \wedge (k \lk \kaff_\infty)$ is contradictory;
\item $\SE (\Loc) = \tau_2\tarr{k}{\tau_1}$ and $\Cleq{\klin}{k}$;
\item $\SE (\Loc) = \tyPair[k]{\tau_1}{\tau_2}$ and $\Cleq \klin
  k$;
\item $\SE (\Loc) = \tapp{\tcon}{\Multi\tau}$.
\end{itemize}
\item
It remains to characterize unrestricted resources.
Define $\Unrestricted\SE\Loc$ if neither $\Affine\SE\Loc$ nor
$\Linear\SE\Loc$ holds.
\item
  Relating environments to contexts\\
  $\SE \vdash \Active\VEnv, \MutableBorrows\VEnv,
\ImmutableBorrows\VEnv, \Suspended[\kaff]\VEnv, \Suspended[\kun]\VEnv : \E$.

Here we consider an
environment $\VEnv = (\Active\VEnv, \MutableBorrows\VEnv,
\ImmutableBorrows\VEnv, \Suspended\VEnv)$ as a tuple
consisting of the active entries in $\Active\VEnv$ and the
entries for exclusive borrows in $\MutableBorrows\VEnv$ and for
shared borrows in $\ImmutableBorrows\VEnv$, and suspended entries
in $\Suspended\VEnv = \Suspended[\kaff]\VEnv, \Suspended[\kun]\VEnv$
for affine and unrestricted entries. The
suspended entries cannot be used directly, but they can be activated
by appropriate borrowing on entry to a region.
\end{itemize}


\begin{mathpar}
  \inferrule{}{\SE \vdash \Sempty, \Sempty, \Sempty, \Sempty : \Eempty}
  \\
  \inferrule{
    \SE \vdash \Active\VEnv, \MutableBorrows\VEnv,
    \ImmutableBorrows\VEnv, \Suspended\VEnv : \E
    \\ \SE \vdash r : \schm
    \\ \Linear\SE r}
  {\SE \vdash \Active\VEnv[ x\mapsto r], \MutableBorrows\VEnv ,
    \ImmutableBorrows\VEnv, \Suspended\VEnv : \E;\bvar x\schm }

  \inferrule{
    \SE \vdash \Active\VEnv, \MutableBorrows\VEnv,
    \ImmutableBorrows\VEnv, \Suspended\VEnv : \E
    \\ \SE \vdash r : \schm
    \\ \Affine\SE r}
  {\SE \vdash \Active\VEnv, \MutableBorrows\VEnv[ x\mapsto r] ,
    \ImmutableBorrows\VEnv, \Suspended\VEnv : \E;\bvar x\schm }

  \inferrule{
    \SE \vdash \Active\VEnv, \MutableBorrows\VEnv,
    \ImmutableBorrows\VEnv, \Suspended\VEnv : \E
    \\ \SE \vdash r : \schm
    \\ \Unrestricted\SE r}
  {\SE \vdash \Active\VEnv, \MutableBorrows\VEnv,
    \ImmutableBorrows\VEnv[ x\mapsto r], \Suspended\VEnv : \E;\bvar
    x\schm }

  \inferrule{\SE \vdash \Active\VEnv, \MutableBorrows\VEnv,
    \ImmutableBorrows\VEnv, \Suspended[\kaff]\VEnv, \Suspended[\kun]\VEnv : \E \\
    \SE \vdash r : \schm}
  { \SE \vdash \Active\VEnv, \MutableBorrows\VEnv,
    \ImmutableBorrows\VEnv, \Suspended[\kaff]\VEnv, \Suspended[\kun]\VEnv[ x\mapsto r] :
    \E;\svar[\IBORROW] x\schm^n }

  \inferrule{\SE \vdash \Active\VEnv, \MutableBorrows\VEnv,
    \ImmutableBorrows\VEnv, \Suspended[\kaff]\VEnv, \Suspended[\kun]\VEnv : \E \\
    \SE \vdash r : \schm}
  { \SE \vdash \Active\VEnv, \MutableBorrows\VEnv,
    \ImmutableBorrows\VEnv, \Suspended[\kaff]\VEnv[ x\mapsto r], \Suspended[\kun]\VEnv :
    \E;\svar[\MBORROW] x\schm^n }

  \inferrule{\SE \vdash \Active\VEnv, \MutableBorrows\VEnv,
    \ImmutableBorrows\VEnv, \Suspended\VEnv : \E \\ \SE \vdash
    \IBORROW\Addr : \schm}
  {\SE \vdash \Active\VEnv, \MutableBorrows\VEnv,
    \ImmutableBorrows\VEnv[ x\mapsto \IBORROW\Addr], \Suspended\VEnv :
    \E;\bvar{\borrow[\IBORROW] x}{\borrowty[\IBORROW] k{ \schm}} }

  \inferrule{\SE \vdash \Active\VEnv, \MutableBorrows\VEnv,
    \ImmutableBorrows\VEnv, \Suspended\VEnv : \E \\ \SE \vdash
    \MBORROW\Addr : \schm}
  {\SE \vdash \Active\VEnv, \MutableBorrows\VEnv[ x\mapsto \MBORROW\Addr],
    \ImmutableBorrows\VEnv, \Suspended\VEnv
    : \E;\bvar{\borrow[\MBORROW] x}{\borrowty[\MBORROW] k{ \schm}} }
\end{mathpar}
\paragraph{Extending environments and stores}
\begin{mathpar}
  \inferrule{}{\SE \le \SE}

  \inferrule{\SE \le \SE' \\ \ell \notin \Dom\Store}{\SE \le \SE' (\ell : \schm)}
\\
  \inferrule{}{\Store\le\Store}

  \inferrule{\Store \le \Store' \\ \ell \notin \Dom\Store
  }{\Store \le \Store'[ \ell \mapsto w] }
\end{mathpar}

\begin{lemma}[Store Weakening]\label{lemma:store-weakening}
  $\SE \vdash \VEnv : \E$ and $\SE \le \SE'$ implies $\SE' \vdash
  \VEnv : \E$.
\end{lemma}

\begin{lemma}[Store Extension]\label{lemma:store-extension-transitive}
  \begin{itemize}
  \item If $\SE_1 \le \SE_2$ and $\SE_2 \le \SE_3$, then $\SE_1 \le
    \SE_3$.
  \item If $\Store_1 \le \Store_2$ and $\Store_2 \le \Store_3$, then
    $\Store_1 \le \Store_3$.
  \end{itemize}
\end{lemma}

We write $\Rawloc\cdot$ for the function that extracts a multiset of
\emph{raw locations} from a result or from the range of the variable
environment.

\begin{align*}
  \Rawloc{\Multi\BQ\Loc} &= \{\Loc\} \\
  \Rawloc{c} &= \{ \} \\
  \Rawloc\Eempty &= \{\} \\
  \Rawloc{\VEnv( x \mapsto r)} &= \Rawloc\VEnv \cup \Rawloc r
\end{align*}

We write $\Reach\Store\VEnv$ for the multiset of all \emph{addresses}
reachable from $\Rawloc\VEnv$
assuming that $\Rawloc\VEnv \subseteq \Dom\Store$\footnote{In
  mixed comparisons between a multiset and a set, we tacitly convert
  a multiset $M$ to its supporting set $\{ x \mid \MultiNumber x M \ne 0\}$.}.
The function $\Reach\Store\cdot$ is defined in
two steps. First a helper function
for results, storables, and environments.

\begin{align*}
  \RS\Store\Eempty &= \Eempty \\
  \RS\Store{\VEnv (x \mapsto r)} &= \RS\Store\VEnv \cup
                                      \RS\Store r \\
  \RS\Store\Addr &= \{ \Addr \}  \\
  \RS\Store c &= \{ \} \\
  \RS\Store{\StPClosure \VEnv {\Multi\kvar} C k x e} &=
                                       \RS\Store\VEnv
  \\
  \RS\Store{\StClosure \VEnv k x e} &=
                                                   \RS\Store\VEnv
  \\
  \RS\Store{\StPair k {r_1} {r_2}} &=
                                                   \RS\Store{r_1}
                                                   \cup \RS\Store{r_2}
  \\
  \RS\Store{\StRes r} &=
                                   \RS\Store r
  \\
  \RS\Store{\StFreed} &= \{ \}
\end{align*}

This multiset is closed transitively by store lookup. We define
$\Reach\Store\VEnv$ as the smallest multiset $\REACH$ that fulfills
the following inequations. We assume a nonstandard
model of multisets such that an element $\Loc$ may occur infinitely often as in
$\MultiNumber\Loc\REACH = \infty$.
\begin{align*}
  \REACH &\supseteq \RS\Store\VEnv \\
  \REACH &\supseteq \RS\Store w & \text{if }
                                     \Multi\BQ\Loc
                                     \in \REACH \wedge w = \Store (\Loc)
\end{align*}

\begin{definition}[Wellformed permission]
  A permission $\Perm$ is \emph{wellformed} if it contains at most one
  address for each raw location.
\end{definition}

\begin{lemma}[Containment]\label{lemma:containment}
  Suppose that $\vdash \Store : \SE$,
  $\SE \vdash r : \tau$, $\entail C
  {\Cleq{\tau}{k} \Cand \Cleq{k}{\klin_{m-1}}}$.
  Then $\Reach\Store r$ cannot contain addresses $\Addr$ such that
  $\Addr = \BORROW_n\Addr'$ with $n\ge m$.
\end{lemma}
\begin{proof}
  By inversion of result typing there are three cases.

  \textbf{Case }$\ruleResultConstant$. Immediate: reachable set
  is empty.

  \textbf{Case }$\ruleResultBorrow$. The typing constraint enforces
  that $n < m$.

  \textbf{Case }$\ruleResultLocation$. We need to continue by
  dereferencing $\Loc$ and inverting store typing.

  \textbf{Case }$\ruleStorableFreed$. Trivial.

  \textbf{Case }$\ruleStorableResource$. We assume the implementation
  type of a result to be unrestricted.

  \textbf{Case }$\ruleStorablePair$.

  The typing constraint yields that $k \le \klin_{m-1}$.
  By induction and transitivity of $\le$, we find that
  $\Reach\Store{r_1}$ and $\Reach\Store{r_2}$ cannot contain offending addresses.

  \textbf{Case }$\ruleStorableClosure$.

  The typing constraint yields that $k\le\klin_{m-1}$.
  By transitivity of $\le$ and $\SE \vdash \VEnv:\E$, we find that the
  types of all addresses in 
  $\VEnv$ have types bounded by $\klin_{m-1}$ and, by induction, they
  cannot contain offending addresses.
\end{proof}
\clearpage{}
\lstMakeShortInline[keepspaces,style=rule]@

\SoundnessThm

Some explanations are in order for the resource-related assumptions
and statements.

Incoming resources are always active (i.e., not freed).
Linear and affine resources as well as suspended affine borrows have
exactly one pointer in the environment.

The Frame condition states that only store locations reachable from
the current environment can change and that all permissions outside
the reachable locations remain the same.

Unrestricted values, resources, and borrows do not change their
underlying resource and do not spend their permission.

Affine borrows and resources may or may not spend their
permission. Borrows are not freed, but resources may be freed. The
permissions for suspended entries remain intact.

% Incoming suspended borrows have no permission attached to them and
% their permission has been retracted on exit of their region.

A linear resource is always freed.

Outgoing permissions are either inherited from the caller or they
refer to newly created values.



\newpage
\begin{proof}
  By induction on the evaluation of
  @eval \Store \Perm \VEnv i e@.

  The base case is trivial as
  @eval \Store \Perm \VEnv 0 e = \TimeOut@.

  For $i>0$ consider the different cases for expressions. For lack of
  spacetime, we only give details on some important cases.

  \textbf{Case $e$ of}
  \lstsemrule{slet}
  We need to invert rule \TirName{Let} for monomorphic let:
  \begin{gather*}
    \ruleSDILet
    % \ruleSDLet
  \end{gather*}
  %% PJT: separate lemma about splitting needed
  As $\Sp$ is the evidence for the splitting judgment and @vsplit@
  distributes values according to $\Sp$, we obtain
  \begin{gather}
    \label{eq:16}
    \SE \vdash \VEnv_1 : \E_1
    \\\label{eq:17}
    \SE \vdash \VEnv_2 : \E_2
  \end{gather}
  Moreover (using $\uplus$ for disjoint union),
  \begin{itemize}
  \item $\Active\VEnv = \Active{\VEnv_1} \uplus\Active{\VEnv_2}$,
  \item $\MutableBorrows\VEnv = \MutableBorrows{\VEnv_1} \uplus
    \MutableBorrows{\VEnv_2}$,
  \item $\ImmutableBorrows\VEnv =
    \ImmutableBorrows{\VEnv_1} = \ImmutableBorrows{\VEnv_2}$,
  \item $\Suspended\VEnv = \Suspended{\VEnv_1} \uplus
    \Suspended{\VEnv_2}$ (this splitting does not distinguish potentially
    unrestricted or affine bindings)
  \end{itemize}
  We establish the assumptions for the call
  @eval \Store \Perm \VEnv_1  i' e_1@.
  \begin{enumerate}[({A1-}1)]
  \item From inversion: $\inferS{C \Cand D}{\E_1}{e_1}{\tau_1} $
  \item From~\eqref{eq:16}: $\SE \vdash \VEnv_1 : \E_1$
  \item From assumption
  \item From assumption
  \item From assumption because $\VEnv_1$ is projected from $\VEnv$.
  \item From assumption because $\VEnv_1$ is projected from $\VEnv$.
  \item From assumption because $\VEnv_1$ is projected from $\VEnv$.
  \end{enumerate}
  Hence, we can apply the induction hypothesis and obtain
  \begin{enumerate}[({R1-}1)]
  \item \resultOk{}{_1}
  \item\label{item:8} \resultEnv{}{_1}
  \item\label{item:9} $\SE_1 \vdash r_1 : \tau_1$
  \item\label{item:14} \resultPermDom{}{_1}
  \item\label{item:13} \resultReachPerm{}{_1}
  \item\label{item:16} \resultFrame{}{_1}{_1}
  \item\label{item:24} \resultImmutables{}{}{_1}
  \item\label{item:26} \resultMutables{}{}{_1}
  % \item\label{item:28} \resultSuspended{}{}{_1}
  \item\label{item:30} \resultResources{}{}{_1}
  \item\label{item:10} \resultThinAir{}{_1}
  \end{enumerate}
  To establish the assumptions for the call \\
  @eval delta_1 pi_1 gamma_2(x-:>r_1) i' e_2@,
  we write $\VEnv_2' = \VEnv (x \mapsto r_1)$.
  \begin{enumerate}[({A2-}1)]
  \item From inversion: $\inferS{C}{\E;\bvar{x}{\tau_1}}{e_2}{\tau_2}$
  \item From~\eqref{eq:17} we have $\SE \vdash \VEnv_2 : \E_2$.
    By store weakening (Lemma~\ref{lemma:store-weakening}) and
    using~\ref{item:8}, we have
    $\SE_1 \vdash \VEnv_2 : \E_2$.
    With~\ref{item:9}, we obtain
    $\SE_1 \vdash \VEnv_2 (x \mapsto r_1 ) : \E_2;\bvar x {\tau_1}$.
  \item Immediate from~\ref{item:8}.
  \item Immediate from~\ref{item:14}.
  \item $\Reach{\Store_1}{\Active{\VEnv_2'}, \MutableBorrows{\VEnv_2'},
      \ImmutableBorrows{\VEnv_2'}} \subseteq \Perm_1$ \\
    From~\ref{item:12}, we have
     $\Reach{\Store_1}{\Active{\VEnv_2}, \MutableBorrows{\VEnv_2},
      \ImmutableBorrows{\VEnv_2}} \subseteq \Perm_1$. The extra
    binding $(x \mapsto r_1)$ goes into one of the compartments
    according to its type. We conclude by~\ref{item:13}.
  \item Disjointness holds by assumption for $\VEnv_2$ and it remains
    to discuss $r_1$. But $r_1$ is either a fresh resource, a
    linear/affine resource from $\VEnv_1$ (which is disjoint), or
    unrestricted. In each case, there is no overlap with another
    compartment of the environment.
  \item We need to show
    \assumeIncoming{_1}{_2'}
    The first item holds by assumption, splitting, and (for $r_1$)
    by~\ref{item:14} and~\ref{item:13}.

    The second and third items hold by assumption~\ref{item:15},
    splitting, and framing~\ref{item:16}.
  \end{enumerate}
  Hence, we can apply the induction hypothesis and obtain
  \begin{enumerate}[({R2}-1)]
  \item\label{item:17} \resultOk{_1}{_2}
  \item\label{item:18} \resultEnv{_1}{_2}
  \item\label{item:19} $\SE_2 \vdash r_2 : \tau_2$
  \item\label{item:20} \resultPermDom{_1}{_2}
  \item\label{item:21} \resultReachPerm{_1}{_2}
  \item\label{item:22} \resultFrame{_1}{_2'}{_2}
  \item\label{item:23} \resultImmutables{_1}{_2'}{_2}
  \item\label{item:25} \resultMutables{_1}{_2'}{_2}
  % \item\label{item:27} \resultSuspended{_1}{_2'}{_2}
  \item\label{item:29} \resultResources{_1}{_2'}{_2}
  \item\label{item:31} \resultThinAir{_1}{_2}
  \end{enumerate}

  It remains to establish the assertions for the let expression.
  \begin{enumerate}[({R}-1)]
  \item \resultOk{_1}{_2}
    \\ Immediate from~\ref{item:17}.
  \item \resultEnv{}{_2}
    \\ Transitivity of store extension
    (Lemma~\ref{lemma:store-extension-transitive}), \ref{item:18},
    and~\ref{item:8}.
  \item  $\SE_2 \vdash r_2 : \tau_2$
    \\ Immediate from~\ref{item:19}.
  \item \resultPermDom{}{_2}
    \\ Immediate from~\ref{item:20}.
  \item \resultReachPerm{}{_2}
    \\ Immediate from~\ref{item:21}
    \textcolor{red}{because $\Reach{\Store_2}{\VEnv_1}
    \subseteq \Reach{\Store_2}{\VEnv} $ and
    $\Reach{\Store_2}{\Suspended{\VEnv_1}} \subseteq
    \Reach{\Store_2}{\Suspended\VEnv}$.
    Moreover, $\Dom\Store \subseteq \Dom{\Store_1}$.
  }
\item \resultFrame{}{}{_2}
    ~\\ Suppose that $\Loc \in \Dom\Store \setminus
    \Rawloc{\Reach{\Store_2}{\VEnv}}$.
    \\ Then $\Loc \in \Dom\Store \setminus
    \Rawloc{\Reach{\Store_1}{\VEnv_1}}$.
    \\ By~\ref{item:16}, $\Store_1 (\Loc) = \Store (\Loc)$ and for any
    $\Addr$ with $\Rawloc\Addr = \{\Loc\}$: $\Addr \in \Perm
    \Leftrightarrow \Addr \in \Perm_1$.
    \\ But also $\Loc \in \Dom{\Store_1} \setminus
    \Rawloc{\Reach{\Store_2}{\VEnv_2'}}$.
    \\ By~\ref{item:22}, $\Store_2 (\Loc) = \Store_1 (\Loc)$ for
    applicable $\Addr$, $\Addr \in \Perm_1
    \Leftrightarrow \Addr \in \Perm_2$.
    \\ Taken together, we obtain the claim.
  \item \resultImmutables{}{}{_2}
    \\ Follows from~\ref{item:23} or~\ref{item:24} because
    $\ImmutableBorrows\VEnv = \ImmutableBorrows{\VEnv_1} =
    \ImmutableBorrows{\VEnv_2}$.
  \item \resultMutables{}{}{_2}
    \\ Follows from~\ref{item:25},~\ref{item:26}, and framing.
  % \item \resultSuspended{}{}{_2}
  %   ~\\ Follows from~\ref{item:27},~\ref{item:28}, and framing.
  \item \resultResources{}{}{_2}
    \\ Follows from disjoint splitting of
    $\Active\VEnv$,~\ref{item:29},~\ref{item:30}, and framing.
  \item \resultThinAir{}{_2}
    \\ Immediate from~\ref{item:31}.
  \end{enumerate}

  \newpage{}
  \textbf{Case $e$ of}
  \lstsemrule{sappanf}

  We need to invert rule \TirName{App}:
  \begin{mathpar}
    \ruleSDAIApp
    %\ruleSDVApp
  \end{mathpar}

  We need to establish the assumptions for the recursive call
  @eval \Store' \Perm' \VEnv'(x'-:>r_2) i' e'@.
  We write $\VEnv_2' = \VEnv' (x'\mapsto r_2)$.
  \begin{enumerate}[({A1-}1)]
  \item $\inferS{C'}{\E'; \bvar{x'}{\tau_2}}{e'}{\tau_1}$, for some
    $C'$ and $\E'$\\
    Applying the first premise of \TirName{App} to $r_1 = \VEnv
    (x_1)$, $\SE \vdash \VEnv : \E$, and inversion of result typing
    yields that $r_1 = \Loc_1$ with $\SE (\Loc_1) = \tau_2 \tarr{k}
    \tau_1$.
    By inversion of store typing and storable typing, we find that
    there exist $\E'$ and $C'$ such that
    \begin{enumerate}
    \item $\Store (\Loc_1) = (\VEnv', \lam[k]{x'}{e'})$
    \item $\Disjoint{\E'}$
    \item\label{item:32} $\SE \vdash \VEnv', \E'$
    \item $\inferS{C'}{\E';\bvar {x'}{\tau_2}}{e'}{\tau_1}$
    \item $\addlin{\entail{C'}{\Cleq{\E'}{k}}}$
    \end{enumerate}
  \item $\SE \vdash \VEnv' (x' \mapsto r_2) : \E';
    \bvar{x'}{\tau_2}$\\
    By~\ref{item:32}, assumption on $\VEnv$, and the subtyping premise.
  \item $\vdash \Store' : \SE'$ \\
    by assumption and the released rule of store typing (where we
    write $\SE'=\SE$ henceforth)
  \item \assumeWellformed{'} \\
    the possible removal of a permission does not violate
    wellformedness; the permission is taken away exactly when the
    closure is destroyed
  \item $\Reach\Store{\VEnv_2'} \subseteq \Perm$\\
    as the reach set is a subset of the incoming environment's reach
  \item $\Rawloc{\Active{\VEnv_2'}}$,
    $\Rawloc{\MutableBorrows{\VEnv_2'}}$,
    $\Rawloc{\ImmutableBorrows{\VEnv_2'}}$, and
    $\Rawloc{\Suspended{\VEnv_2'}}$ are all disjoint follows from
    $\Disjoint{\E'}$ and since $r_2= \E (x_2)$ which is an entry
    disjoint from the closure $\E(x_1)$.
  \item \assumeIncoming{'}{_2'}
    The first item holds because of assumption~\ref{item:15}.

    The second item holds because $\REACH' \subseteq \REACH$ from
    assumption~\ref{item:15}.
  \end{enumerate}
  The inductive hypothesis yields that
    $\exists$ $\Store_3$, $\Perm_3$, $r_3$, $\SE_3$ such that
  \begin{enumerate}[({R1-}1)]
  \item \resultOk{}{_3}
  \item \resultEnv{'}{_3}
  \item $\SE_3 \vdash r_3 : \tau_1$
  \item \resultPermDom{}{_3}
  \item \resultReachPerm{}{_3}
  \item \resultFrame{'}{_2'}{_3}
  \item \resultImmutables{'}{_2'}{_3}
  \item \resultMutables{'}{_2'}{_3}
    % \item \resultSuspended{}{}{'}
  \item \resultResources{'}{_2'}{_3}
  \item \resultThinAir{'}{_3}
  \end{enumerate}
  The desired results are immediate because $\Dom\Store = \Dom{\Store'}$.

  \newpage{}
  \textbf{Case $e$ of}
  \lstsemrule{sregion}

  We need to invert rule \TirName{Region}
  \begin{mathpar}
    \ruleSDRegion
  \end{mathpar}

  We need to establish the assumptions for the recursive call
  @eval \Store \Perm' \VEnv' i' e'@ where  $\VEnv' =
  \VEnv (x\mapsto \Addr')$.

  \begin{enumerate}[({A1-}1)]
  \item $\inferS{C}{\E'}{e}{\tau}$ \\
    immediate from premise
  \item $\SE \vdash \VEnv' : \E'$ \\
    the only change of the environments is at $x$; adding the borrow
    modifier $\BORROW$ succeeds due to the second premise; the address $\Addr'$
    stored into $x$ is compatible to its type by store typing
  \item $\vdash \Store : \SE$ \\
    Immediate by outer assumption
  \item\label{item:11} \assumeWellformed{} \\
    Immediate by outer assumption; adding the modifier does not change
    the underlying raw location
  \item\label{item:12} $\Reach\Store\VEnv' \subseteq \Perm$ \\
    locations were swapped simultaneously
  \item $\Rawloc{\Active{\VEnv'}}$,
    $\Rawloc{\MutableBorrows{\VEnv'}}$,
    $\Rawloc{\ImmutableBorrows{\VEnv'}}$, and
    $\Rawloc{\Suspended{\VEnv'}}$ are all disjoint \\
    Immediate by assumption
  % \item  $\VEnv'$ with $\Rawloc{\VEnv'}
  %   \subseteq \Dom\Store$ and $\Dom\VEnv \cap \Dom{\VEnv'}=\emptyset$
  \item\label{item:15} \assumeIncoming{}{'}
    Immediate by assumption.
  \end{enumerate}

  The induction hypothesis yields the following statements.
  $\exists$ $\Store_1$, $\Perm_1$, $r_1$, $\SE_1$ such that
  \begin{enumerate}[({R1-}1)]
  \item \resultOk{}{_1}
  \item \resultEnv{}{_1}
  \item $\SE_1 \vdash r_1 : \tau$
  \item \resultPermDom{}{_1}
  \item \resultReachPerm{}{_1}
  \item \resultFrame{}{'}{_1}
  \item \resultImmutables{}{'}{_1}
  \item \resultMutables{}{'}{_1}
  % \item \resultSuspended{}{}{'}
  \item \resultResources{}{'}{_1}
  \item \resultThinAir{}{_1}
  \end{enumerate}

  It remains to derive the induction hypothesis in the last line. The
  only additional action is the exchange of permissions which
  withdraws the borrow.
  \begin{enumerate}[({R}1)]
  \item \resultOk{}{_1}\\
    Immediate
  \item \resultEnv{}{_1}\\
    Immediate
  \item $\SE_1 \vdash r_1 : \tau$\\
    Immediate
  \item \resultPermDom{}{_1} \\
    The addresses $\Addr$ and $\Addr'$ have the same raw location, so
    exchanging them does not affect wellformedness. The underlying set
    of locations does not change.
  \item \resultReachPerm{}{_1} \\
    This case is critical for region encapsulation. Here we need to
    argue that $\Addr'$ is not reachable from $r_1$ because its type
    $\tau$ is bounded by $\klin_{n-1}$ according to the fourth
    premise. We conclude with Lemma~\ref{lemma:containment}.
    %%% need lemma
  \item \resultFrame{}{'}{_1}
    Immediate
  \item \resultImmutables{}{'}{_1} \\
    Immediate
  \item \resultMutables{}{'}{_1} \\
    Immediate
  % \item \resultSuspended{}{}{'}
  \item \resultResources{}{'}{_1} \\
    Immediate
  \item \resultThinAir{}{_1} \\
    Immediate
  \end{enumerate}

  %%%%%%%%%%%%%%%%%%%%%%%%%%%%%%%%%%%%%%%%%%%%%%%%%%%%%%%%%%%%%%%%%%%%%%%%%%%%%%%%
  \newpage{}
  \textbf{Case $e$ of}
  \lstsemrule{screateanf}

  We need to invert the corresponding rule
  \begin{mathpar}
    \ruleSDCreate
  \end{mathpar}

  It is sufficient to show that there is some $\SE_1 = \SE(\ell_1 :
  \tapp\tres\tau)$ such that  
  $\Store_1$, $\Perm_1$, and $r_1 = \ell_1$ fulfill the following requirements.
  \begin{enumerate}[({R}1)]
  \item \resultOk{}{_1}
  \item \resultEnv{}{_1} \\
    For the last item, we need to show that $\SE (\ell_1) :
    \tapp\tres\tau$, but this follows from the setting of $w$ to a
    resource storable in the semantics.
  \item $\SE_1 \vdash r_1 : \tapp\tres\tau$ \\
    Immediate from the discussion of the preceding case
  \item \resultPermDom{}{_1} \\
    Follows from the assumption on $\Perm$ and for $\ell_1$ from the
    allocation of the resource.
  \item \resultReachPerm{}{_1} \\
    Immediate from the assignment to $\Perm_1$.
  \item \resultFrame{}{}{_1} 
    Obvious as no existing location is changed.
  \item \resultImmutables{}{}{_1} \\
    Obvious as no existing location has changed and no permission is withdrawn.
  \item \resultMutables{}{}{_1} \\
    Obvious as no existing location has changed and no permission is
    withdrawn.
    % \item \resultSuspended{}{}{'}
  \item \resultResources{}{}{_1} \\
    By the constraint on $\E$ in the \TirName{Create} rule, $\Active\VEnv = \emptyset$.
  \item \resultThinAir{}{_1} \\
    Immediate
  \end{enumerate}
  
  \newpage
  \textbf{Case $e$ of}
  \lstsemrule{sdestroyanf}

  We need to invert rule \TirName{Destroy}.
  \begin{mathpar}
    \ruleSDDestroy
  \end{mathpar}
  It is sufficient to show that $\SE_1 = \SE$,
  $\Store_1$, $\Perm_1$, and $r_1 = ()$ fulfill the following
  requirements.
  \begin{enumerate}[({R}1)]
  \item \resultOk{}{_1}
  \item \resultEnv{}{_1} \\
    Immediate: $\ell$ was updated to void, which has any type.
  \item $\SE_1 \vdash () : \tunit$
  \item \resultPermDom{}{_1}\\
    By assumption on $\Perm$ and because $\Loc$ was removed.
  \item \resultReachPerm{}{_1} \\
    Immediate because the reach set is empty
  \item \resultFrame{}{}{_1} 
    Only $\Store (\Loc)$ was changed, which is not reachable from the frame.
  \item \resultImmutables{}{}{_1} \\
    Immediate because we updated (destroyed) a resource (in $\Active\VEnv$).
  \item \resultMutables{}{}{_1} \\
    Immediate because we updated (destroyed) a resource (in $\Active\VEnv$).
  % \item \resultSuspended{}{}{'}
  \item \resultResources{}{}{_1} \\
    By the constraint on $\E$, $\Loc$ was the only resource passed to
    this invocation of eval. The claimed condition holds as $\Loc$ was
    removed from $\Perm_1$ and the location's contents cleared.
  \item \resultThinAir{}{_1}
    \\ Immediate
  \end{enumerate}

  \newpage
  \textbf{Case $e$ of}
  \lstsemrule{var}

  We need to invert rule \TirName{Var}.
  \begin{mathpar}
    \ruleSDIVar
  \end{mathpar}

  We establish that the claims hold for
  $\Store' = \Store$, $\Perm' = \Perm$, $r = \VEnv (x)$, and $\SE' = \SE$.
  \begin{enumerate}[({R}1)]
  \item \resultOk{}{}
  \item \resultEnv{}{}
    \\ Immediate by reflexivity and assumption.
  \item $\SE \vdash r : \tau$
    \\ Immediate by assumption~\ref{item:32}.
  \item \resultPermDom{}{}
    \\ Immediate by assumption~\ref{item:11}
  \item \resultReachPerm{}{}
    \\ Immediate
  \item \resultFrame{}{}{}
    Immediate as permissions and store stay the same.
  \item \resultImmutables{}{}{}
    \\ Immediate
  \item \resultMutables{}{}{}
    \\Immediate
  % \item \resultSuspended{}{}{'}
  \item \resultResources{}{}{}
    \\ As $\Perm$ remains the same, a linear resource in $x$ is
    returned untouched.
  \item \resultThinAir{}{}
    \\ Immediate
  \end{enumerate}

  \newpage
  \textbf{Case $e$ of}
  \lstsemrule{const}

  We need to invert rule \TirName{Const}.
  \begin{mathpar}
    \ruleSDConst
  \end{mathpar}

  We need to establish the claims for $\Store' = \Store$,
  $\Perm'=\Perm$, $r' = c$, and $\SE' = \SE$:
  \begin{enumerate}[({R}1)]
  \item \resultOk{}{}
  \item \resultEnv{}{} \\
    By assumption~\ref{item:33}.
  \item $\SE \vdash c : \CType c$
    \\by result typing.
  \item \resultPermDom{}{}
    \\ By assumption~\ref{item:11}.
  \item \resultReachPerm{}{}
    \\ As $\Reach\Store c = \emptyset$.
  \item \resultFrame{}{}{}
    \\ Immediate
  \item \resultImmutables{}{}{}
    \\ Immediate
  \item \resultMutables{}{}{}
  % \item \resultSuspended{}{}{'}
  \item \resultResources{}{}{}
    \\ Immediate as $\Active\VEnv = \emptyset$.
  \item \resultThinAir{}{}
    \\ Immediate
  \end{enumerate}

\end{proof}

%%% Local Variables:
%%% mode: latex
%%% TeX-master: "main"
%%% End:

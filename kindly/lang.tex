\section{The \lang language}

We now present a formalization of the \lang language featuring
a minimal ML language with let-polymorphism, abstract types, pairs, and our novel
linear type system with borrows which uses kinds and qualified types.
For ease of presentation,
our formalization makes the following simplifications.
First, we only consider matching on pairs, instead of arbitrary algebraic
datatypes. We consider separate operators for borrowing and reborrowing (taking
the borrow of a borrow). We consider regions to be fully annotated
and present a set of rules to infer their placement.
Finally we emulates region variables
as a series of nesting indices.

In the rest of this section, we present the syntax (\cref{syntax}),
a linearity-aware semantics (\cref{sem}), syntax directed typing rules
(\cref{sdtyping}) and automatic rules for region annotations (\cref{regionannot}).

\subsection{Syntax}
\label{syntax}

Let us first define our conventions and meta-syntactic variables.
Expressions, types and kinds are noted respectively $e$, $\tau$, and $k$.
Variables, type variables and kind variables are noted
respectively $x$, $\tvar$ and $\kvar$. Types and kind schemes
are noted $\schm$ and $\kschm$.
In the rest of the text, we will write lists with a bar and an index, for
instance $\Multi[i]{\tau}$.
Kind constants are noted in bold capital letters, such as $\kaff$ for affine.

The syntax of \lang is presented in \cref{grammar} and follows
traditional ML languages with let bindings.
Borrows, noted $\borrow{e}$, are annotated with a specification indicating
if a borrow is immutable ($\IBORROW$) or mutable ($\MBORROW$).
We also distinguish reborrows, which are noted $\reborrow{e}$.
Match are indexed with a specification which can be either $\&^b$
(which $b$ a borrow specification) or $id$, indicating if the match
should operate on borrows or not.
Regions, noted $\region{x}{e}$ are annotated with their nesting ($n$)
and the variable whose borrow they enclose ($x$).
%
$\tapp{t}{\Multi{\tau}}$ denotes a type application where
$\T t$ is an abstract type constructor.
Arrow and borrow types are annotated with a kind, noted $\tau \tarr{k} \tau$
and $\borrowty{k}{\tau}$ respectively.
%
Kinds can be either a kind variable $\kvar$, or constants
(linear $\klin$, affine $\kaff$ or unrestricted $\kun$) indexed
by a nesting level represented by a member of $N \cup \{\infty\}$.
Type schemes and kind schemes are guarded by a set of constraints.
%
Finally, constraints are composed of a list of kind inequalities.


\TODO{Do we explain environments here, or push that for typing, later on ?}

Environments are composed of value and type bindings, along with type
declarations. Type declarations, of the form
$\tydecl{t}{\kschm}{K}{\tau}$, introduce both a new type constructor $\T{t}$ and
a data constructor $K$.

\begin{figure*}[t]
  \centering
  \begin{subfigure}[t]{0.45\linewidth}
\begin{align*}
  \htag{Expressions}
  e ::=&\ c \mid x \mid \app{e}{e'} \mid \lam{x}{e} \mid \letin{x}{e}{e'}\\
  % |&\ \fix{e}\tag{Fixpoint}\\
  |&\ \introPair{e}{e'} \mid \matchin{x,y}{e}{e'}\tag{Pairs}\\
  |&\ \region{x}{e}\tag{Region}\\
  |&\ \borrow{x}\tag{Borrows}\\
  % |&\ \introK{K}{e}\tag{Data Constructor introduction}\\
  % |&\ \elimK{K}{e}\tag{Data Constructor elimination}\\
  \htag{Environments}
  \E ::=&\ b^*\\
  b ::=&\ \bvar{x}{\schm}\tag{Variable bindings}\\
  |&\ \bvar{\tvar}{\kschm}\tag{Type bindings}\\
  % |&\ \tydecl{t}{\kschm}{K}{\tau}\tag{Type Declaration}\\
  \htag{Type and kind Schemes}
  \schm ::=&\ \forall\kvar^*\forall\bvar{\alpha}{k}^*.(\qual{C}{\tau})\\
  \kschm ::=&\ \forall\kvar^*.(\qual{C}{k_i^* \karr k})\\
\end{align*}
\end{subfigure}\hfill
\begin{subfigure}[t]{0.5\linewidth}
\begin{align*}
  \htag{Type Expressions}
  \tau ::=&\ \tvar\tag{Type variables}\\
  |&\ \tau\tarr{k}\tau\tag{Function types}\\
  |&\ \tapp{t}{\tau^*}\tag{Type constructors}\\
  |&\ \borrow{\tau}\tag{Borrowed Type}\\
  \htag{Kinds}
  k ::=&\ \kvar\tag{Kind variables}\\
  |&\ \klin_n\tag{Linear kind}\\
  |&\ \kaff_n\tag{Affine kind}\\
  |&\ \kun_n\tag{Unrestricted kind}\\
  n \in&\ \mathbb N \cup \{\infty\}\\
  \htag{Constraints}
  C ::= (\Cleq{k}{k'})^*
\end{align*}
\end{subfigure}

%%% Local Variables:
%%% mode: latex
%%% TeX-master: "main"
%%% End:

  \caption{Syntax}
  \label{grammar}
\end{figure*}
\begin{figure}[!tbp]
%   \begin{align*}
% \end{align*}
% \begin{minipage}[t]{0.49\linewidth}
  \begin{align*}
    \htag{Elaborated expressions}
    v ::=~& x \mid \ivar x {\Multi k} {\Multi\tau} \mid \lam[k]{x}{e} \mid \introPair[k]{v}{v'}\\
    e ::=~& x \mid \ivar x {\Multi k} {\Multi\tau} \mid \lam[k]{x}{e} \mid \introPair[k]{e}{e'}\\
    \mid~& \matchin{x,y}{e}{e'} \tag{Tagged pairs}\\
    \mid~& \letin{x}{e}{e'}\\
    \mid~& \letin{\bvar x \schm}{v}{e'}\\
    \mid~& \iapp{\Sp}{e}{e'} \tag{El. application}\\
    \mid~& \region{\Sone x\BORROW}{e}\tag{Region}\\
    \mid~& \borrow{x} \mid \reborrow{x}\tag{Borrows}\\
    \mid~& \create \mid \observe \mid \update \mid \destroy \tag{Resources}\\
    % e &::= \dots \\
    % &\mid \ilam{\Multi\kvar}{\Multi{\tvar : k}}Ckx{\tau}e \tag{El. poly abstraction} \\
    % &\mid \imlam kx{\tau}e \tag{El. mono abstraction} \\
    % &\mid \ivar x {\Multi k} {\Multi\tau} \tag{Instantiation} \\
    % &\mid \introPair[k]{e}{e} \tag{Tagged pairs}\\
    % &\mid \iapp{\Sp}{e}{e} \tag{El. application}\\
    \htag{Elaborated A-normal expressions}
    % TODO: what about constructors applied to values?
    X ::=~& x \mid \ivar x {\Multi k} {\Multi\tau}\\
    v ::=~& c \mid X \mid \lam[k]{x}{e} \mid \introPair[k]{X}{X'}\\
    E ::=~& c \mid X \mid \lam[k]{x}{e} \mid \introPair[k]{x}{x'}\\
    \mid~& \borrow{x} \mid \reborrow{x}\tag{Borrows}\\
    \mid~& \create \mid \observe \mid \update \mid \destroy \tag{Resources}\\
    \mid~& \app{x}{x'} \tag{El. application}\\
    e ::=~& \letin{x}{E}{e'}\\
    \mid~& \letin{\bvar x \schm}{v}{e'}\\
    \mid~& \matchin{x,y}{z}{e} \tag{Tagged pairs}\\
    \mid~& \region{\Sone x\BORROW}{e}\tag{Region}\\
    \htag{Environment}
    \Addr ::=~& \Multi\IBORROW\Multi\MBORROW\Loc \tag{Locations}
    % \\           &\mid \borrow{\Addr} \tag{Borrows}
    \\
    \Perm ::=~& \{\} \mid \Perm + \Addr \tag{Permissions}
    \\
    r ::=~& \Addr \mid c \tag{Results}\\
    \VEnv ::=~& \Eempty \mid \VEnv( x \mapsto r) \tag{Enviroments} \\
%   \end{align*}
% \end{minipage}
% \hfill
% \begin{minipage}[t]{0.49\linewidth}
%   \begin{align*}
    \htag{Storables}
    w ::=~& \StPClosure \VEnv {\Multi\kvar} C k x e \tag{Poly Closures}\\
    \mid~& \StClosure \VEnv k x e \tag{Closures} \\
    \mid~& \StPair k r r \tag{Pairs} \\
    \mid~& \StRes r \tag{Resources} \\
    \mid~& \StFreed \tag{Freed Resource}
    \\
    \htag{Store}
    \Store ::=~& \Eempty \mid \Store( \Loc \mapsto w)
    \\
    \htag{Splittings}
    \Sp ::=~& \Multi{\SpBoth \mid \SpBorrow \mid \SpLeft \mid \SpRight \mid \SpSusp \mid \SpSuspB}
  \end{align*}
% \end{minipage}
\caption{Syntax of internal language}
\label{fig:syntax-internal-language}
\end{figure}

%%% Local Variables:
%%% mode: latex
%%% TeX-master: "main"
%%% End:


\clearpage
\subsection{Semantics}
\label{sem}

\begin{figure*}[ht]
  % \begin{mathpar}
  \inferrule[Lam]
  { j \fresh }
  { \ered{\closure{j}{x}{e}}{\emptyset}{\lam{x}{e}}{\closure{j}{x}{e}} }
  \and
  \inferrule[App]
  { \ered{I}{E}{f}{\closure{j}{x}{e_f}} \\
    \ered{I'}{E'}{e}{v} \\
    \ered{I''}{E''}{\subst{x}{v}{e_f}}{v_f}
  }
  { \ered{I,I',I''}{E,E',E'',\closure{j}{x}{e_f}}{\app{f}{e}}{v_f} }
  \and
  \inferrule[Let]
  { \ered{I}{E}{e}{v} \\
    \ered{I'}{E'}{\subst{x}{v}{e'}}{v'}
  }
  { \ered{I,I'}{E,E'}{\letin{x}{e}{e'}}{v'} }
\end{mathpar}
%%% Local Variables:
%%% mode: latex
%%% TeX-master: "main"
%%% End:

    \begin{mathpar}
    \inferrule{}{ \Sigma, \rho \vdash c \Downarrow \Sigma, c}

    \inferrule{}{\Sigma, \rho \vdash x \Downarrow \Sigma, \rho(x)}

    \inferrule{
      \Sigma, \rho \vdash e \Downarrow \Loc \\
      \Sigma (\Loc) = (\rho, \ilam {\kvar^*}{\tvar^*}Ckx{e}) \\\\
      \Sigma' =  \entail C {(\kaff \le k)} \Rightarrow
      \Sigma[\Loc\mapsto\blob] ; \Sigma \\
      \Loc'\notin\Dom\Sigma'  \\
      \Sigma'' = \Sigma[\Loc' \mapsto (\rho,\lam[{k[\kvar^*\mapsto k^*]}]xe) ]
    }{\Sigma, \rho \vdash  \ivar e{k^*}{\tau^*}  \Downarrow \Sigma'', \Loc'
    }

    \inferrule{
      \Loc\notin\Dom\Sigma \\
      \Sigma' = \Sigma[\Loc \mapsto (\rho, \ilam  {\kvar^*}{\tvar^*}Ck xe)]
    }{
      \Sigma, \rho \vdash \ilam  {\kvar^*}{\tvar^*}Ck xe \Downarrow \Sigma', \Loc
    }
    
    \inferrule{
      \Sigma, \rho \vdash e \Downarrow \Sigma', \Loc \\
      \Sigma' (\Loc) = (\rho'',\lam[k]{x}{e''}) \\
      \Sigma'' = \entail {} {(\kaff \le k)} \Rightarrow \Sigma'[\Loc\mapsto\blob];\Sigma'\\
      \Sigma'', \rho \vdash e' \Downarrow \Sigma''', r' \\
      \Sigma''', \rho''[x\mapsto r'] \vdash e'' \Downarrow \Sigma'''', r
    }{\Sigma, \rho \vdash \app{e}{e'} \Downarrow \Sigma'''', r}

    \inferrule{
      \Sigma, \rho \vdash e \Downarrow \Sigma', r \\
      \Sigma', \rho[x \mapsto r] \vdash e' \Downarrow \Sigma'', r'
    }{
      \Sigma, \rho \vdash \letin{x}{e}{e'} \Downarrow \Sigma'', r'
    }

    \inferrule{
      \Sigma, \rho \vdash e \Downarrow \Sigma', r \\
      \Sigma', \rho \vdash e' \Downarrow \Sigma'', r' \\
      \Loc\notin\Dom{\Sigma''} \\
      \Sigma''' = \Sigma''[\Loc \mapsto (r, r')]
    }{
      \Sigma, \rho \vdash \introPair{e}{e'} \Downarrow \Sigma''', \Loc
    }

    \inferrule{
      \Sigma, \rho \vdash e \Downarrow \Sigma', \Loc \\
      \Sigma' (\Loc) = (r, r') \\
      \Sigma', \rho[x,y \mapsto r, r'] \vdash e' \Downarrow \Sigma'', r''
    }{
      \Sigma, \rho \vdash \matchin{x,y}{e}{e'} \Downarrow  \Sigma'', r''
    }

    \inferrule{
      \rho (x) = \alpha \\
      \Sigma (\alpha) = W \\
      (\Sigma\setminus\alpha)[\borrow{\alpha} \mapsto W], \rho \vdash e
      \Downarrow \Sigma', r \\
      \Sigma'' = (\Sigma' \setminus\borrow{\alpha})[\alpha \mapsto W]
    }{
      \Sigma, \rho \vdash \region{x}{e} \Downarrow \Sigma', r
    }

    \inferrule{\rho (x) = \alpha}{
      \Sigma, \rho \vdash \borrow{x} \Downarrow \Sigma, \borrow\alpha
    }
    \\
    \inferrule{
      \Sigma, \rho \vdash e \Downarrow \Sigma', r\\
      \Loc\notin \Dom\Sigma' }{
      \Sigma,\rho \vdash \create e \Downarrow \Sigma'[\Loc \mapsto \rss{r}], \Loc
    }

    \inferrule{
      \Sigma, \rho \vdash e \Downarrow \Sigma', \Loc \\
      \Sigma' (\Loc) = \rss{r}
    }{
      \Sigma, \rho \vdash \destroy e \Downarrow \Sigma'[\Loc\mapsto \blob], ()
    }

    \inferrule{
      \Sigma, \rho \vdash e \Downarrow \Sigma', \borrow[i]\alpha \\
      \Sigma' (\borrow[i]\alpha) = \rss{r}
    }{
      \Sigma, \rho \vdash \observe e \Downarrow \Sigma', r
    }

    \inferrule{
      \Sigma, \rho \vdash e \Downarrow \Sigma', \borrow[m]\alpha \\
      \Sigma', \rho \vdash e' \Downarrow \Sigma'', r' \\
      \Sigma'' (\borrow[m]\alpha) = \rss{r} \\
      \Sigma''' = \Sigma''[\borrow[m]\alpha \mapsto r']
    }{
      \Sigma, \rho \vdash \update e {e'} \Downarrow \Sigma''', ()
    }

  \end{mathpar}
  \caption{Reduction rules -- $\Sigma, \rho \vdash e \Downarrow
    \Sigma', r$ }
  \label{fig:reduction}
\end{figure*}


%%% Local Variables:
%%% mode: latex
%%% TeX-master: "main"
%%% End:

\begin{figure}
  \lstsemrule{varinst}
  \medskip
  \lstsemrule{sapp}
  \medskip
  \lstsemrule{spair}
  \medskip
  \lstsemrule{sregion}
  \caption{Big-step Interpretation}
\end{figure}


%%% Local Variables:
%%% mode: latex
%%% TeX-master: "main"
%%% End:


\clearpage
\subsection{Typing}
\label{sdtyping}

\subsubsection{Bindings and Weakening}


\subsubsection{Regions and Borrows}


\begin{figure}
  \centering
  \begin{subfigure}{0.4\linewidth}
    \begin{mathpar}
      \ruleSDRegion
    \end{mathpar}
    \caption{The {\sc Region} rule}
  \end{subfigure}\hfill
  \begin{subfigure}{0.6\linewidth}
    \centering
    \begin{tabular}
      {@{}>{$}r<{$}@{ $\Lleftarrow$ }
      >{$}c<{$}@{ $\rightsquigarrow_n^{x}$ }
      >{$}l<{$}
      r}

      {(\kun_n\lk k\lk\kun_\infty)}
      &{\svar[\IBORROW]{x}{\tau}^n}
      &{\bvar{\borrow[\IBORROW]{x}}{\borrowty[\IBORROW] k{\tau}}}
      &Immut\\

      (\kaff_n\lk k\lk\kaff_\infty)
      &\svar[\MBORROW]{x}{\tau}^n
      &\bvar{\borrow[\MBORROW]{x}}{\borrowty[\MBORROW] k{\tau}}
      &Mut
    \end{tabular}
    \caption{Borrowing rules for bindings}
  \end{subfigure}
\end{figure}


\subsubsection{Copying and Splitting}


\begin{figure}
  \centering
  \begin{subfigure}{0.35\linewidth}
    \begin{mathpar}
      \ruleSDPair
    \end{mathpar}
    \caption{The {\sc Pair} rule}
  \end{subfigure}\hfill
  \begin{subfigure}{0.6\linewidth}
    \centering
    \begin{tabular}
      {@{}>{$}r<{$}@{ $\Lleftarrow$ }
      >{$}c<{$}@{ $=$ }
      >{$}c<{$}@{ $\ltimes$ }
      >{$}c<{$}r}
      
      \Cleq{\schm}{\kun_\infty}
      &\bvar{x}{\schm}&\bvar{x}{\schm}&\bvar{x}{\schm}
      &Both\\[2mm]

      {\Cempty}&{\bvar{x}{\schm}}&{\bvar{x}{\schm}}&{\bnone}
      &Left\\
      {\Cempty}&{\bvar{x}{\schm}}&{\bnone}&{\bvar{x}{\schm}}
      &Right\\[2mm]

      {\Cempty}&{\bvar x \schm}&{\svar x \schm^n}&{\bvar x \schm}
      &Susp\\

      {\Cempty}&
      {\bvar{\borrow x} \schm}&{\svar[\IBORROW] x \schm^n}&{\bvar{\borrow x} \schm}
      &SuspB\\[2mm]

      {\Cempty}&
      {\bvar{\borrow[i]{x}}{\schm}}&
      {\bvar{\borrow[i]{x}}{\schm}}&{\bvar{\borrow[i]{x}}{\schm}}
      &Borrow\\

    \end{tabular}
    \caption{Splitting rules for bindings}
  \end{subfigure}
\end{figure}



\subsubsection{Constraints}




\clearpage

\subsection{Annotating regions}
\label{regionannot}

So far, all \lang programs have been fully annotated with regions information.
We now show how to infer these regions annotations based on
optionally-annotated programs.
First, we extend the region annotation to $\region{S}{E}$ where $S$ is
a set of variables. This annotation, defined below, is equivalent to nested
region annotations for each individual variable.

\begin{align*}
  \region{x;S}{e} &= \region{x}{\region{S}{e}}& \region{\emptyset}{e} &= e\\
\end{align*}

\Cref{fig:region-annotation} define a rewriting relation $\RannotT{e}{e'}$
which indicates that an optionally annotated term $e$ can be written
in a fully annotated term $e'$.
Through the rule \textsc{Rewrite-Top}, this is defined
in term of an inductively defined relation
$\Rannot{e}{e'}{S}$ where $n$ is the current nesting and $S$ is a set of
variable that are not enclosed in a region yet.
The base cases are constants, variables and borrows.
The general idea is to start from the leafs of the syntax tree, create a
region for each borrow, and enlarge the region as much as possible.
This is implemented by a depth-first walk of the syntax
tree which collects each variable that has a corresponding borrow.
At each step, it rewrites the inner subterms,
consider which borrow must be enclosed by a region now, and
return the others for later enclosing. Binders force immediate
enclosing of the bound variables, as demonstrated in rule \textsc{Rewrite-Lam}.
For nodes with multiple children, we
use a scope merge operator to decide if regions should be placed and where.
This is shown in rule \textsc{Rewrite-Pair}.
The merge operator, written $\getBorrows{B_l}{B_r}{(S_l,S,S_r)}$, takes
the sets $B_l$ and $B_r$ returned by rewriting the subterms
and returns three sets: $S_l$ and $S_r$ indicates the variables
that should be immediately enclosed by a region on the left and right
subterms and $S$ indicates the set of the yet-to-be-enclosed variables.
As an example, the rule \textsc{AnnotRegion-MutLeft} is applied
when there is an immutable borrow and a mutable borrow. In that case, a
region is created to enclose the immutable borrow, while the mutable
borrow is left to be closed later. This is coherent with the rules
for environment splitting and suspended bindings from \cref{sdtyping}.
%
Explicitly annotated regions are handled specially through
rule \textsc{Rewrite-Region}. In that case, we assume that all inner
borrows should be enclosed immediately.

\begin{figure*}[!hbt]
  \centering
  \begin{mathpar}
  \inferrule[AnnotRegion-Empty]{}{
    \getBorrows{\Sempty}{\Sempty}{\Sempty,\Sempty,\Sempty}
  }

  \inferrule[AnnotRegion-Nonempty]{
    \getBorrows{B_1}{B_2}{S_1,S,S_2}\\
    \getBorrows{b_1}{b_2}{S'_1,S',S'_2}
  }{
    \getBorrows{B_1;b_1}{B_2;b_2}
    {S_1\Sunion S'_1,S\Sunion S',S_2\Sunion S'_2}
  }
\end{mathpar}
\hrulefill
\begin{mathpar}
  \inferrule[AnnotRegion-Left]{}{
    \getBorrows
    {\Sone{x}{b}}
    {\Cempty}
    {\Cempty,\Sone{x}{b},\Cempty}
  }
  
  \inferrule[AnnotRegion-Right]{}{
    \getBorrows
    {\Cempty}
    {\Sone{x}{b}}
    {\Cempty,\Sone{x}{b},\Cempty}
  }
  
  \inferrule[AnnotRegion-Immut]{}{
    \getBorrows
    {\Sone{x}{\IBORROW}}
    {\Sone{x}{\IBORROW}}
    {\Cempty,\Sone{x}{\IBORROW},\Cempty}
  }
  
  \inferrule[AnnotRegion-MutLeft]{}{
    \getBorrows
    {\Sone{x}{\IBORROW}}
    {\Sone{x}{\MBORROW}}
    {\Sone{x}{\IBORROW},\Sone{x}{\MBORROW},\Cempty}
  }
  
  \inferrule[AnnotRegion-MutRight]{}{
    \getBorrows
    {\Sone{x}{\MBORROW}}
    {\Sone{x}{b}}
    {\Sone{x}{\MBORROW},\Cempty,\Sone{x}{b}}
  }
  
\end{mathpar}
\hrulefill
\begin{mathpar}
  \inferrule
  { e = \borrow{x} \mid \reborrow{x}}
  { \Rannot{e}{e}{\Sone{x}{b}} }

  \inferrule{e = c\ |\ x}
  { \Rannot{e}{e}{\Sempty} }

  \inferrule
  { \forall i,\ \Rannot{e_i}{e'_i}{B_i} \\
    \getBorrows{B_1}{B_2}{S_1,S,S_2}
  }
  { \Rannot{\app{e_1}{e_2}}{\app{\region{S_1}{e'_1}}{\region{S_2}{e'_2}}}{S} }

  \inferrule
  { \forall i,\ \Rannot{e_i}{e'_i}{B_i} \\
    \getBorrows{B_1}{(B_2\Sdel{x})}{S_1,S,S_2} \\
    S'_2 = S_2\Sunion B_2\Sonly{x}
  }
  { \Rannot
    {\letin{x}{e_1}{e_2}}
    {\letin{x}{\region{S_1}{e'_1}}{\region{S'_2}{e'_2}}}{S} }
  

  \inferrule
  { \forall i,\ \Rannot{e_i}{e'_i}{B_i} \\
    \getBorrows{B_1}{(B_2\Sdel{x,y})}{S_1,S,S_2} \\
    S'_2 = S_2\Sunion B_2\Sonly{x,y}
  }
  { \Rannot
    {\matchin{x,y}{e_1}{e_2}}
    {\matchin{x,y}{\region{S_1}{e'_1}}{\region{S'_2}{e'_2}}}{S} }

  \inferrule[Rewrite-Region]
  { \Rannot{e}{e'}{B} }
  { \Rannot{\regionS{e}}{\region{B}{e'}}{\Sempty} }

  \inferrule[Rewrite-Lam]
  { \Rannot{e}{e'}{B} \\
  }
  { \Rannot{\lam{x}{e}}{\lam{x}{\region{B}{e'}}}{\Sempty} }

  \inferrule[Rewrite-Pair]
  { \forall i,\ \Rannot{e_i}{e'_i}{B_i} \\
    \getBorrows{B_1}{B_2}{S_1,S,S_2}
  }
  { \Rannot
    {\introPair{e_1}{e_2}}
    {\introPair{\region{S_1}{e'_1}}{\region{S_2}{e'_2}}}
    {S} }

  \inferrule[Rewrite-Top]
  { \Rannot[1]{e}{e'}{S} }
  { \RannotT{e}{\region[1]{S}{e'}} }
\end{mathpar}

%%% Local Variables:
%%% mode: latex
%%% TeX-master: "main"
%%% End:

  \caption{Automatic region annotation --- $\RannotT{e}{e'}$}
  \label{fig:region-annotation}
\end{figure*}

%%% Local Variables:
%%% mode: latex
%%% TeX-master: "main"
%%% End:

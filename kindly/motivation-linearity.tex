\subsection{A session on linearity}
\label{sec:session-linearity}

Session typing \cite{Honda1993,DBLP:conf/esop/HondaVK98} is a type
discipline for checking protocols statically. A session type is
ascribed to a communication channel and describes
a sequence of interactions. For instance, the type
\lstinline{int!int!int?end} specifies a protocol
where the program must send two integers, receive an integer, and then
close the channel.
%
In this context, channels must be used linearly, because every use of
a channel ``consumes'' one interaction and  changes the type of the
channel. That is, after sending two integers on the channel,
the remaining channel has type \lstinline{int?end}.
%
Here are some standard type operators for session types \SType:
\begin{center}
  \begin{tabular}[t]{rl}
    $\tau ! \SType$ & Send a value of type $\tau$ then continue with $\SType$.\\
    $\tau ? \SType$& Receive a value of type $\tau$ then continue with $\SType$.\\
    $\SType \oplus \SType'$& Internal choice between protocols $\SType$ and $\SType'$.\\
    $\SType \operatorname{\&} \SType'$
                    & Offer a choice between protocols $\SType$ and $\SType'$.
  \end{tabular}
\end{center}

\citet{DBLP:journals/jfp/Padovani17}  has shown how to encode this
style of session typing in ML-like languages, but his implementation
downgrades linearity to a run-time check for affinity. Building on that
encoding we can provide a safe API in \lang that statically enforces
linear handling of channels:
%
\begin{lstlisting}
type 'S st : lin (*@\label{line:session1}*)
val receive: ('a ? 'S) st -> 'a * 'S st (*@\label{line:session2}*)
val send : ('a ! 'S) st -> 'a -{lin}> 'S st
val create : unit -> 'S st * (dual 'S) st
val close : end st -> unit
\end{lstlisting}
% val send : ('a:'k). 'a -> ('a ! 'S) st -{'k}> 'S st

Line~\ref{line:session1} introduces a parameterized abstract
type \lstinline{st} which is linear as indicated
by its kind \lstinline{lin}. Its low-level implementation would wrap a
handle for a socket, for example. The \lstinline{receive}  operation
in Line~\ref{line:session2} takes a channel that is ready to receive a
value of type \lstinline{'a} and returns a pair of the value a the
channel at its residual type \lstinline{'S}. It does not matter
whether \lstinline{'a} is restricted to be linear, in fact
\lstinline{receive} is polymorphic in the kind of \lstinline{'a}.
This kind polymorphism is the
default if no constraints are specified.
%
The \lstinline{send} operation takes a linear channel and returns a
single-use function that takes a value of type \lstinline{'a} suitable
for sending and returns the channel with updated type.
%
The \lstinline{create} operation returns a pair of channel
endpoints. They follow dual communication protocols, where the
\lstinline{dual} operator swaps sending and receiving operations.
%
Finally, \lstinline{close} closes the channel.

In \cref{fig:sessiontype} we show how to use these primitives to
implement client and server for an addition service.
No linearity annotations are needed in the code, as all linearity
properties can be inferred from the linearity of the \texttt{st} type.

The inferred type of the server, \lstinline{add_service}, is
\lstinline{(int ! int ! int ? end) st -> unit}.
The client operates by sending two messages
and receiving the result.
This code is polymorphic in both argument and return types, so it could
be used with any binary operator.
%
\begin{figure}[!h]
  \begin{subfigure}[t]{\linewidth}
\begin{lstlisting}
let add_service ep =
  let x, ep = receive ep in
  let y, ep = receive ep in
  let ep = send ep (x + y) in
  close ep
# add_service : (int ! int ! int ? end) st -> unit
\end{lstlisting}
    \vspace{-5pt}
    \caption{Addition server}
  \end{subfigure}

  \begin{subfigure}[t]{\linewidth}
\begin{lstlisting}
let op_client ep x y =
  let ep = send ep x in
  let ep = send ep y in
  let result, ep = receive ep in
  close ep;
  result
# op_client : ('a_1 ? 'a_2 ? 'b ! end) st -> 'a_1 -{lin}> 'a_2 -{lin}> 'b
\end{lstlisting}
% The actual answer from the typechecker looks like this:
% # ('m2:^k2), ('m1:^k1).
% # (^k1 < ^k) & (lin < ^k) =>
% # ('m1 ?'m2 ? 'result ! end) st -> 'm1 -{lin}> 'm2 -{^k}> 'result
    \vspace{-5pt}
    \caption{Binary operators client}
  \end{subfigure}
  \vspace{-10pt}
  \caption{Corresponding session type programs in \lang}
  \label{fig:sessiontype}
\end{figure}
%
Moreover, the \lstinline/op_client/ function can be partially applied
to a channel. Since the closure returned by such a partial application
captures the channel, it can only be used once.  This restriction is
reflected by the arrow of kind \lstinline{lin}, \lstinline/-{lin}>/,
which is the type of a single-use function.
The general form of
arrow types in \lang is \lstinline/-{k}>/, where \lstinline/kk/ is a
kind that restricts the number of uses of the function.  For
convenience, we shorten \lstinline/-{un}>/ to \lstinline/->/.  \lang
infers the
single-use property of the arrows without any user
annotation. In fact, the only difference between the
code presented here and Padovani's examples
\cite{DBLP:journals/jfp/Padovani17} is the kind annotation on the
type definition of \lstinline/st/.

To run client and server, we can create a channel and apply
\lstinline{add_service} to one end and \lstinline{op_client} to the other.
Failure to consume either channel endpoints (\lstinline/a/ or \lstinline/b/)
would result in a type error.
\begin{lstlisting}
let main () =
  let (a, b) = create () in
  fork add_service a;
  op_client b 1 2
# main : unit -> int
\end{lstlisting}


% \begin{figure}
%   \centering
%   \begin{lstlisting}
% val select : ('S st -{'k}> 'a) -> 'a ot -{'k}> ('S dual) st
% val branch : 'm it -> 'm
%   \end{lstlisting}
%   \caption{Session primitives}
%   \label{api:session}
% \end{figure}


%%% Local Variables:
%%% mode: latex
%%% TeX-master: "main"
%%% End:

\clearpage{}
\section{Metatheory}
\label{sec:metatheory}

\begin{itemize}
\item Borrow compatibility
  $\Multi\MBORROW\Multi\IBORROW \Bcompatible \BORROW$,
  \begin{mathpar}
  \inferrule{}{
    \Multi\MBORROW\Multi\IBORROW\IBORROW \Bcompatible \IBORROW
  }

  \inferrule{}{
    \Multi\MBORROW\MBORROW \Bcompatible \MBORROW
  }
  \end{mathpar}
\item Store typing $ \vdash \Store : \SE$,
  \begin{mathpar}
    \inferrule{
      (\forall \Loc \in \Dom\Store)~
      \SE \vdash \Store (\Loc) : \SE (\Loc)
    }{ \vdash \Store : \SE }
  \end{mathpar}
\item Relating storables to type schemes $\SE \vdash W : \schm$
  \begin{mathpar}
  \inferrule{
    (\exists \E)~ \SE \vdash \VEnv : \E
    \\
    \inferS{C}{\E, (x:\tau_2)}{e}{\tau_1}
  }{
    \SE \vdash (\VEnv, \ilam {\Multi\kvar}{\Multi\tvar}Ckx{e})
    : \forall\Multi\kvar\forall\Multi{\bvar{\alpha}{k}}.(\qual{C}{\tau_2\tarr{k}\tau_1})
  }
  \end{mathpar}
\item Relating storables to types $ \SE \vdash W : \tau$
  \begin{mathpar}
    \inferrule{
      (\exists \E, C)~ \SE \vdash \VEnv : \E
      \\
      \inferS{C}{\E, (x:\tau_2)}{e}{\tau_1}
      \\
      \addlin{\entail{C}{\Cleq{\E}{k}}}
    }{
      \SE \vdash (\VEnv, \lam[k]xe) : \tau_2\tarr{k}\tau_1
    }

    \inferrule{
      \SE \vdash r_1 : \tau_1 \\
      \SE \vdash r_2 : \tau_2
    }{
      \SE \vdash \introPair[k]{r_1}{r_2} : \tyPair[k]{\tau_1}{\tau_2}
    }

    \inferrule{
      \SE \vdash r : \IType{t}{\Multi\tau}
    }{
      \SE \vdash {[r]} : \tapp{t}{\Multi\tau}
    }

    \inferrule{}{
      \SE \vdash \blob : \tau
    }
  \end{mathpar}
\item Relating  results to types $ \SE \vdash r : \tau$, 
\item Relating results to type schemes $\SE \vdash r : \schm$
  \begin{mathpar}
  \inferrule{}{ \SE \vdash c : \CType{c} }

  \inferrule{}{ \SE \vdash \ell : \SE (\ell) }

  \inferrule{
    \Multi\MBORROW\Multi\IBORROW \Bcompatible \BORROW \\
    \SE \vdash \Loc  : \tau
  }{  \SE \vdash
    \Loc\Multi\MBORROW\Multi\IBORROW : \borrow{\tau}}
  \end{mathpar}
\item Relating environments to contexts
  $\SE \vdash \VEnv, \Suspended\VEnv : \E$\\
  Here $\VEnv$ is the active environment and $\Suspended\VEnv$ is the
  suspended environment.
\end{itemize}
\begin{mathpar}
  \inferrule{}{\SE \vdash \Sempty, \Sempty : \Eempty}

  \inferrule{\SE \vdash \VEnv, \Suspended\VEnv : \E \\ \SE \vdash r : \schm}
  {\SE \vdash \VEnv[ x\mapsto r], \Suspended\VEnv : \E\bvar x\schm }

  \inferrule{\SE \vdash \VEnv, \Suspended\VEnv : \E \\ \SE \vdash r : \schm}
  {\SE \vdash \VEnv, \Suspended\VEnv[ x\mapsto r] : \E\svar x\schm }

  \inferrule{\SE \vdash \VEnv, \Suspended\VEnv : \E \\ \SE \vdash rb : \tau}
  {\SE \vdash \VEnv[ x\mapsto rb], \Suspended\VEnv : \E\bvar{\borrow
      x}{\borrowty k{ \tau}} }
\end{mathpar}
\clearpage{}
We write $\Addresses{\VEnv}$ for the function that extracts addresses
from the range of the variable environment.

We write $\Reach\Store\VEnv \subseteq \Dom\Store$ for the function
that computes all store addresses reachable from $\Addresses\VEnv$
assuming that $\Addresses\VEnv \subseteq \Dom\Store$. 

We write $\Writeable\SE\Loc$ to express that the type of location
$\Loc$ permits changing its content. For example,
\begin{itemize}
\item if $\SE (\Loc) =
  \forall\Multi\kvar\forall\Multi{\bvar{\alpha}{k}}.(\qual{C}{\tau_2\tarr{k}\tau_1})$,
  then $\Loc$ is not writeable.
\item if $\SE (\Loc) = \tau_2\tarr{k}{\tau_1}$, then $\Loc$ is not
  writeable.
\item if $\SE (\Loc) = \tyPair{\tau_1}{\tau_2}$, then $\Loc$ is not writeable.
\item if $\SE (\Loc) = \tapp{t}{\Multi\tau}$, then $\Loc$ is
  writeable. (IT RATHER SEEMS LIKE $\Loc$'s DECENDANT IS WRITEABLE)
\item if $\SE (\Loc) = \borrow\tau$, then \dots (THAT SHOULDN'T REALLY
  BE A STORE TYPE)
\end{itemize}
\begin{theorem}
  Suppose that
  \begin{itemize}
  \item $\inferS{C}{\E}{e}{\tau}$
  \item $\SE \vdash \VEnv : \E$
  \item $\vdash \Store : \SE$
  \item $\Reach\Store{\VEnv} \subseteq \Perm \subseteq \Dom\Store$
  \item  $\VEnv'$ with $\Addresses{\VEnv'}
    \subseteq \Dom\Store$ and $\Dom\VEnv \cap \Dom{\VEnv'}=\emptyset$
  \item  $i\in\Nat$ and $\Store, \Perm, \VEnv + \VEnv' \vdash {e}
    \Downarrow^i R$ and $R\ne \TimeOut$.
  \end{itemize}
  Then,
  $\exists$ $\Store'$, $\Perm'$, $r'$, $\SE'$ such that
  \begin{itemize}
  \item
    $R = \Ok{\Store', \Perm', r'}$  
  \item $\SE \le \SE'$, $\Store \le \Store'$,
    $\vdash \Store' : \SE'$ 
  \item $\Perm' \subseteq \Dom{\Store'}$
  \item $\SE' \vdash r' : \tau$
  \item $\Reach{\Store'}{r'} \subseteq \Perm'$)
  \item For all $\Loc \in \Reach{\Store'}{\VEnv'}$ it must be that
    $\Store' (\Loc) = \Store (\Loc)$
    and $\Loc\in\Perm \Leftrightarrow \Loc\in\Perm'$ 
  \item For all $\Loc \in \Reach{\Store'}{\VEnv}$,
    if $\neg \Writeable{\SE'}\Loc $ then
    $\Store' (\Loc) = \Store (\Loc)$
    and $\Loc\in\Perm \Leftrightarrow \Loc\in\Perm'$ 
  \end{itemize}
\end{theorem}
    
\begin{proof}
  The proof is by induction on the evaluation judgment $\Store, \Perm,
  \VEnv + \VEnv' \vdash {e} \Downarrow^i R$.
  
  The base case is trivial as $\Store, \Perm, \VEnv \vdash {e}
  \Downarrow^0 \TimeOut$.

  For indexes greater than zero we have a case distinction over all applicable execution
  rules.

  \textbf{Case }$\ruleSConst$.
  The claim is immediate.

  \textbf{Case }$\ruleSVar$.
  The claim is immediate.

  \textbf{Case }$\ruleSTApp$.
  
  By assumption, we have that $\SE \vdash \Loc : \E (x)$ so that
  $\E(x) =
  \forall\Multi\kvar\forall\Multi{\bvar{\alpha}{k}}.(\qual{C}{\tau_2\tarr{k}\tau_1})$.
  All conclusions hold: $\SE \le \SE' = \SE[\Loc' \mapsto
  (\tau_2\tarr{k}\tau_1)[\Multi[i]{\kvar} \mapsto \Multi[i]{ k},
  \Multi[j]{\alpha} \mapsto \Multi[j]{t}]]$;
  $\vdash \Store' : \SE'$;
  $\Perm' \subseteq \Dom{\Store'}$;
  $\SE' \vdash \Loc' : \tau$;
  and $\Loc' \in \Perm' \Sadd{\Loc'}$.

  \textbf{Case }$\ruleSPLam$.

  By assumption $\SE \vdash     \ilam {\Multi[i]{\kvar}}{\Multi[j]{\tvar}}Ck xe
  : \schm$ where $\schm =
  \forall\Multi\kvar\forall\Multi{\bvar{\alpha}{k}}.(\qual{C}{\tau_2\tarr{k}\tau_1})$. Let
  $\SE'  = \SE[\Loc \mapsto \schm]$.
  Then $\SE \le \SE'$ and $\vdash\Store' : \SE'$.
  Moreover, $\SE' \vdash \Loc : \schm$ and $\Loc \in \Perm
  \Sadd{\Loc}$ as required.
  
  \clearpage
  \textbf{Case }$\ruleSApp$.

  By assumption $\ruleSDApp$.

  \begin{itemize}
  \item By splitting of $\E$, we have that $\VEnv = \VEnv_1 +
    \VEnv_1'$ such that  $\SE \vdash \VEnv_1 : \E_1$
  \item $\vdash \Store : \SE$
  \item $\Reach\Store{\VEnv_1} \subseteq \Reach\Store{\VEnv} \subseteq
    \Perm \subseteq \Dom\Store$ 
  \item $\Addresses{\VEnv_1'} \subseteq \Dom\Store$ and
    $\Dom{\VEnv_1} \cap \Dom{\VEnv_1'}=\emptyset$
  \item $\Store, \Perm, \VEnv_1 + \VEnv_1' \vdash {e_1}
    \Downarrow^i R_1$
  \end{itemize}
  If $R_1 = \TimeOut$, then $R = \TimeOut$, too.
  Otherwise induction yields that
  $\exists$ $\Store_1$, $\Perm_1$, $r_1$, $\SE_1$ such that
  \begin{itemize}
  \item
    $R_1 = \Ok{\Store_1, \Perm_1, r_1}$  
  \item $\SE \le \SE_1$, $\Store \le \Store_1$,
    $\vdash \Store_1 : \SE_1$ 
  \item $\Perm_1 \subseteq \Dom{\Store_1}$
  \item $\SE_1 \vdash r_1 : \tau_2\tarr{k}\tau_1$
  \item $\Reach{\Store_1}{r_1} \subseteq \Perm_1$ and by inverting the
    previous typing, we 
    know that $r_1 = \Loc$ hence $\Loc \in \Perm_1$ 
  \item For all $\Loc \in \Reach{\Store_1}{\VEnv_1'}$ it
    must be that $\Store_1 (\Loc) = \Store (\Loc)$ and
    $\Loc\in\Perm \Leftrightarrow \Loc\in\Perm_1$
  \end{itemize}
  By inversion of the store typing we also know that $\Store_1 (\Loc)
  = (\VEnv'',\lam[k]{x}{e})$, for some $\VEnv''$, $k$, $x$, and
  $e$.
  To establish the assumptions for the evaluation for $e_2$, we reason
  as follows
  \begin{itemize}
  \item By splitting of $\E$, we have that $\VEnv = \VEnv_2 +
    \VEnv_2'$ such that  $\SE_1 \vdash \VEnv_2 : \E_2$ (because $\SE\le\SE_1$)
  \item $\vdash \Store_1 : \SE_1$ (by the previous IH)
  \item $\Perm_1' \subseteq \Dom{\Store_1}$ (by the previous IH)
  \item To see that
    $\Reach{\Store_1}{\VEnv_2} \subseteq \Perm_1'$ \\
    Suppose that $\Loc' \in \Reach{\Store_1}{\VEnv_2}$.
    \begin{itemize}
    \item If $\Loc' \in \Reach{\Store_1}{\VEnv_1'} \subseteq
      \Reach\Store{\VEnv_1'} \subseteq \Reach\Store\VEnv$, then $\Loc'
      \in \Perm_1'$ (as $\Loc' \ne \Loc$ which is either new or taken
      from $\Reach{\Store_1}{\VEnv_1}$).
    \item If
      $\Loc \in \Reach{\Store_1}{\VEnv_1'} \cap
      \Reach{\Store_1}{\VEnv_1}$, then ${\entail {} {k \le \kun}}$
      must hold by splitting, which means that $\Loc$ is not writeable
      so that $\Loc\in\Perm_1'$.
    \item If $\Loc' \in \Reach{\Store_1}{\VEnv_1}$, then splitting
      enforces that $\neg\Writeable{\SE_1}{\Loc'}$, hence $\Loc' \in \Perm_1'$.
    \end{itemize}
  \item $\Addresses{\VEnv_2'} \subseteq \Dom{\Store} \subseteq \Dom{\Store_1}$ and
    $\Dom{\VEnv_2} \cap \Dom{\VEnv_2'}=\emptyset$ by construction
  \item $\Store_1, \Perm_1', \VEnv_2 + \VEnv_2' \vdash {e_2}
    \Downarrow^i R_2$
  \end{itemize}
  If $R_2 = \TimeOut$, then $R = \TimeOut$, too.
  Otherwise induction yields that
  $\exists$ $\Store_2$, $\Perm_2$, $r_2$, $\SE_2$ such that
  \begin{itemize}
  \item
    $R_2 = \Ok{\Store_2, \Perm_2, r_2}$  
  \item $\SE_1 \le \SE_2$, $\Store_1 \le \Store_2$,
    $\vdash \Store_2 : \SE_2$ 
  \item $\Perm_2 \subseteq \Dom{\Store_2}$
  \item $\SE_2 \vdash r_2 : \tau_2$
  \item $\Reach{\Store_2}{r_2} \subseteq \Perm_2$
  \item For all $\Loc \in \Reach{\Store_2}{\VEnv_2'}$ it must be that
    $\Store_2 (\Loc) = \Store_1 (\Loc)$
    and $\Loc\in\Perm_1' \Leftrightarrow \Loc\in\Perm_2$ 
  \item For all $\Loc \in \Reach{\Store_2}{\VEnv_2}$,
    if $\neg \Writeable{\SE_2}\Loc $ then
    $\Store_2 (\Loc) = \Store_1 (\Loc)$
    and $\Loc\in\Perm_1' \Leftrightarrow \Loc\in\Perm_2$ 
  \end{itemize}

  It remains to establish the assumptions for the evaluation of $e$.
  \begin{itemize}
  \item $\exists C'', \E''$ such that
    $\inferS{C''}{\E'';\bvar{x}{\tau_2}}{e}{\tau_1}$ and
    $\addlin{\entail{C''}{\Cleq{\E''}{k}}}$
  \item Moreover $\SE_2 \vdash \VEnv'' : \E''$ because $\SE \le \SE_2$
  \item $\vdash \Store_2 : \SE_2$ \quad (by previous IH)
  \item $\Perm_2 \subseteq \Dom{\Store_2}$
  \item To see that $\Reach{\Store_2}{\VEnv''}\subseteq \Perm_2$:
  \\
  Suppose that $\Loc' \in \Reach{\Store_2}{\VEnv''}$.
  Clearly $\Loc' \in \Reach{\Store_1}{\Loc}$, the address of the
  closure, and $\Loc' \in \Perm_1'$ by the inductive hypothesis for
  evaluating $e_1$.
  \begin{itemize}
  \item If $\neg\Writeable{\SE_2}{\Loc'}$,
    then $\Loc' \in \Perm_2$ (because it was never removed from
    $\Perm$)
  \item If $\Writeable{\SE_2}{\Loc'}$, then $k$ must be restricted and
    evaluation of $e_2$ could not remove it from $\Perm_2$ because
    $\Loc' \notin \Reach{\Store_1}{\VEnv_1}$.
  \end{itemize}
  \item For $\VEnv'$ we take $\emptyset$, which fulfills all assumptions
  \item $\Store_2, \Perm_2, \VEnv'' + \emptyset \vdash {e}\Downarrow^i R_3$
  \end{itemize}
  If $R_3 = \TimeOut$, then $R = \TimeOut$, too.
  Otherwise induction yields that
  $\exists$ $\Store_3$, $\Perm_3$, $r_3$, $\SE_3$ such that
  \begin{itemize}
  \item
    $R_3 = \Ok{\Store_3, \Perm_3, r_3}$  
  \item $\SE_2 \le \SE_3$, $\Store_2 \le \Store_3$,
    $\vdash \Store_3 : \SE_3$ 
  \item $\Perm_3 \subseteq \Dom{\Store_3}$
  \item $\SE_3 \vdash r_3 : \tau_1$
  \item $\Reach{\Store_3}{r_3} \subseteq \Perm_3$
  \item Frame condition is void as the ignored part of the environment
    is $\emptyset$.
  \end{itemize}

  From the above, we need to conclude that
  $\exists$ $\Store' = \Store_3$, $\Perm' = \Perm_3$, $r' = r_3$,
  $\SE' = \SE_3$ such that
  \begin{itemize}
  \item
    $R = \Ok{\Store', \Perm', r'}$  
  \item $\SE \le \SE'$, $\Store \le \Store'$,
    $\vdash \Store' : \SE'$, the first two by transitivity of $\le$
    and the last by the IH for evaluating $e$
  \item $\Perm' \subseteq \Dom{\Store'}$ (by last IH)
  \item $\SE' \vdash r' : \tau_1$ (by last IH)
  \item $\Reach{\Store'}{r'} \subseteq \Perm'$ (by last IH)
  \item For all $\Loc \in \Reach{\Store'}{\VEnv'}$ it must be that
    $\Store' (\Loc) = \Store (\Loc)$
    and $\Loc\in\Perm \Leftrightarrow \Loc\in\Perm'$ 
  \item For all $\Loc \in \Reach{\Store'}{\VEnv}$,
    if $\neg \Writeable{\SE'}\Loc $ then
    $\Store' (\Loc) = \Store (\Loc)$
    and $\Loc\in\Perm \Leftrightarrow \Loc\in\Perm'$ 
  \end{itemize}
  

  \clearpage
  \textbf{Case }$\ruleSLet$.

  \textbf{Case }$\ruleSPair$.

  \textbf{Case }$\ruleSMatchLocation$.

  \textbf{Case }$\ruleSMatchBorrow$.

  \clearpage
  \textbf{Case }$\ruleSRegion$.

  We assume that
  \begin{itemize}
  \item $\inferS{C}{\E}{\region{x}{e}}{\tau}$
  \item $\SE \vdash \VEnv, \Suspended\VEnv : \E$ \\
    ($\Suspended\VEnv$ contains suspended
    bindings which we cannot use, although we have the permission)
  \item $\vdash \Store : \SE$
  \item $\Reach\Store{\VEnv, \Suspended\VEnv} \subseteq \Perm \subseteq \Dom\Store$\\
    (we have permission for suspended bindings)
  \item  $\VEnv'$ with $\Addresses{\VEnv'}
    \subseteq \Dom\Store$ and $\Dom\VEnv \cap \Dom{\VEnv'}=\emptyset$
    \\
    (that would be the enclosing frame)
  \item  $i\in\Nat$ and $\Store, \Perm, \VEnv + \Suspended\VEnv + \VEnv' \vdash {\region{x}{e}}
    \Downarrow^{i+1} R$ and $R\ne \TimeOut$.
  \end{itemize}
  
  We apply inversion to the typing: 
  
  $\ruleSDRegion$.

  We see that $x \in \Dom{\Suspended\VEnv}$.
  
  Hence, $\E'$ is mostly equal to $\E$ but $\svar x{\tau_x}$ is
  replaced by $\bvar {\borrow x}{\borrowty{k} {\tau_x}}$.

  (PT: but how is $k$ chosen? Is it $k_\tau$?)

  So $\VEnv' = \VEnv[x \mapsto \Suspended\VEnv (x)b]$ and
  $\Suspended\VEnv' = \Suspended\VEnv \Sdel x$

\clearpage
  \textbf{Case }$\ruleSBorrow$.

  \textbf{Case }$\ruleSCreate$.

  \textbf{Case }$\ruleSDestroy$.

  \textbf{Case }$\ruleSObserve$.

  \textbf{Case }$\ruleSUpdate$.
\end{proof}

%%% Local Variables:
%%% mode: latex
%%% TeX-master: "main"
%%% End:

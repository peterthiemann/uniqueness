\clearpage{}
\section{Metatheory}
\label{sec:metatheory}

\begin{itemize}
\item Borrow compatibility
  $\Multi\IBORROW\Multi\MBORROW \Bcompatible \BORROW$,
  \begin{mathpar}
  \inferrule{}{
    \IBORROW\Multi\IBORROW\Multi\MBORROW \Bcompatible \IBORROW
  }

  \inferrule{}{
    \MBORROW\Multi\MBORROW \Bcompatible \MBORROW
  }
  \end{mathpar}
\item Store typing $ \vdash \Store : \SE$,
  \begin{mathpar}
    \inferrule{
      (\forall \Loc \in \Dom\Store)~
      \SE \vdash \Store (\Loc) : \SE (\Loc)
    }{ \vdash \Store : \SE }
  \end{mathpar}
\item Relating storables to type schemes $\SE \vdash W : \schm$
  \begin{mathpar}
  \inferrule{
    (\exists \E)~ \SE \vdash \VEnv : \E
    \\
    \inferS{C}{\E, (x:\tau_2)}{e}{\tau_1}
  }{
    \SE \vdash (\VEnv, \ilam {\Multi\kvar}{\Multi\tvar}Ckx{e})
    : \forall\Multi\kvar\forall\Multi{\bvar{\alpha}{k}}.(\qual{C}{\tau_2\tarr{k}\tau_1})
  }
  \end{mathpar}
\item Relating storables to types $ \SE \vdash W : \tau$
  \begin{mathpar}
    \inferrule{
      (\exists \E, C)~ \SE \vdash \VEnv : \E
      \\
      \inferS{C}{\E, (x:\tau_2)}{e}{\tau_1}
      \\
      \addlin{\entail{C}{\Cleq{\E}{k}}}
    }{
      \SE \vdash (\VEnv, \lam[k]xe) : \tau_2\tarr{k}\tau_1
    }

    \inferrule{
      \SE \vdash r_1 : \tau_1 \\
      \SE \vdash r_2 : \tau_2
    }{
      \SE \vdash \introPair[k]{r_1}{r_2} : \tyPair[k]{\tau_1}{\tau_2}
    }

    \inferrule{
      \SE \vdash r : \IType{t}{\Multi\tau}
    }{
      \SE \vdash {[r]} : \tapp{t}{\Multi\tau}
    }

    \inferrule{}{
      \SE \vdash \blob : \tau
    }
  \end{mathpar}
\item Relating  results to types $ \SE \vdash r : \tau$, 
\item Relating results to type schemes $\SE \vdash r : \schm$
  \begin{mathpar}
  \inferrule{}{ \SE \vdash c : \CType{c} }

  \inferrule{}{ \SE \vdash \ell : \SE (\ell) }

  \inferrule{
    \Multi\IBORROW\Multi\MBORROW \Bcompatible \BORROW \\
    \SE \vdash \Loc  : \tau
  }{  \SE \vdash
    \Multi\IBORROW\Multi\MBORROW\Loc : \borrow{\tau}}
  \end{mathpar}
\item Relating environments to contexts. Here we consider an
  environment $\VEnv = (\Active\VEnv, \MutableBorrows\VEnv,
  \ImmutableBorrows\VEnv, \Suspended\VEnv)$ as a quadruple
  consisting of the active entries in $\Active\VEnv$ and the
  entries for mutable borrows in $\MutableBorrows\VEnv$ and for
  immutable borrows in $\ImmutableBorrows\VEnv$, and suspended entries
  in $\Suspended\VEnv$. The
  suspended entries cannot be used directly, but they can be activated
  by borrowing.
  
  $\SE \vdash \Active\VEnv, \MutableBorrows\VEnv,
  \ImmutableBorrows\VEnv, \Suspended\VEnv : \E$
\end{itemize}
\begin{mathpar}
  \inferrule{}{\SE \vdash \Sempty, \Sempty, \Sempty, \Sempty : \Eempty}

  \inferrule{
    \SE \vdash \Active\VEnv, \MutableBorrows\VEnv,
    \ImmutableBorrows\VEnv, \Suspended\VEnv : \E
    \\ \SE \vdash r : \schm}
  {\SE \vdash \Active\VEnv[ x\mapsto r], \MutableBorrows\VEnv ,
    \ImmutableBorrows\VEnv, \Suspended\VEnv : \E\bvar x\schm }

  \inferrule{\SE \vdash \Active\VEnv, \MutableBorrows\VEnv,
    \ImmutableBorrows\VEnv, \Suspended\VEnv : \E \\
    \SE \vdash r : \schm}
  { \SE \vdash \Active\VEnv, \MutableBorrows\VEnv,
    \ImmutableBorrows\VEnv, \Suspended\VEnv[ x\mapsto r] : \E\svar x\schm }

  \inferrule{\SE \vdash \Active\VEnv, \MutableBorrows\VEnv,
    \ImmutableBorrows\VEnv, \Suspended\VEnv : \E \\ \SE \vdash
    \IBORROW r : \tau} 
  {\SE \vdash \Active\VEnv, \MutableBorrows\VEnv,
    \ImmutableBorrows\VEnv[ x\mapsto \IBORROW r], \Suspended\VEnv : \E\bvar{\borrow[\IBORROW] x}{\borrowty[\IBORROW] k{ \tau}} }

  \inferrule{\SE \vdash \Active\VEnv, \MutableBorrows\VEnv,
    \ImmutableBorrows\VEnv, \Suspended\VEnv : \E \\ \SE \vdash
    \MBORROW r : \tau}
  {\SE \vdash \Active\VEnv, \MutableBorrows\VEnv[ x\mapsto \MBORROW r],
    \ImmutableBorrows\VEnv : \E\bvar{\borrow[\MBORROW] x}{\borrowty[\MBORROW] k{ \tau}} }
\end{mathpar}
\clearpage{}
We write $\Addresses{\VEnv}$ for the function that extracts addresses
from the range of the variable environment.

We write $\Reach\Store\VEnv \subseteq \Dom\Store$ for the function
that computes all store addresses reachable from $\Addresses\VEnv$
assuming that $\Addresses\VEnv \subseteq \Dom\Store$. 

We write $\Writeable\SE\Loc$ to express that the type of location
$\Loc$ permits changing its content. For example,
\begin{itemize}
\item if $\SE (\Loc) =
  \forall\Multi\kvar\forall\Multi{\bvar{\alpha}{k}}.(\qual{C}{\tau_2\tarr{k}\tau_1})$,
  then $\Loc$ is not writeable.
\item if $\SE (\Loc) = \tau_2\tarr{\kun}{\tau_1}$, then $\Loc$ is not
  writeable.
\item if $\SE (\Loc) = \tyPair[\kun]{\tau_1}{\tau_2}$, then $\Loc$ is not writeable.
\item if $\SE (\Loc) = \tapp{t}{\Multi\tau}$, then $\Loc$ is
  writeable. (IT RATHER SEEMS LIKE $\Loc$'s DECENDANT IS WRITEABLE)
\item if $\SE (\Loc) = \borrow\tau$, then \dots (THAT SHOULDN'T REALLY
  BE A STORE TYPE)
\end{itemize}
We write $\Linear\SE\Loc$ to express that $\Loc$ points to a linear
resource.
\begin{itemize}
\item if $\SE (\Loc) =
  \forall\Multi\kvar\forall\Multi{\bvar{\alpha}{k}}.(\qual{C}{\tau_2\tarr{k}\tau_1})$,
  then $\Loc$ is $\dots$
\item if $\SE (\Loc) = \tau_2\tarr{k}{\tau_1}$ and $\Cleq{\klin}{k}$, then $\Loc$ is linear.
\item if $\SE (\Loc) = \tyPair[k]{\tau_1}{\tau_2}$ and $\Cleq \klin
  k$, then $\Loc$ is linear.
\item if $\SE (\Loc) = \tapp{t}{\Multi\tau}$, then $\Loc$ is linear.
\item if $\SE (\Loc) = \borrow\tau$, then \dots (THAT SHOULDN'T REALLY
  BE A STORE TYPE)
\end{itemize}
We write $\Affine\SE\Loc$ to express that $\Loc$ points to an affine
resource.
\begin{itemize}
\item if $\SE (\Loc) =
  \forall\Multi\kvar\forall\Multi{\bvar{\alpha}{k}}.(\qual{C}{\tau_2\tarr{k}\tau_1})$,
  then $\Loc$ is $\dots$
\item if $\SE (\Loc) = \tau_2\tarr{k}{\tau_1}$ and $\Cleq{\kaff}{k}$, then $\Loc$ is affine.
\item if $\SE (\Loc) = \tyPair[k]{\tau_1}{\tau_2}$ and $\Cleq \kaff
  k$, then $\Loc$ is affine.
\item if $\SE (\Loc) = \tapp{t}{\Multi\tau}$, then $\Loc$ is affine.
\item if $\SE (\Loc) = \borrow[\MBORROW]\tau$, then $\Loc$ is affine
  \dots (THAT SHOULDN'T REALLY  BE A STORE TYPE)
\item if $\SE (\Loc) = \borrow[\IBORROW]\tau$, then $\Loc$ is not affine
  \dots (THAT SHOULDN'T REALLY  BE A STORE TYPE)
\end{itemize}
TODO consider using a multiset for $\Reach\Store{}$. For a multiset
$M$, let $\MultiNumber x M$ be the number of times $x$ occurs in $M$.
\clearpage{}
\begin{theorem}
  Suppose that
  \begin{itemize}
  \item $\inferS{C}{\E}{e}{\tau}$
  \item $\SE \vdash \VEnv : \E$
  \item $\vdash \Store : \SE$
  \item $\Reach\Store{\Active\VEnv, \MutableBorrows\VEnv, \ImmutableBorrows\VEnv} \subseteq \Perm$
  \item $\Addresses\Perm \subseteq \Dom\Store$
  % \item  $\VEnv'$ with $\Addresses{\VEnv'}
  %   \subseteq \Dom\Store$ and $\Dom\VEnv \cap \Dom{\VEnv'}=\emptyset$
  \item Incoming Resources: Let $\REACH = \Reach\Store\VEnv$.
    \begin{itemize}
    \item 
      For all $\Loc$ such that $\MultiNumber\Loc{\Reach{\Store}{\Active\VEnv}} >0$,
      if $\Affine{\SE}\Loc$ then $\MultiNumber\Loc\REACH= 1$
    \item For all $\Loc$ such that $
      \MultiNumber\Loc{\Reach{\Store}{\MutableBorrows\VEnv}} >0$, it
      must be that $\MultiNumber\Loc\REACH=1$.
    \end{itemize}
  \item  $i\in\Nat$ and $\Store, \Perm, \VEnv \vdash {e}
    \Downarrow^i R$ and $R\ne \TimeOut$.
  \end{itemize}
  Then,
  $\exists$ $\Store'$, $\Perm'$, $r'$, $\SE'$ such that
  \begin{itemize}
  \item
    $R = \Ok{\Store', \Perm', r'}$  
  \item $\SE \le \SE'$, $\Store \le \Store'$,
    $\vdash \Store' : \SE'$ 
  \item $\Addresses{\Perm'} \subseteq \Dom{\Store'}$
  \item $\SE' \vdash r' : \tau$
  \item $\Reach{\Store'}{r'} \subseteq \Perm'$
  \item Outside: For all $\Loc \in \Dom{\Store} \setminus
    \Reach{\Store'}{\VEnv}$ it must be that 
    $\Store' (\Loc) = \Store (\Loc)$
    and $\Loc\in\Perm \Leftrightarrow \Loc\in\Perm'$ 
    %% must be \Store because \Perm has no idea of \Perm'
  \item Immutables: For all $\Loc \in
    \Reach{\Store'}{\ImmutableBorrows\VEnv}$ it must be that
    $\Loc\in\Dom\Store$, 
    $\Store' (\Loc) = \Store (\Loc)$
    and $\Loc\in\Perm \Leftrightarrow \Loc\in\Perm'$ 
  \item Resources:
    Let $\REACH' =\Reach{\Store'}{\Active\VEnv}$.
    For all $\Loc$ such that $n= \MultiNumber\Loc\REACH >0$,
    \begin{itemize}
    \item if $\Affine{\SE}\Loc$ then $n\le 1$; if $n=0$, then $\Loc\notin\Perm'$.
    \item if $\Linear{\SE}\Loc$ then $n=0$ and $\Loc\notin\Perm'$.
    \end{itemize}
  \item Immutables: For all $\Loc \in \Reach
    {\Store'}{\Active\VEnv}$, if $\neg\Writeable{\SE}\loc$ then
    $\Store' (\ell) = \Store (\ell)$ and $\Loc\in \Perm'$.
  \item No thin air permission: For all $\Loc\in \Perm'$, either $\Loc
    \in \Perm \cup  \Dom{\Store'} \setminus \Dom{\Store}$.
  \end{itemize}
\end{theorem}

\begin{proof}
  The base case is trivial as
  \lstinline[style=rule]{eval \Store \Perm \VEnv 0 e = \TimeOut}. 
  
  Let now $i>0$ and consider the different cases for expressions.

  \textbf{Case} $e$
  \lstsemrule{sapp}

  By assumption $\ruleSDApp$.

  As \lstinline{split}  is evidence for the split in the typing rule,
  we find that
  \begin{align}
    \label{eq:1}
    \SE \vdash \VEnv_1 : \E_1 && \SE \vdash \VEnv_2 : \E_2
  \end{align}
\end{proof}
\clearpage
%%%%%%%%%%%%%%%%%%%%%%%%%%%%%%%%%%%%%%%%%%%%%%%%%%%%%%%%%%%%%%%%%%%%%%
\begin{proof}
  The proof is by induction on the evaluation judgment $\Store, \Perm,
  \VEnv \vdash {e} \Downarrow^i R$.
  
  The base case is trivial as $\Store, \Perm, \VEnv \vdash {e}
  \Downarrow^0 \TimeOut$.

  For indexes greater than zero we have a case distinction over all applicable execution
  rules.

  \textbf{Case }$\ruleSConst$.
  The claim is immediate.

  \textbf{Case }$\ruleSVar$.
  The claim is immediate.

  \textbf{Case }$\ruleSTApp$.
  
  By assumption, we have that $\SE \vdash \Loc : \E (x)$ so that
  $\E(x) =
  \forall\Multi\kvar\forall\Multi{\bvar{\alpha}{k}}.(\qual{C}{\tau_2\tarr{k}\tau_1})$.
  All conclusions hold: $\SE \le \SE' = \SE[\Loc' \mapsto
  (\tau_2\tarr{k}\tau_1)[\Multi[i]{\kvar} \mapsto \Multi[i]{ k},
  \Multi[j]{\alpha} \mapsto \Multi[j]{t}]]$;
  $\vdash \Store' : \SE'$;
  $\Perm' \subseteq \Dom{\Store'}$;
  $\SE' \vdash \Loc' : \tau$;
  and $\Loc' \in \Perm' \Sadd{\Loc'}$.

  \textbf{Case }$\ruleSPLam$.

  By assumption $\SE \vdash     \ilam {\Multi[i]{\kvar}}{\Multi[j]{\tvar}}Ck xe
  : \schm$ where $\schm =
  \forall\Multi\kvar\forall\Multi{\bvar{\alpha}{k}}.(\qual{C}{\tau_2\tarr{k}\tau_1})$. Let
  $\SE'  = \SE[\Loc \mapsto \schm]$.
  Then $\SE \le \SE'$ and $\vdash\Store' : \SE'$.
  Moreover, $\SE' \vdash \Loc : \schm$ and $\Loc \in \Perm
  \Sadd{\Loc}$ as required.
  
  \clearpage
  \textbf{Case }$\ruleSApp$.

  By assumption $\ruleSDApp$.

  \begin{itemize}
  \item By splitting of $\E$, we have that $\VEnv = \VEnv_1 +
    \VEnv_1'$ such that  $\SE \vdash \VEnv_1 : \E_1$
  \item $\vdash \Store : \SE$
  \item $\Reach\Store{\VEnv_1} \subseteq \Reach\Store{\VEnv} \subseteq
    \Perm \subseteq \Dom\Store$ 
  \item $\Addresses{\VEnv_1'} \subseteq \Dom\Store$ and
    $\Dom{\VEnv_1} \cap \Dom{\VEnv_1'}=\emptyset$
  \item $\Store, \Perm, \VEnv_1 + \VEnv_1' \vdash {e_1}
    \Downarrow^i R_1$
  \end{itemize}
  If $R_1 = \TimeOut$, then $R = \TimeOut$, too.
  Otherwise induction yields that
  $\exists$ $\Store_1$, $\Perm_1$, $r_1$, $\SE_1$ such that
  \begin{itemize}
  \item
    $R_1 = \Ok{\Store_1, \Perm_1, r_1}$  
  \item $\SE \le \SE_1$, $\Store \le \Store_1$,
    $\vdash \Store_1 : \SE_1$ 
  \item $\Perm_1 \subseteq \Dom{\Store_1}$
  \item $\SE_1 \vdash r_1 : \tau_2\tarr{k}\tau_1$
  \item $\Reach{\Store_1}{r_1} \subseteq \Perm_1$ and by inverting the
    previous typing, we 
    know that $r_1 = \Loc$ hence $\Loc \in \Perm_1$ 
  \item For all $\Loc \in \Reach{\Store_1}{\VEnv_1'}$ it
    must be that $\Store_1 (\Loc) = \Store (\Loc)$ and
    $\Loc\in\Perm \Leftrightarrow \Loc\in\Perm_1$
  \end{itemize}
  By inversion of the store typing we also know that $\Store_1 (\Loc)
  = (\VEnv'',\lam[k]{x}{e})$, for some $\VEnv''$, $k$, $x$, and
  $e$.
  To establish the assumptions for the evaluation for $e_2$, we reason
  as follows
  \begin{itemize}
  \item By splitting of $\E$, we have that $\VEnv = \VEnv_2 +
    \VEnv_2'$ such that  $\SE_1 \vdash \VEnv_2 : \E_2$ (because $\SE\le\SE_1$)
  \item $\vdash \Store_1 : \SE_1$ (by the previous IH)
  \item $\Perm_1' \subseteq \Dom{\Store_1}$ (by the previous IH)
  \item To see that
    $\Reach{\Store_1}{\VEnv_2} \subseteq \Perm_1'$ \\
    Suppose that $\Loc' \in \Reach{\Store_1}{\VEnv_2}$.
    \begin{itemize}
    \item If $\Loc' \in \Reach{\Store_1}{\VEnv_1'} \subseteq
      \Reach\Store{\VEnv_1'} \subseteq \Reach\Store\VEnv$, then $\Loc'
      \in \Perm_1'$ (as $\Loc' \ne \Loc$ which is either new or taken
      from $\Reach{\Store_1}{\VEnv_1}$).
    \item If
      $\Loc \in \Reach{\Store_1}{\VEnv_1'} \cap
      \Reach{\Store_1}{\VEnv_1}$, then ${\entail {} {k \le \kun}}$
      must hold by splitting, which means that $\Loc$ is not writeable
      so that $\Loc\in\Perm_1'$.
    \item If $\Loc' \in \Reach{\Store_1}{\VEnv_1}$, then splitting
      enforces that $\neg\Writeable{\SE_1}{\Loc'}$, hence $\Loc' \in \Perm_1'$.
    \end{itemize}
  \item $\Addresses{\VEnv_2'} \subseteq \Dom{\Store} \subseteq \Dom{\Store_1}$ and
    $\Dom{\VEnv_2} \cap \Dom{\VEnv_2'}=\emptyset$ by construction
  \item $\Store_1, \Perm_1', \VEnv_2 + \VEnv_2' \vdash {e_2}
    \Downarrow^i R_2$
  \end{itemize}
  If $R_2 = \TimeOut$, then $R = \TimeOut$, too.
  Otherwise induction yields that
  $\exists$ $\Store_2$, $\Perm_2$, $r_2$, $\SE_2$ such that
  \begin{itemize}
  \item
    $R_2 = \Ok{\Store_2, \Perm_2, r_2}$  
  \item $\SE_1 \le \SE_2$, $\Store_1 \le \Store_2$,
    $\vdash \Store_2 : \SE_2$ 
  \item $\Perm_2 \subseteq \Dom{\Store_2}$
  \item $\SE_2 \vdash r_2 : \tau_2$
  \item $\Reach{\Store_2}{r_2} \subseteq \Perm_2$
  \item For all $\Loc \in \Reach{\Store_2}{\VEnv_2'}$ it must be that
    $\Store_2 (\Loc) = \Store_1 (\Loc)$
    and $\Loc\in\Perm_1' \Leftrightarrow \Loc\in\Perm_2$ 
  \item For all $\Loc \in \Reach{\Store_2}{\VEnv_2}$,
    if $\neg \Writeable{\SE_2}\Loc $ then
    $\Store_2 (\Loc) = \Store_1 (\Loc)$
    and $\Loc\in\Perm_1' \Leftrightarrow \Loc\in\Perm_2$ 
  \end{itemize}

  It remains to establish the assumptions for the evaluation of $e$.
  \begin{itemize}
  \item $\exists C'', \E''$ such that
    $\inferS{C''}{\E'';\bvar{x}{\tau_2}}{e}{\tau_1}$ and
    $\addlin{\entail{C''}{\Cleq{\E''}{k}}}$
  \item Moreover $\SE_2 \vdash \VEnv'' : \E''$ because $\SE \le \SE_2$
  \item $\vdash \Store_2 : \SE_2$ \quad (by previous IH)
  \item $\Perm_2 \subseteq \Dom{\Store_2}$
  \item To see that $\Reach{\Store_2}{\VEnv''}\subseteq \Perm_2$:
  \\
  Suppose that $\Loc' \in \Reach{\Store_2}{\VEnv''}$.
  Clearly $\Loc' \in \Reach{\Store_1}{\Loc}$, the address of the
  closure, and $\Loc' \in \Perm_1'$ by the inductive hypothesis for
  evaluating $e_1$.
  \begin{itemize}
  \item If $\neg\Writeable{\SE_2}{\Loc'}$,
    then $\Loc' \in \Perm_2$ (because it was never removed from
    $\Perm$)
  \item If $\Writeable{\SE_2}{\Loc'}$, then $k$ must be restricted and
    evaluation of $e_2$ could not remove it from $\Perm_2$ because
    $\Loc' \notin \Reach{\Store_1}{\VEnv_1}$.
  \end{itemize}
  \item For $\VEnv'$ we take $\emptyset$, which fulfills all assumptions
  \item $\Store_2, \Perm_2, \VEnv'' + \emptyset \vdash {e}\Downarrow^i R_3$
  \end{itemize}
  If $R_3 = \TimeOut$, then $R = \TimeOut$, too.
  Otherwise induction yields that
  $\exists$ $\Store_3$, $\Perm_3$, $r_3$, $\SE_3$ such that
  \begin{itemize}
  \item
    $R_3 = \Ok{\Store_3, \Perm_3, r_3}$  
  \item $\SE_2 \le \SE_3$, $\Store_2 \le \Store_3$,
    $\vdash \Store_3 : \SE_3$ 
  \item $\Perm_3 \subseteq \Dom{\Store_3}$
  \item $\SE_3 \vdash r_3 : \tau_1$
  \item $\Reach{\Store_3}{r_3} \subseteq \Perm_3$
  \item Frame condition is void as the ignored part of the environment
    is $\emptyset$.
  \end{itemize}

  From the above, we need to conclude that
  $\exists$ $\Store' = \Store_3$, $\Perm' = \Perm_3$, $r' = r_3$,
  $\SE' = \SE_3$ such that
  \begin{itemize}
  \item
    $R = \Ok{\Store', \Perm', r'}$  
  \item $\SE \le \SE'$, $\Store \le \Store'$,
    $\vdash \Store' : \SE'$, the first two by transitivity of $\le$
    and the last by the IH for evaluating $e$
  \item $\Perm' \subseteq \Dom{\Store'}$ (by last IH)
  \item $\SE' \vdash r' : \tau_1$ (by last IH)
  \item $\Reach{\Store'}{r'} \subseteq \Perm'$ (by last IH)
  \item For all $\Loc \in \Reach{\Store'}{\VEnv'}$ it must be that
    $\Store' (\Loc) = \Store (\Loc)$
    and $\Loc\in\Perm \Leftrightarrow \Loc\in\Perm'$ 
  \item For all $\Loc \in \Reach{\Store'}{\VEnv}$,
    if $\neg \Writeable{\SE'}\Loc $ then
    $\Store' (\Loc) = \Store (\Loc)$
    and $\Loc\in\Perm \Leftrightarrow \Loc\in\Perm'$ 
  \end{itemize}
  

  \clearpage
  \textbf{Case }$\ruleSLet$.

  \textbf{Case }$\ruleSPair$.

  \textbf{Case }$\ruleSMatchLocation$.

  \textbf{Case }$\ruleSMatchBorrow$.

  \clearpage
  \textbf{Case }$\ruleSRegion$.

  We assume that
  \begin{itemize}
  \item $\inferS{C}{\E}{\region{x}{e}}{\tau}$
  \item $\SE \vdash \VEnv, \MutableBorrows\VEnv : \E$ \\
    ($\MutableBorrows\VEnv$ contains suspended
    bindings which we cannot use, although we have the permission)
  \item $\vdash \Store : \SE$
  \item $\Reach\Store{\VEnv, \MutableBorrows\VEnv} \subseteq \Perm \subseteq \Dom\Store$\\
    (we have permission for suspended bindings)
  \item  $\VEnv'$ with $\Addresses{\VEnv'}
    \subseteq \Dom\Store$ and $\Dom\VEnv \cap \Dom{\VEnv'}=\emptyset$
    \\
    (that would be the enclosing frame)
  \item  $i\in\Nat$ and $\Store, \Perm, \VEnv + \MutableBorrows\VEnv + \VEnv' \vdash {\region{x}{e}}
    \Downarrow^{i+1} R$ and $R\ne \TimeOut$.
  \end{itemize}
  
  We apply inversion to the typing: 
  
  $\ruleSDRegion$.

  We see that $x \in \Dom{\MutableBorrows\VEnv}$.
  
  Hence, $\E'$ is mostly equal to $\E$ but $\svar x{\tau_x}$ is
  replaced by $\bvar {\borrow x}{\borrowty{k} {\tau_x}}$.

  (PT: but how is $k$ chosen? Is it $k_\tau$?)

  So $\VEnv' = \VEnv[x \mapsto \MutableBorrows\VEnv (x)b]$ and
  $\MutableBorrows\VEnv' = \MutableBorrows\VEnv \Sdel x$

\clearpage
  \textbf{Case }$\ruleSBorrow$.

  \textbf{Case }$\ruleSCreate$.

  \textbf{Case }$\ruleSDestroy$.

  \textbf{Case }$\ruleSObserve$.

  \textbf{Case }$\ruleSUpdate$.
\end{proof}

\section{Metatheory}
\label{sec:metatheory}

\lstMakeShortInline[keepspaces,style=rule]@

There are several connections between the type system and
the operational semantics, which we state as a single type soundness
theorem.
%
The theorem relies on several standard notions like store typing
$\vdash \Store : \SE$ and agreement of the results in the value environment
with the type environment $\SE \vdash \VEnv:\E$ that we define
formally in~\cref{sec:metatheory:proofs} where we also present selected cases of the
proofs.
%
The non-standard part is the handling of permissions. With
$\Rawloc\Perm$ we extract the underlying raw locations from the
permissions as in $\Rawloc{\Multi\IBORROW\hspace{0.5mm}\Multi\MBORROW\hspace{0.5mm}\Loc} = \Loc$
and with $\Reach\Store\VEnv$ we transitively trace the
addresses reachable from $\VEnv$ in store $\Store$. We write
$\SE\le\SE'$ and $\Store \le \Store'$ for extending the domain of the
store type and of the store, respectively.
%
The permission set contains the set
of addresses that can be used during evaluation. It is managed by the
region expression as well as by creation and use of resources as
shown in \cref{sec:sem}.
%
We distinguish several parts of the value
environment $\VEnv$ that correspond to the different kinds of bindings in the
type environment: $\Active\VEnv$ for active entries of direct
references to linear resources, closures, etc; $\MutableBorrows\VEnv$ for
affine borrows or resources;
$\ImmutableBorrows\VEnv$ for unrestricted values including
unrestricted borrows;
and $\Suspended\VEnv$ for suspended entries. The judgment
$\SE \vdash \VEnv:\E$ is defined in terms of this structure.
We treat
$\Reach\Store\VEnv$ as a multiset to properly discuss linearity and
affinity. We use the notation $\MultiNumber x M$ for the number of
times $x$ occurs in multiset $M$.

\newcommand\resultOk[2]{
  $R#2 = \Ok{\Store#2, \Perm#2, r#2}$
}
\newcommand\resultEnv[2]{
  $\SE#1 \le \SE#2$, $\Store#1 \le \Store#2$, $\vdash \Store#2 : \SE#2$
}
\newcommand\resultPermDom[2]{
  $\Perm#2$ is wellformed and
  $\Rawloc{\Perm#2} \subseteq \Dom{\Store#2} \setminus {\Store#2}^{-1}
  (\StFreed)$.
}
\newcommand\resultReachPerm[2]{
  $\Reach0{r#2} \subseteq \Perm#2$,
  $\Reach{\Store#2}{r#2} \subseteq \Sclos{\Perm#2}$
  % \textcolor{red}{
    $\cap (\Reach{\Store#2}{\VEnv#1} \setminus
    \Reach{\Store#2}{\Suspended{\VEnv#1}}
    \cup \Dom{\Store#2} \setminus \Dom{\Store#1})$.
  % }
}
\newcommand\resultFrame[3]{
  % input, inputenv, output
  Frame: \\
  For all $\Loc \in \Dom{\Store#1} \setminus
  \Rawloc{\Reach{\Store#3}{\VEnv#2}}$ it must be that
  \begin{itemize}
  \item $\Store#3 (\Loc) = \Store#1 (\Loc)$ and
  \item  for any $\Addr$ with $\Rawloc\Addr = \{\Loc\}$,
    $\Addr \in \Perm#1 \Leftrightarrow \Addr\in\Perm#3$.
  \end{itemize}
}
\newcommand\resultImmutables[3]{
  % input, inputenv, output
  Unrestricted values, resources, and borrows: \\
  For all $\Addr \in
  \Reach{\Store#3}{\ImmutableBorrows{\VEnv#2}, \Suspended[\kun]{\VEnv#2}}$ with
  $\Rawloc\Addr = \{\Loc\}$, it must be that
  $\Loc\in\Dom{\Store#1}$,
  $\Store#3 (\Loc) = \Store#1 (\Loc) \ne \StFreed$
  and $\Addr\in\Perm#3$.
}
\newcommand\resultMutables[3]{
  % input, inputenv, output
  Affine borrows and resources:\\
  For all $\Addr \in  \Reach{\Store#3}{\MutableBorrows{\VEnv#2}, \Suspended[\kaff]{\VEnv#2}}$
  with $\Rawloc\Addr = \{\Loc\}$, it must be that
  $\Loc\in\Dom{\Store#1}$. If $\Addr\ne\Loc$, then
  $\Store#3 (\Loc) \ne \StFreed$.
  If $\Addr \in \Reach{\Store#3}{\Suspended[\kaff]{\VEnv#2}}$, then
  $\Addr \in \Perm#3$.
}
\newcommand\resultSuspendedXXX[3]{
  % input, inputenv, output
  Suspended borrows: \\
  For all $\Addr \in \Reach{\Store#3}{\Suspended{\VEnv#2}}$ with
  $\Rawloc\Addr = \{\Loc\}$ it must be that $\Addr \in \Perm#3$
  and  $\Store#3 (\Loc) \ne \StFreed$.
  % \begin{itemize}
  % \item If $\Addr = \IBORROW\Addr'$, then $\Loc\in\Dom{\Store#1}$ and
  %   $\Store#3 (\Loc) = \Store#1 (\Loc) \ne \StFreed$.
  % \item If $\Addr = \MBORROW\Addr'$ and
  %   $\Loc\in\Dom{\Store#1}$, then
  %   $\Store#3 (\Loc) \ne \StFreed$.
  % \end{itemize}
}
\newcommand\resultResources[3]{
  % input, inputenv, output
  Resources: Let $\REACH#1 = \Reach{\Store#1}{\Active{\VEnv#2}}$.  Let
  $\REACH#3 =\Reach{\Store#3}{\Active{\VEnv#2}}$.
  \\
  For all $\Loc\in \REACH#1$ it must be that
  $\MultiNumber\Loc{\REACH#1}=\MultiNumber\Loc{\REACH#3}=1$,
  % \textcolor{red}{
  %   either $\Loc\notin\Perm#3$ and $\Store#3 (\Loc) =\StFreed$
  %   or $\Loc\in\Perm#3$ and $\Loc \in \Reach{\Store#3}{r#3}$.
  % }
  % \textcolor{blue}{
    $\Loc\notin\Perm#3$, and if $\Store#1(\Loc)$ is a resource, then
    $\Store#3 (\Loc) = \StFreed$.
  % }
}
\newcommand\resultThinAir[2]{
  No thin air permission: \\
  $\Perm#2 \subseteq
  \Perm#1 \cup ( \Dom{\Store#2} \setminus \Dom{\Store#1})$.
  % For all $\Addr\in {\Perm#2}$, $\Addr
  % \in \Perm#1 \cup ( \Dom{\Store#2} \setminus \Dom{\Store#1})$.
}
\newcommand\assumeWellformed[1]{
  $\Perm#1$ is wellformed and $\Rawloc{\Perm#1} \subseteq
  \Dom{\Store#1} \setminus {\Store#1}^{-1} (\StFreed)$
}
\newcommand\assumeIncoming[2]{
  Incoming Resources:
  \begin{enumerate}
  \item $\forall \Loc\in \Rawloc{\Reach{\Store#1}{\VEnv#2}}$,
    $\Store#1 (\Loc) \ne \StFreed$.
    % suspended bindings must not point to freed resources
  \item $\forall \Loc \in \REACH#1
    =\Rawloc{\Reach{\Store#1}{\Active{\VEnv#2},\MutableBorrows{\VEnv#2},
      \Suspended[\kaff]{\VEnv#2}}}$,
    % if $\Affine\SE\Loc$ then
    $\MultiNumber\Loc{\REACH#1}= 1$.
  \end{enumerate}
}
\newcommand\assumeReachable[2]{
  $\Reach0{\VEnv#2} \subseteq {\Perm#1}$, 
  $\Reach{\Store#1}{\VEnv#2} \subseteq \Sclos{\Perm#1}$.
}

\begin{restatable}[Type Soundness]{theorem}{SoundnessThm}\label{thm:soundness}
  Suppose that
  \begin{enumerate}[({A}1)]
  \item $\inferS{C}{\E}{e}{\tau}$
  \item\label{item:32} $\SE \vdash \VEnv : \E$
  \item\label{item:33} $\vdash \Store : \SE$
  \item\label{item:11} \assumeWellformed{}
  \item\label{item:12} \assumeReachable{}{}
  \item $\Rawloc{\Active\VEnv}$,
    $\Rawloc{\MutableBorrows\VEnv}$,
    $\Rawloc{\ImmutableBorrows\VEnv}$, and
    $\Rawloc{\Suspended\VEnv}$ are all disjoint
  % \item  $\VEnv'$ with $\Rawloc{\VEnv'}
  %   \subseteq \Dom\Store$ and $\Dom\VEnv \cap \Dom{\VEnv'}=\emptyset$
  \item\label{item:15} \assumeIncoming{}{}
  \end{enumerate}
  For all $i\in\Nat$, if
  % Sorry for the ugly fix... I can't get lstinline to behave like
  % \lstMakeShortInline, but copy pasting would also be bad.
  \ \lstinline[style=rule]{R' = eval} $\Store$ $\Perm$ $\VEnv$ \lstinline[style=rule]{i e}
  %$\Store, \Perm, \VEnv \vdash {e}  \Downarrow^i R'$
  and $R'\ne \TimeOut$,
  then
  $\exists$ $\Store'$, $\Perm'$, $r'$, $\SE'$ such that
  \begin{enumerate}[({R}1)]
  \item \resultOk{}{'}
  \item \resultEnv{}{'}
  \item $\SE' \vdash r' : \tau$
  \item \resultPermDom{}{'}
  \item \resultReachPerm{}{'}
  \item \resultFrame{}{}{'}
  \item \resultImmutables{}{}{'}
  \item \resultMutables{}{}{'}
  % \item \resultSuspended{}{}{'}
  \item \resultResources{}{}{'}
  \item \resultThinAir{}{'}
  \end{enumerate}
\end{restatable}

% As customary with the functional style of semantics,
The proof of the
theorem is by functional induction on the evaluation judgment, which
is indexed by the strictly decreasing counter $i$.

The assumptions A1-A3 and results R1-R3 state the standard soundness properties
for lambda calculi with references.

The rest of the statement accounts for the substructural properties and
borrowing in the presence of explicit resource management.
%
% The rest of the statement accounts for the properties and interactions of
% borrowing with explicit resource management via substructural types.
%
% Some explanations are in order for the resource-related assumptions
% and statements.
%
Incoming resources are always active (i.e., not freed).
Linear and affine resources as well as suspended affine borrows have
exactly one pointer in the environment.
%
The Frame condition states that only store locations reachable from
the current environment can change and that all permissions outside
the reachable locations remain the same.
%
Unrestricted values, resources, and borrows do not change their
underlying resource and do not spend their permission.
%
Affine borrows and resources may or may not spend their
permission. Borrows are not freed, but resources may be freed.
%
Incoming suspended borrows have no permission attached to them and
their permission has been retracted on exit of their region.
%
A linear resource is always freed.
%
Outgoing permissions are either inherited from the caller or they
refer to newly created values.

\lstDeleteShortInline@


%%% Local Variables:
%%% mode: latex
%%% TeX-master: "main"
%%% End:


%%% Local Variables:
%%% mode: latex
%%% TeX-master: "main"
%%% End:

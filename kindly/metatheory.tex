\section{Metatheory}
\label{sec:metatheory}

\lstMakeShortInline[keepspaces,style=rule]@

There are several connections between the type system and
the operational semantics, which we state as a single type soundness
theorem.
%
The theorem relies on several standard notions like store typing
$\vdash \Store : \SE$ and agreement of the results in the value environment
with the type environment $\SE \vdash \VEnv:\E$ that we define
formally in~\cref{sec:metatheory:proofs} along with selected cases of the
proofs.
%
The non-standard part is the handling of permissions. With
$\Rawloc\Perm$ we extract the underlying raw locations from the
permissions as in $\Rawloc{\Multi\IBORROW\hspace{0.5mm}\Multi\MBORROW\hspace{0.5mm}\Loc} = \Loc$
and with $\Reach\Store\VEnv$ we transitively trace the
addresses reachable from $\VEnv$ in store $\Store$. We write
$\SE\le\SE'$ and $\Store \le \Store'$ for extending the domain of the
store type and of the store, respectively.
%
The permission set contains the set
of addresses that can be used during evaluation. It is managed by the
region expression as well as by creation and use of resources as
shown in \cref{sec:sem}.
%
We distinguish several parts of the value
environment $\VEnv$ that correspond to the different kinds of bindings in the
type environment: $\Active\VEnv$ for active entries of direct
references to linear resources, closures, etc; $\MutableBorrows\VEnv$ for
affine borrows or resources;
$\ImmutableBorrows\VEnv$ for unrestricted values including
unrestricted borrows;
and $\Suspended\VEnv$ for suspended entries. The judgment
$\SE \vdash \VEnv:\E$ is defined in terms of this structure.
We treat
$\Reach\Store\VEnv$ as a multiset to properly discuss linearity and
affinity. We use the notation $\MultiNumber x M$ for the number of
times $x$ occurs in multiset $M$.

\begin{theorem}[Type Soundness]\label{theorem:type-soundness}
  \SoundnessThmBody
\end{theorem}

% As customary with the functional style of semantics,
The proof of the
theorem is by functional induction on the evaluation judgment, which
is indexed by the strictly decreasing counter $i$.

The assumptions A1-A3 and results R1-R3 state the standard soundness properties
for lambda calculi with references.

The rest of the statement accounts for the substructural properties and
borrowing in the presence of explicit resource management.
%
% The rest of the statement accounts for the properties and interactions of
% borrowing with explicit resource management via substructural types.
%
% Some explanations are in order for the resource-related assumptions
% and statements.
%
Incoming resources are always active (i.e., not freed).
Linear and affine resources as well as suspended affine borrows have
exactly one pointer in the environment.
%
The Frame condition states that only store locations reachable from
the current environment can change and that all permissions outside
the reachable locations remain the same.
%
Unrestricted values, resources, and borrows do not change their
underlying resource and do not spend their permission.
%
Affine borrows and resources may or may not spend their
permission. Borrows are not freed, but resources may be freed.
%
Incoming suspended borrows have no permission attached to them and
their permission has been retracted on exit of their region.
%
A linear resource is always freed.
%
Outgoing permissions are either inherited from the caller or they
refer to newly created values.

\lstDeleteShortInline@


%%% Local Variables:
%%% mode: latex
%%% TeX-master: "main"
%%% End:

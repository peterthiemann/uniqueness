\lstMakeShortInline[keepspaces,basicstyle=\small\ttfamily]@
\section{Limitations and Extensions}

\subsection{Flow sensitivity}

The type system defined so far does not support any
form of flow sensitivity. This means that code patterns that relies on
subtle flow-sensitive usage of permissions and linearity likely will not
typecheck in \lang. As an example the following piece of code,
which defines the merge function on linear lists,
cannot be expressed directly.

\begin{lstlisting}
let rec merge l1 l2 = match l1, l2 with
  | h1::t1, h2::t2 ->
    if &h1 < &h2 
    then h1::(merge t1 l2) (* Must expand l2 to h2::t2 here *)
    else h2::(merge l1 t2)
  | ....
\end{lstlisting}

Such pattern usually requires richer logic, such as
Mezzo~\citep{DBLP:phd/hal/Protzenko14}.
However
\citet{DBLP:journals/corr/abs-1903-00982} formalize Rust's
notion of non-lexical lifetimes which allow to partially support such
code patterns. We believe this notion can be adapted to \lang's notion of
regions

\subsubsection{Non-Lexical Regions}

A recent important addition to Rust is the notion of non-lexical lifetimes.
With this feature code is acceptable even if borrowing does not respect
lexical scoping as in this example:

\begin{lstlisting}
let a = &x in
f a;
g (&!x)
\end{lstlisting}

Such code patterns are dynamically safe since $a$ is not used after
the second line.
Non-lexical lifetimes handle this by removing expressions
that do not mention $a$ from its region; in this example,  the
expression on the last line.
%
In \lang, regions are purely lexical and marked by the
expression $\region{b}{e}$.
During inference, we introduce kind constraints to prevent escaping from
a region.

To add support for non-lexical lifetimes, we could replace the
lexical region by an annotation on each expression indicating which borrows are
live in this expression.
When exploring a subexpression, we would compare the annotations, and automatically
apply the $\textsc{Region}$ rule when they differ.
%
This approach is equivalent to inlining the $\textsc{Region}$ rule in all the other
rules.

Applied to the program above, only the first two lines would be annotated
to be ``in the region associated with @&x@'', but not the last line.
Thanks to these annotations, when type checking the sequence we would check
that the borrow does not escape the left-hand side (i.e., the second line).

\subsection{Capabilities and Identity}

One primary feature of \lang is that its tracking of linearity does
not use any notion of ``identity'': the type system can't specify that two
objects are the same, simply that they share the same usage pattern with
regards to linearity.
Languages such as Alms~\citep{DBLP:conf/popl/TovP11}, on the other hand,
often relies on a notion of identity to express capabilities.
For instance, the Alms function
@Array.create@ of type @int -> 'a -> \E 'b. ('a, 'b) array@ uses
@'b@ to uniquely identifies the array.
Functions such as @Array.acquire : ('a, 'b) array -> 'b cap@
are then used to manipulate capabilities pertaining to the array.

While such uses are partially covered by borrows and regions,
a notion of identity associated to regions
would allow expressing regions directly in type signatures.
For instance, the @get_eb@ function shown in
\cref{tuto:vector} could be made safe
by creating a restricted inner region on function application,
with the signature:
@&!('k, 'a Array.t) -> int -> \E ('k' < 'k) &!('k', 'a)@


This notion of identity often relies on existential types.
\lang does not support existentials so far as it would
forgo principal type inference.
However existentials
are compatible with the HM(X) framework~\citep{DBLP:conf/icfp/Simonet03}
and would make a very desirable addition to \lang. OCaml also show
that existential types can be made usable without compromising
inference in the rest of the language, by integrating unpacking
and pattern matching.

\subsection{Ad-hoc Polymorphism and Borrows}

In our formalization, we use two operators, $\borrow{x}$ and $\reborrow{x}$ to
distinguish between borrows and borrows of borrows.
Such a distinction is inconvenient for programming.
Using a typeclass-like mechanism, we can replace these operators
by a single overloaded operator, $\borrow{x}$, which expects $x$ to be @Borrowable@ and
would then desugar to the more precise operators.
A similar solution is used in Rust through the @Borrow@ and @Defer@
traits.
This approach also enables method calls on objects without
explicit borrows, such as @foo.len()@ where len expects a shared borrow.

Ad-hoc polymorphism fits demonstrably in the \hmx framework of constrained
types and preserves all properties of our language such
as principal type inference. Its soundness is orthogonal to linear types
and has been explored in the literature~\citep{DBLP:conf/fpca/OderskyWW95}.

\subsection{Traditional functional features}

\subsubsection{Algebraic Datatypes}

Algebraic data types are a staple of functional programming and fit nicely
in our paradigm. Indeed, it is sufficient to ensure  that the kinds of
the constructor arguments are less than or equal to the kind of the datatype.
Hence, it is forbidden to include affine elements in an unrestricted
datatype, whereas the elements in a linear list may be linear or
unrestricted. 
Here is the definition of a datatype for linear lists.

\begin{lstlisting}
type ('a : 'k) llist : lin = Nil | Cons of ('a * 'a llist)
\end{lstlisting}

This extension is implemented in our prototype.

\subsubsection{Branching constructs}

Our formalization of \lang does not contain branching constructs. As a consequence,
subexpressions are always executed sequentially. This is of great importance
for the splitting relation introduced in \cref{sdtyping}, as it must
coincide with the order of evaluation. To account for borrows and linear
values in different branches, we need a new (symmetric) join relation.
This relation immediately follows from the semantics of borrows and linearity
shown so far, and allow to easily define branching constructs such as if-then-else
and pattern matching.
This extension is implemented in our prototype.

\lstDeleteShortInline@

%%% Local Variables:
%%% mode: latex
%%% TeX-master: "main"
%%% End:

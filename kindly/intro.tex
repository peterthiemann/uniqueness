\section{Introduction}

A large proportion of systems programming is focused on the proper
handling of resources, like file handles, network connections, or
dynamically allocated memory. Each of these resources comes with a
protocol that prescribes the correct use of its API.
For examples, a file handle appears as the result of opening a
file. If it was opened for reading, then read operations will succeed,
but write operations will fail. Once the handle is closed, it cannot
be used for reading or writing, anymore.
%
Dynamic allocation of memory is similar. An API call returns a
pointer to a memory area, which can then be read and written to until
the area is released by another API call.

In both cases, a resource is created in a certain state and a resource
handle is returned to the program. Depending on this state, certain API calls
can safely be applied to it. Finally, there is another API call to
release the resource, which renders the handle invalid.
Taken to the extreme, each API call changes the state so that a
different set of API calls is enabled afterwards.
Ignoring such life cycle protocols is a common source of errors.
%% this is about typestate \cite{DBLP:conf/ecoop/BeckmanKA11}.


Most type systems provide type soundness and memory safety, but neglect the
protocol aspect. Systems that can support reasoning about protocols
build on linear types \cite{DBLP:journals/tcs/Girard87} and/or 
uniqueness types~\cite{DBLP:conf/plilp/BarendsenS95}. A value of linear
type is guaranteed to be consumed
exactly once. That is, a file that has been opened must be closed and
memory that has been allocated must be released. A value of unique
type is guaranteed to have a single reference to it. Thus, memory can
be reused on consumption of the value. 

These systems work well if one is prepared to write programs
functionally in resource-passing style. In this style, all operations
in the resource's API take the resource as a parameter and return it
in possibly modified state~\cite{DBLP:journals/jfp/AchtenP95}. In
typestate-oriented programming, they would also modify its
type~\cite{DBLP:conf/oopsla/AldrichSSS09}. Functional session types
represent a popular
example~\cite{DBLP:journals/jfp/GayV10,lindley17:_light_funct_session_types}. 

Explicit resource passing places a heavy burden on the programmer and
complicates the program structure. For imperative APIs,
resource-passing style is not an option at all. To this end,
\citet{DBLP:conf/popl/BoylandR05}  proposed the notion of
\emph{borrowing} a resource. The idea is that a linear resource can be
borrowed to a function call. The function can work with a borrow of
the resource, but it cannot release the resource. Only the original
owner of the resource has all rights to it and can release it.

The concepts of ownership and borrowing have grown popular over time
and they form the foundation of the type system of the Rust language
\cite{rust}, which considers any memory-allocated data structure a
resource. Rust supports two kinds of borrows,
shared and exclusive ones.
Exclusive borrows enable modification of the data structure
whereas shared borrows only grant read access.
At any given time, either a single exclusive borrow is active or 
any number of shared borrows can be active.
Moreover, Rust makes sure that the lifetime of a
borrow is properly contained in the lifetime of its lender.

The design of Rust is geared towards programmers with a low-level
imperative programming background, like C or C++. Its management of
lifetimes supports the manual way of memory management customary in
these languages very well and makes it safe. However, programmers with
a background in managed languages feel alienated from the lack of garbage
collected data. They would prefer a setting where automatic memory
management with garbage collection was the default, but where they
could seemlessly switch to safe, manual resource management if that
was required.
%
As a concrete example, consider a functional programmer who wants to
safely interact with a C library. Invoking a C function is easy via
the existing foreign function interface, but managing the underlying
resources like malloc'd storage is not: It cannot be left to the
garbage collector, but proper release of the storage via calls to \texttt{free()} must be
ensured by programming conventions.

Our work provides a safe solution to programmers in this situation. We
propose an extended type system for ML-like languages that comes with
linear and affine types, exclusive and shared borrows, but all that
integrated with full principal type inference, garbage collected data, and
automatic placement of borrowing regions.
In our system, it is a type error to omit the call to release the
storage given a suitably typed API for storage allocation.

The most closely related contenders in this design space are Linear Haskell
\cite{DBLP:journals/pacmpl/BernardyBNJS18}, henceforth LH,
Quill \cite{DBLP:conf/icfp/Morris16}, and ALMS \cite{DBLP:conf/popl/TovP11}.
Compared to LH and Quill, the goals and means are
similar as these systems also permit abstraction over the number of uses of
values and retain type inference, but the details are different.
\begin{enumerate}
\item Multiplicities in LH and Quill are either linear or unrestricted whereas
  we also distinguish affine values.
\item In \lang{} and in Quill multiplicities are directly attached to the type of a
  value. For example, in \lang{} the function type \lstinline/'a-{lin}>'b/
  denotes the type of a \emph{single-use function} that can be called
  just once, whereas the multiplicities in LH choose
  between $\alpha\to\beta$ and $\alpha \multimap\beta$ where the
  latter is a function that promises to \emph{use its argument exactly
    once}.
\item \lang{} makes use of multiplicity contraints (like Quill) and kind
  subsumption (unlike Quill). Kind subsumption results in significantly simpler, more readable
  inferred types.
\item Neither LH nor Quill have borrowing whereas \lang{} supports
  two flavors: affine (exclusive, mutable) and unrestricted (shared, immutable) borrows.
\end{enumerate}
See \cref{sec:related-work} for further in-depth discussion of these
and other related works.

\subsection{First examples}
\label{sec:first-example}


\lstMakeShortInline[keepspaces,basicstyle=\small\ttfamily]@

\begin{figure}[tp]
  \begin{subfigure}[t]{0.53\linewidth}
    \lstinputlisting{code/writefiles.affe}
    \vspace{-15pt}
    \caption{File API}
    \label{fig:writing-files-api}
  \end{subfigure}~
  \begin{subfigure}[t]{0.43\linewidth}
\begin{lstlisting}
let main () =
  let h = File.fopen "foo" in
  File.write &!h "Hello ";(*@\label{line:hello}*)
  File.write &!h "world!";(*@\label{line:world}*)
  File.close h
\end{lstlisting}
    \vspace{-10pt}
    \caption{File example}
    \label{fig:writing-files-example}
  \end{subfigure}
  \vspace{-10pt}
  \caption{Writing files}
  \label{fig:writing-files}
\end{figure}



As a first, well-known example we consider a simplified API for
writing files shown in \cref{fig:writing-files-api}.  It introduces a
linear abstract type @File.t@. A call like
@File.fopen "foo"@ returns a linear handle to a newly created
file, which \emph{must} be released later on with @File.close@
as shown in \cref{fig:writing-files-example}. Failing to do so is a
static type error.  To write to the file, we must take an exclusive
borrow @&!h@ of the handle and pass it to the
@File.write@ function. Exclusive borrows are affine:
they must not be duplicated, but there is no requirement to use
them. This affinity shows up in the annotation @-{aff}>@ of the second arrow in
the type of @File.write@: a partial application like
@File.write &!h@ captures the affine borrow and hence the
resulting function is also affine. It
would be an error to use the affine closure twice as in
\begin{lstlisting}
let w = File.write &!h in w "Hello "; w "world!" (*type error*)
\end{lstlisting}
The remaining arrows in
the API are unrestricted and we write @->@ instead of the
explicitly annotated @-{un}>@.

Every borrow is restricted to a \emph{region}, i.e., a lexically
scoped program fragment from which the borrow must not escape. In
\cref{fig:writing-files-example}, there are two regions, one 
consisting of \cref{line:hello} and another consisting of
\cref{line:world}. Both are fully contained in the scope of the linear
handle @h@, hence we can take one exclusive borrow @&!h@ in each
region. In both regions the borrow is consumed immediately by passing
it to @File.write@. Regions are generally inferred by \lang.

This example demonstrates three features of our system:
\begin{enumerate}
\item type and region inference without annotations in user code (\cref{fig:writing-files-example}),
\item types carry multiplicity annotations in the form of kinds,
\item resource APIs can be written in direct style as linearity is a
  property of the type @File.t@.
\end{enumerate}

Direct style means that there is a function like @fopen@ that creates
and returns a linear resource. In contrast, LH forces programmers to
use resource-passing style because, in LH, linearity is a property of
a function, rather than a property of a value that restricts the way
that value can be handled (as in \lang). An LH API analogous to @File@
might provide functions like
\begin{itemize}
\item @withFile : path -> (handle o-> r) -> r@, which creates a new
  file handle and takes a continuation that uses the @handle@ linearly,
\item @writeFile : string -> handle o-> handle@, which
  returns the transformed resource @handle@, and
\item @closeFile : handle o-> unit@, which consumes the @handle@ by
  closing the file.
\end{itemize}

In general, kinds can be polymorphic and constrained. Function
application and composition are the archetypical
examples for functions exploiting that feature.\footnote{Compared to
  Quill \cite{DBLP:conf/icfp/Morris16} the signatures of application and
  composition are simpler because \lang{} supports kind subsumption.}
For application, \lang{} infers the following type.
\begin{lstlisting}
let app f x = f x
# app : ('a -{'k}> 'b) -> ('a -{'k}> 'b)
\end{lstlisting}
The reading of the type inferred by our system is straightforward. If
@f@ is a @'k@-restricted function, then so is
@app f@. The multiplicities of @'a@ and
@'b@ play no role. As usual in ML-like languages, we
implicitly assume prenex quantification by
$\forall\kappa\forall\alpha\forall\beta$. Internally, the
type checker also quantifies over the kinds of $\alpha$ and $\beta$,
but the full prefix
$\forall\kappa\kappa_1\kappa_2\forall(\alpha:\kappa_1)\forall(\beta:\kappa_2)$
of the type of @app@ is only revealed as much as necessary for
understanding the type. 

For @compose@, \lang{} infers this type.
\begin{lstlisting}
let compose f g x = f (g x)
# compose : ('k <= 'k_1) => ('b -{'k}> 'c) -> ('a -{'k_1}> 'b) -{'k}> ('a -{'k_1}> 'c)
\end{lstlisting}
Like in @app@, the multiplicities of the type variables
@'a,'b,'c@ do not matter. However, the multiplicity
@'k@ of @f@ reappears on the second to last arrow
because @compose f@ is a closure that inherits
@f@'s multiplicity. The multiplicities of @g@  and
@f@ both influence the multiplicity of the last arrow, so
we would expect its annotation to be the least upper bound
$\kappa \sqcup \kappa_1$. Thanks to subsumption of multiplicities, it
is sufficient to assume $\kappa \le \kappa_1$ and @g@'s
actual multiplicity gets subsumed to $\kappa_1$. This constraint
simplification is part of our type inference algorithm. As before,
printing the type scheme only mentions the required constraint 
$\kappa\le\kappa_1$ and omits the prenex quantification over $\kappa,
\kappa_1$ as well as the kinds of @'a,'b,'c@.

\lstDeleteShortInline@

\subsection{Contributions}
\label{sec:contributions}

\begin{itemize}
\item A type system encoding linearity and affinity with
  borrowing. The type system is polymorphic for types and
  for kinds that express constraints on the number of uses of a value.
\item An extension of the \hmx framework
  \cite{DBLP:journals/tapos/OderskySW99} with kinds, regions, and
  multiplicity constraints that supports full, principal type inference.
\item Soundness proof of the inference algorithm.
\item Big-step linearity aware semantics with an expressive soundness theorem.
\item Constraint simplification techniques.
\item Automatic inference of regions for borrows.
\item A prototype implementation of the type inference algorithm, including all
  constraint simplification and extended with algebraic datatypes and
  pattern matching,
  available at \url{https://affe.netlify.com/}\footnote{This link is safe for double-blind.}.
\end{itemize}

As \lang{} is built on top of the \hmx{} framework, which is a general
framework for expressing constraint-based typing and type inference,
the extension of our work with features like typeclasses, ad-hoc overloading,
traits, etc is possible and orthogonal to the presentation in this paper. 
While the system is geared towards type inference, it is nonetheless
compatible with type annotations and thereby amenable to extensions
where type inference may no longer be possible.

%%% Local Variables:
%%% mode: latex
%%% TeX-master: "main"
%%% End:

\section{Introduction}

A large proportion of systems programming is focused on the proper
handling of resources, like file handles, network connections, or
dynamically allocated memory. Each of these resources comes with a
protocol that prescribes the correct use of its API.
For examples, a file handle appears as the result of opening a
file. If it was opened for reading, then read operations will succeed,
but write operations will fail. Once the handle is closed, it cannot
be used for reading or writing, anymore.

Dynamic allocation of memory is analogous. An API call returns a
pointer to a memory area, which can then be read and written to until
the memory is released by another API call.

In both cases, a resource is created in a certain state and a resource
handle is returned to the program. Depending on this state, certain API calls
can safely be applied to it. Finally, there is another API call to
release the resource, which yields the handle invalid.
Taken to the extreme, each API call changes the state so that a
different set of API calls is enabled afterwards.
Ignoring such life cycle protocols is a common source of errors.
%% this is about typestate \cite{DBLP:conf/ecoop/BeckmanKA11}.


Most type systems provide just type soundness, but neglect the
protocol aspect. The systems that do build on linear types \cite{DBLP:journals/tcs/Girard87} and/or
uniqueness types~\cite{DBLP:conf/plilp/BarendsenS95}. A value of linear
type is guaranteed to be consumed
exactly once. That is, a file that has been opened must be closed and
memory that has been allocated must be released. A value of unique
type is guaranteed to have a single reference to it.

These systems work well if one is prepared to write programs
functionally in resource-passing style. In this style, all operations
in the resource's API take the resource as a parameter and return it
in possibly modified state~\cite{DBLP:journals/jfp/AchtenP95}. In
typestate-oriented programming, they would also modify its
type~\cite{DBLP:conf/oopsla/AldrichSSS09}. Functional session types
represent a popular example~\cite{DBLP:journals/jfp/GayV10}.

Explicit resource passing places a heavy burden on the programmer and
complicates the program structure. For imperative APIs,
resource-passing style is not an option at all. To this end,
\citet{DBLP:conf/popl/BoylandR05}  proposed the notion of
\emph{borrowing} a resource. The idea is that a linear resource can be
borrowed to a function call, which can work with the resource, but
cannot release it. Only the original owner of the resource has all
rights to it and can release it.

The concepts of ownership and borrowing have grown popular over time
and they form the foundation of the type system of the Rust language
\cite{rust}. Rust supports two kinds of borrows of data structures, shared and
exclusive ones.
Exclusive borrows generally enable modification of the data structure
whereas shared borrows only grant read-only access.
Several shared borrows can be active at the same time, but
can not be active while an exclusive borrow is.
Moreover, Rust makes sure that the lifetime of a
borrow is properly contained in the lifetime of its lender.

The design of Rust is geared towards programmers with a low-level
imperative programming background, like C or C++. Its management of
lifetimes supports the manual way of memory management customary in
these languages very well and makes it safe. However, programmers with
a background in managed languages feel alienated from the lack of garbage
collected data. They would prefer a setting where automatic memory
management with garbage collection was the default, but where they
could seemlessly switch to safe, manual resource management if that
was required.

As a concrete example, consider a functional programmer who wants to
safely interact with a C library. Invoking the C functions is easy via
the existing foreign function interface, but managing the underlying
resources like malloc'd storage is not: It cannot be left to the
garbage collector, but the proper calls to \texttt{free()} must be
ensured by programming conventions.

Our work provides a solution to programmers in this situation. We
propose an extended type system for ML-like languages that comes with
linear and affine types, mutable and immutable borrows, but all that
integrated with full principal type inference, garbage collected data, and
automatic placement of borrowing regions.

The most closely related contenders in this design space are Linear Haskell
\cite{DBLP:journals/pacmpl/BernardyBNJS18}, henceforth called LH,
Quill \cite{DBLP:conf/icfp/Morris16}, and ALMS \cite{DBLP:conf/popl/TovP11}.
Compared to LH, the goals and means are
similar as both systems permit abstraction over the number of uses of
values and retain type inference, but the details are quite different.
\begin{enumerate}
\item LH's multiplicities are either linear or unrestricted whereas
  we also permit affine values.
\item Our system supports mutable and immutable borrows, hence the
  need for affinity.
\item Our multiplicities are directly attached to the type of the
  value. For example, in our system the function type \lstinline/'a-{lin}>'b/
  denotes the type of a \emph{single-use function} that can be called
  just once, whereas the multiplicities in LH choose
  between $\alpha\to\beta$ and $\alpha \multimap\beta$ where the
  latter is a function that promises to \emph{use its argument exactly
    once}.
\item Similar to Quill, we make use of multiplicity contraints. We
  rely on kind subsumption to simplify the inferred types.
\end{enumerate}
See \cref{sec:related-work} for further in-depth discussion of these
and other related works.
\subsection{First examples}
\label{sec:first-example}
\begin{figure}[tp]
  \begin{subfigure}[t]{0.45\linewidth}
    \lstinputlisting{code/writefiles.affe}
    \vspace{-15pt}
    \caption{File API}
    \label{fig:writing-files-api}
  \end{subfigure}
  ~
  \begin{subfigure}[t]{0.5\linewidth}
\begin{lstlisting}
let main () =
  let h = File.fopen "foo" in
  File.write &!h "Hello ";
  File.write &!h "world!";
  File.close h
\end{lstlisting}
    \vspace{-10pt}
    \caption{File example}
    \label{fig:writing-files-example}
  \end{subfigure}
  \vspace{-10pt}
  \caption{Writing files}
  \label{fig:writing-fiules}
\end{figure}

As a first, well-known example we consider a simplified API for
writing files shown in \cref{fig:writing-files-api}.  It introduces a
linear abstract type \lstinline/File.t/. A call like
\lstinline/fopen "foo"/ returns a linear handle to a newly created
file, which \emph{must} be released later on through \lstinline/close/
as shown in \cref{fig:writing-files-example}. Failing to do so is a
static type error.  To write to the file, we must take a (mutable)
borrow \lstinline/&!h/ of the handle and pass it to the
\lstinline/write/ function. Mutable borrows are classified as affine:
they must not be duplicated, but there is no requirement to use
them. This affinity shows up in the annotation \lstinline/-{aff}>/ of the second arrow in
the type of \lstinline/write/: a partial application like
\lstinline/write &!h/ captures the affine borrow and hence the
resulting function must also be handled in an affine way. That is, it
would be an error to write
\begin{lstlisting}
let w = File.write &!h in w "Hello "; w "world!" (*error*)
\end{lstlisting}
as we would be using the affine closure twice. The remaining arrows in
the API are unrestricted and we write \lstinline/->/ instead of the
explicitly annotated \lstinline/-{un}>/.


This example demonstrates three features of our system:
\begin{enumerate}
\item full type inference without annotations in user code,
\item types carry multiplicity annotations in the form of kinds,
\item resource APIs can be written in direct style as linearity is a
  property of the type \lstinline/File.t/.
\end{enumerate}

In general, kinds can be polymorphic and constrained. Function
application and composition are the archetypical
examples for functions exploiting that feature.
\begin{lstlisting}
let app f x = f x
# app : ('a -{'k}> 'b) -> ('a -{'k}> 'b)
\end{lstlisting}
The reading of the type inferred by our system is straightforward. If
\lstinline/f/ is a \lstinline/'k/-restricted function, then so is
\lstinline/app f/. The multiplicities of \lstinline/'a/ and
\lstinline/'b/ play no role. As usual in ML-like languages, we
implicitly assume prenex quantification by
$\forall\kappa\forall\alpha\forall\beta$. In reality, the
type checker also quantifies over the kinds of $\alpha$ and $\beta$,
hence the full prefix is
$\forall\kappa\kappa_1\kappa_2\forall(\alpha:\kappa_1)\forall(\beta:\kappa_2)$,
which is only revealed as much as necessary for understanding the type.
\begin{lstlisting}
let compose f g x = f (g x)
# compose : ('k <= 'k_1) => ('b -{'k}> 'c) -> ('a -{'k_1}> 'b) -{'k}> ('a -{'k_1}> 'c)
\end{lstlisting}
Like in \lstinline/app/, the multiplicities of the type variables
\lstinline/'a,'b,'c/ do not matter. However, the multiplicity
\lstinline/'k/ of \lstinline/f/ reappears on the second to last arrow
because \lstinline/compose f/ is a closure that inherits
\lstinline/f/'s multiplicity. The multiplicities of \lstinline/g/  and
\lstinline/f/ both influence the multiplicity of the last arrow, so
that we would expect its annotation to be the least upper bound
$\kappa \sqcup \kappa_1$. Thanks to subsumption of multiplicities, it
is sufficient to assume $\kappa \le \kappa_1$ and \lstinline/g/'s
actual multiplicity gets subsumed to $\kappa_1$. This constraint
simplification technique is built into our type inference algorithm. Again, the printout
of the type scheme only mentions the required constraint
$\kappa\le\kappa_1$ and omits the prenex quantification over $\kappa,
\kappa_1$ as well as the kinds of \lstinline/'a,'b,'c/.

\subsection{Contributions}
\label{sec:contributions}

\begin{itemize}
\item A type system that encodes linearity and affinity with
  borrowing. The type system is polymorphic with respect to types as
  well as kinds that express constraints on the number of uses of a value.
\item An extension of the HM(X) framework
  \cite{DBLP:journals/tapos/OderskySW99} with kinds, regions, and
  multiplicity constraints that supports full, principal type inference.
\item Full account of the metatheory for type inference and type soundness.
\item Constraint simplification techniques.
\item Automatic inference of regions for borrows.
\item An implementation of the type inference algorithm including all
  constraint simplification and extended with pattern matching. A
  web implementation of the typechecker is available, its URL is given in the
  non-anonymous supplementary material, and we plan to submit it for
  artifact evaluation.
\end{itemize}

%%% Local Variables:
%%% mode: latex
%%% TeX-master: "main"
%%% End:

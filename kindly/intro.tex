\section{Introduction}

A large proportion of systems programming is focused on the proper
handling of resources, like file handles, network connections, or
dynamically allocated memory. Each of these resources comes with a
protocol that prescribes the correct use of its API.
For examples, a file handle appears as the result of of opening a
file. If it was opened for reading, then read operations will suceed,
but write operations will fail. Once the handle is closed, it cannot
be used for reading or writing, anymore.

Dynamic allocation of memory is analogous. An API call returns a
pointer to a memory area, which can then be read and written to until
the memory is released by another API call.

In both cases, a resource is created in a certain state and a resource
handle is returned to the program. Depending on this state, certain API calls
can safely be applied to it. Finally, there is another API call to
release the resource, which yields the handle invalid.
Taken to the extreme, each API call changes the state so that a
different set of API calls is enabled afterwards. 

Ignoring this protocol is a common source of errors \cite{is there
  some paper?}.

Most statically typed languages do not address this issue.

But linear types do.

But typestate does it. A popular example is session types.

Typestate needs to be supported by affinity (or linearity, or
uniqueness).

But these approaches are restrictive in pratice because [here is an
example that doesn't work well with existing systems].

%%% Local Variables:
%%% mode: latex
%%% TeX-master: "main"
%%% End:

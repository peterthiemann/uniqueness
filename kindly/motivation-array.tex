
\subsection{Imperative programming with linear arrays}
\label{sec:imper-progr}

The API for mutable linear arrays (shown in \cref{sig:array})
aims to safely handle manual allocation and
deallocation of arrays that may contain affine elements.
One would first use @create n v@ to create
an (unrestricted) array of size @n@ initialized with value
@v@. This value must be unrestricted as it is duplicated to
initialize all array elements. Using the @map@ function we
can transform the unrestricted elements into linear (affine)
ones. This array can be processed further using @map@ (or @set@ if
affine). The @get@ function is only applicable if the element type
is unrestricted as one element is duplicated. The @length@
function is always applicable. To @free@ an array the elements must be
affine.

To manage the different kinds of accessing the array we distinguish between constructors, destructors,
observers, and mutators.
Constructors and destructors like @create@ and @free@ manipulate the whole
array. %As with channels in \cref{sec:session-linearity},
The constructor
@create@ yields a linear resource which is consumed by @free@.
During the lifetime of the array resource @a@, we can split
off \emph{shared borrows} @&a@ that provide a read-only
view or \emph{exclusive borrows} @&!a@ for read-write views.
Observer functions such as @length@ and @get@ expect an shared borrow
argument, mutator functions such a @set@ expect
a exclusive borrow.

Each borrow is tied to a region whose lifetime is properly
contained in the lifetime of the resource.
In a region, we can split off as many shared borrows of a resource as
we like, but we can take only one exclusive borrow. In a 
subsidiary region, we can take shared borrows of any
borrow or we can take an exclusive borrow of an exclusive borrow from an
enclosing region. Borrows are confined to their regions. Inside the region,
shared borrows are unrestricted (@un@) whereas exclusive
borrows are affine (@aff@).

The code in \cref{ex:array} demonstrates a use of the API to create an
array of Fibonacci numbers as the programmer would write it. After
creation of the array, the presence of a borrow in the for loop
prevents access to the ``raw'' resource inside the loop's
body. \cref{ex:array:get} of the example contains two shared borrows
in the same expression, which forms a region by itself. These borrows
are split off the exclusive borrow used in \cref{ex:array:set}, which
belongs to the next enclosing region corresponding to the loop body.
The whole array can be returned in \cref{ex:array:return} because  the
borrows are no longer in scope. 

\begin{figure}[tp]
  \centering
  \begin{subfigure}[t]{1\linewidth}
\begin{lstlisting}
module type ARRAY = sig
  type ('a : 'k) t : lin
  val create : ('a : un) => int -> 'a -> 'a t
  val free : ('a : aff) => 'a t -> unit
  val length : &('a t) -> int
  val get : ('a : un) => &('a t) * int -> 'a
  val set : ('a : aff) => &!('a t) * int * 'a -> unit
  val map : (&'a -> 'b) -> &('a t) -> 'b t
end

module Array : ARRAY
\end{lstlisting}
    \vspace{-15pt}
    \caption{Signature}
    \label{sig:array}
  \end{subfigure}

  \begin{subfigure}[t]{1\linewidth}
\begin{lstlisting}
let mk_fib_array n =
  let a = create n 1 in
  for i = 2 to n - 1 do
    let x = get (&a, i-1) + get (&a, i-2) in(*@\label{ex:array:get}*)
    set (&!a, i, x)(*@\label{ex:array:set}*)
  done;
  a(*@\label{ex:array:return}*)
# mk_fib_array : int -> int Array.t
\end{lstlisting}
    \vspace{-10pt}
    \caption{Example of use}
    \label{ex:array}
\begin{lstlisting}[firstnumber=2]
  let a = create n 1 in
  for i = 2 to n - 1 do {|
    let x = {| get &a (i-1) + get &a (i-2) |} in(*@\label{ex:array:region:get}*)
    set &!a i x
  |} done;
\end{lstlisting}
    \vspace{-10pt}
    \caption{Excerpt of \cref{ex:array} with explicit regions}
    \label{ex:array:region}
  \end{subfigure}
  \vspace{-5pt}
  \caption{Linear arrays}
  \label{ex:array}
\end{figure}

\cref{ex:array:region} uses braces @{| ... |}@ to make the
regions of the example in \cref{ex:array} explicit.  One
region consists of the header expression of the @let@ in
\cref{ex:array:region:get}. It is contained in another region that
comprises the body of the @for@ loop. \lang guarantees that borrows
never escape the smallest enclosing region. It employs a system of
\emph{indexed kinds} like @aff_r@ and @un_r@ where
$r$ is a positive integer that corresponds to the lexical nesting
depth of regions. For instance, the type of @&!a@ in
\cref{ex:array:set} has kind @aff_1@ whereas the type of
@&a@ in \cref{ex:array:region:get} has kind
@un_2@ and the typing of the inner region is such that types with
kind indexes greater than or equal to $2$ cannot escape.
In the example, borrows cannot escape  because they are consumed
immediately by @get@ and @set@.

Region annotations are automatically inserted using
syntactic cues such as borrows and binders (see
\cref{regionannot}).  Users may manually insert regions to further restrict the range of a
borrow. In practice, region annotations are
rarely needed with common programming idioms.
% All regions for the examples in this section have been inferred.


% \begin{figure}
%   \centering
%   % \begin{subfigure}{0.6\linewidth}
% \begin{lstlisting}
% let mk_fib_array n =
%   let a = create n 1 in
%   for i = 2 to n - 1 do {|
%     let x = {| get &a (i-1) + get &a (i-2) |} in(*@\label{ex:array:region:get}*)
%     set &!a i x
%   |}
%   done;
%   a(*@\label{ex:array:return}*)
% # mk_fib_array : int -> int Array.t
% \end{lstlisting}
%     \caption{Example from \cref{ex:array} with explicit regions}
%     \label{ex:array:region}
%   % \end{subfigure}
% \end{figure}


%%% Local Variables:
%%% mode: latex
%%% TeX-master: "main"
%%% End:

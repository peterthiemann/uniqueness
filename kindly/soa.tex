\section{Related work}

\subsection{Substructural type systems}

Haller, P., Odersky, M.: Capabilities for Uniqueness and Borrowing
\cite{DBLP:conf/ecoop/HallerO10}

Sing\#

John Tang Boyland and William Retert. Connecting effects and
uniqueness with adoption. In POPL, pages 283–295. ACM, 2005.
\cite{DBLP:conf/popl/BoylandR05}

Manuel Fahndrich and Robert DeLine. Adoption and focus: Practical
linear types for imperative programming. In PLDI, pages 13–24, 2002.

\subsubsection{Functional linear of affine type systems}

Many systems have attempted to combine
functional programming and linear types in a practical setting.
One contribution of \lang is to combine several ingredients
from these different languages and combines them, while still preserving
complete type inference.
In addition, none of the following language supports borrows.

System F\degree~\citep{DBLP:conf/tldi/MazurakZZ10}
extends System F with uses kinds to distinguish
between linear and unrestricted types.
\citet{DBLP:conf/tldi/MazurakZZ10} also provide
(and proves soundness) of a linearity-aware semantics.
Unlike \lang, System F\degree does not allow
quantification over kinds which limits its expressivity. For instace, it
does not admit a most general type for function composition.
Furthermore, since it is based on System F, it does not admit
principal type inference.

Quill~\citep{DBLP:conf/icfp/Morris16} is an Haskell-like language with affine
types through the use of qualified types.
Quill does not exposes a kind language, but
uses typeclass-like mechanism and annotation on arrows.
It supports both a most general type for function composition and
principal type inference.
The combination of type and kind schemes
and subkinding ocasionally yields significantly simpler
type in \lang and we posit it strictly subsumes Quill's type system.
Unlike Quill,
we provide a linearity-aware semantics and prove its soundness.
Quill does not support borrows.

% For instance, the type of the constructor
% in Quill is $\qual{t \geq f}{t \to u \to t * u}$.
% In Affe, it is simply
% $\qual{(\alpha:\kvar)\implies \alpha \to \beta \tarr{\kvar} \alpha * \beta$
% with the kind of $*$ being $\kvar\to\kvar\to\kvar$.

Alms~\citep{DBLP:conf/popl/TovP11} is an ML-like language with rich, kind-based
affine types and supports abstraction through modules.
It relies on meet and join operators instead of qualified types, but still
uses subkinding.
Their system often relies on existential types to track the identity
of objects. For instance, they present an array creation
functions of the type \lstinline/int -> 'a -> \E 'b. ('a, 'b) array/ where
\lstinline/'a/ are the types of the elements in the array and \lstinline/'b/
is a type variable that uniquely identifies the array. Unfortunately, due
to the reliance on existential types, Alms does not support type inference.
Furthermore, Alms does not support borrow and often rely
on manual passing of capabilities.
In our experience, \affe's limited support for existential types through
regions is sufficient for most of their examples.
%
Apart from existential types, Alms' kind system is very complex and
rely on union and intersection operators and dependent kinds. We believe
such complexity is unnecessary and that any Alms' kind signature
can be translated into \lang. For instance the pair type constructor
has kind $\Pi\alpha\Pi\beta. \langle\alpha\rangle \sqcup \langle\beta\rangle$
(where $\alpha$ and $\beta$ are types and $\Pi$ is the dependent function)
in Alms, but simply $\kvar\to\kvar\to\kvar$ in \lang.


\paragraph{Linear Haskell~\citep{DBLP:journals/pacmpl/BernardyBNJS18}}

\paragraph{Vault\citep{DBLP:conf/pldi/DeLineF01,DBLP:conf/pldi/FahndrichD02}}

\subsubsection{Borrows}

Rust, Non Lexical lifetimes

\subsubsection{Object systems}
\TODO{}

Plaid \&Co by Aldrich and others

\subsubsection{Uniqueness}

\subsubsection{Others}

\paragraph{Mezzo}

\subsection{Qualified Types}

HM(X)~\citep{DBLP:journals/tapos/OderskySW99} with extensions to
bounded universal and existentials~\citep{DBLP:conf/icfp/Simonet03}
and GADTs~\citep{DBLP:journals/toplas/SimonetP07},
and syntactic proof of soundness~\citep{DBLP:journals/entcs/SkalkaP02}.

\subsection{Subtyping}



Flowcaml~\citep{DBLP:conf/popl/PottierS02},
\citet{DBLP:conf/sas/TrifonovS96}.


%%% Local Variables:
%%% mode: latex
%%% TeX-master: "main"
%%% End:

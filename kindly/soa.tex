\section{Related Work}
\label{sec:related-work}

% Haller, P., Odersky, M.: Capabilities for Uniqueness and Borrowing
% \cite{DBLP:conf/ecoop/HallerO10}

% Sing\#

% John Tang Boyland and William Retert. Connecting effects and
% uniqueness with adoption. In POPL, pages 283–295. ACM, 2005.
% \cite{DBLP:conf/popl/BoylandR05}

\newcommand\YC{\color{Green}}
\newcommand\NC{\color{Red}}
\newcommand\MC{\color{Orange}}
\newcommand\Y{{\YC{\ding{52}}}\xspace}
\newcommand\N{{\NC{\ding{56}}}\xspace}
\newcommand\M{{\MC{\textasciitilde}}\xspace} % "Meh"
\begin{figure}[tp]
  \begin{center}

    % Little bit of magic to make angled headers
    \newcolumntype{R}[2]{%
      >{\adjustbox{angle=#1,lap=\width-(#2)}\bgroup}%
      c%
      <{\egroup}%
    }
    \newcommand*\rot{\multicolumn{1}{R{45}{1em}}}
    
    \begin{tabular}{l|c@{~}cccccccccc}
      \textbf{Language}
      & UAL
      & \rot{State}
      & \rot{Borrows}
      & \rot{Multiplicity Subsumption}
      & \rot{Multiplicity Polymorphism}
      & \rot{Identity}
      & \rot{Concurrency}
      & \rot{Escape hatch}
      & \rot{Inference} & \rot{Formalisation}
      & \rot{Basis}
      \\\hline
      System F\degree%~\citep{DBLP:conf/tldi/MazurakZZ10}
      & UL & \N & \N & \N & \N & \N & \N & \N & \N & \YC Coq & System F
      \\
      Alms%~\citep{DBLP:conf/popl/TovP11}
      & UA & \Y & \N & \M & \Y & \Y & \Y & \Y & \MC Local & \N & ML \\
      Quill%~\citep{DBLP:conf/icfp/Morris16}
      & UL & \N & \N & \N & \Y & \N & \N & \N & \YC Principal & \MC Manual & Qual. types\\
      Lin. Haskell%~\citep{DBLP:journals/pacmpl/BernardyBNJS18}
      & UL & \M & \N & \N & \Y & \N & \M & \M & \MC Non-pr. & \MC Manual & Haskell \\
      Mezzo%~\citep{DBLP:phd/hal/Protzenko14,DBLP:journals/toplas/BalabonskiPP16}
      & UA & \Y & \M & \M & \Y & \Y & \Y & \Y & \MC Local & \YC Coq & ML \\
      \hline
      Rust%~\citep{rust,DBLP:journals/pacmpl/0002JKD18}
      & UA & \Y & \Y & \Y & \M & \N & \Y & \Y & \MC Local & \YC Coq & --- \\
      Plaid%~\citep{DBLP:conf/oopsla/AldrichSSS09,DBLP:journals/toplas/GarciaTWA14}
      & UA & \Y & \N & \Y & \Y & \Y & \N & \Y & \N & \MC Manual & Java \\
      \hline
      \lang
      & UAL & \Y & \Y & \Y & \Y & \N & \N & \M & \YC Principal & \MC Manual & ML/HM(X)
      \\
    \end{tabular}
  \end{center}
  \caption{Comparison matrix}
  \label{fig:comparison-matrix}
\end{figure}

The comparison matrix in \cref{fig:comparison-matrix} gives an
overview over the systems discussed in this section.
Each column indicates whether a feature is present (\Y), absent (\N), or
partially supported (\M), i.e., if the feature is limited or can
only be obtained through a non-trivial encoding.
Features are selected according to their relevance for type-based resource
management and programmer convenience.

The column UAL
specifies the substructural features supported (Unrestricted, Affine,
Linear). The columns ``State'' and ``Borrows'' indicate support for the respective
feature. In an ideal world, the presence of linearity and state
indicates that the system is able to support safe manual memory
management as linearity enforces manual deallocation. True affinity
and state only works with garbage collection, which eventually
automatically finalizes an object no longer referenced. In practice, this
distinction is often watered down. For example, Rust automatically
destructs objects at the end of their lifetime, creating the
illusion of affinity while the low-level code is strictly linear.
However, there are ways to consume an object at the source
level without invoking its destructor (using
\lstinline/mem::forget/)\footnote{See
  \url{https://doc.rust-lang.org/nomicon/leaking.html} which contains
  further examples and discussion. Thanks to
  Derek Dreyer and Ralf Jung for pointing this out.}
where the high-level code exhibits linearity, but the low-level code
is affine.

``Multiplicity Subsumption'' indicates that unrestricted elements 
can be promoted to affine and then linear. This promotion applies to
objects, resources, borrows, and closures. ``Multiplicity Polymorphism'' refers to polymorphism over
substructural features: a function can be parameterized over the multiplicity
restriction of an object. For instance, the type of function
composition should express that applies to functions with linear,
affine, and unrestricted multiplicity and returns a function with the
same multiplicity. 
``Identity'' indicates that the language supports a notion
of identity, usually through existential types, as described in
\cref{identity}.
``Concurrency'' indicates whether the language supports
concurrency. For example, the implementation of Linear Haskell
supports state and concurrency, but its theory covers neither.
``Escape hatch'' indicates whether a programmer can (locally) opt out of
resource management through language-integrated means such
as Rust's \texttt{unsafe}. Partial support ``\M'', in the case of \lang
for instance, indicates that this feature is available, but not formalized.
Type ``inference''  can be local, principal, or non-principal (if the
inferred type is not necessarily the most general one). ``Formalization''
refers to the existence of a formal semantics and type soundness
proof. ``Basis'' indicates the heritage or inspiration of the language.

\subsection{Substructural Type-systems in Functional Languages}

Many systems propose combinations of
functional programming and linear types in a practical setting.
The goal of \lang is to combine key ingredients
from these proposals while still preserving
complete type inference.
Many of the following languages support linear or affine types, but rarely
both. In many cases, it is easy to adapt a system to support both, as
\lang does.
None of the following languages support borrows.


System F\degree~\citep{DBLP:conf/tldi/MazurakZZ10}
extends System F with kinds to distinguish
between linear and unrestricted types.
The authors provide
a linearity-aware semantics with a soundness proof.
Unlike \lang, System F\degree{} does not allow
quantification over kinds which limits its expressivity. For instance, it
does not admit a most general type for function composition.
Being based on System F, it does not admit
principal type inference.

Quill~\citep{DBLP:conf/icfp/Morris16} is a Haskell-like language with linear
types.
Quill does not expose a kind language, but
uses the framework of qualified types to govern linearity annotations on arrows.
Its type inference algorithm is proven sound and complete.
\lang infers type signatures for all Quill examples, but often with
simpler types because Quill does not support subkinding.
Quill comes with a linearity-aware semantics and soundness proof.
Quill does not support borrows.

% For instance, the type of the constructor
% in Quill is $\qual{t \geq f}{t \to u \to t * u}$.
% In Affe, it is simply
% $\qual{(\alpha:\kvar)\implies \alpha \to \beta \tarr{\kvar} \alpha * \beta$
% with the kind of $*$ being $\kvar\to\kvar\to\kvar$.

Alms~\citep{DBLP:conf/popl/TovP11} is an ML-like language with rich, kind-based
affine types and ML modules, similar to \lang.
Alms examples often rely on existential types to track the identity
of objects. For instance, consider the signature
\lstinline/Array.create : int -> 'a -> \E 'b. ('a, 'b) array/ where
\lstinline/'b/ uniquely identifies the array.
Due to the reliance on existentials, Alms does not support complete type inference.
Furthermore, Alms does not support borrows and often relies
on explicit capability passing.
In our experience, \affe's limited support for existential types through
regions is sufficient to express many of Alms' examples and leads to
a more convenient programming style for imperative code.
%
Alms kind structure features unions, intersections and dependent kinds while
\lang uses constrained types.
We believe most of Alms' kind signatures can be expressed equivalently in
our system: for instance the pair type constructor
has kind $\Pi\alpha\Pi\beta. \langle\alpha\rangle \sqcup \langle\beta\rangle$
(where $\alpha$ and $\beta$ are types and $\Pi$ is the dependent function)
in Alms compared to $\kvar\to\kvar\to\kvar$ in \lang thanks
to subkinding.
%
Finally, Alms provides excellent support for abstraction through
modules by allowing to keep some type unrestricted inside a module, but
exposing it as affine. \lang supports
such programming style thanks to subsumption.

The goal of Linear Haskell~\citep{DBLP:journals/pacmpl/BernardyBNJS18}
(LH) is to retrofit linear types to Haskell.
Unlike the previously discussed approaches, LH relies on ``linear
arrows'', written $\multimap$ as in linear logic, 
which are functions that \emph{use} their argument exactly once.
This design is easy to retrofit on top of an existing compiler
such as GHC, but has proven quite controversial\footnote{
  See the in-depth discussion attached to the GHC proposal for LH on GitHub: \url{https://github.com/ghc-proposals/ghc-proposals/pull/111\#issuecomment-403349707}.}.
Most relevant to \lang:
\begin{itemize}[leftmargin=*]
\item LH does not admit subtyping for arrows and requires
  $\eta$-expansion to pass unrestricted functions in linear
  contexts. This approach is acceptable in a non-strict language such as
  Haskell but changes the semantics in a strict setting.
\item
  While the LH paper specifies a full type system along with a
  linearity-aware soundness proof, there is neither formal description of
  the type inference algorithm nor a proof of the properties of inference.
  Subsequent work~\cite{DBLP:journals/corr/abs-1911-00268}
  formalizes the inference for rank 1 qualified-types.
  However, there is an implementation of the inference as part of GHC.
\item
  LH promotes a continuation-passing style with functions such as
  \lstinline/withFile : path -> (file ->. Unrestricted r) ->. r/
  to ensure linear use of resources. This style leads to problems with
  placing the annotation on, e.g., the IO monad.
  \lang follows System F\degree, Quill, and Alms, all of which support
  resource handling in direct style, where types themselves are
  described as affine or linear. (Of course, continuation-passing
  style is also supported.)
  We expect that the direct approach eases modular reasoning about linearity.
  In particular, using abstraction through modules,
  programmers only need to consider the module
  implementation to ensure that linear
  resources are properly handled.
% \item
%   Linear Haskell introduces a notion of linear monad to express
%   imperative code conveniently. Again, this solution is suitable in Haskell,
%   but less appropriate in other contexts. Our borrow system allow
%   to provide a more free-form programming style for imperative code.
% \item
%   Due to the previous remark, ensuring that a given object is never
%   aliased is fairly difficult in Linear Haskell, as one would need to ensure
%   that no non-linear function can manipulate it. On the other
\end{itemize}

Mezzo~\citep{DBLP:phd/hal/Protzenko14,DBLP:journals/toplas/BalabonskiPP16} is an ML-like language
with a rich capability system which is able to encode numerous
properties akin to separation logic~\citep{DBLP:conf/lics/Reynolds02}.
Mezzo explores the  boundaries of the design space of type systems for
resources. Hence, it is more expressive than \lang, but
much harder to use. The Mezzo typechecker relies on explicit
annotations and it is not known whether type inference for Mezzo is possible.

\citet{DBLP:journals/corr/abs-1803-02796} presents
an extension of OCaml for resource management in the style of C++'s RAII
and Rust's lifetimes. This system assumes
the existence of a linear type system and develops the associated compilation
and runtime infrastructure. We believe our approach is
complementary and aim to combine them in the future.

\subsection{Other substructural type-systems}

% Linear and affine disciplines have  been used in non-functional
% settings, notably in the context of low-level imperative programming
% and object systems. In particular,
\lang uses borrows and regions
which were initially developed in the context of linear and affine
typing for  imperative and
object-oriented
programming~\citep{DBLP:conf/popl/BoylandR05,DBLP:conf/pldi/GrossmanMJHWC02}.

Rust~\citep{rust} is the first
mainstream language that builds on the concepts of borrowing and ownership
to enable safe low-level programming.
\lang is inspired by Rust's borrowing system and transfers some of its
ideas  to a functional setting with type inference, garbage collection, and
an ML-like module system.
Everything is affine in Rust and marker traits like \lstinline/Copy/,
\lstinline/Send/, and \lstinline/Sync/ are used to modulate the characteristics 
of types. \lang relies on kinds to express substructural properties of
types and marker traits may be considered as implementing a
fine-grained kind structure on Rust types.
Rust's lifetime system is more explicit and more expressive than
\lang's regions.
While Rust provides partial lifetime inference, it
does not support full type inference.
Moreover, Rust programmers have full control over memory allocation
and memory layout of objects; they can pass arguments by value or by reference.
These features are crucial for the efficiency goals of Rust.
In contrast, \lang is garbage collected, assumes a uniform object
representation, and all arguments are passed by reference. This choice
forgoes numerous  
issues regarding interior mutability and algebraic data types.
In particular, it
allows us to easily nest mutable references inside objects, regardless
whether they are linear or unrestricted.

In Rust, programmers can implement their low-level abstractions by
using unsafe code fragments. Unsafe code is not typechecked with the
full force of the Rust type system, but with a watered down version
that ignores ownership and lifetimes. This loophole is needed to implement
datastructures like doubly-linked lists or advanced concurrency
abstractions. When unsafe code occurs as part of a 
function body, the Rust typechecker leaves the adherence of the unsafe
code to the function's type signature as a proof obligation to
the programmer. The RustBelt project
\cite{DBLP:journals/pacmpl/0002JKD18} provides a formal 
foundation for creating such proofs by exhibiting a framework for
semantic soundness of the Rust type system in terms of a low-level
core language that incorporates aspects
of concurrency (i.e., data-race freedom). Similar proof obligations
would be needed in \lang to check that an implementation of the module
types or the type of fold shown in \cref{motivation} matches the
semantics of the typings. We aim to develop a suitable framework for
this task for \lang. At present, the metatheory of \lang does not
cover concurrency.

\citet{DBLP:journals/corr/abs-1903-00982}
formalize Rust's ownership discipline from a source-level
perspective. Their approach is purely syntactic and is therefore not
able to reason about unsafe fragments of Rust code. However, their
flow-sensitive type discipline enables soundness proofs for
non-lexical lifetimes, which have been adopted in Rust, but cannot be
expressed in \lang at present.


Vault~\citep{DBLP:conf/pldi/DeLineF01}
and Plaid~\citep{DBLP:conf/oopsla/AldrichSSS09,DBLP:journals/toplas/GarciaTWA14}
leverage typestate and capabilities
to express rich properties in objects and protocols.
These systems are designed for either low-level or object-oriented
programming and do not immediately lend themselves to a more functional
style. While these systems are much more
powerful than \affe's, they require programmer annotations
and do not support inference.
It  would be interesting to extend \lang with limited
forms of typestate as a local, opt-in feature to provide
more expressivity at the cost of inference.

\subsection{Type-system Features}
%
\lang relies on constrained types
to introduce the kind inequalities required for linear types.
\hmx~\citep{DBLP:journals/tapos/OderskySW99} 
allows us to use constrained types in an ML-like language with complete
type inference.
\hmx has been shown to be compatible with subtyping,
bounded quantification and existentials~\citep{DBLP:conf/icfp/Simonet03},
GADTs~\citep{DBLP:journals/toplas/SimonetP07},
and there exists a syntactic soundness proof~\citep{DBLP:journals/entcs/SkalkaP02}.
These results make us confident that the system developed in \lang
could be applied to larger and more complex languages such as OCaml
and the full range of features based on ad-hoc polymorphism.
% Alternatively, we could have based \lang on qualified
% types~\cite{DBLP:journals/scp/Jones94}, similarly to Quill.
% This
% choice would also be sustainable as qualified types is part of the
% foundation of Haskell's type system.

\lang's  subtyping discipline is similar
to structural subtyping, where the only subtyping (or here, subkinding)
is at the leaves.
Such a discipline is known to be friendly to inference and has been used in many
contexts, including OCaml, and has been combined
with constraints~\citep{DBLP:journals/tapos/OderskySW99,DBLP:conf/sas/TrifonovS96}.
% In particular, Flowcaml~\citep{DBLP:conf/popl/PottierS02}
% extends OCaml with security levels forming a lattice and supports type inference.
% In \lang, we repurpose these ideas to check for linearity and affinity.
It also admits classical simplification rules
\citep{DBLP:conf/aplas/Simonet03,DBLP:conf/popl/PottierS02} which we partially use
in our constraint solving algorithm.
\affe's novelty is a kind language
sufficiently simple to make
all simplification rules complete, which allows us to keep type signatures simple.


%%% Local Variables:
%%% mode: latex
%%% TeX-master: "main"
%%% End:

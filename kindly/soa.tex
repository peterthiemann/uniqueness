\section{Related work}
\label{sec:related-work}

% Haller, P., Odersky, M.: Capabilities for Uniqueness and Borrowing
% \cite{DBLP:conf/ecoop/HallerO10}

% Sing\#

% John Tang Boyland and William Retert. Connecting effects and
% uniqueness with adoption. In POPL, pages 283–295. ACM, 2005.
% \cite{DBLP:conf/popl/BoylandR05}


\subsection{Substructural type-systems in functional languages}

Many systems have attempted to combine
functional programming and linear types in a practical setting.
One contribution of \lang is to combine several ingredients
from these different languages while still preserving
complete type inference.
Many of the following languages supports linear or affine types, but rarely
both. In many cases, it is easy to adapt a system to support both, as
\lang does.
None of the following languages support borrows.

System F\degree~\citep{DBLP:conf/tldi/MazurakZZ10}
extends System F with kinds to distinguish
between linear and unrestricted types.
\citet{DBLP:conf/tldi/MazurakZZ10} provide
a linearity-aware semantics with a soundness proof.
Unlike \lang, System F\degree{} does not allow
quantification over kinds which limits its expressivity. For instance, it
does not admit a most general type for function composition.
Being based on System F, it does not admit
principal type inference.

Quill~\citep{DBLP:conf/icfp/Morris16} is a Haskell-like language with affine
types through the use of qualified types.
Quill does not expose a kind language, but
uses typeclass-like mechanism and annotation on arrows.
It supports both a most general type for function composition and
principal type inference.
We were able to express the type signatures of all Quill examples
in \lang by leveraging kind inequalities.
Unlike Quill,
we provide a linearity-aware semantics and prove its soundness.
Quill does not support borrows.

% For instance, the type of the constructor
% in Quill is $\qual{t \geq f}{t \to u \to t * u}$.
% In Affe, it is simply
% $\qual{(\alpha:\kvar)\implies \alpha \to \beta \tarr{\kvar} \alpha * \beta$
% with the kind of $*$ being $\kvar\to\kvar\to\kvar$.

Alms~\citep{DBLP:conf/popl/TovP11} is an ML-like language with rich, kind-based
affine types and ML modules, similar to \lang.
Their system often relies on existential types to track the identity
of objects. For instance
\lstinline/Array.create : int -> 'a -> \E 'b. ('a, 'b) array/ where
\lstinline/'b/ uniquely identify the array.
Due to the reliance on existentials, Alms does not support complete type inference.
Furthermore, Alms does not support borrow and often rely
on manual passing of capabilities.
In our experience, \affe's limited support for existential types through
regions is sufficient to express Alms' examples and leads to
a more convenient programming style for imperative code.
%
Kind-wise, Alms relies
on unions, intersections and dependent kinds while
\lang uses qualified types.
We believe most of Alms' kind signatures can be expressed equivalently in
our system: for instance the pair type constructor
has kind $\Pi\alpha\Pi\beta. \langle\alpha\rangle \sqcup \langle\beta\rangle$
(where $\alpha$ and $\beta$ are types and $\Pi$ is the dependent function)
in Alms and $\kvar\to\kvar\to\kvar$ in \lang thanks
to subkinding.
%
Finally, Alms provides excellent support for abstraction through
modules by allowing to keep some type unrestricted inside a module, but
exposing it as affine. We support a similar programming style, which
we showcase in \cref{motivation}.

The goal of Linear Haskell~\citep{DBLP:journals/pacmpl/BernardyBNJS18}
(LH) is to retrofit linear types to Haskell.
LH introduces a new notion
of ``linear arrows'', written $\multimap$ similar to linear logic,
which are functions that \emph{use} their argument exactly once.
% By contrast, in \lang linearity is only decided
% by the kinds, and $\tarr{\klin}$ simply denotes \emph{single-use functions}.
This design is easy to retrofit on top of an existing compiler
such as GHC, but there are quite controversial discussions about it\footnote{
  See the in-depth discussion attached to the GHC proposal starting here: \url{https://github.com/ghc-proposals/ghc-proposals/pull/111\#issuecomment-403349707}}.
Most relevant to \lang:
\begin{itemize}[leftmargin=*]
\item LH does not admit subtyping for arrows and requires
  $\eta$-expansion to pass unrestricted functions in linear
  contexts. This approach is acceptable in a non-strict language such as
  Haskell but changes the semantics in a strict setting.
\item
  The GHC implementation of LH performs type inference, but
  it is neither formalized nor complete.
\item
  LH promotes a continuation-passing style with functions such as
  \lstinline/withFile : path -> (file ->. file) ->. unit/
  to ensure linear use of resources. This style leads to problems with
  placing the annotation on, e.g., the IO monad.
  \lang follows System F\degree, Quill, and Alms, all of which support
  resource handling in direct style, where types themselves are
  described as affine or linear. (Of course, continuation-passing
  style is also supported.)
  We expect that the direct approach eases modular reasoning about linearity.
  In particular, using abstraction through modules,
  programmers only need to consider the module
  implementation to ensure that linear
  resources are properly handled.
% \item
%   Linear Haskell introduces a notion of linear monad to express
%   imperative code conveniently. Again, this solution is suitable in Haskell,
%   but less appropriate in other contexts. Our borrow system allow
%   to provide a more free-form programming style for imperative code.
% \item
%   Due to the previous remark, ensuring that a given object is never
%   aliased is fairly difficult in Linear Haskell, as one would need to ensure
%   that no non-linear function can manipulate it. On the other
\end{itemize}

Mezzo~\citep{DBLP:phd/hal/Protzenko14} is an ML-like language
with a rich capability system which is able to encode numerous
properties akin to separation logic~\citep{DBLP:conf/lics/Reynolds02}.
Mezzo explores the  boundaries of the design space of type systems for
resources. Hence, it is more expressive than \lang, but
much harder to use. The Mezzo typechecker relies on explicit
annotations and it is not known whether type inference for Mezzo is possible.

\citet{DBLP:journals/corr/abs-1803-02796} presents
an extension of OCaml for resource management in the style of C++'s RAII
and Rust's lifetimes. This system assumes
the existence of a linear type system and develops the associated compilation
and runtime infrastructure. We believe our approach is
complementary and aim to combine them in the future.

\subsection{Other substructural type-systems}

% Linear and affine disciplines have  been used in non-functional
% settings, notably in the context of low-level imperative programming
% and object systems. In particular,
\lang uses borrows and regions
which were initially developed in the context of linear and affine
typing for  imperative and
object-oriented
programming~\citep{DBLP:conf/popl/BoylandR05,DBLP:conf/pldi/GrossmanMJHWC02}.

Rust~\citep{rust} is the first
mainstream language to popularize the idea of borrowing and ownership
for safe low-level programming.
\lang is inspired from Rust's borrowing system to apply
it in a functional setting with type inference, a GC, and
an ML module system.
Rust's lifetime system is more explicit and more expressive than \lang,
but Rust does not provide type inference
and only provides \emph{partial} lifetime inference.
Recently, \citet{DBLP:journals/corr/abs-1903-00982}
formalized Rust's ownership discipline, including non-lexical lifetimes.
We believe this work
could help us improve our handling of regions and of successive exclusive borrows.

Vault~\citep{DBLP:conf/pldi/DeLineF01}
and Plaid~\citep{DBLP:conf/oopsla/AldrichSSS09}
leverage typestate and capabilities
to express rich properties in objects and protocols.
These systems are designed for either low-level or object-oriented
programming and do not immediately lend themselves to a more functional
style. While these systems are much more
powerful than \affe's, they require programmer annotations
and do not support inference.
It  would be interesting to extend \lang with limited
forms of typestate as a local, opt-in feature that would provide
more expressivity at the cost of inference.

\subsection{Type-system features}
%
\lang relies on constrained types
to introduce the kind inequalities required for linear types.
HM(X)~\citep{DBLP:journals/tapos/OderskySW99} 
allows us to use constrained types in an ML-like language with complete
type inference.
HM(X) has been shown to be compatible with subtyping,
bounded quantification and existentials~\citep{DBLP:conf/icfp/Simonet03},
GADTs~\citep{DBLP:journals/toplas/SimonetP07},
and there exists a syntactic soundness proof~\citep{DBLP:journals/entcs/SkalkaP02}.
These results make us confident that the system developed in \lang
could be applied to larger and more complex languages such as OCaml.
Alternatively, we could have based \lang on qualified
types~\cite{DBLP:journals/scp/Jones94}, similarly to Quill.
% This
% choice would also be sustainable as qualified types is part of the
% foundation of Haskell's type system.

\lang's  subtyping discipline is similar
to structural subtyping, where the only subtyping (or here, subkinding)
is at the leaves.
Such discipline is known to be friendly to inference and has been used in many
contexts, including OCaml, and has been combined
with constraints~\citep{DBLP:journals/tapos/OderskySW99,DBLP:conf/sas/TrifonovS96}.
In particular, Flowcaml~\citep{DBLP:conf/popl/PottierS02}
extends OCaml with security levels forming a lattice and supports type inference.
In \lang, we repurpose these ideas to check for linearity and affinity.
\citet{DBLP:conf/aplas/Simonet03} also presents various algorithmic
notions to solve lattice-based constraints, which we partially use
in our constraint solving algorithm. Our main contribution
is the design of a kind language specialized for checking linearity
and sufficiently simple to make
all simplification rules complete, which allows us to keep type signatures simple.


%%% Local Variables:
%%% mode: latex
%%% TeX-master: "main"
%%% End:

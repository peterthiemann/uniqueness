\subsection{Pool of linear resources}
\label{tuto:pool}

We present an interface and implementation of a pool of linear resources where the
extended scope of the region enforces proper use of the resources.

\cref{intf:pool} contains the interface of the @Pool@ module.
A pool is parameterized by its content. The kind of the pool
depends on the content: linear content implies
a linear pool while unrestricted content yields an unrestricted pool.
The functions @Pool.create@ and @Pool.consume@
build/destroy a pool given creators/destructors for the elements
of the pool.
The function @Pool.use@ is the workhorse of the API, which
borrows a resource from the pool to a callback.
It takes a shared borrow of a pool (to enable concurrent access) and a
callback function.
The callback receives a exclusive borrow of an arbitrary resource from the pool.
The typing of the callback ensures
that this borrow is neither captured nor returned by the function.

This encapsulation is implemented with a universally quantified \emph{kind index variable} $r$.
The signature prescribes the type @&!(aff_r1,'a_1)@ for the
exclusive borrow of the resource with an affine kind at region nesting $r+1$. The return
type of the callback is constrained to kind @'k_2 <= aff_r@
so that the callback certainly cannot return the borrowed argument.
In a specific use of @Pool.use@, the index $r$ gets unified
with the current nesting level of regions so that the region for the
callback effectively gets ``inserted'' into the lexical nesting at the callsite.
%
\cref{ex:pool} shows a simple example using the @Pool@ module.

The implementation in \cref{impl:pool} represents a bag of resources
using a concurrent queue with atomic add and remove operations.
The implementation of the @Pool.create@ and @Pool.consume@
functions is straightforward.
The function @Pool.use@ first draws
an element from the pool (or creates a fresh element),
passes it down to the callback function @f@, and returns
it to the pool afterwards.
For clarity,
we explicitly delimit the region in \cref{line:pool:region} to ensure that
the return value of @f &!o@ does not capture @&!o@.
In practice, the type checker inserts this region automatically.

\begin{figure}[tp]
  \centering
  \begin{subfigure}[t]{1\linewidth}
\begin{lstlisting}
type ('a:'k) pool : 'k
create : (unit -> 'a) -> 'a pool
consume : ('a -> unit) -> 'a pool -> unit
use : ('a_1:'k_1),('a_2:'k_2),('k_2 <= aff_r) =>
  &('a_1 pool) -> (&!(aff_r1,'a_1) -{lin}> 'a_2) -{'k_1}> 'a_2
\end{lstlisting}
    \vspace{-10pt}
    \caption{Signature}
    \label{intf:pool}

\begin{lstlisting}
(* Using the pool in queries. *)
let create_user pool name =
  Pool.use &pool (fun connection ->
    Db.insert "users" [("name", name)] connection)

let uri = "postgresql://localhost:5432"
let main users =
  (* Create a database connection pool. *)
  let pool = Pool.create (fun _ -> Db.connect uri) in
  List.parallel_iter (create_user &pool) users;
  Pool.consume (fun c -> Db.close c)
\end{lstlisting}
    \vspace{-10pt}
    \caption{Example of use}
    \label{ex:pool}
  \end{subfigure}

  \begin{subfigure}[t]{1\linewidth}
\begin{lstlisting}
type ('a:'k) pool : 'k = {spawn: unit -> 'a; queue: 'a CQueue.t}
let create spawn = { spawn ; queue = CQueue.create () }
let consume f c = CQueue.iter f c.queue
let use { spawn ; queue } f =
  let o = match CQueue.pop &queue with
    | Some x -> x
    | None () -> spawn ()
  in
  let r = {| f &!o |} in(*@\label{line:pool:region}*)
  Queue.push o &queue;
  r
\end{lstlisting}
    \vspace{-15pt}
    \caption{Implementation}
    \label{impl:pool}
  \end{subfigure}

  \caption{The \texttt{Pool} module}
  \label{fig:pool}
\end{figure}

%%% Local Variables:
%%% mode: latex
%%% TeX-master: "main"
%%% End:

\newcommand\mcase[1]{\noindent\textbf{Case }#1\\\noindent}

\subsection{Soundness}

\begin{lemma}
  \label{lemma:constrsubst}
  Given constraints $C$ and $D$ and substitution $\unif$, if $\entail{D}{C}$
  then $\entail{\unif D}{\unif C}$.
  \begin{proof}
    By induction over the entailment judgement.
  \end{proof}
\end{lemma}

\begin{lemma}
  \label{lemma:constrimply}
  Given a typing derivation $\inferS{C}{\E}{e}{\tau}$ and
  a constraint $D \in \mathcal S$ in solved form such that $\entail{D}{C}$, then
  $\inferS{D}{\E}{e}{\tau}$
  \begin{proof}
    By induction over the typing derivation
  \end{proof}
\end{lemma}


\begin{lemma}
  \label{lemma:typ:weakening}
  Given a type environment $\E$, $\E' \subset \E$, a term $e$ and a variable $x\in\E$,\\
  if $\inferS{C}{\E'}{e}{\tau}$
  then $\inferS{C\Cand \Weaken_{x}(\E')}{\E';\bvar{x}{\E(x)}}{e}{\tau}$

  \begin{proof}
    Trivial if $x \in \Sv$. Otherwise, by induction over the typing derivation.
  \end{proof}
\end{lemma}

% We say that a use map $\Sv$ and an environment $\E$ are equivalent,
% noted $\Sv \equivC \E$, if, for
% any kind $k$, $\Cleq{\Sv}{k} \equivC \Cleq{\E|_{dom(\Sv)}}{k}$.

We define a flattened environments, noted $\Eflat\E$ as the environment
where all the binders are replace by normal ones. More formally:
\begin{align*}
  \Eflat\E
  &= \left\{ \bvar{x}{\tau} \mid
    \bvar{x}{\tau}\in\E
    \vee \bvar{\borrow{x}}{\borrowty{k}{\tau}}\in\E
    \vee \svar{x}{\tau}^n\in\E
    \right\} \cup \left\{ \bvar{\tvar}{k} \mid \bvar{\tvar}{k}\in\E \right\}
\end{align*}

\begin{lemma}
  \label{lemma:env:flat}
  Given a type environment $\E$ and a term $e$ such
  that $\inferW{\Sv}{(C,\unif)}{\E}{e}{\tau}$
  then $\Eflat\Sv \subset \E$.
  \begin{proof}
    By induction over the typing derivation.
  \end{proof}
\end{lemma}


% \begin{lemma}
%   \label{lemma:equivsplit}
%   Given a type environment $\E$ and
%   use maps $\Sv_1$, $\Sv_2$,
%   if $\bsplit{C}{\Sv}{\Sv_1}{\Sv_2}$, then
%   there exists $\E_1$, $\E_2$ such that $\Sv_1 \equivC \E_1$,
%   $\Sv_2 \equivC \E_2$,
%   $\Sv \equivC \E$, $\bsplit{C'}{\E}{\E_1}{\E_2}$ and $C\equivC C'$.
%   \begin{proof}
%     TODO
%   \end{proof}
% \end{lemma}


\begin{theorem}[Soundness of inference]
  Given a type environment $\E$ containing only value bindings,
  $\E|_\tau$ containing only a type binding and a term $e$,\\
  if $\inferW{\Sv}{(C,\unif)}{\E;\E_\tau}{e}{\tau}$
  then $\inferS{C}{\unif(\Sv;\E_\tau)}{e}{\tau}$, $\unif C = C$ and $\unif \tau = \tau$
\begin{proof}
  We proceed by induction over the derivation of $\vdash_w$.
  Most of the cases follow the proofs from HM(X) closely.
  For brevity, we showcase three rules: the treatment of binders
  and weakening, where our inference algorithm differ significantly
  from the syntax-directed rule, and the $Pair$ case
  which showcase the treatment of environment splitting.

  \mcase{$\ruleIVar$}

  We have $\Sv = \Sone{x}{\sigma}$.
  Without loss of generality, we can consider $\unif_x = \unif'|_{\fv{\E}} = \unif'|_{\fv{\sigma}}$.
  Since $\Sv\Sdel{x}$ is empty and by definition of normalize, we
  have that
  $\entail C \unif'(C_x) \Cand \Cleq{\unif_x\Sv\Sdel{x}}{\kaff_\infty}$,
  $\unif' \leq \unif$ and $\unif'C = C$.
  By definition, $\unif_x\unif' = \unif'$.
  By rule $Var$, we obtain
  $\inferS{C}{\unif_x(\Sv;\E_\tau)}{x}{\unif_x\unif'\tau}$, which concludes.
  \\

  \mcase{$\ruleIAbs$}

  By induction, we have
  $\inferS{C'}{\Sv_x;\E_\tau}{e}{\tau}$, $\unif C = C$
  and $\unif \tau = \tau$.\\
  By definition of normalize and \cref{lemma:constrsubst}, we have
  $\entail{C}{C'\Cand\Cleq{\unif'\Sv}{\unif'\kvar} \Cand \Weaken_{x}(\unif'\Sv_x)}$ and
  $\unif \leq \unif'$.
  By \cref{lemma:constrsubst}, we have $\entail{C}{\Cleq{\Sv}{\unif\kvar}}$.

  We now consider two cases:
  \begin{enumerate}[leftmargin=*,noitemsep,topsep=5pt]
  \item If $x\in\Sv_x$, then $\Weaken_{x}(\unif\Sv_x) = \Ctrue$
    and by \cref{lemma:env:flat}, $\Sv_x = \Sv;\bvar{x}{\alpha}$.
    We can deduce that
    $\inferS{C'\Cand \Weaken_{\bvar{x}{\tvar}}(\unif\Sv_x)}{\unif\Sv;\E_\tau;\bvar{x}{\unif(\tvar)}}{e}{\tau}$.
  \item If $x\notin\Sv_x$, then $\Sv = \Sv_x$ and
    $\Weaken_{\bvar{x}{\tvar}}(\unif\Sv_x) = \Cleq{\unif\tvar}{\kaff_\infty}$.
    By \cref{lemma:typ:weakening},
    we have that
    $\inferS{C'\Cand \Weaken_{\bvar{x}{\tvar}}(\unif\Sv_x)}{\unif\Sv;\E_\tau;\bvar{x}{\unif(\tvar)}}{e}{\tau}$
  \end{enumerate}
  % By \cref{lemma:env:flat}, we know that if $x\in\Sv_x$, then
  % By \cref{lemma:typ:weakening,}, we deduce
  % that
  % $\inferS{C'\Cand \Weaken_{x}(\unif\Sv_x)}{\unif\Sv;\bvar{x}{\unif(\tvar)}}{e}{\tau}$.
  By \cref{lemma:constrimply}, we
  have $\inferS{C}{\unif(\Sv;\E_\tau);\bvar{x}{\unif(\tvar)}}{e}{\tau}$.\\
  By rule $Abs$, we obtain
  $\inferS{C}{\unif(\Sv;\E_\tau)}{\lam{x}{e}}{\unif(\tvar)\tarr{\unif(\kvar)}\tau}$ which concludes.
  \\

  % $\ruleSDLam$

  % \mcase{$\ruleIApp$}

  % By induction, we have
  % $\inferS{C_1}{\unif_1\Sv_1}{e_1}{\tau_1}$, $\unif_1 C_1 = C_1$,
  % and $\unif_1 \tau_1 = \tau_1$
  % and
  % $\inferS{C_2}{\unif_2\Sv_2}{e_2}{\tau_2}$, $\unif_2 C_2 = C_2$
  % and $\unif_2 \tau_2 = \tau_2$.
  % %
  % By normalization, $\entail{C}{D}$, $\unif \leq \unif'$ and $\unif C = C$.
  % By \cref{lemma:constrimply} and by substitution, we have
  % $\inferS{C}{\unif\Sv_1}{e_1}{\unif\tau_1}$ and
  % $\inferS{C}{\unif\Sv_2}{e_2}{\unif\tau_2}$.
  % We directly have that
  % $\bsplit{\unif C_s}{\unif \Sv}{\Sv_1}{\unif\Sv_2}$
  % and by \cref{lemma:constrimply}, $\entail{\unif C}{\unif C_s}$.
  % %
  % Finally, since the constraint $\Cleq{\tau_1}{\tau_2\tarr{\kvar}\tvar}$
  % has been solved by normalization,
  % we have $\unif'\tau_1 = \unif\tau_2\tarr{k}\unif\tvar$ for some kind $k$.
  % By rule $App$, we obtain
  % $\inferS{C}
  % {\unif\Sv}{\app{e_1}{e_2}}{\unif(\tvar)}$,
  % which concludes.
  % \\


  % \mcase{$\ruleILet$}

  % By induction, we have
  % $\inferS{C_1}{\unif_1\Sv_1}{e_1}{\tau_1}$, $\unif_1 C_1 = C_1$,
  % and $\unif_1 \tau_1 = \tau_1$
  % and
  % $\inferS{C_2}{\unif_2\Sv_2}{e_2}{\tau_2}$, $\unif_2 C_2 = C_2$
  % and $\unif_2 \tau_2 = \tau_2$.
  % %
  % By definition of $\text{gen}$, $\entail{C_\schm}{C_1}$.
  % %
  % By normalization, $\entail{C}{D}$, $\unif \leq \unif'$ and $\unif C = C$.
  % By \cref{lemma:constrimply} and by substitution, we have
  % $\inferS{C}{\unif\Sv_1}{e_1}{\unif\tau_1}$ and
  % $\inferS{C}{\unif\Sv_2}{e_2}{\unif\tau_2}$.
  % %

  % \TODO{}

  % $
  % \inferrule[Let]
  % { \inferS{C \Cand D}{\E_1}{e_1}{\tau_1} \\
  %   (C_\schm,\schm) = \generalize{D}{\E}{\tau_1}\\
  %   \entail{C}{C_\schm} \\
  %   \inferS{C}{\E;\bvar{x}{\sigma}}{e_2}{\tau_2} \\
  %   \addlin{\lsplit{C}{\E}{\E_1}{\E_2}}\\
  % }
  % { \inferS{C}
  %   {\E}{\letin{x}{e_1}{e_2}}{\tau_2} }$

  \mcase{$\ruleIPair$}

  By induction, we have
  $\inferS{C_1}{\unif_1(\Sv_1;\E^1_\tau)}{e_1}{\tau_1}$, $\unif_1 C_1 = C_1$,
  and $\unif_1 \tau_1 = \tau_1$
  and
  $\inferS{C_2}{\unif_2(\Sv_2;\E^2_\tau)}{e_2}{\tau_2}$, $\unif_2 C_2 = C_2$
  and $\unif_2 \tau_2 = \tau_2$.
  %
  Wlog, we can rename the type $\E^1_\tau$ and $\E^2_\tau$ to be disjoint
  and define $\E_\tau = \E^1_\tau \cup \E^2_\tau$.
  By normalization, $\entail{C}{D}$, $\unif \leq \unif'$ and $\unif C = C$.
  By \cref{lemma:constrimply} and by substitution, we have
  $\inferS{C}{\unif\Sv_1}{e_1}{\unif\tau_1}$ and
  $\inferS{C}{\unif\Sv_2}{e_2}{\unif\tau_2}$.
  We directly have that
  $\bsplit{\unif C_s}{\unif \Sv}{\Sv_1}{\unif\Sv_2}$
  and by \cref{lemma:constrimply}, $\entail{\unif C}{\unif C_s}$.
  %
  By rule $Pair$, we obtain
  $\inferS{C}
  {\unif(\Sv;\E_\tau)}{\app{e_1}{e_2}}{\unif(\tyPair{\tvar_1}{\tvar_2}))}$,
  which concludes.
  \\

  % \mcase{$\ruleIMatch$}

  % \mcase{$\ruleIBorrow$}

  % We trivially have
  % $\bvar{\borrow x}{\borrowty{\unif(k)}{\unif(\tau)}} \in \unif\Sv$.
  % By induction, we have $\unif C = C$ and $\unif(\tau) = \tau$.
  % Since $\kvar$ is new, $\unif(\kvar) = \kvar$.
  % By rule $Borrow$, we obtain
  % $\inferS{C}{\unif\Sv}{\borrow{x}}{\borrowty{\kvar}{\tau}}$,
  % which concludes.
  % \\

  % \mcase{$\ruleIRegion$}

\end{proof}
\end{theorem}

\subsection{Completeness}

We now state our algorithm is complete: for any given
syntax-directed typing derivation, our inference algorithm can find
a derivation that gives a type at least as general.
For this, we need first to provide a few additional definitions.

\begin{definition}[More general unifier]
  Given a set of variable $U$ and $\unif$, $\unif'$ and $\phi$
  substitutions. \\
  Then
  $\unif \leq^{\phi}_{U} \unif'$ iff $(\phi \circ \unif)|_{U} = \unif'|_U$.
\end{definition}

\begin{definition}[Instance relation]
  Given a constraints $C$ and two schemes
  $\schm = \forall \Multi\tvar. \qual{D}{\tau}$ and
  $\schm' = \forall \Multi\tvar'. \qual{D'}{\tau'} $.
  Then $\entail{C}{\schm \preceq \schm'}$
  iff $\entail{C}{\subst{\tvar}{\tau''} D}$
  and $\entail{C\Cand D'}{\Cleq{\subst{\tvar}{\tau''}\tau}{\tau'}}$
\end{definition}

We also extend the instance relation to environments $\E$.


We now describe the interactions between splitting and the
various other operations.

\begin{lemma}
  \label{split:flat}
  Given $\bsplit{C}{\E}{\E_1}{\E_2}$, Then $\Eflat\E = \Eflat\E_1 \cup \Eflat\E_2$.
  \begin{proof}
    By induction over the splitting derivation.
  \end{proof}
\end{lemma}


\begin{lemma}
  \label{split:gen}
  Given $\lsplit{C}{\E_1}{\E_2}{\E_3}$,
  $\bsplit{C'}{\E'_1}{\E'_2}{\E'_3}$
  and $\unif$ such that
  $\E'_i\subset\E''_i$ and $\entail{}{\unif\E''_i \preceq \E_i}$
  for $i\in\{1;2;3\}$.

  Then $\entail{C}{\unif C'}$.
\begin{proof}
  By induction over the derivation of $\bsplit{C'}{\E'_1}{\E'_2}{\E'_3}$.
\end{proof}
\end{lemma}


We can arbitrarily extend the initial typing environment in an inference
derivation, since
it is not used to check linearity.

\begin{lemma}
  \label{infer:extend}
  Given $\inferW{\Sv}{(C,\unif)}{\Eflat\E}{e}{\tau}$ and $\E'$ such
  that $\E \subseteq \E'$, then $\inferW{\Sv}{(C,\unif)}{\Eflat\E'}{e}{\tau}$
  \begin{proof}
    By induction over the type inference derivation.
  \end{proof}
\end{lemma}

Finally, we present the completeness theorem.

\begin{theorem}[Completeness]
  Given $\inferS{C'}{\E'}{e}{\tau'}$ and
  $\entail{}{\unif'\E \preceq \E'}$.
  Then $$\inferW{\Sv}{(C,\unif)}{\Eflat\E}{e}{\tau}$$
  for some environment $\Sv$,
  substitution $\unif$, constraint $C$ and type $\tau$ such
  that
  \begin{align*}
    \unif &\leq^{\phi}_{\fv{\E}} \unif'
    &\entail{C'&}{\phi C}
    &\entail{&}{\phi \schm \preceq \schm'}
    &\Sv&\subset\E
    % &( C, \schm, \unif) &\leq (C',\schm',\unif')
  \end{align*}
  where $\schm' = \generalize{C'}{\E'}{\tau'}$
  and $\schm = \generalize{C}{\E}{\tau}$
\end{theorem}
\begin{proof}
  Most of the difficulty of this proof comes from proper handling of
  instanciation and generalization for type-schemes.
  This part is already proven
  by \citet{sulzmann1997proofs} in the context
  of HM(X). As before, we will only present few cases
  which highlights the handling of bindings and environments.
  For clarity, we will only present the part of the proof that
  only directly relate to the new aspect introduced by \lang.
  %
  \\

  % \mcase{$\ruleSDVar$}

  % $\ruleIVar$


  \mcase{$
    \inferrule[Abs]
    {
      \inferS{C'}
      {\E'_x;\bvar{x}{\tau'_2}}{e}{\tau'_1} \\
      \addlin{\entail{C}{\Cleq{\E'_x}{k}}}
    }
    { \inferS{C'}{\E'_x}
      {\lam[k]{x}{e}}{\tau'_2\tarr{k}\tau'_1} }
    $ and $\entail{}{\unif'\E \preceq \E'}$.}

  Let us pick $\tvar$ and $\kvar$ fresh.
  Wlog, we choose $\unif'(\tvar) = \tau_2$ and $\unif'(\kvar) = k$
  so that $\entail{}{\unif'\E_x \preceq \E'_x}$.
  By induction:
  \begin{align*}
    \inferW{\Sv_x}{(C,\unif)}{\Eflat\E_x;\bvar{x}{\tvar}&}{e}{\tau}
    &\unif &\leq^{\phi}_{\fv{\E_x}\cup\{\tvar;\kvar\}} \unif'
    &\entail{C'&}{\phi C}
    &\entail{&}{\phi \schm \preceq \schm'}
  \end{align*}
  \begin{align*}
    \Sv_x&\subset\E_x;\bvar{x}{\tvar}
    &\schm' &= \generalize{C'}{\E_x';\bvar{x}{\tau'_2}}{\tau'_1}
    &\schm &= \generalize{C}{\E_x;\bvar{x}{\tvar}}{\tau_1}
  \end{align*}

  Let $C_a = C\Cand \Cleq{\Sv}{\kvar} \Cand \Weaken_{\bvar{x}{\tvar}}(\Sv_x)$
  and
  By definition, $\unif_D\Sdel{\alpha;\kvar} \leq^{\phi_D}_{\fv{\E_x}} \unif$
  which means we have
  $\unif_D\Sdel{\alpha;\kvar} \leq^{\phi\circ\phi_D}_{\fv{\E_x}} \unif'$.
  We also have that $\Sv_x\Sdel{x}\subset \E_x$.

  Since $\entail{C}{\Cleq{\E'_x}{k}}$, we have $\entail{C}{\unif'\Cleq{\Sv}{\kvar}}$.
  If $x\in\Sv_x$, then $\Weaken_{\bvar{x}{\tvar}}(\Sv_x) = \Ctrue$.
  Otherwise we can show by induction
  that $\entail{C'}{\unif'\Weaken_{\bvar{x}{\tvar}}(\Sv_x)}$.
  We also have $\unif C = C$, which gives us $\entail{C'}{\unif'(C_a)}$.
  We can deduce
  $\entail{C'}{\unif'(C_a)}$.\\
  This means $(C',\unif')$ is a normal form of $C_a$, so a principal normal form
  exists. Let $(D,\unif_D) = \normalize{C_a}{\unif\Sdel{\alpha;\kvar}}$.
  By the property of principal normal forms,
  we have $\entail{C'}{\rho D}$ and
  $\unif_D \leq^{\rho}_{\fv{\E_x}} \unif'$.

  By application of {\sc Abs$_I$}, we have
  $\inferW{\Sv_x\Sdel{x}}{(C,\unif_D\Sdel{\tvar,\kvar})}{\Eflat\E_x}
  {\lam{x}{e}}{\unif_D(\tvar)\tarr{\unif_D(\kvar)}\tau_1}$.\\
  The rest of the proof proceeds as in the original HM(X) proof.
  \\

  % $\ruleIAbs$

  \mcase{$
    \inferrule[Pair]
    {
      \inferS{C'}{\E'_1}{e_1}{\tau'_1} \\
      \inferS{C'}{\E'_2}{e_2}{\tau'_1} \\
      \lsplit{C}{\E'}{\E'_1}{\E'_2}
    }
    { \inferS{C'}
      {\E'}{\app{e_1}{e_2}}{\tyPair{\tau'_1}{\tau'_2}} }
    $ and $\entail{}{\unif'\E \preceq \E'}$}

  The only new elements compared to HM(X) is
  the environment splitting.
  By induction:
  \begin{align*}
    \inferW{\Sv_1}{(C_1,\unif_1)}{\Eflat\E_1&}{e}{\tau_1}
    &\unif_1 &\leq^{\phi_2}_{\fv{\E}} \unif'_1
    &\entail{C'&}{\phi C_1}
    &\entail{&}{\phi_2 \schm_1 \preceq \schm'_1}
  \end{align*}
  \begin{align*}
    \Sv_1&\subset\E_1
    &\schm'_1 &= \generalize{C'}{\E'_1}{\tau'_1}
    &\schm_1 &= \generalize{C}{\E_1}{\tau_1}
  \end{align*}
  and
  \begin{align*}
    \inferW{\Sv_2}{(C_2,\unif_2)}{\Eflat\E_2&}{e}{\tau_2}
    &\unif_2 &\leq^{\phi_2}_{\fv{\E}} \unif'_2
    &\entail{C'&}{\phi C_2}
    &\entail{&}{\phi_2 \schm_2 \preceq \schm'_2}
  \end{align*}
  \begin{align*}
    \Sv_2&\subset\E_2
    &\schm'_2 &= \generalize{C'}{\E'_2}{\tau'_2}
    &\schm_2 &= \generalize{C}{\E_2}{\tau_2}
  \end{align*}


  By \cref{split:flat,infer:extend}, we have
  \begin{align*}
    \inferW{\Sv_1}{(C_1,\unif_1)}{\Eflat\E&}{e}{\tau_1}
    &\inferW{\Sv_2}{(C_2,\unif_2)}{\Eflat\E&}{e}{\tau_2}
  \end{align*}

  Let $\bsplit{C_s}{\Sv}{\Sv_1}{\Sv_2}$.
  We know that $\entail{}{\unif'\E \preceq \E'}$,
  $\entail{}{\unif'_i\E_i \preceq \E'_i}$ and $\Sv_i\subset\E_i$.
  By \cref{split:gen},
  we have $\entail{C}{\unif' C_s}$.
  The rest of the proof follows HM(X).

\end{proof}



%%% Local Variables:
%%% mode: latex
%%% TeX-master: "../main"
%%% End:

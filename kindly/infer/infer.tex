\section{Type inference}

We now formulate type inference for the \lang language. Our inference technique is based on the HM(X) framework~\citep{DBLP:journals/tapos/OderskySW99} which
presents how to merge Hindley-Milner type inference for let-polymorphism and
constraint solving. We first present our constraint language, we then show
our slight extension of HM(X) that support a kind system and affine types.

\subsection{Constraint language}

We consider $\mathcal C$ the set of constrains.
The grammar of constrains, presented
in \cref{grammar:constraint}, follows the traditional HM(X) formulation
with conjunctions,projections and type inequalities. The only new
element specific to our approach are kind equalities.
Entailment is noted $\entail{C}{D}$, where $D$ is a consequence of the
constraints $C$. The entailment rules are provided in \cref{rules:entail}.

\begin{figure}[h]
  \centering
  \begin{align*}
    C ::=&\ \Cleq{\tau_1}{\tau_2}\ |\ \Cleq{k_1}{k_2}\ |\ C_1 \Cand C_2\ |\ \Cproj{\alpha}{C}
  \end{align*}
  \caption{The constraint language}
  \label{grammar:constraint}
\end{figure}

\begin{figure*}[h]
  \begin{mathpar}
  \inferrule
  {}{ \entail{}{\Cleq{\kvar}{\kaff}} }
  \and
  \inferrule
  {}{ \entail{}{\Cleq{\kun}{\kvar}} }
  \and
  \inferrule
  {}{ \entail{}{\Cleq{\kvar}{\kvar}} }
  \and
  % \inferrule
  % {\Cleq{k}{k'} \in C}{ \entail{C}{\Cleq{k}{k'}} }
  % \and
  % \inferrule
  % { \entail{C}{\Cleq{x_1}{x}}\\
  %   \entail{C}{\Cleq{x}{x_2}}
  % }
  % { \entail{C}{\Cleq{x_1}{x_2}} }
  % \and
  % \inferrule
  % { \entail{C}{D} }
  % { \entail{C}{\Cproj{x}{D}} }
  % \\
  \inferrule
  { \entail{C}{\Cleq{\tau'_1}{\tau_1}}\\
    \entail{C}{\Cleq{\tau_2}{\tau'_2}}\\
    \entail{C}{\Cleq{k}{k'}}
  }
  { \entail{C}{\Cleq{\tau_1\tarr{k}\tau_2}{\tau'_1\tarr{k'}\tau'_2}} }
  \and
  \inferrule
  { \forall i,\ \entail{C}{\Cleq{\tau_i}{\tau_i}}\\
  }
  { \entail{C}{\Cleq{\tapp{t}{(\tau_i)}}{\tapp{t}{(\tau'_i)}}} }
  \and
  
  % \and
  % \inferrule
  % { \entail{C}{\Cleq{k}{k'}} \\
  %   \entail{C}{\Cleq{k'}{k}} }
  % { \entail{C}{\Ceq{k}{k'}} }
  % \and
  % \inferrule
  % { \entail{C}{\Cleq{k}{k'}} }
  % { \entail{C}{\Ckind{\tau_0\tarr{k}\tau_1}}{k'}}
  % \and
  % \text{Completion to form a cylindric constraint system.}
\end{mathpar}

%%% Local Variables:
%%% mode: latex
%%% TeX-master: "../main"
%%% End:

  \caption{Entailment rules -- $\entail{C}{D}$ }
  \label{rules:entail}
\end{figure*}


\paragraph{Solved forms}

The set of solved forms, noted $\mathcal S$,
is the set of constraints than can be used inside
type and kind schemes. $\mathcal S$ is composed only of kind
inequalities \emph{over variables}. For convenience, if $C\in\mathcal S$, we
note $C$ as a list of kind inequalities: $\Cleq{\kvar_i}{\kvar_{i'}}^n$.
\TODO{Extend the properties of solved forms}


We consider the existence of a function $\normalize$ which takes
a constraint in $\mathcal C$ and a unifier $\psi$ and returns a constraint
in solved form $C'$, and an updated unifier.


\subsection{Kind inference}

We note $\inferK{(\unif,C)}{\E}{\tau}{k}$ the fact that type $\tau$ has kind $k$
in environment $\E$ with constraints $C$ and unifier $\unif$. From an
algorithmic point of view, $\E$ and $\tau$ are the input parameters of
our inference procedure.


\subsection{Type inference}

\begin{figure*}[h]
  \begin{mathpar}
  \inferrule[TyVar]
  { \bvar{\alpha}{k}\in\E }
  { \tyval{C}{\E}{\alpha} }
  \and
  \inferrule[TyArrow]
  { \tyval{C}{\E}{\tau_0} \\
    \tyval{C}{\E}{\tau_1}
  }
  { \tyval{C}{\E}{\tau_0\tarr{k}\tau_1} }
  \and
  \inferrule[TyApp]
  { \bvar{t}{\forall\kvar_j^*.(\qual{D}{k^* \karr k})}\in\E \\
    \entail{C}{\subst{\kvar_j}{k_i}{D}} \\
    \forall i, \tyval{C}{\E}{\tau_i}
  }
  { \tyval{C}{\E}{\tapp{t}{k_i^* \tau_i^*}} }
  \\
  \inferrule[TyScheme]
  { \tyval{C \Cand D}
    {\E;\bkind{\kvar_i}^*;\bvar{\alpha_j}{k_j}^*}{\tau} }
  { \schval{C}{\E}{\forall\kvar_i^*\forall\bvar{\alpha_j}{k_j}^*.(\qual{D}{\tau})} }
\end{mathpar}

%%% Local Variables:
%%% mode: latex
%%% TeX-master: "main"
%%% End:

  \caption{Type validity rules -- $\tyval{C}{\E}{\tau}$ }
  \centering
\begin{mathpar}
  \inferrule[Var]
  { \bvar{x}{
      \forall \kvar_i \forall (\alpha_i:\kvar_i).\ \qual{D}{\tau}}
    \in \E \\
    (\beta_i) \text{ new} \\
    \kvar_x = \operatorname{kind}(x)\\
    (C,\unif) =
    \normalize{D\Cand\Ckind{\tau}{\kvar_x}}{\subst{\alpha_i}{\beta_i}{}}
  }
  { \inferW{\Sone{\kvar_x}}{(C,\unif|_{\fv{\E}})}{\E}{x}{\unif\tau} }

  \inferrule[Abs]
  { \inferW{\Sv}{(C',\unif')}{\E;\bvar{x}{\alpha}}{e}{\tau} \\
    \tvar,\kvar\text{ new}\\
    D = C'\Cand\Cleq{\Sv}{\kvar}\Cand\addlin{\Weaken(x,\Sv)}\\
    (C,\unif) = \normalize{D}{\unif'}
  }
  { \inferW{\Sv}{(C,\unif)}{\E}{\lam{x}{e}}{\unif(\tvar)\tarr{\kvar}\tau} }

  \inferrule[App]
  { \inferW{\Sv_1}{(C_1,\unif_1)}{\E}{e_1}{\tau_1} \\
    \inferW{\Sv_2}{(C_2,\unif_2)}{\E}{e_2}{\tau_2} \\\\
    \unif' = \unif_1 \mostgeneral \unif_2 \\
    D =
    C_1 \Cand C_2 \Cand \Ceq{\tau_1}{\tau_2\tarr{\kvar}\tvar}
    \Cand \Cleq{\Sv_1 \Sinter \Sv_2}{\kun} \\
    \tvar,\kvar\text{ new}\\
    (C,\unif) = \normalize{D}{\unif'}\\
  }
  { \inferW{\Sv_1 \Sunion \Sv_2}{(C,\unif)}{\E}{\app{e_1}{e_2}}{\unif(\tvar)} }

  \inferrule[Let]
  { \inferW{\Sv_1}{(C_1,\unif_1)}{\E}{e_1}{\tau_1} \\
    (C'_1,\sigma) = \generalize{C_1}{\unif_1\E}{\tau_1} \\
    \inferW{\Sv_2}{(C_2,\unif_2)}{\E;\bvar{x}{\sigma}}{e_2}{\tau_2} \\\\
    \unif' = \unif_1 \mostgeneral \unif_2 \\
    D =
    C'_1 \Cand C_2 \Cand \Cleq{\Sv_1 \Sinter \Sv_2}{\kun}
    \Cand \addlin{\Weaken(x,\Sv)} \\
    (C,\unif) = \normalize{D}{\unif'}\\
  }
  { \inferW{\Sv_1 \Sunion \Sv_2}{(C,\unif|_{\fv{\E}})}{\E}{\letin{x}{e_1}{e_2}}
    {\unif\tau_2} }
\end{mathpar}

\begin{align*}
  \Weaken(x,\Sv)
  &\equiv \begin{cases}
    \operatorname{kind}(x)\lk\kun &\text{if } \operatorname{kind}(x)\in\Sv\\
    \Cempty &\text{otherwise}
  \end{cases}\\
  \Cleq{\Sv}{k}
  &\equiv \bigwedge_{\kvar\in\Sv} \Cleq{\kvar}{k}
\end{align*}

%%% Local Variables:
%%% mode: latex
%%% TeX-master: "main"
%%% End:

  \caption{Inference rules -- $\inferW{\Sigma}{(C,\psi)}{\bf{\E}}{\bf{e}}{\tau}$ }
\end{figure*}

\section*{Todo}

\begin{itemize}
\item Add type constructors
\item Properly ensure that it respects HM(X) (cylindric, \dots)
\item Show principal type inference:
  \begin{itemize}
  \item Principal constraint system
  \item Regular constraint system: $\Ceq{\tau}{\tau'} \implies \fv{\tau} = \fv{\tau'}$.
  \item Solved forms are in simplified form.
    $C\in S, \entail{C}{\Ceq{\tau}{\tau'}} \implies \entail{}{\Ceq{\tau}{\tau'}}$.
  \end{itemize}
\item Show equivalence with the logic-based system.

\end{itemize}


\begin{lemma}
  Without loss of generality, we can consider that
  kind inequalities in satisfiable constraints
  are only done on kind variables. 

  \begin{proof}
    Consider the constraint $\Cleq{k}{k'}\Cand C$.
    \begin{itemize}
    \item If $k$ and $k'$ are both constants, it can be removed.
    \item If the constraint is $\Cleq{\kvar}{\kaff}$ or $\Cleq{\kun}{\kvar}$, it can be removed.
    \item If the constraint is $\Cleq{\kvar}{\kun}$ or $\Cleq{\kaff}{\kvar}$, we
      can substitute $\kvar$ by its value in $C$.
    \end{itemize}
  \end{proof}
\end{lemma}

%%% Local Variables:
%%% mode: latex
%%% TeX-master: "../main"
%%% End:

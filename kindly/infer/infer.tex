\section{Type inference}

We now formulate type inference for the \lang language. Our inference technique is based on the HM(X) framework~\citep{DBLP:journals/tapos/OderskySW99} which
presents how to merge Hindley-Milner type inference for let-polymorphism and
constraint solving. We first present our constraint language, we then show
our slight extension of HM(X) that support a kind system and affine types.

\subsection{Preliminaries}

\subsubsection{Constraint language}

We consider $\mathcal C$ the set of constrains.
The grammar of constrains, presented
in \cref{grammar:constraint}, follows the traditional HM(X) formulation
with conjunctions,projections and type inequalities. The only new
element specific to our approach are kind equalities.
Entailment is noted $\entail{C}{D}$, where $D$ is a consequence of the
constraints $C$. The entailment rules are provided in \cref{rules:entail}.

\begin{figure}[h]
  \centering
  \begin{align*}
    C ::=&\ \Cleq{\tau_1}{\tau_2}\ |\ \Cleq{k_1}{k_2}\ |\ C_1 \Cand C_2\ |\ \Cproj{\alpha}{C}
  \end{align*}
  \caption{The constraint language}
  \label{grammar:constraint}
\end{figure}

\begin{figure*}[h]
  \begin{mathpar}
  \inferrule{l \leq_{\mathcal L} l'}{\entail{}{\Cleq{l}{l'}}}
  \and
  \inferrule{}{\entail{}{\Cleq{k}{\klin_\infty}}}
  \and
  \inferrule{}{\entail{}{\Cleq{\kun_0}{k}}}
  \and
  \inferrule
  {}{ \entail{}{\Cleq{\kvar}{\kvar}} }
  % \and
  % \inferrule
  % {\Cleq{k}{k'} \in C}{ \entail{C}{\Cleq{k}{k'}} }
  % \and
  % \inferrule
  % { \entail{C}{\Cleq{x_1}{x}}\\
  %   \entail{C}{\Cleq{x}{x_2}}
  % }
  % { \entail{C}{\Cleq{x_1}{x_2}} }
  % \and
  % \inferrule
  % { \entail{C}{D} }
  % { \entail{C}{\Cproj{x}{D}} }
  \\
  \inferrule
  { \entail{C}{\Cleq{\tau'_1}{\tau_1}}\\
    \entail{C}{\Cleq{\tau_2}{\tau'_2}}\\
    \entail{C}{\Cleq{k}{k'}}
  }
  { \entail{C}{\Cleq{\tau_1\tarr{k}\tau_2}{\tau'_1\tarr{k'}\tau'_2}} }
  \and
  \inferrule
  { \forall i,\ \entail{C}{\Ceq{\tau_i}{\tau_i}}\\
  }
  { \entail{C}{\Cleq{\tapp{t}{(\tau_i)}}{\tapp{t}{(\tau'_i)}}} }
  % \and
  % \inferrule
  % { \entail{C}{\Cleq{k}{k'}} \\
  %   \entail{C}{\Cleq{k'}{k}} }
  % { \entail{C}{\Ceq{k}{k'}} }
  % \and
  % \inferrule
  % { \entail{C}{\Cleq{k}{k'}} }
  % { \entail{C}{\Ckind{\tau_0\tarr{k}\tau_1}}{k'}}
  % \and
  % \text{Completion to form a cylindric constraint system.}
\end{mathpar}

%%% Local Variables:
%%% mode: latex
%%% TeX-master: "../main"
%%% End:

  \caption{Entailment rules -- $\entail{C}{D}$ }
  \label{rules:entail}
\end{figure*}


\subsubsection{Solved forms}

The set of solved forms, noted $\mathcal S$,
is the set of constraints than can be used inside
type and kind schemes. $\mathcal S$ is composed only of kind
inequalities \emph{over variables}. For convenience, if $C\in\mathcal S$, we
note $C$ as a list of kind inequalities: $\Cleq{\kvar_i}{\kvar_{i'}}^n$.
\TODO{Extend the properties of solved forms}


We consider the existence of a function $\normalize$ which takes
a constraint in $\mathcal C$ and a unifier $\psi$ and returns a constraint
in solved form $C'$, and an updated unifier.


\subsubsection{Usage maps}

\TODO{Explain $\Sigma$}

\subsection{Kind inference}

We note $\inferK{(C,\unif)}{\bf{\E}}{\bf{\tau}}{k}$ when type $\tau$ has kind $k$
in environment $\E$ under constraints $C$ and unifier $\unif$. From an
algorithmic point of view, $\E$ and $\tau$ are the input parameters of
our inference procedure.
We present the kind inference algorithm as a set of syntax-directed rules in
\cref{rules:kinding}.
Since higher-kinded types are not supported, the type application
rule and the type constructor rule are merged in {\sc KApp}.
Additionally, type variables must be of a simple kind in rule {\sc KVar}.
Note that in the case of a type constructor with no argument, the {\sc KApp}
rule degenerates to a simpler form which is similar to the {\sc KVar} rule.
Kind schemes are instantiated in the {\sc KApp} and {\sc KVar} rules by creating
fresh kind variables and the associated substitution.


\TODO{Explain more?} 


\begin{figure}[h]
  \centering
  \begin{mathpar}
  \inferrule[KVar|KCons]
  { \bvar{\tvar|\T t}{
      \forall \kvar_i.\ \qual{D}{(k_j)^* \karr k}}
    \in \E \\
    (\kvar'_i) \text{ new} \\
    (C,\unif) =
    \normalize{D}{\subst{\kvar_i}{\kvar'_i}{}_i}
  }
  { \inferK{(C,\unif|_{\fv{\E}})}{\E}{\tvar}{\unif ((k_j)^* k)} }
  \and
  \inferrule[KArr]
  { }
  { \inferK{(\Ctrue,\emptyset)}{\E}{\tau_1 \tarr{k} \tau_2}{() k} }
  \and
  \inferrule[KApp]
  { \forall j,\
    \inferK{(C_j,\unif_j)}{\E}{\tau_j}{() k'_j} \\
    \inferK{(C_0,\unif_0)}{\E}{\T t}{(k_j) k}
    \\
    (C,\unif) =
    \normalize
    {C_0\Cand (\bigwedge_j C_j) \Cand \Cleq{k'_j}{k_j}_j}
    {\unif_0\meet (\meeti_j \unif_j)}
  }
  { \inferK{(C,\unif|_{\fv{\E}})}{\E}{\tapp{t}{(\tau_j)}}{()\unif k} }
\end{mathpar}


%%% Local Variables:
%%% mode: latex
%%% TeX-master: "../main"
%%% End:

  \caption{Kind inference rules -- $\inferK{(C,\unif)}{\E}{\tau}{k}$}
  \label{rules:kinding}
\end{figure}


\subsection{Type inference}

We reformulate the HM(X) type inference in the context of our affine type
system. The main difference compared to HM(X) are noted in \addlin{blue}.
We note $\inferW{\addlin{\Sigma}}{(C,\unif)}{\bf{\E}}{\bf{e}}{\tau}$ when
$e$\ as type $\tau$ in $\E$ under the constraints $C$ and unifier $\unif$.
$\Sigma$ is a map which associates free variables in $e$ to
their kinds.
As before, $\E$ and $e$ are the input parameters of the inference
algorithm. The syntax-directed rules are shown in \cref{rules:typing}.

\begin{figure*}[h]
  \begin{mathpar}
  \inferrule[Scheme]{
    \inferK{C \Cand C_x} \E \tau {k'} \\
    \entail C {\Cleq{k'}k}
  }{
    \entail C {(\forall \kvar_i \forall (\tvar_j:k_j).\
      \qual{C_x}{\tau}) \le  k}
  }
\end{mathpar}
\hrulefill
\begin{mathpar}
  \ruleIVar
  
  \ruleIAbs

  \ruleIApp

  \ruleILet

  \ruleIPair

  \ruleIMatch

  \ruleIBorrow

  \ruleIRegion
\end{mathpar}

% \begin{align*}
%   \Weaken(x,\Sv)
%   &\equiv \begin{cases}
%     \operatorname{kind}(x)\lk\kun &\text{if } \operatorname{kind}(x)\in\Sv\\
%     \Cempty &\text{otherwise}
%   \end{cases}\\
%   \Cleq{\Sv}{k}
%   &\equiv \bigwedge_{\kvar\in\Sv} \Cleq{\kvar}{k}
% \end{align*}

%%% Local Variables:
%%% mode: latex
%%% TeX-master: "../main"
%%% End:

  \caption{Type Inference rules -- $\inferW{\Sigma}{(C,\psi)}{\bf{\E}}{\bf{e}}{\tau}$ }
  \label{rules:typing}
\end{figure*}

\subsection{Constraint normalization and generalization}

\TODO{}

\subsection{Principality}

\TODO{}

\begin{itemize}
\item Properly ensure that it respects HM(X) (cylindric, \dots)
\item Show principal type inference:
  \begin{itemize}
  \item Principal constraint system
  \item Regular constraint system: $\Ceq{\tau}{\tau'} \implies \fv{\tau} = \fv{\tau'}$.
  \item Solved forms are in simplified form.
    $C\in S, \entail{C}{\Ceq{\tau}{\tau'}} \implies \entail{}{\Ceq{\tau}{\tau'}}$.
  \end{itemize}
\item Show equivalence with the logic-based system.

\end{itemize}


\begin{lemma}
  Without loss of generality, we can consider that
  kind inequalities in satisfiable constraints
  are only done on kind variables. 

  \begin{proof}
    Consider the constraint $\Cleq{k}{k'}\Cand C$.
    \begin{itemize}
    \item If $k$ and $k'$ are both constants, it can be removed.
    \item If the constraint is $\Cleq{\kvar}{\kaff}$ or $\Cleq{\kun}{\kvar}$, it can be removed.
    \item If the constraint is $\Cleq{\kvar}{\kun}$ or $\Cleq{\kaff}{\kvar}$, we
      can substitute $\kvar$ by its value in $C$.
    \end{itemize}
  \end{proof}
\end{lemma}

%%% Local Variables:
%%% mode: latex
%%% TeX-master: "../main"
%%% End:

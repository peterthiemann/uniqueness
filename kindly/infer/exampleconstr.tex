We now demonstrate the constraint solving
algorithm presented in \cref{infer:solving}
on the expression $\lam{f}{\lam{x}{(\app{f}{x},x)}}$.
The application of our inference algorithm yields the following constraints:
%
\begin{align*}
  \E &= (\tvar_f : \kvar_f)
  (\tvar_x : \kvar_x)\dots\\
  C &= (\tvar_f \leq \gamma \tarr{\kvar_1} \beta )
  \Cand
  (\gamma \leq \tvar_x)
  \Cand
  (\beta * \tvar_x \leq \alpha_r)
  \Cand
  (\kvar_x \leq \kun)
\end{align*}

The first step of the algorithm is to use Herbrand unification to obtain
a type skeleton. We thus obtain
$$\alpha_r =
(\gamma \tarr{\kvar_3} \beta) \tarr{\kvar_2} \gamma \tarr{\kvar_1} \beta * \gamma$$

and the following kind constraints: 
\[\begin{aligned}
  \Cproj{\kvar_r}\Cproj{\kvar_f}\Cproj{\kvar_x}{}&
  (\kvar_x \leq \kun)
  \Cand
  (\kvar_\gamma \leq \kvar_x)
  \Cand
  (\kvar_x \leq \kvar_r)
  \Cand\\
  &(\kvar_\beta \leq \kvar_r)
  \Cand
  (\kvar_3 \leq \kvar_f)
  \Cand
  (\kvar_f \leq \kvar_1)
\end{aligned}\]
Since $\kvar_r$, $\kvar_f$ and $\kvar_x$ are not present in the type skeleton,
they are existentially quantified.

We translate these constraints into a relation whose graph
is shown in \cref{example:graph}.
%
The algorithm then proceeds as follow:
\begin{itemize}[noitemsep]
\item[1] From the constraints above, we deduce the graph shown
  with plain arrows.
\item[2] We add all the dashed arrows by saturating
  lattice inequalities. For clarity, we only show $\kun$.
\item[4] We identify the connected component circled in
  {\color{green} green}.
  We deduce $\kvar_\gamma = \kvar_x = \kun$
\item[5] We take the transitive closure, which adds the
  arrow in {\color{blue} blue} from $\kvar_1$ to $\kvar_3$.
\item[6] We remove the remaining existentially quantified node in {\color{red} red}, $\kvar_r$ and $\kvar_f$.
\item[7, 8, 9] We clean up the graph, and obtain the graph shown
  in \cref{example:graph:final}.
  We deduce $\kvar_3 \leq \kvar_1$.
\end{itemize}

The final constraint is thus
$$\kvar_\gamma = \kvar_x = \kun \Cand \kvar_3 \leq \kvar_1$$
If we were to generalize, we would obtain the type scheme:
$$\forall \kvar_\beta \kvar_1 \kvar_2 \kvar_3
(\gamma : \kun) (\beta : \kvar_\beta).\ %
\qual
{\Cleq{\kvar_3}{\kvar_1}}
{(\gamma \tarr{\kvar_3} \beta) \tarr{\kvar_2} \gamma \tarr{\kvar_1} \beta * \gamma}$$

The algorithm we presented does not use variance, however we can use it to
simplify this type further by noticing that $\kvar_1$ and $\kvar_2$ are only
used in covariant position, and can thus be replaced by their lower bound.
By removing the unneeded quantifier, we obtain the equivalent type:
$$
\forall \kvar
(\gamma : \kun).
{(\gamma \tarr{\kvar} \beta) \tarr{} \gamma \tarr{\kvar} \beta * \gamma}$$

\begin{figure}[tbp]
  \centering
  \begin{tikzpicture}[every edge/.style=[link,->,>=latex,thick]]
    \node (U) {$\kun$} ;
    \node[above left=of U] (x) {$\kvar_x$} ;
    \node[above=of x] (r) {$\kvar_r$};
    \node[left=of x] (g) {$\kvar_\gamma$} ;
    \node[right=of x] (b) {$\kvar_\beta$} ;
    \node[right=of b] (3) {$\kvar_3$} ;
    \node[above=of 3] (f) {$\kvar_f$} ;
    \node[above=of f] (1) {$\kvar_1$} ;

    \draw (x) to[bend right] (U) ;
    \draw (x) -> (r) ;
    \draw (b) -> (r) ;
    \draw (g) -> (x) ;
    \draw (3) -> (f) ;
    \draw (f) -> (1) ;

    \draw[blue] (3) to[bend right] (1);
    
    \begin{scope}[dashed,gray]
      \draw (U) to[bend right] (x) ;
      \draw (U) to (b) ;
      \draw (U) to[bend left] (g) ;
      \draw (U) to (r) ;
      \draw (U) to (1) ;
      \draw (U) to (3) ;
      \draw (U) to (f) ;
    \end{scope}
  
    \begin{scope}[on background layer]
      \node[fill=green!20,draw=green,
      inner sep=-1mm,ellipse,rotate fit=-20,fit=(U) (x) (g)] {};
      \node[fill=red!20,draw=red, inner sep=-2pt, circle,fit=(r)] {};
      \node[fill=red!20,draw=red, inner sep=-2pt, circle,fit=(f)] {};
    \end{scope}
  \end{tikzpicture}
    \caption{Graph representing the example constraints}
    \label{example:graph}
    
  \begin{tikzpicture}[every edge/.style=[link,->,>=latex,thick]]
    \node (b) {$\kvar_\beta$} ;
    \node[right=of b] (3) {$\kvar_3$} ;
    \node[above=of 3] (f) {} ;
    \node[above=of f] (1) {$\kvar_1$} ;

    \draw (3) to[bend right] (1);
    
  \end{tikzpicture}
    \caption{Final state of the graph}
    \label{example:graph:final}
\end{figure}
  

%%% Local Variables:
%%% mode: latex
%%% TeX-master: "../main"
%%% End:

\label{solving:example}

\begin{figure}[tbp]
  \centering
  \begin{tikzpicture}[node distance=6mm,xscale=1.2,yscale=0.7,.every edge/.style=[link,->,>=latex,thick]]
    \node (U) {$\kun$} ;
    \node[above left=of U] (x) {$\kvar_x$} ;
    \node[above=of x] (r) {$\kvar_r$};
    \node[left=of x] (g) {$\kvar_\gamma$} ;
    \node[right=of x] (b) {$\kvar_\beta$} ;
    \node[right=of b] (3) {$\kvar_3$} ;
    \node[above=of 3] (f) {$\kvar_f$} ;
    \node[above=of f] (1) {$\kvar_1$} ;

    \draw (x) to[bend right] (U) ;
    \draw (x) -> (r) ;
    \draw (b) -> (r) ;
    \draw (g) -> (x) ;
    \draw (3) -> (f) ;
    \draw (f) -> (1) ;

    \draw[blue] (3) to[bend right] (1);

    \node at (-0.5,-0.5) {Before};
    
    \begin{scope}[dashed,gray]
      \draw (U) to[bend right] (x) ;
      \draw (U) to (b) ;
      \draw (U) to[bend left] (g) ;
      \draw (U) to (r) ;
      \draw (U) to (1) ;
      \draw (U) to (3) ;
      \draw (U) to (f) ;
    \end{scope}
  
    \begin{scope}[on background layer]
      \node[fill=green!20,draw=green,
      inner sep=-1pt,ellipse,rotate fit=-20,fit=(U.south) (x) (g)] {};
      \node[fill=red!20,draw=red, inner sep=-2pt, circle,fit=(r)] {};
      \node[fill=red!20,draw=red, inner sep=-2pt, circle,fit=(f)] {};
    \end{scope}
  \end{tikzpicture}~
  \vrule~
  \begin{tikzpicture}[node distance=7mm,every edge/.style=[link,->,>=latex,thick]]
    \node (b) {$\kvar_\beta$} ;
    \node[right=of b] (3) {$\kvar_3$} ;
    \node[above=of 3] (f) {} ;
    \node[above=of f] (1) {$\kvar_1$} ;

    \draw (3) to[bend right] (1);

    \node at (0.5,-1.5) {After};
  \end{tikzpicture}
  \vspace{-5pt}
    \caption{Graph representing the example constraints}
    \label{example:graph}
    % \caption{Final state of the graph}
    \label{example:graph:final}
\end{figure}

Consider the expression $\lam{f}{\lam{x}{(\app{f}{x},x)}}$.
The inference algorithm yields the following constraints:
%
\begin{align*}
  \E &= (\tvar_f : \kvar_f)
  (\tvar_x : \kvar_x)\dots\\
  C &= (\tvar_f \leq \gamma \tarr{\kvar_1} \beta )
  \Cand
  (\gamma \leq \tvar_x)
  \Cand
  (\beta * \tvar_x \leq \alpha_r)
  \Cand
  (\kvar_x \leq \kun)
\end{align*}

The first step of the algorithm uses Herbrand unification to obtain
a type skeleton. 
$$\alpha_r =
(\gamma \tarr{\kvar_3} \beta) \tarr{\kvar_2} \gamma \tarr{\kvar_1} \beta * \gamma$$

In addition, we obtain the following kind constraints: 
\[\begin{aligned}
    % \Cproj{\kvar_r}\Cproj{\kvar_f}\Cproj{\kvar_x}{}
    &(\kvar_x \leq \kun)
    \Cand
    (\kvar_\gamma \leq \kvar_x)
    \Cand
    (\kvar_x \leq \kvar_r)
    \Cand
    (\kvar_\beta \leq \kvar_r)
    \Cand
    (\kvar_3 \leq \kvar_f)
    \Cand
    (\kvar_f \leq \kvar_1)
\end{aligned}\]
% Since $\kvar_r$, $\kvar_f$ and $\kvar_x$ are not present in the type skeleton,
% they are existentially quantified.

We translate these constraints into a relation whose graph
is shown in \cref{example:graph}.
%
The algorithm then proceeds as follow:
\begin{itemize}[noitemsep]
\item From the constraints above, we deduce the graph shown
  with plain arrows on the left of \cref{example:graph}.
\item We add all the dashed arrows by saturating
  lattice inequalities. For clarity, we only show $\kun$.
\item We identify the connected component circled in
  {\color{green} green}.
  We deduce $\kvar_\gamma = \kvar_x = \kun$.
\item We take the transitive closure, which adds the
  arrow in {\color{blue} blue} from $\kvar_3$ to $\kvar_1$.
\item We remove the remaining nodes not present in the type skeleton (colored in {\color{red} red}): $\kvar_r$ and $\kvar_f$.
\item We clean up the graph (transitive reduction, remove unneeded constants, \dots),
  and obtain the graph shown on the right.
  We deduce $\kvar_3 \leq \kvar_1$.
\end{itemize}

The final constraint is thus
$$\kvar_\gamma = \kvar_x = \kun \Cand \kvar_3 \leq \kvar_1$$
If we were to generalize, we would obtain the type scheme:
$$\forall \kvar_\beta \kvar_1 \kvar_2 \kvar_3
(\gamma : \kun) (\beta : \kvar_\beta).\ %
\qual
{\Cleq{\kvar_3}{\kvar_1}}
{(\gamma \tarr{\kvar_3} \beta) \tarr{\kvar_2} \gamma \tarr{\kvar_1} \beta * \gamma}$$

We can further simplify this type by exploiting variance. As $\kvar_1$
and $\kvar_2$ are only used in covariant position, they can be
replaced by their lower bounds, $\kvar_3$ and $\kun$. 
By removing the unused quantifiers, we obtain a much simplified equivalent type:
$$
\forall \kvar
(\gamma : \kun).
{(\gamma \tarr{\kvar} \beta) \tarr{} \gamma \tarr{\kvar} \beta * \gamma}$$



%%% Local Variables:
%%% mode: latex
%%% TeX-master: "../main"
%%% End:
